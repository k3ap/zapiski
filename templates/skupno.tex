\usepackage{polyglossia}
\usepackage{amsmath}
\usepackage{amsthm}
\usepackage{amssymb}
\usepackage{parskip}
\usepackage{tikz}
\usepackage{tikz-cd}
\usepackage{hyperref}
\usepackage{esint}
\usepackage{siunitx}
\usepackage{nicefrac}
\usepackage[autostyle]{csquotes}
\usepackage{enumitem}
\usepackage{algorithm}
\usepackage{algpseudocode}
\usepackage{bbm}
\usepackage{quiver}
\usepackage{stmaryrd}

\DeclareQuoteAlias{german}{slovene}

\usetikzlibrary{patterns}
\usetikzlibrary{positioning}
\usetikzlibrary{decorations.pathmorphing}
\usetikzlibrary{decorations.pathreplacing}
\usetikzlibrary{quotes}
\usetikzlibrary{angles}

\setmainlanguage{english}
\setotherlanguage{slovene}

\setlength{\parindent}{0pt}

%%% vprašanja %%%
\newcounter{vprasanja}

\newcommand{\predmet}[1]{
  \chapter{#1}
  \setcounter{vprasanja}{0}
  \newpage
}

\newcommand{\naslov}[1]{
  \section{#1}
}

\newcommand{\podnaslov}[1]{
  \subsection{#1}
}

\newcommand{\vprasanje}[1]{
  \stepcounter{vprasanja}
  \iflanguage{slovenian}{
	  {\textbf{Vprašanje \arabic{vprasanja}.} #1}
	}{
	  {\textbf{Question \arabic{vprasanja}.} #1}
	}
}

% prazna vrstica tu je potrebna, da \vspace deluje pravilno
\newcommand{\odgovorstart}{
  
  \vspace*{-5mm}
  \iflanguage{slovenian}{
	  \textit{Odgovor:}
	}{
	  \textit{Answer:}
	}
}

\newcommand{\odgovorend}{
  $\boxtimes$
}

\newcommand{\odgovor}[1]{
  \odgovorstart{}
  #1
  \odgovorend{}
}

\newenvironment{vo}[1]
			   {\vprasanje{#1}\\\odgovorstart{}}
			   {\odgovorend{}}


\theoremstyle{plain}
\newtheorem{izrek}{Izrek}[section]
\newtheorem{trditev}[izrek]{Trditev}
\newtheorem{posledica}[izrek]{Posledica}
\newtheorem{lema}[izrek]{Lema}

\newtheorem{theorem}[izrek]{Theorem}
\newtheorem{proposition}[izrek]{Proposition}
\newtheorem{corollary}[izrek]{Corollary}
\newtheorem{lemma}[izrek]{Lemma}

\theoremstyle{definition}
\newtheorem{definicija}[izrek]{Definicija}

\newtheorem{definition}[izrek]{Definition}

\theoremstyle{remark}
\newtheorem*{opomba}{Opomba}
\newtheorem*{primer}{Primer}

\newtheorem*{remark}{Remark}
\newtheorem*{example}{Example}

% standardne množice
\newcommand{\N}{\mathbb{N}}
\newcommand{\Z}{\mathbb{Z}}
\newcommand{\Q}{\mathbb{Q}}
\newcommand{\R}{\mathbb{R}}
\newcommand{\C}{\mathbb{C}}
\newcommand{\Kvaternioni}{\mathbb{H}}
\newcommand{\F}{\mathbb{F}}  % splošno polje
\newcommand{\Prime}{\mathbb{P}}

% projektivni prostori
\newcommand{\RP}[1]{\mathbb{RP}^{#1}}
\newcommand{\CP}[1]{\mathbb{CP}^{#1}}
\newcommand{\HP}[1]{\mathbb{HP}^{#1}}
\newcommand{\FP}[1]{\mathbb{FP}^{#1}}

% logika
\renewcommand{\iff}{\Leftrightarrow}
\newcommand{\such}{\,|\,}
\renewcommand{\complement}{^\mathsf{c}}
\DeclareMathOperator{\id}{id}
\newcommand{\comp}[1]{\overline{#1}}

% algebra
\DeclareMathOperator{\End}{End}
\DeclareMathOperator{\Aut}{Aut}
\DeclareMathOperator{\jedro}{ker}
\DeclareMathOperator{\im}{im}
\DeclareMathOperator{\rang}{rang}
\DeclareMathOperator{\dimenzija}{dim}
\newcommand{\sk}[1]{\left< #1 \right>}  % skalarni produkt
\newcommand{\norm}[1]{\left\lVert#1\right\rVert}
\newcommand{\normiran}[1]{\frac{#1}{\left\lVert#1\right\rVert}}
\DeclareMathOperator{\sled}{sl}
\DeclareMathOperator{\GL}{GL}
\DeclareMathOperator{\Hom}{Hom}
\newcommand{\edinka}{\triangleleft}
\DeclareMathOperator{\Sym}{Sym}
\DeclareMathOperator{\Alt}{Alt}
\DeclareMathOperator{\Gal}{Gal}

% analiza
\newcommand{\zvodv}[2]{\mathcal{C}^{#1}(#2)}
\newcommand{\zvodvi}[1]{\mathcal{C}^{#1}}
\newcommand{\holo}[1]{\mathcal{O}(#1)}
\newcommand{\vnabla}{\vec{\nabla}}
\renewcommand{\div}{\vnabla{} \cdot}
\newcommand{\grad}{\vnabla{} .}
\newcommand{\rot}{\vnabla{} \times}
\newcommand{\konj}[1]{\overline{#1}}
\DeclareMathOperator{\Real}{Re}
\DeclareMathOperator{\Imag}{Im}
\DeclareMathOperator{\Res}{Res}

% splošnejša topologija
\newcommand{\zvezne}[1]{\mathcal{C}(#1)}
\newcommand{\zveznei}{\mathcal{C}}
\newcommand{\Ti}[1]{\text{T}_{#1}}
\newcommand{\odp}{^\text{\normalfont odp}}
\newcommand{\zap}{^\text{\normalfont zap}}
\newcommand{\odpp}{^\text{\normalfont odp}\subseteq}
\newcommand{\zapp}{^\text{\normalfont zap}\subseteq}
\newcommand{\cl}[1]{\overline{#1}}

% misc
\newcommand{\abs}[1]{\left| #1 \right|}
\newcommand{\pol}{\frac{1}{2}}
\newcommand{\inv}[1]{\frac{1}{#1}}
\DeclareMathOperator{\sgn}{sgn}
\newcommand{\protislovje}{$\rightarrow\!\leftarrow$}
\newcommand{\nastevanje}[3]{{#1}_1 #2 {#1}_2 #2 \ldots #2 {#1}_#3}
\newcommand{\poudarjenanegacija}[1]{\underline{#1}}
\newcommand{\poudarjenne}{\poudarjenanegacija{ne}}
\newcommand{\pojem}[1]{\textsc{#1}}
\newcommand{\zo}[1]{\left[#1\right)}
\newcommand{\oz}[1]{\left(#1\right]}
\DeclareMathOperator*{\argmin}{arg\,min}
\newcommand{\pvec}[1]{\vec{#1}\mkern2mu\vphantom{#1}}
\newcommand{\ceil}[1]{\lceil #1 \rceil}

% verjetnost in statistika
\DeclareMathOperator{\var}{var}
\DeclareMathOperator{\cov}{cov}

\author{Patrik Žnidaršič}
\date{\today}

