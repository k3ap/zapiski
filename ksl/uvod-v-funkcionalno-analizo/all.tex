\naslov{Normirani prostori}

Vektorski prostor $X$ nad poljem $\F \in \{\R, \C\}$ je \pojem{normiran}, če ima
definirano normo, torej preslikavo $\norm{\cdot}: X \to \zo{0, \infty}$ z
naslednjimi lastnostmi:
\begin{itemize}
\item $\norm{x} = 0 \iff x = 0$ za vsak $x \in X$,
\item $\norm{\lambda x} = \abs{\lambda} x$ za vsak $\lambda \in \F$ in $x \in
  X$,
\item $\norm{x+y} \le \norm{x} + \norm{y}$ za vsaka $x, y \in \F$.
\end{itemize}
Prostor je \pojem{Banachov}, če je normiran in poln za metriko, porojeno z
normo, $d(x,y) = \norm{x-y}$.

Poznamo nekaj standardnih primerov normiranih prostorov:
\begin{itemize}
\item Za $1 \le p < \infty$ je
  \[
	l^p = \left\{ (x_n)_n \in \C^\N \such \sum_{n \in \N} \abs{x_n}^p < \infty
	\right\},
  \]
  opremljen z normo
  \[
	\norm{(x_n)_n}_p = \left( \sum_{n \in \N} \abs{x_n}^p \right)^p
  \]
  Banachov prostor.
\item Prostor
  \[
	l^\infty = \left\{ (x_n)_n \in \C^\N \such \sup_n \abs{x_n} < \infty \right\},
  \]
  opremljen z normo
  \[
	\norm{(x_n)_n}_\infty = \sup_n \abs{x_n},
  \]
  Banachov prostor.
\item Imamo tudi nekaj standardnih podprostorov $l^\infty$;
  \begin{gather*}
	c = \{(x_n)_n \in l^\infty \such \text{$(x_n)_n$ je konvergentno
	  zaporedje}\} \\
	c_0 = \{(x_n)_n \in l^\infty \such \lim x_n = 0\} \\
	c_{00} = \{ (x_n)_n \in l^\infty \such \text{rep $(x_n)_n$ je konstantno
	  enak $0$} \}
  \end{gather*}
  Prva dva prostora sta Banachova, zadnji pa ni.
  Njegovo zaprtje glede na $l^p$-normo je $l^p$, glede na $l^\infty$-normo pa je
  $c_0$.
\end{itemize}

Pri dokazovanju dejstev o $l^p$ prostorih prideta prav sledeči neenakosti.

\begin{trditev}[Hölderjeva neenakost]
  Če za $1 < p, q < \infty$ velja $p^{-1} + q^{-1} = 1$, potem za poljubna $x
  \in l^p$ ter $y \in l^q$ velja
  \[
	\sum_{n=1}^\infty \abs{x_n y_n} \le \norm{x}_p \norm{y}_q.
  \]
\end{trditev}

\begin{trditev}[Minkowski]
  Če je $1 \le p < \infty$ ter $x, y \in l^p$, potem velja $\norm{x+y}_p =
  \norm{x}_p + \norm{y}_p$.
\end{trditev}

Slednja trditev je seveda le trikotniška neenakost za $\norm{\cdot}_p$.
Ker med drugim obravnavamo Banachove prostore, marsikaj delamo z zaporedji, in
se je vredno spomniti naslednjega dejstva iz topologije: (topološki) podprostor
je zaprt natanko tedaj, ko ima vsako konvergentno zaporedje z elementi iz tega
podprostora tudi limito v tem podprostoru.
Iz tega lahko enostavno pokažemo naslednjo trditev.

\begin{trditev}
  Naj bo $Y$ vektorski podprostor v normiranem prostoru $X$.
  Če je $Y$ poln, je zaprt v $X$.
  Če je $X$ Banachov, je $Y$ Banachov natanko tedaj, ko je zaprt v $X$.
\end{trditev}

% LocalWords:  Hölderjeva Minkowski Banachove
