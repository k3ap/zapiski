\naslov{Introduction}

A \pojem{Turing machine} is defined to consist of the following components.
There is an infinite tape divided into cells, each of which contains a symbol
from the chosen alphabet $\Gamma$.
This alphabet must include a \texttt{blank} symbol.
At the start, only a finite number of cells in the tape have a character
different than \texttt{blank}.
The machine also possesses a read-write head positioned at some cell, and an
internal control state, which determines the instruction to be followed.

Instructions are given as a \pojem{transition} (partial) function $f$, which
maps
\[
  (\text{state}, \text{character}) \mapsto (\text{new state}, \text{new
	character}, \text{motion}).
\]
To perform an action, a TM will look for rules matching its current control
state and the character currently written at the position of the read-write
head.
When a matching rule is found, the machine switches to the defined new state,
writes the specified character on the tape and moves according to the
instruction.
We limit its motion to three possibilities: one cell to the left or right, or
no motion at all.
More formally, we may restate the definition as follows:

\begin{definition}
  A \pojem{Turing machine} is specified by the following:
  \begin{itemize}
  \item a finite \pojem{tape alphabet} $\Gamma$ with $\boxdot \in \Gamma$,
  \item a finite set $Q$ of \pojem{states} with $\mathtt{start} \in Q$,
  \item a transition partial function
	\[
	  \delta: Q \times \Gamma \rightharpoonup Q \times \Gamma \times \{-1, 0, +1\}.
	\]
  \end{itemize}
\end{definition}

For a given input alphabet $\Sigma_1 \subseteq \Gamma$ and output alphabet
$\Sigma_2 \subseteq \Gamma$, a TM specifies a partial function $f: \Sigma_1^*
\rightharpoonup \Sigma_2^*$ if for any $w \in \Sigma_1^*$, running the TM on the
input $w$ results in the machine halting in the \texttt{halt} state and it has
$f(w)$ as the word on the tape, with the head at the leftmost character.
We also require $\boxdot \notin \Sigma_1$ or $\Sigma_2$.
If the machine does not halt in the \texttt{halt} state, we say $f(w)$ is not
defined.

\begin{definition}
  A partial function $f: \Sigma_1^* \rightharpoonup \Sigma_2^*$ is
  \pojem{computable} if there exists a TM that computes it.
\end{definition}

\vprasanje{Describe the basic working of a Turing machine. What does it mean for
  a function to be computable?}

\begin{definition}
  A \pojem{language} over an alphabet $\Sigma$ is a subset $L \subseteq
  \Sigma^*$.
\end{definition}

For language recognition we require a subset $\Sigma \subseteq \Gamma$ (the
\pojem{input alphabet}) and two distinguished states, \texttt{accept} and
\texttt{reject}.
A language-recognizing Turing machine accepts a word $w$ if, when the machine is
run on the input $w$, the computation halts in the \texttt{accept} state.
Similarly, $w$ is rejected if the machine halts in the \texttt{reject} state.
We say that a Turing machine $M$ \pojem{decides} or \pojem{computes} $L$ if for
any $w \in \Sigma^*$, $w \in L$ implies that $M$ accepts $w$ and $w \notin L$
implies that $M$ rejects $w$.

If $M$ correctly accepts words of the language, and does not accept words that
are not part of the language, we say that $M$ \pojem{semidecides},
\pojem{semicomputes} or \pojem{recognizes} $L$.
A language is \pojem{decidable} or \pojem{computable} if there is a TM which
decides it, and \pojem{semidecidable}, \pojem{semicomputable} or
\pojem{computably innumerable} if there is a TM which recognizes it.
Clearly, every decidable language is also semidecidable.

\vprasanje{What is a decidable and what is a semidecidable language?}

Given a $k$-tape machine $(\Gamma, Q, \delta)$, we can simulate it by a
single-tape machine.
We will encode the different tapes by separating them with a special symbol $|
\in \tilde{\Gamma}$ and encoding the positions of the read-write heads with
another special symbol $\Delta \in \tilde{\Gamma}$.
When simulating a computational step of the multi-tape machine, we scan for
transitions and implement them manually.

\vprasanje{How can you simulate a multi-tape Turing machine with a single-tape
  machine?}

\begin{proposition}
  There are languages that are not semidecidable.
\end{proposition}

\begin{proof}
  There are only countably many non-equivalent Turing machines, but an
  uncountable number of languages on any alphabet.
\end{proof}

We can also give an example of a language that is semidecidable but not
decidable.
To construct it, we will encode a Turing machine $M = (\Gamma, Q, \delta)$ into
a string.
We encode it in $\sk{M} \in \Sigma_u^*$ for the alphabet
\[
  \Sigma_u = \{0, 1, -1, \verb+[+, \verb+]+, \parallel, \cdot\}.
\]
We encode every state $q \in Q$ as a word $\sk{q} \in \{0, 1, -1\}^l$, where $l
\ge \log_3 \abs{Q}$.
We require that the encoding of the start state is $0^l$, the encoding of the
accepting state is $1^l$, and the encoding of the rejecting state is $(-1)^l$.
Every symbol $a \in \Gamma$ is encoded as $\sk{a} \in \{-1, 0, 1\}^m$ for $m \ge
\log_3 \abs{\Gamma}$.
We require that the encoding of the blank symbol is $0^m$.

Finally, to encode $\delta$, consider an instruction $(q, a) \mapsto (q', b,
d)$.
This will be encoded as a word
\[
  [\sk{q} \cdot \sk{a} \parallel \sk{q'} \cdot \sk{b} \cdot d]
\]
with $d \in \{0, 1, -1\}$.
This encoding has length $2l + 2m + 7$, and allows us to encode the full Turing
machine as the encoding of the start state, followed by a dot $\cdot$, followed
by the encoding of the blank symbol, and then followed by the encodings of all
transitions one after another.
We can encode any word $w \in \Gamma^*$ as a sequence of characters, delimited
by $\cdot$, so that we may define a language $L_{\text{accept}}$ as follows:
the language includes words of the form $\sk{M} \cdot \sk{w}$, where $M$ is a
single-tape Turing machine with tape alphabet $\Gamma \supseteq \Sigma_u$, and
$w \in \Sigma_u^*$ is a word which $M$ accepts.

\vprasanje{How is $L_{\text{accept}}$ defined? Describe the universal encoding
  of a Turing machine.}

\begin{theorem}
  The language $L_{\text{accept}}$ is undecidable.
\end{theorem}

\begin{proof}
  Suppose that it is decidable, so there exists a Turing machine $D$ which
  decides it.
  We define a new Turing machine $N$ with input alphabet $\Sigma_u$.
  This machine reads its input string $v$ and converts it to the string $v \cdot
  \sk{v}$.
  It then proceeds as $D$ on this input, except it switches the accept and
  reject states.

  Consider what $N$ does when given an input of the form $v = \sk{M}$ for some
  Turing machine $M$.
  If $D$ rejects $\sk{M} \cdot \sk{\sk{M}}$, then $N$ terminates on the
  accepting state, and vice versa.
  Because $D$ decides $L_{\text{accept}}$, we get from $N$:
  \begin{itemize}
  \item \texttt{accept} iff $M$ does not accept $\sk{M}$, and
  \item \texttt{reject} iff $M$ accepts $\sk{M}$.
  \end{itemize}
  Now run $N$ on $\sk{N}$.
  We get \texttt{accept} iff $N$ does not accept $\sk{N}$, and \texttt{reject}
  iff $N$ accepts $\sk{N}$.
  This is a contradiction.
\end{proof}

\vprasanje{Show that $L_{\text{accept}}$ is undecidable.}

Using $\Sigma_u$, we can also construct the \pojem{universal Turing machine}.

\begin{theorem}
  There exists a Turing machine $U$ over the tape alphabet $\Sigma_u \cup
  \{\boxdot\}$ that exhibits the following behavior:
  If we run $U$ on the string $\sk{M} \cdot \sk{w}$, then the resulting
  execution satisfies the following.
  \begin{itemize}
  \item It terminates if and only if $M$ terminates on input $w$.
  \item It accepts if and only if $M$ accepts $w$.
  \item It rejects if and only if $M$ rejects $w$.
  \end{itemize}
\end{theorem}

The idea of the proof is to use a three-tape machine, putting the encoding of
$M$ on the first tape, the encoding of the state on the second, and the input on
the third.
Then simulate the execution of $M$.

\begin{theorem}
  The language $L_{\text{accept}}$ is semidecidable.
\end{theorem}

\begin{proof}
  We construct a machine $S$ that does the following.
  It first reads its input word $v \in \Sigma_u^*$ and checks whether $v$ is of
  the form $\sk{M} \cdot \sk{w}$.
  If $v$ is not of this form, then we reject immediately.
  Otherwise, we run the universal machine $U$ on the input $v$ and end in the
  end state of $U$.
\end{proof}

\vprasanje{Show that $L_{\text{accept}}$ is semidecidable.}

\begin{proposition}
  If $f: \Sigma_1^* \rightharpoonup \Sigma_2^*$ and $g: \Sigma_2^*
  \rightharpoonup \Sigma_3^*$ are computable, then so if $g \circ f$.
\end{proposition}

Note that the composite of two partial functions is defined as
\[
  (g \circ f)(w) \simeq
  \begin{cases}
	g(f(w)) & f(w) \downarrow \\
	\uparrow & f(w) \uparrow
  \end{cases}
\]

\begin{definition}
  A $k$-tape Turing machine \pojem{computes} $f: \Sigma_1^* \times \cdots \times
  \Sigma_k^* \rightharpoonup \Sigma^*$ if when we run the Turing machine on the
  configuration with a word on each tape, the machine terminates in the halt
  state if and only if for all $(w_1, \ldots, w_k)$ in the domain of $f$, it
  halts in the configuration with $f(w_1, \ldots, w_k)$ on the first tape and
  all other tapes blank.
\end{definition}

\begin{example}
  To define the computability for partial functions $f: \N \rightharpoonup \N$,
  we can represent numbers using words over $\Sigma_b = \{0,1\}$ and the
  representation
  \[
	\gamma_\N(w) = \sum_{i=0}^{\abs{w}-1} 2^{\abs{w} -i -1} w_i.
  \]
  If we allow for leading zeros, every number is represented by an infinite
  number of words.
  Additionally, zero is also represented by the empty word $\varepsilon$.
  A Turing machine $M$ computes $f: \N \rightharpoonup \N$ if it computes $g:
  \Sigma_b^* \rightharpoonup \Sigma_b^*$ such that for all words $w$ for which
  $\gamma_\N(g(w)) \simeq f(\gamma_\N(w))$.
  We say that $f$ is \pojem{computable} if there is a Turing machine which
  computes it.
\end{example}

We could also restrict our representation, for example requiring words to begin
with a $1$ (we also allow the empty word).
Alternatively, we could use e.g.~a unary representation.

\begin{definition}
  A \pojem{representation} of a set $X$ by words over an alphabet $\Sigma$ is a
  surjective partial function $\gamma: \Sigma^* \rightharpoonup X$.
\end{definition}

\begin{remark}
  Only countable sets can be represented.
\end{remark}

\begin{definition}
  Given representations $\gamma_1: \Sigma_1^* \rightharpoonup X_1$ and
  $\gamma_2: \Sigma_2^* \rightharpoonup X_2$, a partial function $f: X_1
  \rightharpoonup X_2$ is \pojem{$(\gamma_1 \to \gamma_2)$-computable} if there
  exists a computable partial function $g: \Sigma_1^* \rightharpoonup \Sigma_2^*$
  such that for all words $w$ in the domain of $\gamma_1$, if $g(w)$ is defined,
  then $\gamma_2(g(w)) \simeq f(\gamma_1(w))$.
\end{definition}

\begin{definition}
  Two representations $\gamma_1$ and $\gamma_2$ of the same set $X$ are
  \pojem{equivalent} if the identity function $\id_X$ is $(\gamma_1 \to
  \gamma_2)$-computable and $(\gamma_2 \to \gamma_1)$-computable.
\end{definition}

Given representations $(\gamma_i : \Sigma_i^* \rightharpoonup X_i)_{i = 1,
  \ldots, k}$, we construct a product representation $\gamma: \Sigma^*
\rightharpoonup X_1 \times \cdots \times X_k$, where
\[
  \Sigma = (\Sigma_1 \cup \Sigma_2 \cup \ldots \cup \Sigma_k) \amalg \{,\}
\]
and
\[
  \gamma(w_1, \ldots, w_k) = (\gamma_1(w_1), \ldots, \gamma_k(w_k)).
\]

\begin{definition}
  A partial function $f: \N \rightharpoonup \N$ is \pojem{computable} if it is
  $(\gamma_\N \times \cdots \times \gamma_\N \to \gamma_\N)$-computable.
\end{definition}

\begin{definition}
  Given a representation $\gamma: \Sigma^* \rightharpoonup X$, a subset $A
  \subseteq X$ is \pojem{$\gamma$-decidable} if there exists a Turing machine
  $M$ such that for all $w$ in the domain of $\gamma$, if $\gamma(w) \in A$,
  then $M$ accepts $w$, and if $\gamma(w) \notin A$, then $M$ rejects $w$.
  The same subset is \pojem{$\gamma$-semidecidable} if there exists a Turing
  machine $M$ such that for all $w$ in the domain of $\gamma$, $M$ accepts $w$
  if and only if $\gamma(w) \in A$.
\end{definition}

\begin{proposition}
  Given a representation $\gamma$ of $X$ and $A \subseteq X$, $A$ is
  $\gamma$-decidable if and only if its characteristic function is
  $(\gamma \to \gamma_b)$-computable, and $A$ is $\gamma$-semidecidable if and
  only if its partial characteristic function is $(\gamma \to
  \gamma_b)$-computable.
\end{proposition}

\begin{remark}
  Above, $\gamma_b$ is the representation of $\{0,1\}$ which maps $0 \mapsto 0$
  and $1 \mapsto 1$.
\end{remark}