\naslov{Algorithmic randomness}

The law of large numbers states that for a sequence, generated randomly by coin
tosses,
\[
  P\!\left( \lim_{n \to \infty} \inv{n} \sum_{i=0}^{n-1} q_i = \pol \right)
  = 1.
\]
We say that a sequence $q$ satisfies the law of large numbers if
\[
  \lim_{n \to \infty} \inv{n} \sum_{i=0}^{n-1} q_i = \pol.
\]
If $q$ does not satisfy the law, then there exists a $\delta > 0$ such that
$\abs{m_n(q) - \pol} < \delta$ for infinitely many $n$, where
\[
  m_n(q) = \inv{n} \sum_{i=0}^{n-1} q_i.
\]
Given $\delta$, we then have the following non-randomness test.
Define $T_n$ as the set of all sequences $q \in \{0,1\}^\omega$ for which there
are at least $n$ different values of $i$ where $\abs{m_n(q) - \pol} > \delta$.
Note that we can semidecide $T_n$.

\begin{definition}
  A \pojem{naive non-randomness test} is a sequence $(T_n)_n$ of subsets of
  $\{0,1\}^\omega$ such that
  \begin{itemize}
  \item every $T_n$ is open,
  \item $\lim \lambda(T_n) = 0$ where $\lambda(T_n)$ is the probability that a
	fair randomly generated sequence lands in $T_n$.
  \end{itemize}
  Then $q \in 2^\omega$ \pojem{satisfies $(T_n)_n$} if $q \in \bigcap T_n$.
\end{definition}

\begin{definition}
  A \pojem{Martin-Löf non-randomness test} is a sequence $(T_n)_n$ of subsets of
  $2^\omega$ such that
  \begin{itemize}
  \item $(T_n)_n$ is a computable sequence of computably open sets,
  \item $\lim \lambda(T_n) = 0$ with a computable rate of convergence.
  \end{itemize}
  A sequence $q$ is \pojem{Martin-Löf random} if it fails every ML test.
\end{definition}

\begin{proposition}
  Computable sequences are not ML random.
\end{proposition}

\begin{proof}
  Let $p$ be a computable sequence.
  Then $(\sk{\left. p \right|_n})_n$ is an ML test which $p$ satisfies.
\end{proof}

\begin{definition}
  An ML test $(T_n^u)_n$ is \pojem{universal} if any non-ML random $p$ satisfies
  $(T_n^u)_n$.
\end{definition}

It holds that every open set $U \subseteq 2^\omega$ can be written as a disjoint
union of cylinder sets $\sk{w_i}$ for some $(w_i)_{i \in I}$.
Define $\lambda(\sk{w}) = 2^{-\abs{w}}$-
Then the probability of an open set is
\[
  \lambda(U) = \sum_{i \in I} \lambda(w_i)
\]
for the disjoint union above.
We can prove that $\lambda(U)$ is well-defined.

We represent open sets as infinite sequences $w_0, w_1, \ldots$ where every
$w_i$ is either a word in $2^*$ or the symbol $\varnothing$.
We say that an open set is \pojem{computable} if it is represented by some
computable infinite sequence $w_0, w_1, \ldots$.
Note that this definition coincides with the semidecidable sets.

Then a sequence of open sets $(T_n)_n$ is represented by a sequence of
sequences, $T_i \sim (w_{ij})_j$.
We say that $(T_n)_n$ is a \pojem{computable sequence of computable opens} if
the single sequence representing $w_{ij}$ above, using the pairing function, is
computable.

A computable total function $r: \N \to \N$ is a \pojem{computable rate of
  convergence} for a Cauchy sequence $(x_n)_n \subseteq [0,1]$ converging to $0$
if for any $n$ and $m \ge r(n)$, we have $x_m \le 2^{-n}$.

\begin{theorem}[Martin-Löf]
  There exists a universal ML test.
\end{theorem}

\begin{proof}
  Outline.
  We can show the following:
  \begin{itemize}
  \item There is a computable function from $\N$ to the representations of
	computable sequences of computably open sets, and it finds some
	representation of any sequence.
  \item There is a function from the computable sequences of computably open
	sets to ML tests, which preserves rapid ML tests in its domain
	(an ML test is rapid if $\lambda(T_n) \le 2^{-n}$).
  \item There is a function from ML tests to rapid decreasing ML tests which
	preserves the set of sequences that satisfy the test
	(an ML test is decreasing if $T_{n+1} \subseteq T_n$ for all $n$).
  \end{itemize}
  So a sequence $p$ is ML random if and only if it fails every rapid decreasing
  test.
  From all three points above, we have a computable enumeration of the rapid
  decreasing ML tests.
  It is surjective in the sense that for every rapid decreasing ML test, at
  least one representation is found.
  Let $(T_n^0)_n, (T_n^1)_n, \ldots$ be this enumeration.
  Take
  \[
	T_n^u = \bigcup_{i=0}^\infty T_{n+i+1}^i.
  \]
  This is clearly a rapid decreasing sequence, and it is computable.
\end{proof}

\begin{corollary}
  The set $R_{\text{ML}}$ of all ML random sequences is a countable union of
  closed sets ($F_\sigma$).
  Additionally, $\lambda(R_{\text{ML}}) = 1$ and $\abs{R_{\text{ML}}} \ge
  2^{\aleph_0}$.
\end{corollary}

\begin{proof}
  The complement of $R_{\text{ML}}$ is exactly the intersection of all $T_n^u$.
\end{proof}

\begin{theorem}
  The following are equivalent for any infinite sequence $q \in 2^\omega$.
  \begin{itemize}
  \item $q$ is ML random,
  \item $q$ is prefix-free incompressible.
  \end{itemize}
\end{theorem}

% LocalWords:  semidecide Löf incompressible
