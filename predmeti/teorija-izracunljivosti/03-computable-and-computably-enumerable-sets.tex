\naslov{Computable and computably enunmerable sets}

\begin{definition}
  A subset $A \subseteq \N$ is \pojem{computable} if the characteristic function
  $\chi_A: \N \to \N$ is computable.
  It is \pojem{computably enumerable} if the partial characteristic function
  $\chi_A^p: \N \to \N$ is computable as a partial function.
\end{definition}

\begin{definition}
  \pojem{Kleene's halting set} is the set
  \[
	K = \{e \in \N \such \phi_e(e) \downarrow \}.
  \]
\end{definition}

\begin{proposition}
  Kleene's halting set $K$ is not computable.
\end{proposition}

\begin{proof}
  Suppose it is.
  Then so is the partial function
  \[
	h(e) \simeq
	\begin{cases}
	  \uparrow, & \chi_K(e) = 1, \\
	  0, & \chi_K(e) = 0.
	\end{cases}
  \]
  Therefore there exists an index $e$ such that $h = \phi_e$.
  Then $\phi_e(e) \downarrow \iff e \in K \iff h(e) \uparrow \iff \phi_e(e)
  \uparrow$.
  This is a contradiction.
\end{proof}

\begin{proposition}
  A set $A \subseteq \N$ is computably enumerable if and only if it is the
  domain of some computable partial function.
\end{proposition}

\begin{proof}
  If a set $A$ is computable enumerable, it is the domain of $\chi_A^p$.
  Suppose that $A = \operatorname{dom}(f)$ for some computable $f: \N \to \N$.
  We can compute $\chi_A^p$ by
  \[
	\chi_A^p(n) \simeq
	\begin{cases}
	  1, & f(n) \downarrow, \\
	  \uparrow & \text{otherwise.} \qedhere
	\end{cases}
  \]
\end{proof}

\begin{proposition}
  The halting set $K$ is computably enumerable.
\end{proposition}

\begin{proof}
  It is the domain of the computable partial function $e \mapsto \phi_e(e) =
  \psi_u(e,e)$.
\end{proof}

We can enumerate the computably enumerable sets by
\[
  W_e := \operatorname{dom}(\phi_e).
\]

\begin{lemma}
  Suppose that $A \subseteq \N$ is a set for which there exists a total
  computable function $t: \N^2 \to \N$ such that $x \in A$ if and only if there
  exists $z \in \N$ with $t(x,z) = 0$.
  Then $A$ is computably enumerable.
  The reverse also holds.
\end{lemma}

\begin{proof}
  Suppose that $A$ is computably enumerable, so by Kleene's normal form theorem,
  $x \in A$ if and only if $T(e,x,z) = 0$ for some $z$, and for $A = W_e$.
  We can take $t(x,z) = T(e,x,z)$.

  As for the reverse, given a computable $t$, the partial function
  $\mu t: \N \to \N$  has $A$ as its domain, so $A$ is computably enumerable.
\end{proof}

\begin{theorem}
  A set $A \subseteq \N$ is computable if and only if both $A$ and $\cl{A}$ are
  computably enumerable.
\end{theorem}

\begin{proof}
  The left-to-right implication is trivial.
  Suppose both $A$ and $\cl{A}$ are computably enumerable.
  By the lemma, there exist total computable $s,s': \N \to \N$ such that $x \in
  A$ if and only if there exists $z \in \N$ for which $s(x,z) = 0$, and
  similarly $x \notin A$ if and only if there is a number $z$ such that $s'(x,z)
  = 0$.

  Then the following function is both computable and total:
  \[
	g := \mu ((x,z) \mapsto \min(s(x,z), s'(x,z))).
  \]
  So
  \[
	\chi_A(x) =
	\begin{cases}
	  1, & s(x,g(x)) = 0, \\
	  0, & \text{otherwise},
	\end{cases}
  \]
  is computable.
\end{proof}

\begin{corollary}
  The set $\cl{K}$ is not computably enumerable.
\end{corollary}

\begin{theorem}
  The following are equivalent for a set $A \subseteq \N$:
  \begin{itemize}
  \item $A$ is computably enumerable,
  \item $A = \varnothing$ or $A$ is the range of a total computable function,
  \item $A$ is the range of a computable partial function.
  \end{itemize}
\end{theorem}

\begin{proof}
  One to two:
  Suppose $A \ne \varnothing$, and select any $a_0 \in A$.
  By the above lemma, there exists a computable $s: \N^2 \to \N$ satisfying
  \[
	y \in A \iff \exists z. s(y,z) = 0.
  \]
  Using the pairing function $p: \N^2 \to \N$, we have that $A$ is the range of
  the total computable
  \[
	x \mapsto
	\begin{cases}
	  y, & s(y,z) = 0, \\
	  a_0 & \text{otherwise}.
	\end{cases}
  \]
  We used $(y, z) = p^{-1}(x)$ above.

  Two to three is trivial.
  Three to one:
  Suppose $A$ is the range of $\phi_e$.
  Let $T$ and $U$ be as in Kleene's normal form theorem.
  Then the total function $s: \N^2 \to \N$ is computable:
  \[
	s(w,y) =
	\begin{cases}
	  0 & T(e, x, z) = 0  \land w = U(z), \\
	  1 & \text{otherwise},
	\end{cases}
  \]
  where $y = p(x,z)$.
  Then $w \in A$ if and only if there exists a $y$ such that $s(w,y) = 0$.
\end{proof}

\begin{theorem}
  The following are equivalent for a set $A \subseteq \N$:
  \begin{itemize}
  \item $A$ is computable,
  \item $A = \varnothing$ or $A$ is the range of an increasing total computable
	function,
  \item $A$ is finite or $A$ is the range of a strictly increasing computable
	total function.
  \end{itemize}
\end{theorem}

\begin{proof}
  We will only prove the equivalence of the first and third statement.
  One to three:
  Suppose $A$ is computable and infinite.
  Then $A$ is the image of the function which maps $n$ to the $n$-th smallest
  element of $A$, by searching through $m = 0, 1, 2, \ldots$ using $\chi_A$.

  Three to one:
  Suppose $A$ is the range of $f: \N \to \N$, which is strictly increasing.
  Then we can compute $\chi_A(n)$ by checking whether $n$ is in the image
  of $f(x)$ for $x = 0, 1, 2, \ldots$, until we find either $n$ or a number
  larger than it.
\end{proof}

\begin{corollary}
  Every infinite computably enumerable set has an infinite computable subset.
\end{corollary}

\begin{proof}
  Let $A$ be infinite and computably enumerable.
  Then $A$ is the image of a computable total function $f$.
  Define $g(0) = f(0)$ and $g(n+1) = f(m)$ where $m$ is the smallest number such
  that $f(m) > g(n)$.
\end{proof}

% LocalWords:  Kleene's
