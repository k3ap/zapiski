\naslov{Computable and computably enunmerable sets}

\begin{definition}
  A subset $A \subseteq \N$ is \pojem{computable} if the characteristic function
  $\chi_A: \N \to \N$ is computable.
  It is \pojem{computably enumerable} if the partial characteristic function
  $\chi_A^p: \N \to \N$ is computable as a partial function.
\end{definition}

\begin{definition}
  \pojem{Kleene's halting set} is the set
  \[
	K = \{e \in \N \such \phi_e(e) \downarrow \}.
  \]
\end{definition}

\begin{proposition}
  Kleene's halting set $K$ is not computable.
\end{proposition}

\begin{proof}
  Suppose it is.
  Then so is the partial function
  \[
	h(e) \simeq
	\begin{cases}
	  \uparrow, & \chi_K(e) = 1, \\
	  0, & \chi_K(e) = 0.
	\end{cases}
  \]
  Therefore there exists an index $e$ such that $h = \phi_e$.
  Then $\phi_e(e) \downarrow \iff e \in K \iff h(e) \uparrow \iff \phi_e(e)
  \uparrow$.
  This is a contradiction.
\end{proof}

\vprasanje{What are computable and what are computably enumerable sets? Define
  Kleene's halting set and show it is not computable.}

\begin{proposition}
  A set $A \subseteq \N$ is computably enumerable if and only if it is the
  domain of some computable partial function.
\end{proposition}

\begin{proof}
  If a set $A$ is computably enumerable, it is the domain of $\chi_A^p$.
  Suppose that $A = \dom(f)$ for some computable $f: \N \to \N$.
  We can compute $\chi_A^p$ by
  \[
	\chi_A^p(n) \simeq
	\begin{cases}
	  1, & f(n) \downarrow, \\
	  \uparrow & \text{otherwise.}
	\end{cases}
	\qedhere
  \]
\end{proof}

\vprasanje{Show that a set is computably enumerable if and only if it is the
  domain of some computable partial function.}

\begin{proposition}
  The halting set $K$ is computably enumerable.
\end{proposition}

\begin{proof}
  It is the domain of the computable partial function $e \mapsto \phi_e(e) =
  \psi_u(e,e)$.
\end{proof}

We can enumerate the computably enumerable sets by
\[
  W_e := \dom(\phi_e).
\]

\begin{lemma}
  Suppose that $A \subseteq \N$ is a set for which there exists a total
  computable function $t: \N^2 \to \N$ such that $x \in A$ if and only if there
  exists $z \in \N$ with $t(x,z) = 0$.
  Then $A$ is computably enumerable.
  The reverse also holds.
\end{lemma}

\begin{proof}
  Suppose that $A$ is computably enumerable, so by Kleene's normal form theorem,
  $x \in A$ if and only if $T(e,x,z) = 0$ for some $z$, and for $A = W_e$.
  We can take $t(x,z) = T(e,x,z)$.

  As for the reverse, given a computable $t$, the partial function
  $\mu t: \N \to \N$  has $A$ as its domain, so $A$ is computably enumerable.
\end{proof}

\begin{theorem}
  A set $A \subseteq \N$ is computable if and only if both $A$ and $\cl{A}$ are
  computably enumerable.
\end{theorem}

\begin{proof}
  The left-to-right implication is trivial.
  Suppose both $A$ and $\cl{A}$ are computably enumerable.
  By the lemma, there exist total computable $s,s': \N \to \N$ such that $x \in
  A$ if and only if there exists $z \in \N$ for which $s(x,z) = 0$, and
  similarly $x \notin A$ if and only if there is a number $z$ such that $s'(x,z)
  = 0$.

  Then the following function is both computable and total:
  \[
	g := \mu ((x,z) \mapsto \min(s(x,z), s'(x,z))).
  \]
  So
  \[
	\chi_A(x) =
	\begin{cases}
	  1, & s(x,g(x)) = 0, \\
	  0, & \text{otherwise},
	\end{cases}
  \]
  is computable.
\end{proof}

\vprasanje{Show that a set is computable if and only if both it and its
  complement are computably enumerable.}

\begin{corollary}
  The set $\cl{K}$ is not computably enumerable.
\end{corollary}

\begin{theorem}
  The following are equivalent for a set $A \subseteq \N$:
  \begin{itemize}
  \item $A$ is computably enumerable,
  \item $A = \varnothing$ or $A$ is the range of a total computable function,
  \item $A$ is the range of a computable partial function.
  \end{itemize}
\end{theorem}

\begin{proof}
  One to two:
  Suppose $A \ne \varnothing$, and select any $a_0 \in A$.
  By the above lemma, there exists a computable $s: \N^2 \to \N$ satisfying
  \[
	y \in A \iff \exists z. s(y,z) = 0.
  \]
  Using the pairing function $p: \N^2 \to \N$, we have that $A$ is the range of
  the total computable
  \[
	x \mapsto
	\begin{cases}
	  y, & s(y,z) = 0, \\
	  a_0 & \text{otherwise}.
	\end{cases}
  \]
  We used $(y, z) = p^{-1}(x)$ above.

  Two to three is trivial.
  Three to one:
  Suppose $A$ is the range of $\phi_e$.
  Let $T$ and $U$ be as in Kleene's normal form theorem.
  Then the total function $s: \N^2 \to \N$ is computable:
  \[
	s(w,y) =
	\begin{cases}
	  0 & T(e, x, z) = 0  \land w = U(z), \\
	  1 & \text{otherwise},
	\end{cases}
  \]
  where $y = p(x,z)$.
  Then $w \in A$ if and only if there exists a $y$ such that $s(w,y) = 0$.
\end{proof}

\vprasanje{Characterize computably enumerable sets and prove the
  characterization.}

\begin{theorem}
  The following are equivalent for a set $A \subseteq \N$:
  \begin{itemize}
  \item $A$ is computable,
  \item $A = \varnothing$ or $A$ is the range of an increasing total computable
	function,
  \item $A$ is finite or $A$ is the range of a strictly increasing computable
	total function.
  \end{itemize}
\end{theorem}

\begin{proof}
  We will only prove the equivalence of the first and third statement.
  One to three:
  Suppose $A$ is computable and infinite.
  Then $A$ is the image of the function which maps $n$ to the $n$-th smallest
  element of $A$, by searching through $m = 0, 1, 2, \ldots$ using $\chi_A$.

  Three to one:
  Suppose $A$ is the range of $f: \N \to \N$, which is strictly increasing.
  Then we can compute $\chi_A(n)$ by checking whether $n$ is in the image
  of $f(x)$ for $x = 0, 1, 2, \ldots$, until we find either $n$ or a number
  larger than it.
\end{proof}

\vprasanje{Characterize the computable sets and prove the characterization.}

\begin{corollary}
  Every infinite computably enumerable set has an infinite computable subset.
\end{corollary}

\begin{proof}
  Let $A$ be infinite and computably enumerable.
  Then $A$ is the image of a computable total function $f$.
  Define $g(0) = f(0)$ and $g(n+1) = f(m)$ where $m$ is the smallest number such
  that $f(m) > g(n)$.
\end{proof}

We define $E_e$ as the domain of $\phi_e$, and
\[
  \mathcal{C} = \{ f: \N \rightharpoonup \N \such \text{$f$ is computable} \}.
\]

\begin{definition}
  A subset $B \subseteq \mathcal{C}$ is \pojem{decidable} if
  \[
	I_B = \{ e \in \N \such \phi_e \in B \}
  \]
  is a computable set.
  Also, $B$ is \pojem{semidecidable} if $I_B$ is computably enumerable.
\end{definition}

\begin{lemma}[Reduction lemma]
  If $A$ is reducible to $B$ and $B$ is computable, then so is $A$.
  If $A$ is reducible to $B$ and $B$ is computably enumerable, then so it $A$.
\end{lemma}

\vprasanje{When is a subset of $\mathcal{C}$ decidable or semidecidable? State
  the reduction lemma.}

\begin{theorem}[Rice]
  A set $B \subseteq \mathcal{C}$ is decidable if and only if $B = \varnothing$
  or $B = \mathcal{C}$.
\end{theorem}

\begin{proof}
  The right-to-left implication is trivial.
  Left-to-right:
  Assume $B$ is neither $\varnothing$ nor $\mathcal{C}$.
  We'll show that $I_B$ is not computable.
  Without loss of generality, we assume the everywhere undefined partial
  function $f_\varnothing$ is in $\mathcal{C} \setminus B$.

  Choose some function $g \in B$.
  Define
  \[
	f(x,y) \simeq
	\begin{cases}
	  g(y), & x \in K \\
	  \uparrow, & \text{otherwise}
	\end{cases}
  \]
  Clearly $f$ is computable.
  By the s-m-n theorem, there is a total computable $S: \N \to \N$ such that
  $\phi_{s(x)}(y) \simeq f(x,y)$.
  Then for $x \in K$, $\phi_{s(x)}(y) \simeq g(y)$ and for $x \notin K$,
  $\phi_{s(x)}(y) \downarrow$.
  Since $g \in B$, we have $s(x) \in I_B$ for any $x \in K$.
  As $f_\varnothing$ is in $\mathcal{C} \setminus B$, we have $\phi_{s(x)}
  \notin I_B$ for any $x \notin K$.
  So $s$ is a reduction of $K$ to $I_B$.
  But $K$ is not computable, so $I_B$ isn't either.
\end{proof}

\vprasanje{State and prove Rice's theorem.}

\begin{definition}
  For partial functions $\N \rightharpoonup \N$, $f' \subseteq f$ if the graph
  of $f'$ is a subset of the graph of $f$.
  We say that $f$ is \pojem{finite} if its domain is finite.
\end{definition}

\begin{theorem}[Rice-Shapiro]
  If $B \subseteq \mathcal{C}$ is semidecidable, then for all $f \in
  \mathcal{C}$, then $f \in B$ if and only if there exists $f' \in B$ such that
  $f'$ is finite and $f' \subseteq f$.
\end{theorem}

\begin{proof}
  We prove that if either implication of the equivalence fails, then $B$ is not
  semidecidable.

  Suppose that the left-to-right implication fails, so there exists a function
  $f \in B$ such that all finite subfunctions of $f$ aren't in $B$.
  Note that $f$ is necessarily infinite.
  Because $K$ is computably enumerable, there is some total computable $t: \N^2
  \to \N$ such that $x \in K$ if and only if there exists a number $z$ for which
  $t(x,z) = 0$.
  Define
  \[
	g(x,z) \simeq
	\begin{cases}
	  f(z), & \text{if $t(x,y) \ne 0$ for all $0 \le y \le z$} \\
	  \uparrow, & \text{otherwise}
	\end{cases}
  \]
  By the s-m-n theorem, there is a total computable function $s: \N \to \N$ such
  that $\phi_{s(x)}(z) \simeq g(x,z)$.
  If $x \in K$, then $\phi_{s(x)}$ is finite and a subfunction of $f$, so by
  assumption $s(x) \notin I_B$ (since $\phi_{s(x)} \notin B$).
  If $x \notin K$, then $\phi_{s(x)} = f$, so $s(x) \in I_B$ because $f \in B$.
  Then $s$ is a reduction of $\cl{K}$ to $I_B$.
  By the reduction lemma, $I_B$ is not computably enumerable (as $\cl{K}$
  isn't).

  Now suppose that the right-to-left implication fails.
  Let $f'$ be a finite subfunction of $f$ such that $f' \in B$ and $f \notin B$.
  Define a computable partial function
  \[
	g(x,z) \simeq
	\begin{cases}
	  f(z), & \text{if $z \in \dom(f')$ or $x \in K$} \\
	  \uparrow, & \text{otherwise}
	\end{cases}
  \]
  By the s-m-n theorem, there is a total computable $s: \N \to \N$ such that
  $\phi_{s(x)}(z) = g(x,z)$.
  Since $f' \subseteq f$, we have:
  \begin{itemize}
  \item if $x \in K$, then $\phi_{s(x)} = f$, but $f \notin B$, so $s(x) \notin
	I_B$,
  \item if $x \notin K$, then $\phi_{s(x)} = f'$, but since $f' \in B$, so $s(x)
	\in I_B$.
  \end{itemize}
  So $s$ is a reduction of $\cl{K}$ to $I_B$, and $I_B$ can't be computably
  enumerable.
\end{proof}

\vprasanje{State and prove the Rice-Shapiro theorem.}

\podnaslov{Varieties of non-computable sets}

Observe that the following are equivalent for a set $A \subseteq \N$:
\begin{itemize}
\item $A$ is not computably enumerable,
\item for any $W_e \subseteq A$ there exists some $n \in A \setminus W_e$.
\end{itemize}
A set $A \subseteq \N$ is \pojem{productive} if there exists a computable total
function $g: \N \to \N$ such that for all $e \in \N$, if $W_e \subseteq A$, then
$g(e) \in A \setminus W_e$.
We call $g$ an \pojem{outsider finder}.

\begin{proposition}
  Every productive set is not computably enumerable.
\end{proposition}

\begin{proposition}
  The set $\cl{K}$ is productive.
\end{proposition}

\begin{proof}
  We show that $g = \id_\N$ is an outsider finder for $\cl{K}$.
  Suppose that $W_e \subseteq \cl{K}$.
  Also suppose that $g(e) = e \notin \cl{K}$.
  Then $e \in K$, so $\phi_e(e) \downarrow$, which means $e \in W_e \subseteq
  \cl{K}$, which is a contradiction, meaning that $e \in \cl{K}$.
  Now if $e \in \cl{K}$, then $e \in \{ f \such f \notin W_f \}$, so $e \notin
  W_e$.
  All this implies $g(e) \in \cl{K} \setminus W_e$.
\end{proof}

\vprasanje{Define productive sets. Show that $\cl{K}$ is productive.}

\begin{lemma}
  If $A$ reduces to $B$ and $A$ is productive, then so is $B$.
\end{lemma}

\begin{proof}
  Let $g$ be an outsider finder of $A$.
  We claim that $f \circ g \circ h$ is an outsider finder for $B$, where $h$ is
  defined in the following way.
  Consider the map $(e,x) \mapsto \phi_e(f(x))$.
  By the s-m-n theorem, there is a computable $h: \N \to \N$ for which
  $\phi_{h(e)}(x) \simeq \phi_e(f(x))$.
  To prove $f \circ g \circ h$ is an outsider finder for $B$, suppose $W_e
  \subseteq B$.
  Then $f^{-1}(W_e) \subseteq A$, but
  \[
	f^{-1}(W_e) = f^{-1}(\dom \phi_e) = \dom \phi_e
	\circ f = \dom \phi_{h(e)} = W_{h(e)}.
  \]
  So $f(g(h(e))) \subseteq B \setminus W_e$ because $g(h(e)) \in A \setminus
  W_{h(e)}$.
\end{proof}

\vprasanje{Show that if a productive set $A$ reduces to $B$, then $B$ is also
  productive.}

\begin{theorem}
  If $\varnothing \subsetneq B \subsetneq \mathcal{C}$ and the everywhere
  undefined function $f_\varnothing \in B$, then $I_B$ is productive.
\end{theorem}

\begin{proof}
  In the proof of Rice's theorem, we reduced $K$ to $\cl{I_B}$, so $\cl{K}$ to
  $I_B$.
  Because $\cl{K}$ is productive, so is $I_B$.
\end{proof}

\begin{definition}
  A set $A$ is \pojem{creative} if it is computably enumerable and its
  complement is productive.
\end{definition}

\begin{theorem}
  If $\varnothing \subsetneq B \subsetneq \mathcal{C}$ is such that $I_B$ is
  computably enumerable, then is it creative.
\end{theorem}

\begin{theorem}
  Every productive set has an infinite computably enumerable subset.
\end{theorem}

\begin{proof}
  Suppose that $A$ is productive with outsider finder $g$.
  We will define a function $f: \N \to \N$ which satisfies $f(0) = e_0$ with
  $W_{e_0} = \varnothing$, and $f(n+1) = e_{n+1}$ with $W_{e_{n+1}} = W_{e_n}
  \cup \{g(e_n)\}$.
  By the s-m-n theorem, there exists a computable total function $h: \N \to \N$
  such that
  \[
	\phi_{h(x)}(y) = \simeq
	\begin{cases}
	  0 & y \in W_x \lor y = g(x), \\
	  \uparrow & \text{otherwise.}
	\end{cases}
  \]
  So $W_{h(e)} = W_e \cup \{g(e)\}$, and we can find an $f$ such that $W_{f(0)}
  = \varnothing$ and $f(n+1) = h(f(n))$.

  We then have $g \circ f$, a total computable function with an image that is an
  infinite subset of $A$.
  Then that image is a computably enumerable subset of $A$.
\end{proof}

\vprasanje{Show that every productive set has an infinite computably enumerable
  subset.}

\begin{definition}
  A set $A \subseteq \N$ is \pojem{immune} if it is infinite and it has no
  infinite computably enumerable subset.
\end{definition}

\begin{remark}
  If $A$ is immune, it is not computably enumerable, and not productive.
\end{remark}

\begin{definition}
  A set $A \subseteq \N$ is \pojem{simple} if it is computably enumerable and
  its complement is immune.
\end{definition}

\begin{theorem}[Post]
  There exists a simple set.
\end{theorem}

\begin{proof}
  Consider a partial function $f: \N \to \N$, defined as follows:
  $f(e) = \phi_e(z)$ if $z$ is the smallest number such that $\phi_e(x)
  \downarrow$ for any $x \le z$ and $\phi_e(z) \ge 2e$, and $f(e)$ is undefined
  if no such $z$ exists.
  This is clearly computable, so its image is computably enumerable.
  Define $A = \im f$.
  We will prove that $A$ is simple.

  When $f(e) = n$, we have $n \ge 2e$, so the numbers $\{0, 1, \ldots, 2m-1\}$
  can only appear as values $f(e)$ when $e < m$.
  So for every $m \ge 0$, at least $m$ distinct numbers from that set belong to
  $\cl{A}$, meaning that $\cl{A}$ is infinite.

  Let $B$ be an infinite computably enumerable set.
  We know that $B$ must be the image of some total $\phi_e$.
  Then $f(e) \downarrow$, since because $B$ is infinite, it must contain some
  number larger than $2e$.
  So indeed $B \nsubseteq A$.
\end{proof}

\vprasanje{State and prove Post's theorem.}

% LocalWords:  Kleene's subfunctions subfunction
