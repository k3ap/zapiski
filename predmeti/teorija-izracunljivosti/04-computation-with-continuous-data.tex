\naslov{Computation with continuous data}

A type 2 Turing machine (T2M for short) with $k$ input tapes, $n$ working tapes
and one output tape is a Turing machine with $k+n+1$ tapes where:
\begin{itemize}
\item Input tapes start with no blank symbols, are only infinite in ore
  direction, and their heads are read-only and only move to the right.
\item Working tapes start blank on all but a finite number of squares.
  They are infinite in both directions, and have regular heads.
\item The output tape is initially blank and has a write-only head that can only
  move right.
\end{itemize}

Formally, a T2M is specified by
\begin{itemize}
\item the tape alphabet $\Gamma$ with $\boxdot \in \Gamma$,
\item an input/output alphabet $\Sigma \subseteq \Gamma \setminus \{\boxdot\}$,
\item a finite set $Q$ of control states,
\item the transition function
  \[
	\delta: Q \times \Sigma^k \times \Gamma^n \rightharpoonup Q \times \{0,1\}^k
	\times \Gamma^n \times \{-1,0,1\}^n \times \{\Sigma \cup \boxdot\}.
  \]
\end{itemize}

We say that a T2M \pojem{computes} an infinite word $p \in \Sigma^\omega$ given
input $(p^1, \ldots, p^k) \in (\Sigma^\omega)^k$ if when we run the machine on
input tapes containing $p^1, \ldots, p^k$, it outputs $p$ on the output tape.
An $\omega$-word is \pojem{computable} if it is computed by some T2M with not
input tapes.

A T2M $M$ \pojem{computes} a partial function $f: (\Sigma^\omega)^k
\rightharpoonup \Sigma^\omega$ if:
\begin{itemize}
\item for any $(p^1, \ldots, p^k) \in \dom f$, $M$ computes
  $f(p^1, \ldots, p^k)$ given input $(p^1, \ldots, p^k)$,
\item if $M$ is given $(p^1, \ldots, p^k)$ as input, it will compute some $p \in
  \Sigma^\omega$ only if $(p^1, \ldots, p^k) \in \dom f$.
\end{itemize}

For any recognition machines for an $\omega$-language, we assume distinguished
halting states \texttt{accept} and \texttt{reject}.
A single input tape T2M \pojem{accepts} $p \in \Sigma^\omega$ if, when run on
input $p$, it halts in the accepting state.
Similarly, it rejects $p$, if it halts in the rejecting state.

Given a word $p \in \Sigma^\omega$, we can define $\operatorname{Prefix}(p)
\subseteq \Sigma^*$ as the set of all prefixes of $p$.

\begin{theorem}
  The following are equivalent:
  \begin{itemize}
  \item $p$ is computable via a T2M,
  \item $\operatorname{Prefix}(p)$ is decidable,
  \item $\operatorname{Prefix}(p)$ is semidecidable.
  \end{itemize}
\end{theorem}

\begin{proof}
  1 to 2:
  Suppose that there is a T2M $M$ computing $p$.
  To decide whether $w$ is in $\operatorname{Prefix}(p)$ for some word $w$, we
  may run $M$ until it produces $\abs{w}$ characters, then compare that result
  to $w$.

  2 to 3:
  Trivial.

  3 to 1:
  Suppose $S$ is an ordinary TM that semidecides $\operatorname{Prefix}(p)$.
  We build a T2M that computes $p$ as follows.
  Suppose we have already output $n$ symbols of $p$.
  To find the next symbol, for each of the $m$ symbols $b_1, \ldots, b_m \in
  \Sigma$, we run $S$ (in parallel) on the input $p_0 \ldots p_{n-1} b_i$.
  Exactly one of these will halt in the accepting state, so when it does, output
  that symbol.
\end{proof}

\podnaslov{Topological aspects of computing with $\omega$-words}

\begin{theorem}%
  \label{thm:tizr:topology-1}
  If $L \subseteq \Sigma^\omega$ is semidecidable, then for any $p \in L$, there
  exists $n \ge 0$ such that for any infinite word $q \in \Sigma^\omega$, if
  $\left. q \right|_n = \left. p \right|_n$.
\end{theorem}

\begin{proof}
  Let $M$ be a T2M that semidecides $L$.
  Consider any $p \in L$, and let $n$ be $1$ plus the position of the read head
  when $M$ enters the accept state if run on $p$.
  Note that the read head can only move right, so $M$ could only access the
  first $n$ characters during its execution.
  If we give it another $\omega$-word with the same $n$-prefix, it will take the
  same actions and accept.
\end{proof}

\begin{theorem}%
  \label{thm:tizr:topology-2}
  If a partial function $f: \Sigma^\omega \rightharpoonup \Sigma^\omega$ is
  computable, then for every $p \in \dom f$ and for every $n \ge
  0$ there exists an $m \ge 0$ such that for all $q \in \dom f$,
  if $\left. q \right|_m = \left. p \right|_m$, then $\left. f(q) \right|_n =
  \left. f(p) \right|_n$.
\end{theorem}

\begin{proof}
  Let $M$ be a T2M that computes $f$.
  Consider any $p \in \dom f$ and $n \ge 0$.
  Let $m$ be $1$ plus the position of the input head at the time $M$ writes the
  $n$-th symbol to the output tape.

  Now consider any $q \in \dom f$ which agrees with $p$ on the first $m$
  characters.
  Then the execution of $M$ on $q$ follows the same steps as on $p$, so it
  produces the same first $n$ characters of output.
\end{proof}

We introduce a topology on $\Sigma^\omega$, which is just the infinite product
topology of $\Sigma$ (which is discrete).
This topology is metrizable for the metric
\[
  d(p,q) = 2^{-i},
\]
where $i$ is the smallest number such that $p_i \ne q_i$.
We of course take $d(p,p) = 0$.
Also, for a word $p \in \Sigma^\omega$ and number $n \ge 0$, define the
\pojem{cylinder set} $\sk{\left. p \right|_n}$, where for a finite word $w$,
\[
  \sk{w} = \{ q \in \Sigma^\omega \such \left. q \right|_{\abs{w}} = w \}.
\]
Note that the collection of cylinder sets is a countable basis for
$\Sigma^\omega$.

\begin{remark}
  Theorem~\ref{thm:tizr:topology-1} states:
  A semidecidable language is an open set.
\end{remark}

\begin{remark}
  Theorem~\ref{thm:tizr:topology-2} states:
  Computable functions are continuous with respect to the subspace topology on
  $\dom f$.
\end{remark}

\begin{proposition}
  If $L \subseteq \Sigma^\omega$ is decidable, it is clopen.
\end{proposition}

\begin{proof}
  The complement is semidecidable.
\end{proof}

\begin{theorem}
  An $\omega$-language $L$ is decidable if and only if it is clopen.
\end{theorem}

\begin{definition}
  A subset $Z \subseteq \Sigma^\omega$ is $G_\delta$ if it is a countable
  intersection of open sets.
\end{definition}

\begin{theorem}
  If a partial function $f: \Sigma^\omega \rightharpoonup \Sigma^\omega$ is
  computable, then its domain of definition is a $G_\delta$-subset of
  $\Sigma^\omega$.
\end{theorem}

\begin{proof}
  Suppose that $M$ is a T2M which computes $f$.
  For every $n \ge 0$, define
  \[
	D_n = \{ p \in \Sigma^\omega \such \text{$M$ produces $\ge n$ output
	  characters when run on $p$} \}.
  \]
  Note that $\dom f \subseteq D_n$ and that $D_n$ is semidecidable (and hence
  open).
  Clearly $\dom f$ is the intersection of all $D_n$.
\end{proof}

\begin{theorem}
  The topological space $\Sigma^\omega$ is compact.
\end{theorem}

\begin{proof}
  Tychonoff.
\end{proof}

\begin{corollary}
  The compact subsets of $\Sigma^\omega$ are exactly the closed sets.
\end{corollary}

\begin{lemma}
  Every clopen set is decidable.
\end{lemma}

\begin{proof}
  Let $L$ be a clopen set.
  Since the cylinder sets form a basis and $L$ is open, we have
  \[
	L = \bigcup \{ \sk{w} \such w \in \Sigma^*, \sk{w} \subseteq L \}.
  \]
  So this family of cylinders is an open cover of $L$.
  Because $L$ is closed, it is compact, so there is a finite subcover
  \[
	L = \sk{w_1} \cup \ldots \cup \sk{w_k}.
  \]
  We can decide $L$ by checking whether the prefix of a word is equal to any of
  $w_1, \ldots, w_k$.
\end{proof}

\podnaslov{Computing with real numbers}

We can represent real numbers as $\omega$-words via infinite decimal expansions
(or representations in other bases), so with the alphabet $\{0,1, \ldots, 9, -,
.\}$, with at most one $-$ at the start, and exactly one decimal point.
The problem is that this is a poor representation, as many useful algorithms
cannot be written with it.

\begin{definition}
  A \pojem{type 2 representation} of a set $X$ is a surjective partial function
  $\gamma: \Sigma^\omega \rightharpoonup X$.
  We say that $p$ is a \pojem{name} for an element $x \in X$ if $\gamma(p) = x$,
  and that $x$ is \pojem{computable} if it has a computable name.
\end{definition}

We now introduce the Cauchy representation of $\R$.
First, give a type $1$ representation of the dyadic rationals $\Q_d$, which we
represent by a finite word $\pm d_{m-1} \ldots d_0 . d_{-1} \ldots d_{-n}$
with every $d_i \in \{0, 1\}$ and $m, n \ge 0$.
For a word $u$ representing a dyadic rational, define $q_d(u)$ as its rational
value, interpreted as we usually interpret binary numbers.

We can now define the Cauchy representation $\gamma_c: \Sigma_c^\omega
\rightharpoonup \R$ for $\Sigma_c = \{ -, ., 0, 1, ; \}$.
The domain of $\gamma_c$ consists of $\omega$-words of the form
\[
  p = u_0; u_1; u_2; \ldots,
\]
where $u_i \in \dom q_d$ and the sequence $(q_d(u_i))_i$ is a fast Cauchy
sequence, i.e.~it satisfies
\[
  \abs{q_d(u_m) - q_d(u_n)} \le \inv{2^n}
\]
for all $m \ge n \ge 0$.
For such a name $p$, we define $\gamma_c(p) = \lim q_d(u_n)$.

\begin{proposition}
  The representation $\gamma_c$ is surjective, and if $p \in \dom \gamma_c$ is
  as above, then for any $n \ge 0$, $\abs{q_d(u_n) - \gamma_c(p)} \le 2^{-n}$.
\end{proposition}

\begin{proposition}
  A real number is $\gamma_c$-computable if and only if it is computable with
  respect to the decimal or binary representation.
\end{proposition}

\begin{definition}
  A name $p$ is \pojem{close} if for every $n \ge 0$, we have $\abs{q_d(u_n) -
	\gamma_c(p)} \le 2^{-(n+1)}$.
\end{definition}

\begin{lemma}
  Every $x \in \R$ has a close name.
  If $p$ is a close name for $x$, then for every $n \ge 0$, every $x'$ with
  $\abs{x'-x} < 2^{-(n+1)}$ has a name of the form
  \[
	u_0; u_1; u_2; \ldots; u_n; u_{n+1}'; u_{n+2}'; \ldots
  \]
\end{lemma}

\begin{theorem}[continuity theorem]
  If $f: \R \rightharpoonup \R$ is computable with respect to $\gamma_c$, then
  $f$ is continuous on its domain.
\end{theorem}

\begin{proof}
  Suppose $f$ is computable, so it has a realiser $g: \Sigma_c^\omega
  \rightharpoonup \Sigma_c^\omega$.
  Let $x \in \dom f$ and $\varepsilon > 0$.
  We are searching for a suitable $\delta$.
  Let $p = u_0; u_1; \ldots$ be a close name for $x$, and let $r = g(p)$.
  Since $g$ is a realiser for $f$, $r$ is a name for $f(x)$.
  Therefore $r$ has the form $r = v_0; v_1; \ldots$.

  Let $N$ be such that $2^{-N} < \nicefrac{\varepsilon}{2}$.
  We have $\abs{q_d(v_N) - f(x)} \le 2^{-N} < \nicefrac{\varepsilon}{2}$ by one
  of the preceding propositions.
  Let $n$ be the length of the string $v_0; v_1; \ldots; v_N;$.
  By the topological continuity theorem, there exists an $m \ge 0$ such that for
  all $p' \in \dom g$, if $\left. p' \right|_m = \left. p \right|_m$, then
  $\left. g(p') \right|_n = v_0; v_1; \ldots, v_N;$.
  Now take $M \ge 0$ such that the prefix $u_0; u_1; \ldots; u_M;$ of $p$ has
  length $m' \ge n$, and define $\delta = 2^{-(M+1)}$.

  Then for $x' \in \dom f$ with $\abs{x' - x} < \delta$, we have a name for $x'$
  of the form $p' = u_0; \ldots; u_M; u_{M+1}'; \ldots$, since $p$ is a close
  name for $x$.
  Since $x' \in \dom f$, $g(p')$ is a name for $f(x')$, and $\left. g(p')
  \right|_n = v_0; v_1; \ldots; v_N;$, so $\abs{q_d(v_N) - f(x')} \le 2^{-N} <
  \nicefrac{\varepsilon}{2}$.
  Then $\abs{f(x) - f(x')} < \varepsilon$.
\end{proof}

% LocalWords:  subcover realiser
