\naslov{Computation with continuous data}

A type 2 Turing machine (T2M for short) with $k$ input tapes, $n$ working tapes
and one output tape is a Turing machine with $k+n+1$ tapes where:
\begin{itemize}
\item Input tapes start with no blank symbols, are only infinite in ore
  direction, and their heads are read-only and only move to the right.
\item Working tapes start blank on all but a finite number of squares.
  They are infinite in both directions, and have regular heads.
\item The output tape is initially blank and has a write-only head that can only
  move right.
\end{itemize}

Formally, a T2M is specified by
\begin{itemize}
\item the tape alphabet $\Gamma$ with $\boxdot \in \Gamma$,
\item an input/output alphabet $\Sigma \subseteq \Gamma \setminus \{\boxdot\}$,
\item a finite set $Q$ of control states,
\item the transition function
  \[
	\delta: Q \times \Sigma^k \times \Gamma^n \rightharpoonup Q \times \{0,1\}^k
	\times \Gamma^n \times \{-1,0,1\}^n \times \{\Sigma \cup \boxdot\}.
  \]
\end{itemize}

We say that a T2M \pojem{computes} an infinite word $p \in \Sigma^\omega$ given
input $(p^1, \ldots, p^k) \in (\Sigma^\omega)^k$ if when we run the machine on
input tapes containing $p^1, \ldots, p^k$, it outputs $p$ on the output tape.
An $\omega$-word is \pojem{computable} if it is computed by some T2M with not
input tapes.

A T2M $M$ \pojem{computes} a partial function $f: (\Sigma^\omega)^k
\rightharpoonup \Sigma^\omega$ if:
\begin{itemize}
\item for any $(p^1, \ldots, p^k) \in \operatorname{dom} f$, $M$ computes
  $f(p^1, \ldots, p^k)$ given input $(p^1, \ldots, p^k)$,
\item if $M$ is given $(p^1, \ldots, p^k)$ as input, it will compute some $p \in
  \Sigma^\omega$ only if $(p^1, \ldots, p^k) \in \operatorname{dom} f$.
\end{itemize}

For any recognition machines for an $\omega$-language, we assume distinguished
halting states \texttt{accept} and \texttt{reject}.
A single input tape T2M \pojem{accepts} $p \in \Sigma^\omega$ if, when run on
input $p$, it halts in the accepting state.
Similarly, it rejects $p$, if it halts in the rejecting state.

Given a word $p \in \Sigma^\omega$, we can define $\operatorname{Prefix}(p)
\subseteq \Sigma^*$ as the set of all prefixes of $p$.

\begin{theorem}
  The following are equivalent:
  \begin{itemize}
  \item $p$ is computable via a T2M,
  \item $\operatorname{Prefix}(p)$ is decidable,
  \item $\operatorname{Prefix}(p)$ is semidecidable.
  \end{itemize}
\end{theorem}

\begin{proof}
  1 to 2:
  Suppose that there is a T2M $M$ computing $p$.
  To decide whether $w$ is in $\operatorname{Prefix}(p)$ for some word $w$, we
  may run $M$ until it produces $\abs{w}$ characters, then compare that result
  to $w$.

  2 to 3:
  Trivial.

  3 to 1:
  Suppose $S$ is an ordinary TM that semidecides $\operatorname{Prefix}(p)$.
  We build a T2M that computes $p$ as follows.
  Suppose we have already output $n$ symbols of $p$.
  To find the next symbol, for each of the $m$ symbols $b_1, \ldots, b_m \in
  \Sigma$, we run $S$ (in parallel) on the input $p_0 \ldots p_{n-1} b_i$.
  Exactly one of these will halt in the accepting state, so when it does, output
  that symbol.
\end{proof}