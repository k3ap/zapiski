\naslov{Algorithmic information theory}

A computable partial function $u: \{0,1\}^* \rightharpoonup \{0,1\}^*$ is
\pojem{universal} if for any computable partial function $f: \{0,1\}^*
\rightharpoonup \{0,1\}^*$ there exists a word $v_f \in \{0,1\}^*$ such that for
all $w \in \{0,1\}^*$ we have $f(w) \simeq u(v_f w)$.

\begin{proposition}
  A universal computable partial function exists.
\end{proposition}

Given a universal function $u$, the \pojem{Kolmogorov complexity} $C_u(w)$ of a
word $w$ is
\[
  C_u(w) = \min \{ \abs{v} \such u(v) = w \}.
\]

\begin{proposition}
  Suppose $u,u'$ are universal computable functions.
  Then for all words $w$, $C_{u'}(w) \le C_u(w) + C$ for some constant $C$,
  equal for all words.
\end{proposition}

\begin{proposition}
  For any $n \ge 0$, there exists a word $w$ of length $n$ such that $C_u(w) \ge
  n$.
\end{proposition}

\begin{proof}
  There are only at most $2^{n-1}$ words with complexity $< n$, but there are
  $2^n$ words of length $n$.
\end{proof}

\begin{definition}
  An infinite word $p \in \{0,1\}^\omega$ is \pojem{K-incompressible} if for all
  $n \ge 0$, we have $C(\left. p \right|_n) \ge n - O(1)$.
\end{definition}

\begin{theorem}
  No infinite sequence is K-incompressible.
\end{theorem}

\begin{proof}
  Define a map
  \[
	\ceil{w} = \sum_{i=0}^{\abs{w}-1} w_i 2^i + 2^{\abs{w}} - 1
  \]
  which has an inverse $\operatorname{word}: \N \to \{0,1\}^*$.
  Let $p \in \{0,1\}^\omega$.
  Consider any $d \ge 0$ and let $m = \ceil{\left. p\right|_d}$.
  Define $h: \{0,1\}^* \to \{0,1\}^*$ by $h(w) = \operatorname{word}(\abs{w})
  w$.
  Note that
  \[
	p_0 p_1 \ldots p_{d-1} p_d \ldots p_{d+m-1} = \left. p \right|_{d+m}
	= h(p_d \ldots p_{d+m-1}) = u(v_h p_d \ldots p_{d+m-1}),
  \]
  so $C_u(\left. p \right|_{d+m}) \le \abs{v_h} + \ceil{\left. p \right|_d}$.
\end{proof}

\begin{definition}
  A partial function $f: \{0,1\}^* \rightharpoonup \{0,1\}^*$ is
  \pojem{prefix-free} if $\dom f \subseteq \{0,1\}^*$ satisfies the following
  property:
  for all $w, w' \in \dom f$, $w$ is not a proper prefix of $w'$.
\end{definition}

\begin{definition}
  A computable prefix-free function $u: \{0,1\}^* \rightharpoonup \{0,1\}^*$ is
  \pojem{universal} if for any computable prefix-free function $f: \{0,1\}^*
  \rightharpoonup \{0,1\}^*$ there exists $v_f \in \{0,1\}^*$ such that for all
  $w$, $f(w) \simeq u(v_f w)$.
\end{definition}

\begin{definition}
  The \pojem{prefix-free complexity} $K(w)$ is defined as
  \[
	K(w) = \min \{ \abs{v} \such u(v) = w \}
  \]
  for a universal prefix-free function $u$.
\end{definition}

\begin{proposition}
  For any $d > 1$, there exists a constant $C$ such that $K(w) \le d \abs{w} +
  C$ for any word $w$.
\end{proposition}

\begin{proof}
  Outline.
  For any large $N$, we work with an encoding of $\{0,1\}^*$ as words $\Sigma^*$
  where $\abs{\Sigma} = 2^N -1$.
  We map $w$ to $\ceil{w}$, then represent $\Sigma$ with binary words of length
  $N$.
  We are left with one non-represented word, which is $0^N$ without loss of
  generality.
  Then we assign to $w$ the representation $\sk{\ceil{w}}$ followed by $N$
  zeros.
\end{proof}

\begin{definition}
  An infinite word $p$ is \pojem{prefix-free incompressible} if there exists a
  constant $c$ such that for all $n$, $K(\left. p \right|_n) \ge n - c$.
\end{definition}

\begin{definition}
  Chaitin's halting probability $\Omega$ is the probability that $u$ halts if
  given an input generated randomly by throwing coins, so
  \[
	\Omega = \sum_{w \in \dom u} 2^{-\abs{w}}.
  \]
  Then take $p^\Omega$ as the binary representation of $\Omega$.
  If there are multiple options, choose the one with infinite ones.
\end{definition}

\begin{theorem}
  Chaitin's halting probability $p^\Omega$ is prefix-free incompressible.
\end{theorem}

\begin{proof}
  Note that
  \[
	\Omega = \sum_{i=0}^\infty p_i^\Omega 2^{-(i+1)}.
  \]
  We will define a computable function $f: \{0,1\}^* \rightharpoonup \{0,1\}^*$
  satisfying $f(\left. p^\Omega \right|_n) \downarrow$ and $K(f(\left. p^\Omega
  \right|_n)) > n$ for all $n$.
  If we do, then there is a number $c$ such that for all $n \ge 0$,
  \[
	K(f(\left. p^\Omega \right|_n)) \le K(\left. p^\Omega \right|_n) + c,
  \]
  so $K(\left. p^\Omega \right|_n) > n-c$.

  Let $w_0, w_1, \ldots$ be a computable enumeration of $\dom u$ without
  repetitions.
  Given an input word $p_0 p_1 \ldots p_{n-1}$, the algorithm of $f$ finds the
  smallest $N \ge 0$ (if it exists) with the property
  \[
	\sum_{i=0}^{N-1} 2^{-\abs{w_i}} \ge \sum_{i=0}^{n-1} p_i 2^{-(i+1)}.
  \]
  Then the algorithm returns the first $v \in \{0,1\}^*$ such that $v \notin \{
  u(w_0), \ldots, u(w_{N-1})\}$.
  Define $f(p_0 \ldots p_{n-1}) = v$.

  Clearly $f(\left. p^\Omega \right|_n)$ is defined for any $n \in \N$, as we
  have chosen a representation of $p^\Omega$ with infinite ones.
  From
  \[
	\sum_{i=0}^{n-1} p_i^\Omega 2^{-(i+1)}
	\le \sum_{i=0}^{N-1} 2^{-\abs{w_i}}
	< \Omega
	< \sum_{i=0}^{n-1} p_i^\Omega 2^{-(i+1)} + 2^{-n}
  \]
  we have
  \[
	\Omega - \sum_{i=0}^{n-1} 2^{-\abs{w_i}} < 2^{-n},
  \]
  so every word $w_i$ for $i \ge N$ is such that $\abs{w_i} > n$.
  This means that for every finite word $v \notin \{ u(w_0), \ldots, u(w_{N-1})
  \}$, the smallest $w \in \dom u$ with $u(w) = v$ has $\abs{w} > n$ or $K(v) >
  n$.
\end{proof}

% LocalWords:  Kolmogorov incompressible
