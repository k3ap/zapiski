\naslov{Computability of natural numbers}

\begin{definition}
  The \pojem{computable partial functions} is the set
  \[
	\{f: \N^k \rightharpoonup \N \such k \ge 0, \text{$f$ is computable}\}.
  \]
\end{definition}

\begin{definition}
  The \pojem{primitive recursive functions} are the smallest collection
  $\mathcal{F} \subseteq \{\N^k \rightharpoonup \N \such k \ge 0\}$ of partial
  functions that satisfies the following properties:
  \begin{itemize}
  \item $\mathcal{F}$ contains the zero function $Z: \{()\} \to \N$,
	which maps $Z() = 0$, the successor function $S: \N \to \N$, which maps $x
	\mapsto x+1$ and the projection functions:
	for any $k \ge 1$ and $1 \le i \le k$, $U_i^k: \N^k \to \N$ is defined as
	\[
	  U_i^k(x_1, \ldots, x_k) = x_i.
	\]
  \item $\mathcal{F}$ is closed under composition: if $f: \N^k \rightharpoonup
	\N$ and $g_1, \ldots, g_k: \N^l \rightharpoonup \N$ are in $\mathcal{F}$,
	then $f \circ (g_1, \ldots, g_k)$ is in $\mathcal{F}$.
  \item Primitive recursion: If $f: \N^k \rightharpoonup \N$ and $g: \N^{k+2}
	\rightharpoonup \N$ are both in $\mathcal{F}$, then so is $R_{fg}: \N^{k+1}
	\rightharpoonup \N$, defined with $R_{rg}(x_1, \ldots, x_k, 0) \simeq f(x_1,
	\ldots, x_k)$ and $R_{fg}(x_1, \ldots, x_k, x+1) \simeq g(x_1, \ldots, x_k,
	x, R_{fg}(x_1, \ldots, x_k, x))$.
  \end{itemize}
\end{definition}

\begin{remark}
  Since all basic functions are total, every function in $\mathcal{F}$ is total.
\end{remark}

\begin{remark}
  Not every total computable function is primitive recursive.
  We can show for example that the Ackermann function grows faster than any
  primitive recursive function.
\end{remark}

\begin{definition}
  The \pojem{partial recursive functions} are the smallest collection of partial
  functions $\mathcal{F} \subseteq \{\N^k \rightharpoonup \N\}_{k \ge 0}$, which
  satisfies the axioms of partial recursive functions, with an additional one:
  \begin{itemize}
  \item Minimization: If $f: \N^{k+1} \rightharpoonup \N$ is in $\mathcal{F}$,
	then so is $\mu f: \N^k \rightharpoonup \N$, with $\mu f(x_1, \ldots,
	x_k)$ equal to the smallest number $n \in \N$ such that $f(x_1, \ldots, x_k,
	n) = 0$ if it exists and $f(x_1, \ldots, x_k, m)$ is defined for all $m <
	n$, and $\mu f(x_1, \ldots, x_k)$ undefined otherwise.
  \end{itemize}
\end{definition}

\begin{proposition}
  A partial function $f: \N^k \rightharpoonup \N$ is computable if and only if
  there exists a $(k+1)$-tape Turing machine such that for all $x_1, \ldots, x_k
  \in \N$ and binary words $w_1, \ldots, w_k$ of $x_1, \ldots, x_k$, if we run
  the Turing machine with $w_1, \ldots, w_k$ on the first $k$ tapes, then it
  halts if and only if $f(x_1, \ldots, x_k)$ is defined and it halts with the
  representation of $f(x_1, \ldots, x_k)$ on the last tape, and $w_1, \ldots,
  w_k$ on the first $k$ tapes.
\end{proposition}

\begin{theorem}
  The partial recursive functions coincide with the computable partial functions.
\end{theorem}

\begin{proof}
  A partial recursive function is computable:
  We will show that the family of computable functions satisfies the properties
  of partial computable functions.
  Since the partial recursive functions are the smallest such family, every
  partial recursive function is computable.

  Clearly, computable partial functions satisfy composition and include the
  basic functions.
  Let's show they are closed under primitive recursion.
  Suppose we have a $(k+1)$-tape Turing machine $M_f$ computing $f$ and a
  $(k+3)$-tape Turing machine $M_g$ computing $g$.
  We will find a $(k+4)$-tape Turing machine computing $R_{fg}$, which can then
  be compressed.
  On the first $k+1$ tapes, put the arguments to $R_{fg}$.
  On the $(k+2)$-th tape, put a counter.
  On the next tape, put the result of $R_{fg}$ on the current iteration, and
  finally, on the last tape, put the result being computed.
  The Turing machine then proceeds as follows:
  \begin{enumerate}
  \item initialize tape $k+2$ to $0$
  \item use $M_f$ to compute $f(x_1, \ldots, x_k)$ and write the result on tape
	$k+4$
  \item \label{item:halt-if-equal} if the numbers on tapes $k+1$ and $k+2$ are equal, halt
  \item copy tape $k+4$ to tape $k+3$
  \item apply $M_g$ on tapes $1, \ldots, k, k+2, k+3$ and write the result on
	tape $k+4$
  \item increment tape $k+2$
  \item go to step~\ref{item:halt-if-equal}
  \end{enumerate}
  We may similarly construct a machine which performs minimization, which
  finishes this inclusion.

  Before starting the proof of the other inclusion, consider the following.
  We can encode $\N \times \N$ using $\N$ with the bijection $p(x,y) = \pol
  (x+y)(x+y+1) + x$, which is also primitive recursive.
  The functions $q_1, q_2: \N \to \N$ which reverse $p$ (such that $q_1(p(x,y))
  = x$ and $q_2(p(x,y)) = y$) are also primitive recursive.

  We can also encode finite sequences with $\ceil{\cdot} : \N^* \to \N$, defined
  as $\ceil{n_0, \ldots, n_{k-1}} = 2^{n_0} + 2^{n_0 + n_1 + 1} + \ldots +
  2^{n_0 + \ldots + n_{k-1} + k-1}$, so that numbers are encoded in a binary
  sequence with the number of zeros between a pair of ones denoting the
  sequence.
  This is clearly a bijection.
  Additionally, the functions $\sigma: \N^2 \to \N$ mapping
  \[
	\sigma(\ceil{w}, i) =
	\begin{cases}
	  0 & i \ge \abs{w} \\
	  w_i+1 & i < \abs{w}
	\end{cases}
  \]
  and $l: \N \to \N$ mapping
  \[
	l(\ceil{w}) = \abs{w}
  \]
  are primitive recursive.

  Now we can prove the other inclusion.
  Suppose $f: \N^k \rightharpoonup \N$ is computable and that it is computed by
  a Turing machine $M$.
  We assume $M$ is a single-tape machine computing $f$ via representations.
  Suppose $M$ has a tape alphabet $\Gamma \supseteq \{\boxdot, 0, 1,
  \mathtt{,}\}$ and state set $Q \supseteq \{\text{start}, \text{halt}\}$.
  Choose injective functions $r: \Gamma \to \{\text{odd numbers}\}$ and $s: Q
  \to \{\text{even numbers}\}$.
  Now suppose we are in a configuration $C$ with a finitely many not necessarily
  blank symbols $a_0, \ldots, a_{k-1}$, the tape head at $a_i$, and the current
  state equal to $q$.
  Define an encoding
  \[
	\ceil{C} = \ceil{r(a_0) \ldots r(a_{i-1}) s(q) r(a_i) \ldots r(a_{k-1})}.
  \]
  The following functions are primitive recursive:
  \begin{itemize}
  \item $\text{step}: \N \to \N$, mapping $\ceil{C}$ to $\ceil{C'}$, which is
	the configuration we obtain by taking one step of $M$ on configuration $C$.
	If the input of the function is not a valid configuration, it returns $0$.
  \item $\text{run}: \N^2 \to \N$, mapping $(n,x)$ to $\text{step}^n(x)$.
  \item $\text{extract}: \N \to \N$, mapping $\ceil{s(\text{halt}) r(w_0) \ldots
	  r(w_k)}$ to $n$ if $w$ is a binary representation of $n$, and any other
	input to $0$.
  \item $\text{halt?}: \N \to \N$ mapping $\ceil{s(\text{halt}) r(w_0) \ldots
	  r(w_{k+1})}$ to $0$ and all other inputs to $1$.
  \item $\text{init}: \N^k \to \N$, mapping $(x_1, \ldots, x_k)$ to
	$\ceil{s(\text{start}) y_1 \ldots y_m}$, where $y_i$ is equal to the result
	of $r$ on the $i$-th character of the sequence $(\operatorname{bin}(x_1),
	\ldots, \operatorname{bin}(x_k))$.
  \end{itemize}
  Then, $f$ is partial recursive because
  \[
	f(x_1, \ldots, x_n) \simeq \text{extract}(\text{run}(
	\mu(n \mapsto \text{halt?}(\text{run}(n, \text{init}(x_1, \ldots, x_k)))),
	\text{init}(x_1, \ldots, x_k)
	))
  \]
\end{proof}