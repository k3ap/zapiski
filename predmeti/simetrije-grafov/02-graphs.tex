\naslov{Graphs}

It is implied that every graph is finite, simple and undirected.

\begin{definition}
  The \pojem{arc set} of a graph $\Gamma$ is $A(\Gamma) = \{ (u,v) \such uv
  \in E(\Gamma)\}$.
\end{definition}

\begin{definition}
  A \pojem{morphism} from a graph $\Gamma$ to another graph $\Lambda$ is a
  mapping $\varphi: V(\Gamma) \to V(\Lambda)$ such that for each edge $uv \in
  E(\Gamma)$, there is an edge $\varphi(u) \varphi(v) \in E(\Lambda)$.

\end{definition}

\begin{remark}
  A morphism $\varphi: \Gamma \to \Lambda$ induces a map $E(\Gamma) \to
  E(\Lambda)$ with $uv \in E(\Gamma) \mapsto \varphi(u) \varphi(v) \in
  E(\Lambda)$.
\end{remark}

\begin{definition}
  If $\varphi$ is injective, it is a \pojem{monomorphism}.
  If it is surjective both as a vertex map and an edge map, then it is an
  \pojem{epimorphism}.
\end{definition}

\begin{definition}
  If $\varphi$ is bijective as a vertex map, and $\varphi^{-1}$ is also a graph
  morphism, then $\varphi$ is an \pojem{isomorphism}.
\end{definition}

\begin{definition}
  If $\varphi$ is an isomorphism $\Gamma \to \Gamma$, then $\varphi$ is an
  \pojem{automorphism} of $\Gamma$.
  We denote the set of all automorphisms of $\Gamma$ with $\Aut\Gamma$.
\end{definition}

\begin{lemma}
  Let $\Gamma$ and $\Lambda$ be graphs and let $\varphi: V(\Gamma) \to
  V(\Lambda)$ be a map.
  Then $\varphi$ is an isomorphism if and only if it is bijective and
  $\phi(E(\Gamma)) = E(\Lambda)$.
\end{lemma}

\begin{remark}
  If $g$ is a permutation of $V(\Gamma)$, it is enough to check that $u \sim v$
  implies $u^g \sim v^g$, since the sets $E(\Gamma)$ and $E(\Gamma)^g$ are of
  the same cardinality.
\end{remark}

\begin{remark}
  A permutation $g$ of $V(\Gamma)$ is an automorphism if and only if $g$ is in
  the stabilizer of $E(\Gamma)$ in the action of $\Sym(V(\Gamma))$ on the power
  set of $V(\Gamma)^{\{2\}}$.
  So $\Aut \Gamma = \Sym(V(\Gamma))_{E(\Gamma)}$ in this particular action.
  The orbit-stabiliser theorem then gives us $\abs{\Sym(V)} = \abs{\Aut \Gamma}
  \abs{E(\Gamma)^{\Sym(V)}}$, or for $n = \abs{V(\Gamma)}$,
  \[
	\abs{\Aut \Gamma} = \frac{n!}{\abs{E(\Gamma)^{\Sym V}}}.
  \]
  The number in the denominator is then precisely the number of graphs on vertex
  set $V$ that are isomorphic to $\Gamma$.
\end{remark}

\podnaslov{Finding graphs with prescribed automorphisms}

Consider the following problem.
Given a permutation group $G \le \Sym \Omega$, determine for which sets $E
\subseteq \binom{\Omega}{2}$ does $G \le \Aut (\Omega, E)$ hold.
Or, in particular, how many such sets are there.
Label this number with $\gamma(G)$.

\begin{lemma}
  Let $\Gamma$ be a graph, $V = V(\Gamma)$, $E = E(\Gamma)$ and let $G \le \Sym
  \Omega$.
  Then $G \le \Aut \Gamma$ if and only if $E$ is a union of some orbits of $G$
  in its induced action on $\binom{V}{2}$.
\end{lemma}

\begin{corollary}
  Let $G \le \Sym \Omega$.
  Then
  \[
	\gamma(G) = 2^{\operatorname{orb}_2(G)},
  \]
  where $\operatorname{orb}_2(G)$ is the number of orbits of $G$ on its induced
  action on $\binom{\Omega}{2}$.
\end{corollary}

By the Cauchy-Frobenius lemma, we have
\[
  \operatorname{orb}_2(G) = \inv{\abs{G}} \sum_{g \in G} \abs{\Fix_2(g)},
\]
where $\Fix_2(g)$ is the number of unordered pairs of elements of $\Omega$ fixed
by $g$.

Suppose that $G = \sk{g}$ is cyclic.
Then the above reduces to
\[
  \operatorname{orb}_2(G) = \inv{\abs{g}} \sum_{i=0}^{\abs{g}-1}
  \abs{\Fix_2(g^i)},
\]
where $\abs{g}$ is the order of $g$.
Now consider several cases.
For the first case, suppose $\Omega = \{1, 2, \ldots, m\}$ and $g =
(1\,2\,\ldots\,m)$.
Then a pair $e = \{x,y\} \subseteq \Omega$ is fixed by $g^i$ if either $g^i$
fixes both $x$ and $y$, which happens only in the case $i=0$, or if $g^i$ maps
$x$ to $y$ and vice versa, but this happens only if $m$ is even, $i =
\frac{m}{2}$ and $x$ and $y$ are opposite.
We thus have
\[
  \abs{\Fix_2(g^i)} =
  \begin{cases}
	\binom{m}{2} & i = 0 \\
	\frac{m}{2} & \text{$m$ even, $i = \frac{m}{2}$} \\
	0 & \text{otherwise}
  \end{cases}
\]
so
\[
  \operatorname{orb}_2(\sk{g})
  = \inv{m} \left(
	\binom{m}{2} + \operatorname{even}(m)\,\frac{m}{2}
  \right)
  = \left\lfloor \frac{m}{2} \right\rfloor,
\]
where $\operatorname{even}(m)$ is either $1$ or $0$ if $m$ is even or odd,
respectively.

For the second case, suppose $\Omega = \{ 1, 2, \ldots, m, 1', 2', \ldots, k'
\}$, so $\abs{\Omega} = m+k'$, and $g = (1\,2\,\ldots\,m) (1'\,2'\,\ldots\,k')$.
Note that $\abs{g} = \operatorname{lcm}(m, k)$.
Now choose some $e = \{i, j'\}$ for $1 \le i \le m$ and $1 \le j \le k$.
We can estimate
\[
  \abs{e^{\sk{g}}}
  = \frac{\abs{\sk{g}}}{\abs{\sk{g}_e}}
  = \operatorname{lcm}(m, k)
\]
since if $g^r$ fixes $e$, then it also fixes $i$ and $j'$ (since $g$ cannot map
a non-primed element to a primed element), so $g^r \in \sk{g}_e$ implies $g^r
\in \sk{g}_i$.
Then by the following lemma, since a cyclic group is Abelian, $g^r$ fixes all
points $1, \ldots, m$.
Similarly, it fixes all points $1', \ldots, k'$.
So $g^r$ is the identity (since a permutation group always acts faithfully).

\begin{lemma}
  Suppose $G$ is an Abelian group acting on $\Omega$.
  Then every $g \in G_\omega$ fixes all points in $\omega^G$.
  In other words, $G_\omega$ acts trivially on $\omega^G$.
\end{lemma}

\begin{proof}
  Take $g \in G_\omega$ and $\delta \in \omega^G$.
  Then $\delta = \omega^ĥ$ for some $h \in G$, so
  \[
	\delta^g = \omega^{hg} = \omega^{gh} = \omega^h = \delta
  \]
  which we were trying to prove.
\end{proof}

\begin{remark}
  If $G$ is Abelian and acts transitively and faithfully, then it acts
  regularly.
  Since all regular actions are isomorphic, this means Abelian groups are
  boring.
\end{remark}

Continuing the previous discussion, this implies we have precisely
\[
  \frac{mk}{\operatorname{lcm}(m,k)} = \operatorname{gcd}(m, k)
\]
orbits containing a pair of the form $\{i, j'\}$.

For the general case, suppose $g$ consists of $k$ fixed points $K = \Fix(g)$ and
$r$ cycles $M_1, \ldots, M_r$ of lengths $m_1, \ldots, m_r \ge 2$.
Label $n = k + m_1 + \cdots + m_k$.
If $g \ne \id$, then $k < n$, $r \ge 1$ and $r \le \frac{n}{2}$.
Let $E$ be an orbit of $\sk{g}$ on $\binom{\Omega}{2}$.
Then $E$ is of one of the following types:
\begin{enumerate}
\item $E$ contains a pair $e \subseteq \binom{K}{2}$.
  Then $E = \{e\}$.
  There are $\binom{k}{2}$ orbits of this type.
\item $E$ contains a pair $e = \{x,y\}$, where $x \in K$ and $y \in M_i$ for
  some $i$.
  Then $E = \{ \{x, z\} \such z \in M_i \}$, and there are $rk$ orbits of this
  type.
\item $E$ contains a pair $e \subseteq \binom{M_i}{2}$ for some $i$.
  Then by the first case, there are
  \[
	\left\lfloor \frac{m_1}{2} \right\rfloor
	+ \left\lfloor \frac{m_2}{2} \right\rfloor
	+ \cdots
	+ \left\lfloor \frac{m_r}{2} \right\rfloor
  \]
  such orbits.
\item $E$ contains a pair $e = \{x,y\}$ with $x \in M_i, y \in M_j$ and $i \ne
  j$.
  By the second case above, there are $\operatorname{gcd}(m_i, m_j)$ of these
  orbits for fixed $i$, $j$, so in total
  \[
	\sum_{1 \le i < j \le r} \operatorname{gcd}(m_i, m_j)
  \]
  orbits of this type.
\end{enumerate}
We counted
\[
  \operatorname{orb}_2(\sk{g})
  = \binom{k}{2} + kr + \sum_{i=1}^r \left\lfloor \frac{m_i}{2} \right\rfloor
  + \sum_{1 \le i < j \le r} \operatorname{gcd}(m_i, m_j).
\]
This is an exact formula, but it's horrendous, so we will use it to derive some
less exact bounds.
We can estimate
\[
  \sum_{i=1}^r \left\lfloor \frac{m_i}{2} \right\rfloor
  \le \sum_{i=1}^r \frac{m_i}{2}
  = \frac{n-k}{2}
\]
and, since the minimum is always less than the average,
\begin{align*}
  \sum_{i<j} \operatorname{gcd}(m_i, m_J)
  &\le \sum_{i<j} \min(m_i, m_j)
  \le \sum_{i<j} \frac{m_i+m_j}{2}
  = \pol (r-1) \sum_{i=0}^r m_i \\
  &= \frac{(r-1)(n-k)}{2}.
\end{align*}
We thus see that
\[
  \operatorname{orb}_2(\sk{g})
  \le \binom{k}{2} + kr + \frac{n-k}{2} + \frac{(r-1)(n-k)}{2}
  = \binom{k}{2} + \frac{r(n+k)}{2}.
\]
This is further equal to
\begin{align*}
  \operatorname{orb}_2(\sk{g})
  &= \binom{k}{2} + \frac{r(n+k)}{2} \\
  &\le \binom{n}{2} + \frac{k(k-1)-n(n-1)}{2} + \frac{r(n+k)}{2} \\
  &= \binom{n}{2} - \frac{(n-r-1)(n-k)}{2}.
\end{align*}

Now let us find some useful lower bounds for $(n-r-1)(n-k)$.
Since $n = k + m_1 + \cdots + m_r$, we have $1 \le r \le \frac{n}{2}$ and $n-k
\ge 2r$.
Plugging this into the expression above yields us
\[
  \operatorname{orb}_2(\sk{g})
  \le \binom{n}{2} - r(n - r - 1).
\]
The second term on the right is a quadratic function $r \mapsto -r(n-r-1)$,
with a minimum at $\frac{n-1}{2}$.
In the bounds $1 \le r \le \frac{n}{2}$, the maximum value of this quadratic
function is achieved at $r = 1$, and it is equal to $-(n-2)$.
So
\[
  \operatorname{orb}_2(\sk{g})
  \le \binom{n}{2} - (n-2).
\]

We can similarly show an alternate bound
\[
  \operatorname{orb}_2(\sk{g})
  \le \binom{n}{2} - \frac{(n-2)(n-k)}{4}
\]
by considering $(n-r-1)(n-k)$ and the inequality $n-r \ge \frac{n}{2}$.

\podnaslov{Graph counting}

Consider the following question:
How many non-isomorphic graphs on $n$ vertices are there?
First, how many graphs with vertex set $[n]$ can we find?
The answer is of course precisely the size of the power set of $\binom{[n]}{2}$,
so $2$ to that power.
But some of those graphs are isomorphic, so we've counted them too many times.

Let $F(n)$ denote the number of non-isomorphic graphs on $n$ vertices.
We're interested in
\[
  \lim_{n \to \infty} \frac{F(n)}{2^{\binom{n}{2}}},
\]
if the limit even exists.

Let $\Gamma, \Delta$ be graphs with vertex set $[n]$.
A permutation $g \in \Sym(n)$ is an isomorphism between $\Gamma$ and $\Delta$ if
and only if $E(\Gamma)$ and $E(\Delta)$ are in the same orbit of $\sk{g}$ in its
action on $P(\binom{[n]}{2})$.
Then the number of all graphs $[n]$, isomorphic to $\Gamma$ is
\[
  \abs{[\Gamma]} = \abs{E(\Gamma)^{\Sym(n)}}
  = \frac{\abs{\Sym(n)}}{\abs{\Sym(n)_{E(\Gamma)}}}
  = \frac{n!}{\abs{\Aut \Gamma}}.
\]
So, since $\Aut \Gamma \ge 1$, we find
\[
  \frac{2^{\binom{n}{2}}}{n!} \le F(n) \le 2^{\binom{n}{2}}.
\]
By the Cauchy-Frobenius lemma, we have
\[
  F(n) = \inv{n!} \sum_{g \in \Sym(n)} \abs{\Fix_2(g)},
\]
where $\abs{\Fix_2(g)}$ is the number of fixed points of $g$ in the action on
$P(\binom{[n]}{2})$.
Note that $g$ fixes a point $E \in P(\binom{[n]}{2})$ if and only if the graph
$([n], E)$ has $g$ in its automorphism group.
We know that
\[
  \abs{\Fix_2(g)} = 2^{\operatorname{orb}_2(\sk{g})},
\]
which gives us a formula for $F(n)$:
\[
  F(n) = \inv{n!} \sum_{g \in \Sym(n)} 2^{\operatorname{orb}_2(\sk{g})}.
\]

\begin{theorem}[Erdös, Renyi]
  There exists a function $f: \N \to \N$ with $\lim f(n) = 0$ such that
  \[
	F(n) = (1 + f(n)) \frac{2^{\binom{n}{2}}}{n!},
  \]
  i.e.~$F(n) = (1 + o(1)) 2^{\binom{n}{2}} / {n!}$
\end{theorem}

\begin{proof}
  We estimate $\operatorname{orb}_2(\sk{g})$ for a given $g \in \Sym(n)$.
  Fix some $m \in \{1, \ldots, n\}$.
  Then split the set $\Sym(n)$ into the following classes:
  \begin{gather*}
	A = \{\id\} \\
	B = \{ g \in \Sym(n) \such \text{$g$ moves more than $m$ points} \} \\
	C = \Sym(n) \setminus A \setminus B.
  \end{gather*}
  We've shown that
  \[
	n! F(n) = \sum_{g \in \Sym(n)} 2^{\operatorname{orb}_2(\sk{g})}.
  \]
  If $g = \id$, then $\operatorname{orb}_2(\{\id\}) = \binom{n}{2}$.
  For $g \in B$, we use
  \[
	\operatorname{orb}_2(\sk{g})
	\le \binom{n}{2} - \frac{(n-2)(n-k)}{4}
	\le \binom{n}{2} - \frac{(n-2) m}{4}.
  \]
  Since $\abs{B} \le n! \le n^n$, the contribution of all elements of $B$ to the
  sum is at most
  \[
	n^n 2^{\binom{n}{2} - \frac{(n-2)m}{4}}.
  \]
  Similarly, for $g \in C$, we use the estimates $\operatorname{orb}_2(\sk{g})
  \le \binom{n}{2} - (n-2)$ and $\abs{C} \le \binom{n}{m} m! \le n^m$.
  We then have
  \[
	n! F(n) \le 2^{\binom{n}{2}} \left(
	  1 + \underbrace{n^n 2^{\displaystyle -(n-2)m / 4} + n^m 2^{\displaystyle
		  -(n-2)}}_{f_m(n)}
	\right).
  \]
  Note that
  \[
	f_m(n) = 2^{\displaystyle n \log_2 n - (n-2)m / 4} + 2^{\displaystyle m \log_2 n - (n-2)}
  \]
  We can choose $m = \lfloor 6 \log_2(n) \rfloor$.
  Then for large enough $n$, we have $5 \log_2(n) \le m \le 6 \log_2(m)$, so
  \[
	f(n) = f_m(n)
	\le 2^{\displaystyle n \log_2(n) - 5 (n-2) \log_2(n) / 4}
	+ 2^{\displaystyle 6 \log_2^2(n) - (n-2)}
  \]
  but this converges to $0$ for $n \to \infty$.
\end{proof}

Fix a number $n \in \N$.
Then let $\mathcal{G}(n)$ be the set of isomorphism classes of graphs on $n$
vertices, so $\abs{\mathcal{G}(n)} = F(n)$.
Also introduce $\mathcal{A}(n)$ as the set of all isomorphisms classes for which
the isomorphism group is trivial.
Denote $A(n) = \abs{\mathcal{A}(n)}$.

\begin{theorem}
  The limit
  \[
	\lim_{n \to \infty} \frac{A(n)}{F(n)} = 1,
  \]
  so most graphs have trivial isomorphism groups.
\end{theorem}

% LocalWords:  Frobenius
