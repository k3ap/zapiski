\naslov{Graphs}

It is implied that every graph is finite, simple and undirected.

\begin{definition}
  The \pojem{arc set} of a graph $\Gamma$ is $A(\Gamma) = \{ (u,v) \such uv
  \in E(\Gamma)\}$.
\end{definition}

\begin{definition}
  A \pojem{morphism} from a graph $\Gamma$ to another graph $\Lambda$ is a
  mapping $\varphi: V(\Gamma) \to V(\Lambda)$ such that for each edge $uv \in
  E(\Gamma)$, there is an edge $\varphi(u) \varphi(v) \in E(\Lambda)$.

\end{definition}

\begin{remark}
  A morphism $\varphi: \Gamma \to \Lambda$ induces a map $E(\Gamma) \to
  E(\Lambda)$ with $uv \in E(\Gamma) \mapsto \varphi(u) \varphi(v) \in
  E(\Lambda)$.
\end{remark}

\begin{definition}
  If $\varphi$ is injective, it is a \pojem{monomorphism}.
  If it is surjective both as a vertex map and an edge map, then it is an
  \pojem{epimorphism}.
\end{definition}

\begin{definition}
  If $\varphi$ is bijective as a vertex map, and $\varphi^{-1}$ is also a graph
  morphism, then $\varphi$ is an \pojem{isomorphism}.
\end{definition}

\begin{definition}
  If $\varphi$ is an isomorphism $\Gamma \to \Gamma$, then $\varphi$ is an
  \pojem{automorphism} of $\Gamma$.
  We denote the set of all automorphisms of $\Gamma$ with $\Aut\Gamma$.
\end{definition}

\begin{lemma}
  Let $\Gamma$ and $\Lambda$ be graphs and let $\varphi: V(\Gamma) \to
  V(\Lambda)$ be a map.
  Then $\varphi$ is an isomorphism if and only if it is bijective and
  $\phi(E(\Gamma)) = E(\Lambda)$.
\end{lemma}

\begin{remark}
  If $g$ is a permutation of $V(\Gamma)$, it is enough to check that $u \sim v$
  implies $u^g \sim v^g$, since the sets $E(\Gamma)$ and $E(\Gamma)^g$ are of
  the same cardinality.
\end{remark}

\begin{remark}
  A permutation $g$ of $V(\Gamma)$ is an automorphism if and only if $g$ is in
  the stabilizer of $E(\Gamma)$ in the action of $\Sym(V(\Gamma))$ on the power
  set of $V(\Gamma)^{\{2\}}$.
  So $\Aut \Gamma = \Sym(V(\Gamma))_{E(\Gamma)}$ in this particular action.
  The orbit-stabiliser theorem then gives us $\abs{\Sym(V)} = \abs{\Aut \Gamma}
  \abs{E(\Gamma)^{\Sym(V)}}$, or for $n = \abs{V(\Gamma)}$,
  \[
	\abs{\Aut \Gamma} = \frac{n!}{\abs{E(\Gamma)^{\Sym V}}}.
  \]
  The number in the denominator is then precisely the number of graphs on vertex
  set $V$ that are isomorphic to $\Gamma$.
\end{remark}