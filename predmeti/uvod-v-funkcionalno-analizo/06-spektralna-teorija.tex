\naslov{Spektralna teorija}

Naj bo $A$ kompleksna Banachova algebra z enoto.
Naj bo $a \in A$.
Definiramo
\[
  \rho(a) = \{ \lambda \in \C \such \text{$\lambda - a$ obrnljiv v $A$} \}.
\]

\begin{opomba}
  Identificiramo $\lambda - a = \lambda \cdot 1 - a$.
\end{opomba}

Pravimo, da je $\rho(a)$ \pojem{resolventa} elementa $a$.
Označimo
\[
  R(\lambda, a) = (\lambda - a)^{-1}.
\]
Preslikava $\lambda \mapsto R(\lambda, a)$ je \pojem{resolventna funkcija}.
Množici $\sigma(a) = \C \setminus \rho(a)$ pravimo \pojem{spekter} elementa $a$.

\begin{trditev}
  Naj bo $A \in B(H)$.
  Tedaj je $\sigma(A^*) = \{ \konj{\lambda} \such \lambda \in \sigma(A) \}$.
\end{trditev}

\begin{proof}
  Naj bo $B$ tak, da je $(\lambda I - A) B = B(\lambda I - A) = I$.
  Če obe strani adjungiramo, dobimo $B^* (\konj{\lambda} I - A^*) =
  (\konj{\lambda} I - A^*) B^* = I$.
\end{proof}

\begin{definicija}
  Naj bo $X$ kompleksen Banachov prostor in $A$ omejen operator na $X$.
  Potem je $\lambda$ v \pojem{zveznem} delu spektra ($\lambda \in \sigma_c(A)$),
  če je $\lambda I - A$ injektiven in $\cl{\im(\lambda I - A)} = X$ ter
  $\im(\lambda I - A) \ne X$.
  Pravimo, da je $\lambda$ v \pojem{residualnem} delu spektra ($\lambda \in
  \sigma_r(A)$), če je $\lambda I - A$ injektiven in $\im (\lambda I - A)$ ni
  gosta.
  Pravimo, da je $\lambda$ v \pojem{točkastem} delu spektra ($\lambda \in
  \sigma_p(A)$), če je $\lambda$ lastna vrednost za $A$.
\end{definicija}

\begin{trditev}
  Naj bo $A \in B(H)$.
  Če je $A = A^*$, je $\sigma_p(A) \subseteq \R$.
  Če je $A$ normalen, so lastni vektorji medseboj pravokotni.
\end{trditev}

\begin{trditev}
  Naj bo $H$ Hilbertov prostor in $A \in B(H)$.
  Če je $\lambda \in \sigma_r(A)$, je $\konj{\lambda} \in \sigma_p(A^*)$.
  Če je $\lambda \in \sigma_p(A)$, je $\konj{\lambda} \in \sigma_p(A^*) \cup
  \sigma_r(A^*)$.
\end{trditev}

\begin{proof}
  Če $\im(A - \lambda I)$ ni gost v $H$, potem je
  \[
	\jedro (A^* - \konj{\lambda} I)
	= \jedro (A - \lambda I)^*
	= \im (A - \lambda I)^\bot
	\ne \{0\},
  \]
  torej obstaja lastni vektor za $\konj{\lambda}$ za $A^*$.

  Za drugo trditev velja $\jedro (\lambda I -A) \ne \{0\}$, če vzamemo
  pravokotni komplement obeh strani, dobimo $\cl{\im (\konj{\lambda} I - A^*)}
  \ne H$.
\end{proof}

\begin{posledica}
  Če je $A \in B(H)$ normalen, je $\sigma_r(A) = \varnothing$.
\end{posledica}

\begin{proof}
  Če je $\lambda \in \sigma_r(A)$, potem je $\konj{\lambda} \in \sigma_p(A^*)$
  in zato
  \[
	\jedro (A - \lambda I) = \jedro(A - \konj{\lambda} I)^*
	= \jedro(A^* - \lambda I) \ne \{0\},
  \]
  kjer smo v prvem koraku uporabili normalnost.
  Sledi $\lambda \in \sigma_p(A)$.
  \protislovje{}
\end{proof}

\begin{lema}
  Naj bo $X$ kompleksen Banachov prostor in $A \in B(X)$.
  Naj bo $\lambda \in \sigma_c(A)$.
  Tedaj obstaja $(x_n)_n$ z $\norm{x_n} = 1$, da $A x_n - \lambda x_n \to 0$.
\end{lema}

\begin{izrek}
  Naj bo $A \in B(H)$ sebi adjungiran operator.
  Tedaj $\sigma(A) \subseteq \R$.
\end{izrek}

\begin{proof}
  Ker je $A$ sebi adjungiran, je normalen, zato $\sigma_r(A) = \varnothing$.
  Dokazati moramo torej le še $\sigma_c(A) \subseteq \R$.
  Če je $\lambda in \sigma_c(a)$, obstaja zaporedje $(x_n)_n$ z $\norm{x_n} = 1$
  in $A x_n - \lambda x_n \to 0$.
  Pišimo $\lambda = \alpha + i \beta$.
  Potem velja
  \[
	\abs{\sk{(\lambda I - A) x_n, x_n}}
	\le \norm{x_n} \norm{(\lambda I - A) x_n}
	\xrightarrow[n \to \infty]{} 0,
  \]
  skalarni produkt pa je enak
  \[
	\sk{(\lambda I - A) x_n, x_n}
	= \sk{\alpha x_n - A x_n, x_n} + i \beta \sk{x_n, x_n}
	= \sk{\alpha x_n - A x_n, x_n} + i \beta,
  \]
  torej $\beta = 0$.
\end{proof}

\begin{izrek}
  Naj bo $A$ Banachova algebra z enico in naj bo $\norm{a} < 1$ za nek $a \in
  A$.
  Tedaj je $1 - a$ obrnljiv in
  \[
	(1 - a)^{-1} = \sum_{n=0}^\infty a^n.
  \]
\end{izrek}

\begin{proof}
  Označimo delne vsote z $s_n$.
  Če je $n > m$, velja
  \[
	\norm{s_n - s_m}
	= \norm{a^{m+1} + \cdots + a^n}
	\le \norm{a}^{m+1} + \cdot + \norm{a}^n
	\le \norm{a}^{m+1} \sum_{k=0}^\infty \norm{a}^k
	= \frac{\norm{a}^{m+1}}{1 - \norm{a}},
  \]
  kar konvergira k $0$ za $n, m \to \infty$.
  Torej je zaporedje Cauchyjevo, in obstaja limita $s$.
  Velja $(1 - a) s_n = 1 - a^{n+1}$.
  Leva stran konvergira k $(1-a)s$, desna pa k $1$.
\end{proof}

\begin{lema}
  Naj bosta $\lambda, \mu \in \rho(a)$.
  Velja $R(\mu, a) - R(\lambda, a) = (\lambda - \mu) R(\lambda, a) R(\mu, a) =
  (\lambda - \mu) R(\mu,a) R(\lambda, a)$.
\end{lema}

\begin{proof}
  Ker $(\lambda - a)$ in $(\mu - a)$ komutirata, komutirata tudi njuna inverza.
  Preostalo je preprost račun.
\end{proof}

\begin{trditev}
  Naj bo $a \in A$, kjer je $A$ kompleksna Banachova algebra z enico.
  Potem je $\rho(a)$ odprta v $\C$, $\sigma(a)$ pa kompaktna in vsebovana v
  $B(0, \norm{a})$.
\end{trditev}

\begin{proof}
  Naj bo $\lambda_0 \in \rho(a)$.
  Iščemo $\varepsilon > 0$, da iz $\abs{\lambda - \lambda_0} < \varepsilon$
  sledi $\lambda \in \rho(a)$.
  Izrazimo lahko
  \[
	\lambda - a = (\lambda_0 - a)(1 + (\lambda - \lambda_0)(\lambda_0 - a)^{-1}).
  \]
  Če je $\norm{(\lambda - \lambda_0)(\lambda_0 -a)^{-1}} \le \varepsilon
  \norm{(\lambda_0 - a)^{-1}} < 1$, potem je
  \[
	1 + (\lambda - \lambda_0)(\lambda_0 - a)^{-1}
  \]
  obrnljiv, torej $\lambda - a$ produkt dveh obrnljivih elementov.

  Za drugo trditev je $\sigma(a) = \C \setminus \rho(a)$ zaprta množica.
  Če je $\abs{\lambda} > \norm{a}$, je
  \[
	\lambda - a = \lambda (1 - \frac{a}{\lambda}),
  \]
  ampak potem $\norm{\nicefrac{a}{\lambda}} < 1$, in je $\lambda \in \rho(a)$.
  Torej je $\sigma(a) \subseteq B(0, \norm{a})$.
\end{proof}

\begin{opomba}
  Če je $\abs{\lambda} > \norm{a}$, je $\lambda \in \rho(a)$ in
  \[
	(\lambda - a)^{-1} = \sum_{n=0}^\infty \frac{a^n}{\lambda^{n+1}}.
  \]
\end{opomba}

\begin{opomba}
  Če je $\lambda_0 \in \rho(a)$ in $\abs{\lambda - \lambda_0} < \norm{(\lambda_0
	- a)^{-1}}^{-1}$, potem je $\lambda \in \rho(a)$ in velja
  \[
	(\lambda - a)^{-1} = (\lambda_0 - a)^{-1} \sum_{n=0}^\infty (\lambda_0 -
	\lambda)^n (\lambda_0 - a)^{-n}.
  \]
\end{opomba}

\begin{trditev}
  Naj bo $\mu \in \rho(a)$.
  Tedaj velja
  \[
	\lim_{\lambda \to \mu} \frac{(\lambda - a)^{-1} - (\mu - a)^{-1}}{\lambda -
	  \mu}
	= -(\mu - a)^{-2}.
  \]
\end{trditev}

\begin{trditev}
  Resolventna funkcija je zvezna.
\end{trditev}

\begin{proof}
  Naj bo $\lambda_0 \in \rho(a)$ in $\abs{\lambda - \lambda_0} <
  \norm{R(\lambda_0, a)}^{-1}$.
  Potem velja
  \[
	R(\lambda,a) = \sum_{n=0}^\infty (\lambda - \lambda_0)^n R(\lambda_0,
	a)^{n+1},
  \]
  torej je
  \[
	R(\lambda, a) - R(\lambda_0, a) = \sum_{n=1}^\infty (\lambda_0 - \lambda)^n
	R(\lambda_0, a)^{n+1}.
  \]
  Iz tega sledi
  \begin{align*}
	\norm{R(\lambda, a) - R(\lambda_0, a)}
	&\le \abs{\lambda - \lambda_0} \norm{R(\lambda_0, a)}^2 \sum_{n=0}^\infty \abs{\lambda - \lambda_0}^n \norm{R(\lambda_0, a)}^n \\
	&= \frac{\abs{\lambda_0 - \lambda} \norm{R(\lambda_0, a)}^2}{1 - \abs{\lambda_0 - \lambda} \norm{R(\lambda_0, a)}} \\
	&\xrightarrow[\lambda \to \lambda_0]{} 0.
	  \qedhere
  \end{align*}
\end{proof}

\begin{izrek}
  Naj bo $a \in A$ in $A$ kompleksna Banachova algebra z enico.
  Tedaj je $\sigma(a) \ne \varnothing$.
\end{izrek}

\begin{proof}
  Recimo $\sigma(a) = \varnothing$.
  Potem je $\rho(a) = \C$, ker pa velja
  \[
	\lim_{\lambda \to \mu} \frac{R(\lambda, a) - R(\mu, a)}{\lambda - \mu} =
	-R(\mu, a)^2,
  \]
  imamo za vsak $\varphi \in A^*$
  \[
	\varphi\!\left( \lim_{\lambda \to \mu} \frac{R(\lambda, a) - R(\mu,
		a)}{\lambda - \mu} \right)
	= \lim_{\lambda \to \mu} \frac{\varphi(R(\lambda, a)) - \varphi(R(\mu,
	  a))}{\lambda - \mu}
	= - \varphi(R(\mu, a)^2),
  \]
  torej je $\varphi \circ R(\cdot, a)$ cela holomorfna funkcija.

  Če je $\abs{\lambda} > \norm{a}$, je
  \[
	\norm{R(\lambda, a)} \le \sum_{n=0}^\infty
	\frac{\norm{a}^n}{\abs{\lambda}^{n+1}}
	= \inv{\abs{\lambda}} \inv{1 - \nicefrac{\norm{a}}{\abs{\lambda}}},
  \]
  to pa konvergira k $0$ za $\abs{\lambda} \to \infty$.
  Torej je $R(\cdot, a)$ omejena, zato je tudi $\varphi \circ R(\cdot, a)$
  omejena.
  Ker je cela holomorfna, je konstantna in velja $\varphi(R(\lambda, a)) = 0$ za
  vsak $\lambda \in \C$.
  Za $\lambda \in \rho(a)$ velja $(\lambda - a)R(\lambda, a) = 1$, torej
  $R(\lambda, a) \ne 0$, in smo prišli v protislovje s Hahn-Banachom.
\end{proof}

\begin{definicija}
  Naj bo $a \in A$.
  Definirajmo
  \[
	r(a) = \sup_{\lambda \in \sigma(a)} \abs{\lambda} = \max_{\lambda \in
	  \sigma(a)} \abs{\lambda}.
  \]
  To število imenujemo \pojem{spektralni radij} elementa $a$.
\end{definicija}

\begin{lema}
  če je $\lambda \in \sigma(a)$, je $\lambda^n \in \sigma(a^n)$.
\end{lema}

\begin{proof}
  Dokažemo obratno; če je $\lambda^n \in \rho(a^n)$, je $\lambda \in \rho(a)$.
  Računamo
  \[
	\lambda^n - a^n = (\lambda -a)(\lambda^{n-1} + \cdots + a^{n-1})
  \]
  in desni člen produkta označimo z $b$.
  Obstaja $c \in A$, da je $(\lambda^n - a^n)c = c(\lambda^n - a^n) =1$, torej
  $(\lambda - a)bc = 1$ in ima $\lambda - a$ desni inverz.
  Podobno pokažemo, da ima levi inverz.
\end{proof}

\begin{trditev}
  Naj bo $a \in A$ in $p$ polinom.
  Tedaj je $\sigma(p(a)) = p(\sigma(a))$.
\end{trditev}

\begin{izrek}[Gelfandova formula]
  Naj bo $a \in A$.
  Potem je
  \[
	r(a) = \lim_{n \to \infty} \norm{a^n}^{1/n} = \liminf_{n \to \infty}
	\norm{a^n}^{1/n}
	= \inf_{n \in \N} \norm{a^n}^{1/n}.
  \]
\end{izrek}

\begin{proof}
  Naj bo $\lambda \in \sigma(a)$, da je $\abs{\lambda} = r(a)$.
  Potem je $\lambda^n \in \sigma(a^n)$, torej $\abs{\lambda}^n \le \norm{a^n}$
  in velja $r(a) = \abs{\lambda} \le \norm{a^n}^{1/n}$ za vsak $n \in \N$.
  Zaporedje je navzdol omejeno, torej ima infimum, ki je
  \[
	r(a) \le \inf_{n \in \N} \norm{a^n}^{1/n} \le \liminf_{n \to \infty}
	\norm{a^n}^{1/n}.
  \]
  Če pokažemo, da je $\limsup \norm{a^n}^{1/n} \le r(a)$, potem bo dokaz končan.

  Naj bo $f \in A^*$.
  Definiramo $f\tilde{f}(\lambda) = f(R(\lambda, a))$ kot preslikavo $\rho(a)
  \to \C$.
  To je holomorfna funkcija.
  Če je $\abs{\lambda} > \norm{a}$, lahko $R(\lambda, a)$ eksplicitno izrazimo
  in zato velja
  \[
	\tilde{f}(\lambda) = \sum_{n=0}^\infty \frac{f(a^n)}{\lambda^{n+1}}.
  \]
  To je Laurantov razvoj $\tilde{f}$ v okolici $0$ na komplement krogle $B(0,
  \norm{a})$.
  Razvoj lahko naredimo tudi na $B(0, r(a))\complement$.
  Ker je enoličen, zgornji predpis velja za vsak $\lambda > r(a)$.
  Za $\varepsilon > 0$ in $\lambda = r(a) + \varepsilon$ torej konvergira vrsta
  \[
	\sum_{n=0}^\infty \frac{f(a^n)}{(r(a) + \varepsilon)^{n+1}},
  \]
  torej obstaja $M > 0$, da je $\abs{f(a^n) / (r(a) + \varepsilon)^{n+1}} \le M$
  za vse $n \in \N$.
  Po principu šibke omejenosti je potem množica
  \[
	\left\{ \frac{a^n}{(r(a) + \varepsilon)^{n+1}} \such n \in \N\right\}
  \]
  omejena in obstaja $\tilde{M} > 0$, da
  \[
	\norm{\frac{a^n}{(r(a) + \varepsilon)^{n+1}}} \le \tilde{M}
  \]
  za vse $n \in \N$.
  Sledi
  \[
	\limsup_{n \to \infty} \norm{a^n}^{1/n} \le \lim_{n \to \infty} (\tilde{M}
	(r(a) + \varepsilon))^{1/n} (r(a) + \varepsilon)
	= r(a) + \varepsilon,
  \]
  saj prvi člen produkta konvergira k $1$.
  Torej je $\limsup \norm{a^n}^{1/n} \le r(a)$.
\end{proof}

\begin{posledica}
  Naj bo $A$ sebi adjungiran operator na Hilbertovem prostoru.
  Tedaj velja $r(A) = \norm{A}$.
\end{posledica}

\begin{proof}
  Po izreku velja $r(A) = \lim \norm{A^n}^{1/n}$.
  Ker je $\norm{A^2} = \norm{A^* A} = \norm{A}^2$, lahko z indukcijo pokažemo,
  da velja $\norm{A^{2^n}} = \norm{A}^{2^n}$.
  S tem dobimo podzaporedje zaporedja v limiti, ki bo konvergiralo k $\norm{A}$.
\end{proof}

\podnaslov{Spekter kompaktnega operatorja}

\begin{primer}
  Če je $H = l^2$ in $(e_n)_n$ neka ortonormirana baza.
  Naj bo $(d_n)_n$ zaporedje omejenih števil.
  Definiramo diagonalni operator $D: l^2 \to l^2$ z
  \[
	Dx = \sum_{n=0}^\infty d_n \sk{x, e_n} e_n.
  \]
  Preverimo lahko naslednja dejstva:
  \begin{itemize}
  \item $\norm{D} = \sup_n \abs{d_n}$,
  \item $D$ je sebi adjungiran natanko tedaj, ko je $d_n \in \R$ za vsak $n$,
  \item $D$ je normalen,
  \item $D$ je unitaren natanko tedaj, ko je $\abs{d_n} = 1$ za vsak $n$,
  \item $D$ je kompakten natanko tedaj, ko je $d_n \to 0$.
  \end{itemize}
\end{primer}

\begin{lema}[Rieszova lema o pravokotnici]
  Naj bo $X$ normiran prostor in $Y$ zaprt podprostor v $X$.
  Tedaj za vsak $\varepsilon \in (0,1)$ obstaja $x \in X$, da velja $\norm{x} =
  1$ in $d(x,Y) \ge \varepsilon$.
\end{lema}

\begin{opomba}
  Če je $X$ končnorazsežen, obstaja $x \in X$ z $\norm{x} = 1$ in $d(x,Y) = 1$.
  Skličemo se lahko na kompaktnost.
\end{opomba}

\begin{trditev}
  Naj bo $Y$ končnorazsežen podprostor v normiranem prostoru $X$.
  Tedaj obstaja zaprt podprostor $Z \subseteq X$, da je $X = Y \oplus Z$.
\end{trditev}

\begin{proof}
  Naj bo $\{e_1, \ldots, e_n\}$ baza za $Y$ in $\{f_1, \ldots, f_n\}$ dualna
  baza.
  Ker je dimenzija $Y$ končna, je $f_i$ omejen za vse $i=1, \ldots, n$.
  Naj bo $F_i$ katerakoli Hahn-Banachova razširitev funkcije $f_i$.
  Definiramo
  \[
	Z = \bigcap_{n=1}^n \jedro F_i.
  \]
  Ker je $F_i$ omejen, je $\jedro F_i$ zaprta množica, torej je presek zaprt.

  Če je $x \in X$, lahko izrazimo
  \[
	x = \underbrace{ F_1(x) e_1 + \cdots + F_n(x) e_n }_{\in Y}
	+ \underbrace{ x - F_1(x) e_1 - \cdots - F_n(x) e_n }_{\in Z}.
  \]
  Če je $x \in Y \cap Z$, velja $x = \alpha_1 e_1 + \cdots + \alpha_n e_n$,
  ampak tudi $F_i(x) = 0 = \alpha_i$ za vse $i$, torej $x = 0$.
\end{proof}

\begin{trditev}
  Naj bo $X$ Banachov prostor in $K \in K(X)$.
  Če je $\lambda \ne 0$, je $\dim \jedro (K - \lambda I) < \infty$.
  Dodatno je $\im(K - \lambda I)$ zaprta v $X$.
\end{trditev}

\begin{proof}
  Izrazimo lahko $K - \lambda I = \lambda (\inv{\lambda} K - I)$, torej je
  $\jedro (K - \lambda I) = \jedro (\inv{\lambda} K - I)$.
  Označimo drugo preslikavo z $S$.
  Velja $x \in \jedro (S - I) \iff Sx = x$, torej je $\left. S \right|_{\jedro
	(S-I)}$ identiteta, in hkrati kompakten operator.
  Torej je jedro končnorazsežno.

  Za drugo trditev zapišemo $X = \jedro(K - \lambda I) \oplus Y$, kjer je $Y$
  zaprt podprostor $X$.
  Velja
  \[
	\im (K - \lambda I)
	= (K - \lambda I) (\jedro (K - \lambda I) \oplus Y)
	= (K - \lambda I) Y
	= \im (\left. K - \lambda I \right|_Y).
  \]
  Označimo $T = K - \lambda I$.

  Dovolj je dokazati, da je $\left. T \right|_Y$ navzdol omejen, saj bo potem
  topološki izomorfizem $Y \to \im T$.
  Ker je $Y$ zaprt, je Banachov, zato bo $\im T$ Banachov, polni prostori pa so
  vedno zaprti.

  Recimo, da $\left. T \right|_Y$ ni navzdol omejen.
  Tedaj za vsak $n \in \N$ obstaja $y_n \in Y$, da $\norm{y_n} = 1$ in $\norm{T
	y_n} \le \nicefrac{1}{n}$.
  Ker je $K$ kompakten, obstaja podzaporedje $(y_{n_k})_k$, da $(K
  y_{n_k})_k$ konvergira.
  Velja
  \[
	y_{n_k} = \lambda \inv{\lambda} y_{n_K} = \inv{\lambda} \left( \lambda
	  y_{n_k} - K y_{n_k} + K y_{n_k} \right) \to y
  \]
  za nek $y \in Y$ z $\norm{y} = 1$ saj sta prva dva člena vsote del zaporedja
  $T y_n$, ki konvergira.
  Enostavno lahko pokažemo, da velja $Ky = \lambda y$.
  To pomeni, da je $y \in \jedro (K - \lambda I) \cap Y = \{0\}$, ampak
  $\norm{y} = 1$.
  \protislovje{}
\end{proof}

\begin{lema}
  Naj bo $K$ kompakten operator na kompleksnem Banachovem prostoru $X$.
  Tedaj ima $K$ za vsak $\varepsilon > 0$ le končno mnogo linearno neodvisnih
  lastnih vektorjev za lastne vrednosti $\lambda$ z $\abs{\lambda} \ge
  \varepsilon$.
\end{lema}

\begin{proof}
  Recimo, da imamo $\varepsilon > 0$ in zaporedje $(x_n)_n$ linearno neodvisnih
  lastnih vektorjev, ki zaporedoma pripadajo lastnim vrednostim $\lambda_n$ z
  $\abs{\lambda_n} \ge \varepsilon$.
  Definirajmo $X_n$ kot linearno ogrinjačo prvih $n$ lastnih vektorjev.
  Velja $\dim X_n < \infty$, torej so ti prostori zaprti, hkrati pa velja\
  \[
	\{0\} \subsetneq X_1 \subsetneq X_2 \subsetneq X_3 \subsetneq \cdots
  \]
  Po Rieszovi lemi o pravokotnici obstaja $y_n \in X_n \setminus X_{n-1}$ z
  $\norm{y_n} = 1$, da velja $d(y_n, X_{n-1}) \ge \pol$ za vse $n \ge 2$.
  Pišimo $y_n = \alpha_1 x_1 + \cdots + \alpha_n x_n$ in $z_n = y_n /
  \lambda_n$.
  Potem velja $\norm{z_n} = \abs{\lambda_n}^{-1} \le \inv{\varepsilon}$ ter
  \[
	K z_n - y_n
	= \alpha_1\left(\frac{\lambda_1}{\lambda_n} - 1\right) x_1
	+ \cdots + \alpha_{n-1} \left(\frac{\lambda_{n-1}}{\lambda_n} - 1\right)
	x_{n-1} \in X_{n-1}.
  \]
  Za $m < n$ izračunamo
  \[
	\norm{K z_n - K z_m}
	= \| \underbrace{y_n}_{\in X_n} - \underbrace{(y_n - K z_n + K z_m)}_{\in
	  X_{n-1}} \|
	\ge \pol
  \]
  torej $(K z_n)_n$ nima Cauchyjevega podzaporedja, zato nima stekališča.
  To je protislovje s kompaktnostjo $K$.
\end{proof}

\begin{izrek}
  Za kompakten operator $K$ na kompleksnem Banachovem prostoru $X$ veljajo
  naslednje trditve:
  \begin{itemize}
  \item če je $\dim X = \infty$, je $0 \in \sigma(K)$,
  \item če je $\lambda \in \sigma(K) \setminus \{0\}$, je $\lambda$ lastna
	vrednost $K$,
  \item $\sigma(K)$ je končen ali števno neskončen,
  \item če je $\sigma(K)$ neskončen in so $(\lambda_n)_n$ lastne vrednosti,
	štete z geometrijsko večkratnostjo, potem $\lambda_n \to 0$.
  \end{itemize}
\end{izrek}

\begin{proof}
  Prva točka:
  Če $0 \notin \sigma(K)$, je $K$ obrnljiv in je zato $\id = K K^{-1}$
  kompaktna, torej je $\dim X < \infty$.

  Tretja točka:
  Računamo
  \[
	\sigma(K) \setminus \{0\}
	= \bigcup_{n \in \N} \sigma(K) \cap \left\{ \lambda \in \C \such \abs{\lambda}
	\ge \inv{n} \right\}.
  \]
  To je števna unija končnih množic.

  Četrta točka:
  Iz leme dobimo, da za vsak $\varepsilon > 0$ obstaja $n_\varepsilon$, za
  katerega za poljuben $n \ge n_\varepsilon$ velja $\abs{\lambda_n} <
  \varepsilon$.
  Torej $\lambda_n \to 0$.

  Tretja točka:
  Recimo, da je $0 \ne \lambda \in \sigma(K)$.
  Recimo, da je $K - \lambda I$ injektiven.
  Če dokažemo, da je $K - \lambda I$ surjektiven, bo sledilo, da je obrnljiv z
  omejenim inverzom, oziroma $\lambda \in \rho(K)$.

  Pokažimo prvo, da je $\im (K - \lambda I)^n$ zaprta za vsak $n$.
  Trditev pove, da ima $K - \lambda I$ zaprto sliko.
  Izrazimo lahko
  \[
	(K - \lambda I)^n
	= \sum_{n=0}^\infty \binom{n}{k} (-\lambda)^k K^{n-k}
	=: \tilde{K} + (-\lambda)^n I.
  \]
  Operator $\tilde{K}$ je kompakten, torej ima $(K - \lambda I)^n$ zaprto sliko.

  Definirajmo $X_n = \jedro (K - \lambda I)^n$ za $n \in \N_0$.
  Velja $X_0 = X \supseteq X_1 \supseteq X_2 \supseteq \cdots$.
  Recimo, da so vse inkluzije stroge.
  Po Rieszovi lemi o pravokotnici za vsak $n \in \N$ obstaja $x_n \in X_{n-1}
  \setminus X_n$ z $\norm{x_n} = 1$ in $d(x_n, X_n) \ge \pol$.
  Za $m > n$ je
  \[
	\norm{K x_m - K x_n}
	= \norm{\lambda x_n + K x_n - \lambda x_n - K x_m}
	= \abs{\lambda} \norm{x_n + \inv{\lambda} (K - \lambda I) x_n -
	  \inv{\lambda} K x_m}.
  \]
  Velja $x_m \in X_{m-1}$, torej $K x_m \in X_{m-1}$, iz česar sledi $K x_m \in
  X_n$, saj je $m > n$.
  Hkrati je $(K - \lambda I) x_n \in (K - \lambda I) X_{n-1} =X_n$, torej
  $\inv{\lambda}((K - \lambda I) x_n - K x_m) \in X_{n-1}$.
  Velja torej $\norm{K x_n - K x_m} ge \pol \abs{\lambda}$.
  Kot prej pridemo v protislovje s kompaktnostjo operatorja $K$.
  Torej mora obstajati $n$, za katerega je $X_n = X_{n+1}$.

  Za konec recimo, da $K - \lambda I$ ni surjektiven.
  Potem $X \ne (K - \lambda I) X$, velja pa $(K - \lambda I)^n X = (K - \lambda
  I)^{n+1} X$.
  To je protislovje z injektivnostjo $(K - \lambda I)^n$.
\end{proof}

\begin{izrek}
  Naj bo $K: H \to H$ kompakten sebi adjungiran operator na kompleksnem
  Hilbertovem prostoru $H$.
  Potem obstajata zaporedje $(\lambda_n)_n \subseteq \R$ in ortonormiran sistem
  $(e_n)_n$ (lahko sta oba končna) z
  \begin{itemize}
  \item $\abs{\lambda_1} \ge \abs{\lambda_2} \ge \cdots$, $\lambda_n \ne 0$, in
	če je zaporedje neskončno, velja $\lambda_n \to 0$,
  \item $K e_n = \lambda_n e_n$,
  \item če je $\lambda \in \sigma_p(K) \setminus \{0\}$, se $\lambda$ pojavi v
	$(\lambda_n)_n$ natanko tolikokrat, kot je $\dim \jedro (K - \lambda I)$,
  \item $Kx = \sum \lambda_n \sk{x, e_n} e_n$.
  \end{itemize}
\end{izrek}

\begin{proof}
  Ker je $K$ sebi adjungiran, velja $r(K) = \norm{K}$ in $\sigma(K) \subseteq
  [-\norm{K}, \norm{K}]$.
  Če je $K = 0$, nimamo nič za dokazati.
  Sicer lahko razvrstimo lastne vrednosti $K$ v padajoče zaporedje po absolutni
  vrednosti, pri čemer jih štejemo glede na geometrijske večkratnosti.
  Če je $\lambda \in \sigma_p(K) \setminus \{0\}$, lahko izberemo ortonormirano
  bazo za $\jedro (K - \lambda I)$.
  Če je $\mu \ne \lambda$ tudi neničelna lastna vrednost, bo njena ONB
  pravokotna na ONB za $\lambda$.
  Postopek ponovimo za vsako lastno vrednost v zaporedju.
  Dobljene vektorje zložimo v $(e_n)_n$, da je $K e_n = \lambda_n e_n$.
  Definiramo $Y$ kot zaprtje linearne ogrinjače teh vektorjev.
  Potem je $(e_n)_n$ ONB za $Y$ in $H = Y \oplus Y^\bot$.

  Ker je $K e_n = \lambda_n e_n$, je linearna ogrinjača $(e_n)_n$ invariantna na
  $K$, ki je zvezen, torej enako velja za $Y$.
  Iz sebi adjungiranosti sledi $K(Y^\bot) \subseteq Y^\bot$.
  Če je $\left. K \right|_{Y^\bot} \ne 0$, je njegova norma neničeln lastni
  vektor za $K$, kar je nemogoče, ker smo že našli vse lastne vrednosti.
  Torej za $x \in H$ velja $Kx = \sum \lambda_n \sk{x, e_n} e_n$.
\end{proof}

% LocalWords:  resolventa resolventna adjungiramo medseboj residualnem Rieszova
% LocalWords:  Hahn-Banachom Gelfandova Hahn-Banachova Rieszovi Cauchyjevega
% LocalWords:  večkratnostjo večkratnosti
