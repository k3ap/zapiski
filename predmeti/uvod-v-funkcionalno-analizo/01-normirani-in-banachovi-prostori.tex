\naslov{Normirani in Banachovi prostori}

\begin{definicija}
  Naj bo $X$ vektorski prostor nad poljem $\F \in \{\R, \C\}$.
  Preslikava $\norm{\cdot}: X \to \R$ je \pojem{norma}, če velja:
  \begin{itemize}
  \item $\norm{x} \ge 0$,
  \item $\norm{x} = 0 \iff x = 0$,
  \item $\norm{\lambda x} = \abs{\lambda} \norm{x}$,
  \item $\norm{x + y} \le \norm{x} + \norm{y}$.
  \end{itemize}
\end{definicija}

\begin{opomba}
  Velja $\abs{\norm{x} - \norm{y}} \le \norm{x - y}$, iz česar sledi, da je
  norma zvezna (celo Lipschitzova za $L = 1$).
\end{opomba}

Norma porodi metriko $d(x,y) = \norm{x-y}$ na prostoru $X$, ki je invariantna na
translacije, in za katero velja
\[
  d(\lambda x, \lambda y) = \abs{\lambda} d(x,y).
\]
Zaprto kroglo radija $r$ s središčem v točki $x$ označimo z $B(x, r)$, odprto
kroglo pa z $\mathring{B}(x, r)$.
Zaradi zveznosti norme je zaprtje odprte krogle natanko pripadajoča zaprta
krogla.

\begin{definicija}
  Normiran prostor je \pojem{Banachov}, če je poln za inducirano metriko.
\end{definicija}

\begin{trditev}
  Seštevanje in množenje vektorjev s skalarjem sta zvezni operaciji.
\end{trditev}

\begin{definicija}
  Algebra $A$ je \pojem{normirana algebra}, če je normiran vektorski prostor in
  če velja $\norm{xy} \le \norm{x} \norm{y}$.
  Če ima normirana algebra enoto, zahtevamo še $\norm{e} = 1$.
\end{definicija}

\begin{primer}
  Naj bo $X$ Hausdorffov topološki prostor in $\zveznei_b(X)$ množica zveznih
  omejenih funkcij $X \to \F$.
  Če jo opremimo s supremum normo, postane normirana algebra za seštevanje in
  množenje po točkah.
  Preverimo lahko, da je celo Banachov prostor.
\end{primer}

\begin{posledica}
  Če je $X$ kompakten Hausdorffov prostor, je $\zvezne{X}$ Banachova algebra.
\end{posledica}

\begin{trditev}
  Naj bo $X$ normiran prostor in $Y$ (vektorski) podprostor v $X$.
  Veljata naslednji točki.
  \begin{itemize}
  \item Če je $Y$ poln, potem je $Y$ zaprt v $X$.
  \item Če je $X$ Banachov prostor, potem je $Y$ Banachov natanko tedaj, ko je
	$Y$ zaprt v $X$.
  \end{itemize}
\end{trditev}

\begin{proof}
  Prva točka:
  Naj bo $y \in \cl{Y}$.
  Obstaja zaporedje $(y_n)_n$ v $Y$, ki konvergira k $y$.
  To zaporedje je Cauchyjevo, torej ima limito, ki je enaka $y$.
  Sledi $y \in Y$, zato $Y = \cl{Y}$.

  Druga točka:
  V desno smo ravno dokazali.
  V levo naj bo $(y_n)_n$ Cauchyjevo zaporedje v $Y$.
  Potem je Cauchyjevo tudi v $X$, kjer ima limito, saj je $X$ poln.
  Ker je $Y$ zaprt, je $y \in Y$, torej $y_n \to y$ v $Y$.
\end{proof}

\begin{primer}
  Naj bo $X$ lokalno kompakten Hausdorffov prostor ter $\zveznei_0(X)$ množica
  vseh funkcij v $\zvezne{X}$, za katere za vsak $\varepsilon > 0$ obstaja
  kompakt $K_\varepsilon \subseteq X$, da je $\abs{f}$ zunaj $K_\varepsilon$
  strogo manjša od $\varepsilon$.

  Pri $X = \R$ so to natanko vse funkcije, katerih limita v obeh neskončnostih
  je enaka $0$, pri $X = \N$ pa je to natanko prostor $c_0$ zaporedij, ki
  konvergirajo k $0$.

  Dokažemo lahko, da je $\zveznei_0(X)$ zaprt dvostranski ideal v
  $\zveznei_b(X)$ in zato Banachova algebra.
\end{primer}

\begin{primer}
  Množica $c$ vseh konvergentnih zaporedij je Banachov prostor za supremum
  normo.
\end{primer}

\podnaslov{Napolnitve normiranih prostorov}

Naj bo $X$ normiran prostor, ki ni poln, in naj bo $\tilde{X}$ množica vseh
Cauchyjevih zaporedij v $X$.
To je vektorski prostor za operacije po komponentah.
Definiramo
\[
  \norm{(x_n)_n} = \lim_{n \to \infty} \norm{x_n}.
\]
Ta izraz je dobro definiran, saj velja $\abs{\norm{x_n} - \norm{y_n}} \le
\norm{x_n - y_n}$, torej je zaporedje norm Cauchyjevo in konvergira v $\R$.
To pa ni norma, ker obstaja veliko zaporedij z limito norm enako $0$.
Na $\tilde{X}$ zato vpeljemo ekvivalenčno relacijo
\[
  (x_n)_n \sim (y_n)_n \iff x_n - y_n \xrightarrow[n \to \infty]{} 0.
\]
Sedaj definiramo $\hat{X} = \tilde{X} / \sim$.
V $\hat{X}$ potem vpeljemo operaciji seštevanja in množenja s skalarjem, ki
delujeta na predstavnikih, ter podobno vpeljemo normo.

\begin{izrek}
  Prostor $(\hat{X}, \norm{\cdot})$ je Banachov in vsebuje $X$ kot gost
  podprostor.
\end{izrek}

\begin{proof}
  Dokaz o polnosti izpustimo.
  Definiramo vložitev $j: X \to \hat{X}$ s predpisom $x \mapsto [(x)_n]$.
  To je očitno linearna preslikava, za katero velja $\norm{j(x)} = \norm{x}$,
  torej je tudi izometrija.
  Pokazali bomo, da je $j(X)$ gost podprostor v $\hat{X}$.

  Naj bo $[(x_n)_n] \in \hat{X}$.
  Za poljuben $\varepsilon > 0$ obstaja $n_\varepsilon$, da za vse $n, m \ge
  n_\varepsilon$ velja $\norm{x_n - x_m} < \varepsilon$.
  Za $m = n_\varepsilon$ dobimo $j(x_{n_\varepsilon}) =
  [(x_{n_\varepsilon})_n]$, in velja
  \[
	\norm{j(x_{n_\varepsilon}) - [(x_n)_n]} = \lim_{n \to \infty}
	\norm{x_{n_\varepsilon} - x_n} \le \varepsilon.
	\qedhere
  \]
\end{proof}

\begin{posledica}
  Prostor $X$ je Banachov natanko tedaj, ko je $j(X) = \hat{X}$.
\end{posledica}

\begin{proof}
  Ideja.
  Izometrije ohranjajo polnost, polni podprostori so zaprti.
  Če so gosti, so enaki celoti.
\end{proof}

\podnaslov{Osnovne konstrukcije}

Naj bo $X$ vektorski prostor nad $\F$.
Normi $\norm{\cdot}_1$ in $\norm{\cdot}_2$ sta \pojem{ekvivalentni}, če
obstajata $\alpha, \beta > 0$, da za vse $x \in X$ velja
\[
  \alpha \norm{x}_1 \le \norm{x}_2 \le \beta \norm{x}_1.
\]
Topologiji, ki jih normi porodita, sta enaki, zato je identiteta $(X,
\norm{\cdot}_1) \to (X, \norm{\cdot}_2)$ linearni homeomorfizem.

\begin{definicija}
  Normirana prostora $X$ in $Y$ sta \pojem{izomorfna}, če obstaja linearni
  homeomorfizem med njima.
\end{definicija}

Če sta normi ekvivalentni, je $(X, \norm{\cdot}_1)$ Banachov natanko tedaj, ko
je $(X, \norm{\cdot}_2)$ Banachov.
Če je $Y \subseteq X$ podprostor in $X$ normiran, je tudi $Y$ normiran, če normo
zožimo na $Y$.
Vložitev $Y$ v $X$ je izometrija.

\begin{lema}
  Če je $Y \subseteq X$ podprostor in $X$ normiran, je $\cl{Y}$ podprostor.
\end{lema}

\begin{proof}
  Naj bosta $x, y \in \cl{Y}$ in $\alpha, \beta \in \F$.
  Potem obstajata zaporedji $x_n \to x$ in $y_n \to y$.
  Velja $\alpha x_n + \beta y_n \to \alpha x + \beta y$.
\end{proof}

Če je $X$ normiran in $Y \le X$, lahko na $X$ vpeljemo relacijo $x_1 \sim x_2
\iff x_1 - x_2 \in Y$.
V kvocientni prostor vpeljemo
\[
  \norm{x + Y} = \inf \{ \norm{x+y} \such y \in Y \}.
\]

\begin{trditev}
  Naj bo $X$ normiran prostor in $Y \le X$.
  \begin{itemize}
  \item $\norm{\cdot}$ je polnorma na $X/Y$.
  \item $\norm{\cdot}$ je norma na $X/Y$ natanko tedaj, ko je $Y$ zaprt v $X$.
  \item Če je $X$ Banachov, je kvocient Banachov.
  \end{itemize}
\end{trditev}

\begin{proof} \hfill
  \begin{itemize}
  \item Točka je le vprašanje preverjanja; dokaz izpustimo.
  \item S sklicem na prvo točko moramo dokazati le, da je $\norm{x + Y} = 0 \iff
	x \in Y$.
	Če je $\norm{x + Y} = 0$, je $d(x, Y) = 0$, ker pa je $Y$ zaprta, to pomeni
	$x \in Y$.
	Podobno v obratno smer.

  \item Naj bo $(x_n + Y)_n$ Cauchyjevo zaporedje v $X/Y$.
	Poiskali bomo podzaporedje $(x_{n_k} + Y)_k$, ki bo konvergiralo.
	Ker je prvotno zaporedje Cauchyjevo, bo tudi konvergiralo k isti limiti.

	Vemo, da za poljuben $\varepsilon > 0$ obstaja $n_\varepsilon \in \N$, za
	katerega za vse $n, m \ge n_\varepsilon$ velja $\norm{x_n - x_m + Y} <
	\varepsilon$.
	Sedaj induktivno konstruiramo podzaporedje $(x_{n_k} + Y)_k$, da za vsak $k$
	velja $\norm{x_{n_{k+1}} -x_{n_k} + Y} < 2^{-k}$.
	Ko to zaporedje imamo, po definiciji infimuma obstaja tak $y_k \in Y$, da je
	$\norm{x_{n_{k+1}} - x_{n_k} + y_k} < 2^{-k}$.
	Definiramo $z_1 = 0$ in $z_{k+1} = z_k + y_k \in Y$.
	Tedaj
	\[
	  \norm{(x_{n_{k+1}} + z_{k+1}) - (x_{n_k} + z_k)} = \norm{x_{n_{k+1}} -
		x_{n_k} + y_k} < 2^{-k}.
	\]
	Za $w_k = x_{n_k} + z_k$ velja $\norm{w_{k+1} - w_k} < 2^{-k}$, hkrati pa je
	\[
	  \norm{w_{m+k} - w_m} = \norm{\sum_{i=0}^{k-1} w_{m+i+1} - w_{m'i}}
	  \le \sum_{i=0}^{k-1} \norm{w_{m+i+1} - w_{m+i}}
	  < \sum_{i=0}^{k-1} 2^{-m-i}
	\]
	kar je manjše od $2^{1-m}$.
	Sedaj je $(w_m)_m$ Cauchyjevo v $X$, torej obstaja limita $x \in X$.
	Zato je
	\[
	  \norm{(x_{n_i} + Y) - (x + Y)} \le \norm{x_{n_i} - x + z_i}
	  = \norm{w_i - x} \to 0.
	  \qedhere
	\]
  \end{itemize}
\end{proof}

\begin{trditev}
  Naj bo $Y$ zaprt podprostor normiranega prostora $X$.
  Tedaj je $X$ Banachov natanko tedaj, ko sta tako $Y$ kot $X/Y$ Banachova.
\end{trditev}

\begin{proof}
  V desno smo ravno dokazali.
  V levo:
  Naj bo $(x_n)_n$ Cauchyjevo zaporedje v $X$.
  Vemo, da je $\norm{x + Y} \le \norm{x}$, torej je tudi $(x_n + Y)_n$
  Cauchyjevo zaporedje in ima limito $x + Y$.

  Za vsak $n \in \N$ obstaja tak $y_n$, da je
  \[
	\norm{x_n - x + y_n} < \norm{x_n - x + Y} + \inv{n}.
  \]
  Ker velja
  \[
	\norm{y_n - y_m} \le \norm{y_n + x_n - x} + \norm{y_m + x_m - x} + \norm{x_m
	- x_n},
  \]
  je $(y_n)_n$ Cauchyjevo, torej ima limito $y \in Y$.
  Sledi $\lim x_n = x - y$.
\end{proof}

\begin{posledica}
  Vsak končnorazsežen normiran prostor je Banachov.
\end{posledica}

\begin{proof}
  Indukcija na $d = \dim X$.
  Za $d = 1$ izberimo $x \in X$ z $\norm{x} = 1$.
  Tedaj za vsak $y \in X$ velja $y = \lambda x$ in $\norm{y} = \abs{\lambda}$.
  Naj bo $(y_n)_n$ Cauchyjevo v $X$.
  Potem je $y_n = \lambda_n x$ in $\norm{y_n - y_m} = \abs{\lambda_n -
	\lambda_m}$,
  zato je $(\lambda_n)_n$ Cauchyjevo v $\F$ in ima limito $\lambda$.
  Seveda $\lambda_n x \to \lambda x$.

  Recimo, da so vsi končnorazsežni normirani prostori dimenzije $d-1$ ali manj
  polni.
  Naj bo $Y \le X$ poljuben enorazsežen prostor.
  Po predpostavki sta $Y$ in $X/Y$ Banachova.
\end{proof}

Produkt $X \times Y$ lahko opremimo z eno od spodnjih norm:
\begin{itemize}
\item $\norm{(x,y)}_\infty = \max \{ \norm{x}_X, \norm{y}_Y \}$
\item $\norm{(x,y)}_p = \left( \norm{x}_X^p + \norm{y}_Y^p \right)^{1/p}$
\end{itemize}
Produkt je Banachov natanko tedaj, ko sta $X$ in $Y$ Banachova.

% LocalWords:  Banachovi
