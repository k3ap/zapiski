\naslov{Normirani in Banachovi prostori}

\begin{definicija}
  Naj bo $X$ vektorski prostor nad poljem $\F \in \{\R, \C\}$.
  Preslikava $\norm{\cdot}: X \to \R$ je \pojem{norma}, če velja:
  \begin{itemize}
  \item $\norm{x} \ge 0$,
  \item $\norm{x} = 0 \iff x = 0$,
  \item $\norm{\lambda x} = \abs{\lambda} \norm{x}$,
  \item $\norm{x + y} \le \norm{x} + \norm{y}$.
  \end{itemize}
\end{definicija}

\begin{opomba}
  Velja $\abs{\norm{x} - \norm{y}} \le \norm{x - y}$, iz česar sledi, da je
  norma zvezna (celo Lipschitzova za $L = 1$).
\end{opomba}

Norma porodi metriko $d(x,y) = \norm{x-y}$ na prostoru $X$, ki je invariantna na
translacije, in za katero velja
\[
  d(\lambda x, \lambda y) = \abs{\lambda} d(x,y).
\]
Zaprto kroglo radija $r$ s središčem v točki $x$ označimo z $B(x, r)$, odprto
kroglo pa z $\mathring{B}(x, r)$.
Zaradi zveznosti norme je zaprtje odprte krogle natanko pripadajoča zaprta
krogla.

\begin{definicija}
  Normiran prostor je \pojem{Banachov}, če je poln za inducirano metriko.
\end{definicija}

\begin{trditev}
  Seštevanje in množenje vektorjev s skalarjem sta zvezni operaciji.
\end{trditev}

\begin{definicija}
  Algebra $A$ je \pojem{normirana algebra}, če je normiran vektorski prostor in
  če velja $\norm{xy} \le \norm{x} \norm{y}$.
  Če ima normirana algebra enoto, zahtevamo še $\norm{e} = 1$.
\end{definicija}

\begin{primer}
  Naj bo $X$ Hausdorffov topološki prostor in $\zveznei_b(X)$ množica zveznih
  omejenih funkcij $X \to \F$.
  Če jo opremimo s supremum normo, postane normirana algebra za seštevanje in
  množenje po točkah.
  Preverimo lahko, da je celo Banachov prostor.
\end{primer}

\begin{posledica}
  Če je $X$ kompakten Hausdorffov prostor, je $\zvezne{X}$ Banachova algebra.
\end{posledica}

\begin{trditev}
  Naj bo $X$ normiran prostor in $Y$ (vektorski) podprostor v $X$.
  Veljata naslednji točki.
  \begin{itemize}
  \item Če je $Y$ poln, potem je $Y$ zaprt v $X$.
  \item Če je $X$ Banachov prostor, potem je $Y$ Banachov natanko tedaj, ko je
	$Y$ zaprt v $X$.
  \end{itemize}
\end{trditev}

\begin{proof}
  Prva točka:
  Naj bo $y \in \cl{Y}$.
  Obstaja zaporedje $(y_n)_n$ v $Y$, ki konvergira k $y$.
  To zaporedje je Cauchyjevo, torej ima limito, ki je enaka $y$.
  Sledi $y \in Y$, zato $Y = \cl{Y}$.

  Druga točka:
  V desno smo ravno dokazali.
  V levo naj bo $(y_n)_n$ Cauchyjevo zaporedje v $Y$.
  Potem je Cauchyjevo tudi v $X$, kjer ima limito, saj je $X$ poln.
  Ker je $Y$ zaprt, je $y \in Y$, torej $y_n \to y$ v $Y$.
\end{proof}

\begin{primer}
  Naj bo $X$ lokalno kompakten Hausdorffov prostor ter $\zveznei_0(X)$ množica
  vseh funkcij v $\zvezne{X}$, za katere za vsak $\varepsilon > 0$ obstaja
  kompakt $K_\varepsilon \subseteq X$, da je $\abs{f}$ zunaj $K_\varepsilon$
  strogo manjša od $\varepsilon$.

  Pri $X = \R$ so to natanko vse funkcije, katerih limita v obeh neskončnostih
  je enaka $0$, pri $X = \N$ pa je to natanko prostor $c_0$ zaporedij, ki
  konvergirajo k $0$.

  Dokažemo lahko, da je $\zveznei_0(X)$ zaprt dvostranski ideal v
  $\zveznei_b(X)$ in zato Banachova algebra.
\end{primer}

\begin{primer}
  Množica $c$ vseh konvergentnih zaporedij je Banachov prostor za supremum
  normo.
\end{primer}

\podnaslov{Napolnitve normiranih prostorov}

Naj bo $X$ normiran prostor, ki ni poln, in naj bo $\tilde{X}$ množica vseh
Cauchyjevih zaporedij v $X$.
To je vektorski prostor za operacije po komponentah.
Definiramo
\[
  \norm{(x_n)_n} = \lim_{n \to \infty} \norm{x_n}.
\]
Ta izraz je dobro definiran, saj velja $\abs{\norm{x_n} - \norm{y_n}} \le
\norm{x_n - y_n}$, torej je zaporedje norm Cauchyjevo in konvergira v $\R$.
To pa ni norma, ker obstaja veliko zaporedij z limito norm enako $0$.
Na $\tilde{X}$ zato vpeljemo ekvivalenčno relacijo
\[
  (x_n)_n \sim (y_n)_n \iff x_n - y_n \xrightarrow[n \to \infty]{} 0.
\]
Sedaj definiramo $\hat{X} = \tilde{X} / \sim$.
V $\hat{X}$ potem vpeljemo operaciji seštevanja in množenja s skalarjem, ki
delujeta na predstavnikih, ter podobno vpeljemo normo.

\begin{izrek}
  Prostor $(\hat{X}, \norm{\cdot})$ je Banachov in vsebuje $X$ kot gost
  podprostor.
\end{izrek}

\begin{proof}
  Dokaz o polnosti izpustimo.
  Definiramo vložitev $j: X \to \hat{X}$ s predpisom $x \mapsto [(x)_n]$.
  To je očitno linearna preslikava, za katero velja $\norm{j(x)} = \norm{x}$,
  torej je tudi izometrija.
  Pokazali bomo, da je $j(X)$ gost podprostor v $\hat{X}$.

  Naj bo $[(x_n)_n] \in \hat{X}$.
  Za poljuben $\varepsilon > 0$ obstaja $n_\varepsilon$, da za vse $n, m \ge
  n_\varepsilon$ velja $\norm{x_n - x_m} < \varepsilon$.
  Za $m = n_\varepsilon$ dobimo $j(x_{n_\varepsilon}) =
  [(x_{n_\varepsilon})_n]$, in velja
  \[
	\norm{j(x_{n_\varepsilon}) - [(x_n)_n]} = \lim_{n \to \infty}
	\norm{x_{n_\varepsilon} - x_n} \le \varepsilon.
	\qedhere
  \]
\end{proof}

\begin{posledica}
  Prostor $X$ je Banachov natanko tedaj, ko je $j(X) = \hat{X}$.
\end{posledica}

\begin{proof}
  Ideja.
  Izometrije ohranjajo polnost, polni podprostori so zaprti.
  Če so gosti, so enaki celoti.
\end{proof}

% LocalWords:  Banachovi
