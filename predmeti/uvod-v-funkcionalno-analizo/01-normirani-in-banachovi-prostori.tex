\naslov{Normirani in Banachovi prostori}

\begin{definicija}
  Naj bo $X$ vektorski prostor nad poljem $\F \in \{\R, \C\}$.
  Preslikava $\norm{\cdot}: X \to \R$ je \pojem{norma}, če velja:
  \begin{itemize}
  \item $\norm{x} \ge 0$,
  \item $\norm{x} = 0 \iff x = 0$,
  \item $\norm{\lambda x} = \abs{\lambda} \norm{x}$,
  \item $\norm{x + y} \le \norm{x} + \norm{y}$.
  \end{itemize}
\end{definicija}

\begin{opomba}
  Velja $\abs{\norm{x} - \norm{y}} \le \norm{x - y}$, iz česar sledi, da je
  norma zvezna (celo Lipschitzova za $L = 1$).
\end{opomba}

Norma porodi metriko $d(x,y) = \norm{x-y}$ na prostoru $X$, ki je invariantna na
translacije, in za katero velja
\[
  d(\lambda x, \lambda y) = \abs{\lambda} d(x,y).
\]
Zaprto kroglo radija $r$ s središčem v točki $x$ označimo z $B(x, r)$, odprto
kroglo pa z $\mathring{B}(x, r)$.
Zaradi zveznosti norme je zaprtje odprte krogle natanko pripadajoča zaprta
krogla.

\begin{definicija}
  Normiran prostor je \pojem{Banachov}, če je poln za inducirano metriko.
\end{definicija}

\begin{trditev}
  Seštevanje in množenje vektorjev s skalarjem sta zvezni operaciji.
\end{trditev}

% LocalWords:  Banachovi
