\naslov{Hilbertovi prostori}

Omejimo se na primer $\F \in \{\R, \C\}$.

\begin{definicija}
  Naj bo $X$ vektorski prostor nad $\F$.
  Preslikava $\sk{\cdot, \cdot} : X \times X \to \F$ je \pojem{skalarni
	produkt}, če zadošča
  \begin{itemize}
  \item $\sk{x,x} \ge 0$ (realno in nenegativno),
  \item $\sk{x,x} = 0$ natanko tedaj, ko je $x = 0$,
  \item $\sk{\alpha x + \beta y, z} = \alpha \sk{x,z} + \beta \sk{y,z}$,
  \item $\sk{x,y} = \konj{\sk{y,x}}$.
  \end{itemize}
\end{definicija}

\begin{trditev}[Paralelogramska enakost]
  Naj bo $X$ prostor s polskalarnim produktom.
  Za $x, y \in X$ velja
  \[
	\norm{x+y}^2 + \norm{x-y}^2 = 2 (\norm{x}^2 + \norm{y}^2).
  \]
\end{trditev}

\begin{trditev}
  Skalarni produkt je zvezna preslikava.
\end{trditev}

\begin{izrek}[Jordan, von Neumann]
  Če v normiranem prostoru velja paralelogramska enakost, je norma porojena s
  skalarnim produktom.
\end{izrek}

\begin{definicija}
  Prostor $X$ s skalarnim produktom je \pojem{Hilbertov prostor}, če je za
  porojeno normo Banachov prostor.
\end{definicija}

Naj bo $X$ prostor s skalarnim produktom in $\hat{X}$ napolnitev $X$ kot
normiran prostor.
Ker norma na $X$ ustreza paralelogramski enakosti, zaradi zveznosti norme to
velja tudi na $\hat{X}$ in je norma na $\hat{X}$ porojena s skalarnim produktom.
Torej je Hilbertov prostor.
Če $x_n \to x$ in $y_n \to y$, velja
\[
  \sk{x,y} = \lim_{n \to \infty} \sk{x_n, y_n}.
\]

\begin{definicija}
  Vektorja $x$ in $y$ sta \pojem{pravokotna}, če $\sk{x,y} = 0$.
  Označimo $x \bot y$.
\end{definicija}

\begin{definicija}
  Množici $A$ in $B$ sta \pojem{pravokotni}, če je $\sk{a,b} = 0$ za vsak $a \in
  A$ ter $b \in B$.
\end{definicija}

\begin{izrek}[Pitagora]
  Naj bo $X$ vektorski prostor s skalarnim produktom.
  Če sta vektorja $x$ in $y$ pravokotna, je $\norm{x}^2 + \norm{y}^2 =
  \norm{x+y}^2$.
\end{izrek}

% LocalWords:  polskalarnim von paralelogramska paralelogramski
