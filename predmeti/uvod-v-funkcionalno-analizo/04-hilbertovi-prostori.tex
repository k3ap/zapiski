\naslov{Hilbertovi prostori}

Omejimo se na primer $\F \in \{\R, \C\}$.

\begin{definicija}
  Naj bo $X$ vektorski prostor nad $\F$.
  Preslikava $\sk{\cdot, \cdot} : X \times X \to \F$ je \pojem{skalarni
	produkt}, če zadošča
  \begin{itemize}
  \item $\sk{x,x} \ge 0$ (realno in nenegativno),
  \item $\sk{x,x} = 0$ natanko tedaj, ko je $x = 0$,
  \item $\sk{\alpha x + \beta y, z} = \alpha \sk{x,z} + \beta \sk{y,z}$,
  \item $\sk{x,y} = \konj{\sk{y,x}}$.
  \end{itemize}
\end{definicija}

\begin{trditev}[Paralelogramska enakost]
  Naj bo $X$ prostor s polskalarnim produktom.
  Za $x, y \in X$ velja
  \[
	\norm{x+y}^2 + \norm{x-y}^2 = 2 (\norm{x}^2 + \norm{y}^2).
  \]
\end{trditev}

\begin{trditev}
  Skalarni produkt je zvezna preslikava.
\end{trditev}

\begin{izrek}[Jordan, von Neumann]
  Če v normiranem prostoru velja paralelogramska enakost, je norma porojena s
  skalarnim produktom.
\end{izrek}

\begin{definicija}
  Prostor $X$ s skalarnim produktom je \pojem{Hilbertov prostor}, če je za
  porojeno normo Banachov prostor.
\end{definicija}

Naj bo $X$ prostor s skalarnim produktom in $\hat{X}$ napolnitev $X$ kot
normiran prostor.
Ker norma na $X$ ustreza paralelogramski enakosti, zaradi zveznosti norme to
velja tudi na $\hat{X}$ in je norma na $\hat{X}$ porojena s skalarnim produktom.
Torej je Hilbertov prostor.
Če $x_n \to x$ in $y_n \to y$, velja
\[
  \sk{x,y} = \lim_{n \to \infty} \sk{x_n, y_n}.
\]

\begin{definicija}
  Vektorja $x$ in $y$ sta \pojem{pravokotna}, če $\sk{x,y} = 0$.
  Označimo $x \bot y$.
\end{definicija}

\begin{definicija}
  Množici $A$ in $B$ sta \pojem{pravokotni}, če je $\sk{a,b} = 0$ za vsak $a \in
  A$ ter $b \in B$.
\end{definicija}

\begin{izrek}[Pitagora]
  Naj bo $X$ vektorski prostor s skalarnim produktom.
  Če sta vektorja $x$ in $y$ pravokotna, je $\norm{x}^2 + \norm{y}^2 =
  \norm{x+y}^2$.
\end{izrek}

\begin{izrek}
  Naj bo $H$ Hilbertov prostor in $K$ neprazna zaprta konveksna množica v $H$.
  Tedaj za vsak $x \in H$ obstaja natanko en $k \in K$, da je $d(x,K) =
  \norm{x,k}$.
\end{izrek}

\begin{proof}
  Brez škode za splošnost lahko privzamemo $x = 0$.
  Označimo $d = \inf \{\norm{y} \such y \in K\}$.
  Po definiciji infimuma obstaja zaporedje $(k_n)_n$, da $\norm{k_n} \to d$.
  Izberimo $\varepsilon > 0$.
  Potem obstaja $N \in \N$, da za $n \ge N$ velja $\norm{k_n}^2 \le d^2 +
  \varepsilon^2 /4$.
  Po paralelogramski enakosti velja
  \[
	\norm{k_n - k_m}^2 = 2 \norm{k_n}^2 + 2 \norm{k_m}^2 - 4 \norm{\frac{k_n +
		k_m}{2}}^2
	\le 4 d^2 + \varepsilon^2 - 4 d^2 = \varepsilon^2,
  \]
  torej je $(k_n)_n$ Cauchyjevo in ima limito $k \in \cl{K} = K$.
  Velja $\norm{k} = d$.

  Za enoličnost še enkrat uporabimo paralelogramsko enakost.
  Če je $k' \in K$ še en vektor s $\norm{k'} = d$, potem
  \[
	\norm{k-k'}^2 = 2 \norm{k}^2 + 2 \norm{k'}^2 - 4 \norm{\frac{k+k'}{2}}^2 \le 0,
  \]
  saj je $K$ konveksna in $k, k' \in K$.
\end{proof}

\begin{izrek}
  Naj bo $M$ zaprt podprostor Hilbertovega prostora $H$, $x \in H$ ter $x_0 \in
  M$.
  Tedaj velja
  \[
	x - x_0 \bot M \iff d(x,M) = \norm{x-x_0}.
  \]
\end{izrek}

\begin{proof}
  Recimo, da za $x_0 \in M$ velja $d(x,M) = \norm{x-x_0}$ in da $x - x_0$ ni
  pravokoten na $M$.
  Tedaj obstaja $y \in M$, da je $\sk{x - x_0, y} \ne 0$.
  Brez škode za splošnost lahko privzamemo, da je ta skalarni produkt pozitiven,
  sicer $y$ zavrtimo za potreben kot.
  Potem je
  \[
	\norm{x - (x_0 + \varepsilon y)}^2 = \norm{x - x_0}^2 -
	2 \varepsilon \Real \sk{x - x_0, y} + \abs{\varepsilon}^2 \norm{y}^2
  \]
  za poljuben $\varepsilon \in \C$, saj je $(x_0 + \varepsilon y) \in M$.
  Če izberemo dovolj majhen $\varepsilon \in \R$, dobimo
  \[
	\norm{x - (x_0 + \varepsilon y)}^2 < \norm{x - x_0}^2,
  \]
  kar je protislovje, saj je $x_0$ najbližji vektor iz $M$.
  Torej je $x - x_0 \bot M$.
  Tedaj za vsak $y \in M$ velja
  \[
	\norm{x-y}^2 = \norm{x-x_0}^2 + \norm{x_0 - y}^2 \ge \norm{x - x_0}.
	\qedhere
  \]
\end{proof}

\begin{definicija}
  Naj bo $X$ prostor s skalarnim produktom.
  Za $x \in X$ definiramo $\{x\}^\bot = \{y \in X \such x \bot y\}$, za $A
  \subseteq X$ pa
  \[
	A^\bot = \bigcap_{x \in A} \{x\}^\bot.
  \]
\end{definicija}

\begin{lema}
  Za $A \subseteq X$ je $A^\bot$ vedno zaprt podprostor v $X$.
\end{lema}

\begin{proof}
  Dovolj je pokazati, da je $\{x\}^\bot$ zaprt podprostor za katerikoli $x \in
  X$.
  Očitno je podprostor.
  Za zaprtost vzemimo zaporedje $(y_n)_n$ v $\{x\}^\bot$, ki konvergira k $y \in
  \cl{\{x\}^\bot}$.
  Potem je
  \[
	\sk{y,x} = \sk{y - y_n, x} + \sk{y_n, x} = \sk{y - y_n, x}
  \]
  in zato
  \[
	\abs{\sk{y, x}} \le \norm{y - y_n} \norm{x} \xrightarrow[n \to \infty]{} 0
  \]
  po Cauchy-Schwarzu.
  Torej je $y \in \{x\}^\bot$.
\end{proof}

\begin{izrek}
  Naj bo $M$ zaprt podprostor v Hilbertovem prostoru $H$.
  Za $x \in H$ definiramo $Px \in M$ kot tisti vektor, ki je najbližji $x$ med
  vektorji iz $M$.
  Potem velja:
  \begin{itemize}
  \item $P$ je linearen operator $H \to M$,
  \item $\norm{Px} \le \norm{x}$,
  \item $P^2 = P$,
  \item $\im P = M$ in $\ker P = M^\bot$,
  \item $H = M \oplus M^\bot$ in $M^{\bot \bot} = M$.
  \end{itemize}
\end{izrek}

\begin{proof}
  Prva točka:
  Vzemimo $z \in M$.
  Potem je
  \[
	\sk{(\alpha x + \beta y) - (\alpha Px + \beta Py), z}
	= \alpha \sk{x - Px, z} + \beta \sk{y - Py, z}
	= 0,
  \]
  torej po prejšnjem izreku $P(\alpha x + \beta y) = \alpha P x + \beta P y$.

  Za drugo točko izračunamo
  \[
	\norm{x}^2 = \norm{x - Px + Px}^2 = \norm{x - Px}^2 + \norm{Px}^2 \ge
	\norm{Px}^2.
  \]

  Tretja točka je očitna.
  Za četrto je seveda $\im P = M$, za $x \in \ker P$ velja $x - Px \in M^\bot$,
  torej (ker $Px = 0$) tudi $x \in M^\bot$.
  Če pa je $x \in M^\bot$, je $x = x - 0 \in M^\bot$, torej $Px = 0$ po
  definiciji $P$.

  Za zadnjo točko razcepimo $x = Px + (x -Px)$.
  Ker za $A \subseteq H$ vedno velja $A \cap A^\bot \subseteq \{0\}$ in ker je
  $0 \in M$, je $H = M \oplus M^\bot$.
  Preslikava $I - P$ je pravokoten projektor na $M^\bot$.
  Velja $M^{\bot \bot} = \jedro (I-P) = \im P = M$.
\end{proof}

Idempotent $P$ iz izreka je pravokotni projektor na $M$ vzdolž $M^\bot$.
Množica $M^\bot$ se imenuje \pojem{ortogonalni komplement $M$}.

\begin{posledica}
  Za $A \subseteq H$ je $A^{\bot \bot} = \cl{\operatorname{Lin} A} =: [A]$.
\end{posledica}

\begin{proof}
  Seveda je $A \subseteq [A]$.
  Velja $[A]^\bot \subseteq A^\bot$ in $A^{\bot \bot} \subseteq [A]^{\bot \bot}
  = [A]$ ter celo $[A]^\bot = A^\bot$, saj za $x \in A^\bot$ velja $x \bot A$
  in bo $x$ pravokoten tudi na linearno ogrinjačo $A$ in njeno zaprtje.
  Torej $A^{\bot \bot} = [A]^{\bot \bot}$.
\end{proof}

Naj bo $X$ prostor s skalarnim produktom in $y \in X$.
Definiramo $f_y: X \to \F$ z
\[
  f_y(x) = \sk{x,y}.
\]

\begin{lema}
  Preslikava $f_y$ je omejen linearen funkcional z $\norm{f_y} = \norm{y}$.
\end{lema}

\begin{proof}
  Očitno je linearen.
  Velja
  \[
	\norm{f_y(x)} = \abs{\sk{x,y}} \le \norm{x} \norm{y}
  \]
  po Cauchy-Schwarzu.
  Če je $x$ linearno odvisen od $y$, velja enakost.
\end{proof}

\begin{izrek}[Riesz]
  Naj bo $H$ Hilbertov prostor in $f \in H^*$.
  Tedaj obstaja natanko en $y \in H$, da je $f(x) =\sk{x,y}$ in $\norm{f} =
  \norm{y}$.
\end{izrek}

\begin{proof}
  Za enoličnost: če je $f_y = f_z$, potem za vsak $x$
  \[
	\sk{x, y-z} = 0,
  \]
  torej $y = z$.

  Če je $f = 0$, vzamemo $y = 0$.
  Sicer $f \ne 0$, in je $\jedro f$ zaprt podprostor (praslika zaprte množice),
  torej $H = \jedro f \oplus (\jedro f)^\bot$.
  Obstaja $z \in (\jedro f)^\bot$, da je $f(z) = 1$.
  Za $x \in H$ potem velja
  \[
	x = \underbrace{x - f(x) z}_{\in \jedro f} + \underbrace{f(x) z}_{\in
	  (\jedro f)^\bot},
  \]
  torej $\sk{x,z} = \sk{f(x) z, z} = f(x) \sk{z,z}$.
  Potem je $f(x) = \sk{x, z / \norm{z}^2}$.
\end{proof}

\begin{posledica}
  Za vsak $f \in (l^2)^*$ obstaja natanko en $y \in l^2$, da je
  \[
	f(x) = \sum_{n=1}^\infty x_n \cl{y_n}
  \]
\end{posledica}

\begin{trditev}
  Preslikava $J: H \to H^*$, podana s predpisom $Jy = f_y$ za $f_y(x) =
  \sk{x,y}$, je poševno linearen izometrični izomorfizem.
\end{trditev}

\begin{proof}
  Vemo $\norm{f_y} = \norm{y}$, torej je $J$ izometrija.
  Po Rieszovem izreku je surjektivna in injektivna, poševna linearnost pa je
  preprost račun.
\end{proof}

\begin{izrek}
  Naj bo $H$ Hilbertov prostor.
  Potem je tudi $H^*$ Hilbertov s skalarnim produktom $\sk{f,g}_{H^*} = \sk{y_g,
  y_f}_H$.
\end{izrek}

\begin{proof}
  Preprosto preverjanje.
\end{proof}

\begin{izrek}
  Naj bo $H$ Hilbertov prostor in $K \le H$ podprostor.
  Tedaj ima vsak $f \in K^*$ natanko eno Hahn-Banachovo razširitev na $H$.
\end{izrek}

\begin{proof}
  Omejen funkcional $f: K \to \F$ lahko razširimo do $g: \cl{K} \to \F$, pri
  čemer se ohrani norma.
  Po Rieszovem izreku obstaja natanko en $y \in \cl{K}$, da je $g(x) =
  \sk{x,y}$ za $x \in \cl{K}$.
  Definiramo $F(x) = \sk{x,y}$ za $x \in H$.
  Seveda je $\left. F \right|_K = f$ in $\norm{F} = \norm{y} = \norm{f}$.

  Recimo, da je $F'$ še ena Hahn-Banachova razširitev $f$.
  Po Rieszu obstaja natanko en $y' \in H$, da je $F'(x) = \sk{x,y'}$.
  To velja tudi za $x \in \cl{K}$, torej $F'(x) = g(x)$,
  oziroma $\sk{x,y'} = \sk{x,y}$ za vse $x \in \cl{K}$.
  Sledi $y-y' \bot \cl{K}$.
  Potem je
  \[
	\norm{F'}^2 = \norm{y}^2 = \norm{y' - y + y}^2
	= \norm{y' - y}^2 + \norm{y}^2 = \norm{g}^2 + \norm{y - y'}^2.
  \]
  Torej $\norm{y'-y} = 0$.
\end{proof}

% LocalWords:  polskalarnim von paralelogramska paralelogramski paralelogramsko
% LocalWords:  Riesz Rieszovem Hahn-Banachovo Hahn-Banachova Rieszu
