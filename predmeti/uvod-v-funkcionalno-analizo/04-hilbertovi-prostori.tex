\naslov{Hilbertovi prostori}

Omejimo se na primer $\F \in \{\R, \C\}$.

\begin{definicija}
  Naj bo $X$ vektorski prostor nad $\F$.
  Preslikava $\sk{\cdot, \cdot} : X \times X \to \F$ je \pojem{skalarni
	produkt}, če zadošča
  \begin{itemize}
  \item $\sk{x,x} \ge 0$ (realno in nenegativno),
  \item $\sk{x,x} = 0$ natanko tedaj, ko je $x = 0$,
  \item $\sk{\alpha x + \beta y, z} = \alpha \sk{x,z} + \beta \sk{y,z}$,
  \item $\sk{x,y} = \konj{\sk{y,x}}$.
  \end{itemize}
\end{definicija}

\begin{trditev}[Paralelogramska enakost]
  Naj bo $X$ prostor s polskalarnim produktom.
  Za $x, y \in X$ velja
  \[
	\norm{x+y}^2 + \norm{x-y}^2 = 2 (\norm{x}^2 + \norm{y}^2).
  \]
\end{trditev}

\begin{trditev}
  Skalarni produkt je zvezna preslikava.
\end{trditev}

\begin{izrek}[Jordan, von Neumann]
  Če v normiranem prostoru velja paralelogramska enakost, je norma porojena s
  skalarnim produktom.
\end{izrek}

% LocalWords:  polskalarnim von paralelogramska
