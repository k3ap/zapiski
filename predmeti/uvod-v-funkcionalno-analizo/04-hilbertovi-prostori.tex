\naslov{Hilbertovi prostori}

Omejimo se na primer $\F \in \{\R, \C\}$.

\begin{definicija}
  Naj bo $X$ vektorski prostor nad $\F$.
  Preslikava $\sk{\cdot, \cdot} : X \times X \to \F$ je \pojem{skalarni
	produkt}, če zadošča
  \begin{itemize}
  \item $\sk{x,x} \ge 0$ (realno in nenegativno),
  \item $\sk{x,x} = 0$ natanko tedaj, ko je $x = 0$,
  \item $\sk{\alpha x + \beta y, z} = \alpha \sk{x,z} + \beta \sk{y,z}$,
  \item $\sk{x,y} = \konj{\sk{y,x}}$.
  \end{itemize}
\end{definicija}

\begin{trditev}[Paralelogramska enakost]
  Naj bo $X$ prostor s polskalarnim produktom.
  Za $x, y \in X$ velja
  \[
	\norm{x+y}^2 + \norm{x-y}^2 = 2 (\norm{x}^2 + \norm{y}^2).
  \]
\end{trditev}

\begin{trditev}
  Skalarni produkt je zvezna preslikava.
\end{trditev}

\begin{izrek}[Jordan, von Neumann]
  Če v normiranem prostoru velja paralelogramska enakost, je norma porojena s
  skalarnim produktom.
\end{izrek}

\begin{definicija}
  Prostor $X$ s skalarnim produktom je \pojem{Hilbertov prostor}, če je za
  porojeno normo Banachov prostor.
\end{definicija}

Naj bo $X$ prostor s skalarnim produktom in $\hat{X}$ napolnitev $X$ kot
normiran prostor.
Ker norma na $X$ ustreza paralelogramski enakosti, zaradi zveznosti norme to
velja tudi na $\hat{X}$ in je norma na $\hat{X}$ porojena s skalarnim produktom.
Torej je Hilbertov prostor.
Če $x_n \to x$ in $y_n \to y$, velja
\[
  \sk{x,y} = \lim_{n \to \infty} \sk{x_n, y_n}.
\]

\begin{definicija}
  Vektorja $x$ in $y$ sta \pojem{pravokotna}, če $\sk{x,y} = 0$.
  Označimo $x \bot y$.
\end{definicija}

\begin{definicija}
  Množici $A$ in $B$ sta \pojem{pravokotni}, če je $\sk{a,b} = 0$ za vsak $a \in
  A$ ter $b \in B$.
\end{definicija}

\begin{izrek}[Pitagora]
  Naj bo $X$ vektorski prostor s skalarnim produktom.
  Če sta vektorja $x$ in $y$ pravokotna, je $\norm{x}^2 + \norm{y}^2 =
  \norm{x+y}^2$.
\end{izrek}

\begin{izrek}
  Naj bo $H$ Hilbertov prostor in $K$ neprazna zaprta konveksna množica v $H$.
  Tedaj za vsak $x \in H$ obstaja natanko en $k \in K$, da je $d(x,K) =
  \norm{x,k}$.
\end{izrek}

\begin{proof}
  Brez škode za splošnost lahko privzamemo $x = 0$.
  Označimo $d = \inf \{\norm{y} \such y \in K\}$.
  Po definiciji infimuma obstaja zaporedje $(k_n)_n$, da $\norm{k_n} \to d$.
  Izberimo $\varepsilon > 0$.
  Potem obstaja $N \in \N$, da za $n \ge N$ velja $\norm{k_n}^2 \le d^2 +
  \varepsilon^2 /4$.
  Po paralelogramski enakosti velja
  \[
	\norm{k_n - k_m}^2 = 2 \norm{k_n}^2 + 2 \norm{k_m}^2 - 4 \norm{\frac{k_n +
		k_m}{2}}^2
	\le 4 d^2 + \varepsilon^2 - 4 d^2 = \varepsilon^2,
  \]
  torej je $(k_n)_n$ Cauchyjevo in ima limito $k \in \cl{K} = K$.
  Velja $\norm{k} = d$.

  Za enoličnost še enkrat uporabimo paralelogramsko enakost.
  Če je $k' \in K$ še en vektor s $\norm{k'} = d$, potem
  \[
	\norm{k-k'}^2 = 2 \norm{k}^2 + 2 \norm{k'}^2 - 4 \norm{\frac{k+k'}{2}}^2 \le 0,
  \]
  saj je $K$ konveksna in $k, k' \in K$.
\end{proof}

\begin{izrek}
  Naj bo $M$ zaprt podprostor Hilbertovega prostora $H$, $x \in H$ ter $x_0 \in
  M$.
  Tedaj velja
  \[
	x - x_0 \bot M \iff d(x,M) = \norm{x-x_0}.
  \]
\end{izrek}

\begin{proof}
  Recimo, da za $x_0 \in M$ velja $d(x,M) = \norm{x-x_0}$ in da $x - x_0$ ni
  pravokoten na $M$.
  Tedaj obstaja $y \in M$, da je $\sk{x - x_0, y} \ne 0$.
  Brez škode za splošnost lahko privzamemo, da je ta skalarni produkt pozitiven,
  sicer $y$ zavrtimo za potreben kot.
  Potem je
  \[
	\norm{x - (x_0 + \varepsilon y)}^2 = \norm{x - x_0}^2 -
	2 \varepsilon \Real \sk{x - x_0, y} + \abs{\varepsilon}^2 \norm{y}^2
  \]
  za poljuben $\varepsilon \in \C$, saj je $(x_0 + \varepsilon y) \in M$.
  Če izberemo dovolj majhen $\varepsilon \in \R$, dobimo
  \[
	\norm{x - (x_0 + \varepsilon y)}^2 < \norm{x - x_0}^2,
  \]
  kar je protislovje, saj je $x_0$ najbližji vektor iz $M$.
  Torej je $x - x_0 \bot M$.
  Tedaj za vsak $y \in M$ velja
  \[
	\norm{x-y}^2 = \norm{x-x_0}^2 + \norm{x_0 - y}^2 \ge \norm{x - x_0}.
	\qedhere
  \]
\end{proof}

\begin{definicija}
  Naj bo $X$ prostor s skalarnim produktom.
  Za $x \in X$ definiramo $\{x\}^\bot = \{y \in X \such x \bot y\}$, za $A
  \subseteq X$ pa
  \[
	A^\bot = \bigcap_{x \in A} \{x\}^\bot.
  \]
\end{definicija}

\begin{lema}
  Za $A \subseteq X$ je $A^\bot$ vedno zaprt podprostor v $X$.
\end{lema}

\begin{proof}
  Dovolj je pokazati, da je $\{x\}^\bot$ zaprt podprostor za katerikoli $x \in
  X$.
  Očitno je podprostor.
  Za zaprtost vzemimo zaporedje $(y_n)_n$ v $\{x\}^\bot$, ki konvergira k $y \in
  \cl{\{x\}^\bot}$.
  Potem je
  \[
	\sk{y,x} = \sk{y - y_n, x} + \sk{y_n, x} = \sk{y - y_n, x}
  \]
  in zato
  \[
	\abs{\sk{y, x}} \le \norm{y - y_n} \norm{x} \xrightarrow[n \to \infty]{} 0
  \]
  po Cauchy-Schwarzu.
  Torej je $y \in \{x\}^\bot$.
\end{proof}

\begin{izrek}
  Naj bo $M$ zaprt podprostor v Hilbertovem prostoru $H$.
  Za $x \in H$ definiramo $Px \in M$ kot tisti vektor, ki je najbližji $x$ med
  vektorji iz $M$.
  Potem velja:
  \begin{itemize}
  \item $P$ je linearen operator $H \to M$,
  \item $\norm{Px} \le \norm{x}$,
  \item $P^2 = P$,
  \item $\im P = M$ in $\ker P = M^\bot$,
  \item $H = M \oplus M^\bot$ in $M^{\bot \bot} = M$.
  \end{itemize}
\end{izrek}

\begin{proof}
  Prva točka:
  Vzemimo $z \in M$.
  Potem je
  \[
	\sk{(\alpha x + \beta y) - (\alpha Px + \beta Py), z}
	= \alpha \sk{x - Px, z} + \beta \sk{y - Py, z}
	= 0,
  \]
  torej po prejšnjem izreku $P(\alpha x + \beta y) = \alpha P x + \beta P y$.

  Za drugo točko izračunamo
  \[
	\norm{x}^2 = \norm{x - Px + Px}^2 = \norm{x - Px}^2 + \norm{Px}^2 \ge
	\norm{Px}^2.
  \]

  Tretja točka je očitna.
  Za četrto je seveda $\im P = M$, za $x \in \ker P$ velja $x - Px \in M^\bot$,
  torej (ker $Px = 0$) tudi $x \in M^\bot$.
  Če pa je $x \in M^\bot$, je $x = x - 0 \in M^\bot$, torej $Px = 0$ po
  definiciji $P$.

  Za zadnjo točko razcepimo $x = Px + (x -Px)$.
  Ker za $A \subseteq H$ vedno velja $A \cap A^\bot \subseteq \{0\}$ in ker je
  $0 \in M$, je $H = M \oplus M^\bot$.
  Preslikava $I - P$ je pravokoten projektor na $M^\bot$.
  Velja $M^{\bot \bot} = \jedro (I-P) = \im P = M$.
\end{proof}

Idempotent $P$ iz izreka je pravokotni projektor na $M$ vzdolž $M^\bot$.
Množica $M^\bot$ se imenuje \pojem{ortogonalni komplement $M$}.

\begin{posledica}
  Za $A \subseteq H$ je $A^{\bot \bot} = \cl{\operatorname{Lin} A} =: [A]$.
\end{posledica}

\begin{proof}
  Seveda je $A \subseteq [A]$.
  Velja $[A]^\bot \subseteq A^\bot$ in $A^{\bot \bot} \subseteq [A]^{\bot \bot}
  = [A]$ ter celo $[A]^\bot = A^\bot$, saj za $x \in A^\bot$ velja $x \bot A$
  in bo $x$ pravokoten tudi na linearno ogrinjačo $A$ in njeno zaprtje.
  Torej $A^{\bot \bot} = [A]^{\bot \bot}$.
\end{proof}

Naj bo $X$ prostor s skalarnim produktom in $y \in X$.
Definiramo $f_y: X \to \F$ z
\[
  f_y(x) = \sk{x,y}.
\]

\begin{lema}
  Preslikava $f_y$ je omejen linearen funkcional z $\norm{f_y} = \norm{y}$.
\end{lema}

\begin{proof}
  Očitno je linearen.
  Velja
  \[
	\norm{f_y(x)} = \abs{\sk{x,y}} \le \norm{x} \norm{y}
  \]
  po Cauchy-Schwarzu.
  Če je $x$ linearno odvisen od $y$, velja enakost.
\end{proof}

\begin{izrek}[Riesz]
  Naj bo $H$ Hilbertov prostor in $f \in H^*$.
  Tedaj obstaja natanko en $y \in H$, da je $f(x) =\sk{x,y}$ in $\norm{f} =
  \norm{y}$.
\end{izrek}

\begin{proof}
  Za enoličnost: če je $f_y = f_z$, potem za vsak $x$
  \[
	\sk{x, y-z} = 0,
  \]
  torej $y = z$.

  Če je $f = 0$, vzamemo $y = 0$.
  Sicer $f \ne 0$, in je $\jedro f$ zaprt podprostor (praslika zaprte množice),
  torej $H = \jedro f \oplus (\jedro f)^\bot$.
  Obstaja $z \in (\jedro f)^\bot$, da je $f(z) = 1$.
  Za $x \in H$ potem velja
  \[
	x = \underbrace{x - f(x) z}_{\in \jedro f} + \underbrace{f(x) z}_{\in
	  (\jedro f)^\bot},
  \]
  torej $\sk{x,z} = \sk{f(x) z, z} = f(x) \sk{z,z}$.
  Potem je $f(x) = \sk{x, z / \norm{z}^2}$.
\end{proof}

\begin{posledica}
  Za vsak $f \in (l^2)^*$ obstaja natanko en $y \in l^2$, da je
  \[
	f(x) = \sum_{n=1}^\infty x_n \cl{y_n}
  \]
\end{posledica}

\begin{trditev}
  Preslikava $J: H \to H^*$, podana s predpisom $Jy = f_y$ za $f_y(x) =
  \sk{x,y}$, je poševno linearen izometrični izomorfizem.
\end{trditev}

\begin{proof}
  Vemo $\norm{f_y} = \norm{y}$, torej je $J$ izometrija.
  Po Rieszovem izreku je surjektivna in injektivna, poševna linearnost pa je
  preprost račun.
\end{proof}

\begin{izrek}
  Naj bo $H$ Hilbertov prostor.
  Potem je tudi $H^*$ Hilbertov s skalarnim produktom $\sk{f,g}_{H^*} = \sk{y_g,
  y_f}_H$.
\end{izrek}

\begin{proof}
  Preprosto preverjanje.
\end{proof}

\begin{izrek}
  Naj bo $H$ Hilbertov prostor in $K \le H$ podprostor.
  Tedaj ima vsak $f \in K^*$ natanko eno Hahn-Banachovo razširitev na $H$.
\end{izrek}

\begin{proof}
  Omejen funkcional $f: K \to \F$ lahko razširimo do $g: \cl{K} \to \F$, pri
  čemer se ohrani norma.
  Po Rieszovem izreku obstaja natanko en $y \in \cl{K}$, da je $g(x) =
  \sk{x,y}$ za $x \in \cl{K}$.
  Definiramo $F(x) = \sk{x,y}$ za $x \in H$.
  Seveda je $\left. F \right|_K = f$ in $\norm{F} = \norm{y} = \norm{f}$.

  Recimo, da je $F'$ še ena Hahn-Banachova razširitev $f$.
  Po Rieszu obstaja natanko en $y' \in H$, da je $F'(x) = \sk{x,y'}$.
  To velja tudi za $x \in \cl{K}$, torej $F'(x) = g(x)$,
  oziroma $\sk{x,y'} = \sk{x,y}$ za vse $x \in \cl{K}$.
  Sledi $y-y' \bot \cl{K}$.
  Potem je
  \[
	\norm{F'}^2 = \norm{y}^2 = \norm{y' - y + y}^2
	= \norm{y' - y}^2 + \norm{y}^2 = \norm{g}^2 + \norm{y - y'}^2.
  \]
  Torej $\norm{y'-y} = 0$.
\end{proof}

\begin{posledica}
  Vsak Hilbertov prostor je refleksiven (vložitev v $H^{**}$ je surjektivna).
\end{posledica}

\podnaslov{Ortonormirani sistemi}

\begin{definicija}
  Naj bo $X$ prostor s skalarnim produktom.
  Množica $E \subseteq X$ je \pojem{ortonormiran sistem}, če je $\norm{e} = 1$
  za vsak $e \in E$ ter $e \bot f$ za vsaka $e,f \in E$.
\end{definicija}

\begin{opomba}
  Če velja le druga zahteva, je $E$ \pojem{ortogonalna množica}.
\end{opomba}

\begin{lema}
  Vsaka ortogonalna množica je linearno neodvisna.
\end{lema}

\begin{definicija}
  Naj bo $H$ Hilbertov prostor. Ortonormiran sistem $E \subseteq H$ je
  \pojem{kompleten} ali \pojem{baza} Hilbertovega prostora $H$, če je maksimalen
  v množici vseh ortonormiranih sistemov (glede na $\subseteq$).
\end{definicija}

\begin{trditev}
  Vsak ortonormiran sistem v Hilbertovem prostoru lahko dopolnimo do kompletnega
  ortonormiranega sistema.
\end{trditev}

\begin{proof}
  Če je $(F_\alpha)_\alpha$ veriga v množici vseh ortonormiranih sistemov, ki
  vsebujejo $E$, je
  \[
	\bigcup_\alpha F_\alpha
  \]
  zgornja meja za to verigo, ki je očitno ortonormiran sistem.
  Po Zornovi lemi obstaja kompleten ortonormiran sistem, ki vsebuje $E$.
\end{proof}

\begin{opomba}
  Kaj je rumeno in ekvivalentno aksiomu izbire?
  Zornova limona!
\end{opomba}

\begin{posledica}
  Vsak Hilbertov prostor ima bazo.
\end{posledica}

\begin{trditev}
  Naj bo $\{e_1, \ldots, e_n\}$ ortonormiran sistem v Hilbertovem prostoru $H$.
  Naj bo $P_n$ ortogonalna projekcija na $M_n = \operatorname{Lin} \{e_1,
  \ldots, e_n\}$.
  Tedaj za $x \in H$ velja
  \[
	P_n x = \sum_{k=1}^n \sk{x, e_k} e_k.
  \]
\end{trditev}

\begin{proof}
  Naj bo $x_0$ ta vsota.
  Tedaj za $1 \le j \le n$ velja
  \[
	\sk{x_0, e_j} = \sum_{k=1}^n \sk{x, e_k} \sk{e_k, e_j} = \sk{x, e_j},
  \]
  torej je $x - x_0 \bot M_n$.
  Po definiciji je $x_0 = P_n x$.
\end{proof}

\begin{trditev}[Besselova neenakost]
  Naj bo $(e_n)_n$ števen ortonormiran sistem v prostoru s skalarnim produktom
  $X$.
  Tedaj za vsak $x \in X$ velja
  \[
	\norm{x}^2 \ge \sum_{n=1}^\infty \abs{\sk{x, e_n}}^2.
  \]
\end{trditev}

\begin{proof}
  Definiramo
  \[
	x' = \sum_{k=1}^n \sk{x, e_k} e_k.
  \]
  Enostavno lahko preverimo $x - x' \bot x'$, torej po Pitagori
  \[
	\norm{x}^2 = \norm{x - x'}^2 + \norm{x'}^2 \ge \norm{x'}^2 = \sum_{k=1}^n
	\abs{\sk{x, e_k}}^2.
  \]
  To velja za vsak $n$, torej tudi v limiti.
\end{proof}

\begin{posledica}
  Naj bo $X$ prostor s skalarnim produktom in $E \subseteq X$ ortonormiran
  sistem.
  Naj bo $x \in X$.
  Tedaj je $\{e \in E \such \sk{x,e} \ne 0\}$ kvečjemu števna.
\end{posledica}

\begin{proof}
  Definiramo
  \[
	E_n = \left\{e \in E \such \abs{\sk{x,e}} \ge \inv{n}\right\}.
  \]
  Potem je $\sk{x,e} \ne 0$ natanko tedaj, ko je $e \in E_n$ za vsak $n \in \N$.
  Trdimo, da so vsi $E_n$ končni.
  Sicer obstaja $m \in \N$, da je $E_m$ neskončna, torej vsebuje števno
  neskončno podmnožico $(e_k)_k$.
  Potem je
  \[
	\norm{x}^2 \ge \sum_{k=1}^\infty \abs{\sk{x, e_k}}^2 = \infty,
  \]
  ker je $\abs{\sk{x,e_k}} \ge \nicefrac{1}{m}$ za vse $k$.
\end{proof}

\begin{posledica}
  Če je $E \subseteq X$ kompleten ortonormiran sistem, za vsak $x \in X$ velja
  \[
	\norm{x}^2 \ge \sum_{e \in E} \abs{\sk{x, e}}^2.
  \]
\end{posledica}

\begin{proof}
  Po prejšnji posledici je $\sk{x, e} \ne 0$ za kvečjemu števno mnogo $e \in E$.
  Na njih uporabimo Besselovo neenakost.
\end{proof}

\begin{izrek}
  Za ortonormiran sistem $E \subseteq H$ so naslednje trditve ekvivalentne.
  \begin{itemize}
  \item $E$ je kompleten,
  \item $E^\bot = \{0\}$,
  \item $[E] = H$ (zaprta linearna ogrinjača),
  \item za vsak $x \in H$ velja
	\[
	  x = \sum_{e \in E} \sk{x,e} e,
	\]
  \item za poljubna $x, y \in H$ velja
	\[
	  \sk{x,y} = \sum_{e \in E} \sk{x,e} \sk{y,e},
	\]
  \item (Parsevalova enakost) za vsak $x \in H$ velja
	\[
	  \norm{x}^2 = \sum_{e \in E} \abs{\sk{x, e}}^2.
	\]
  \end{itemize}
\end{izrek}

\begin{proof}
  1 v 2:
  Recimo, da $E^\bot \ne \{0\}$.
  Vzamemo $x \in E^\bot \setminus \{0\}$ in definiramo $E \cup \{x /
  \norm{x}\}$, kar je ortonormiran sistem, ki vsebuje $E$.
  \protislovje{}

  2 v 1:
  Recimo, da $E$ ni KONS\@.
  Potem obstaja KONS $E'$, ki vsebuje $E$, in obstaja $x \in E' \setminus E$.
  Ampak $x \in E^\bot = \{0\}$.
  \protislovje{}

  Ekvivalentnost 2 in 3:
  Velja $[E] = H \iff E^\bot = [E]^\bot = H^\bot = \{0\}$.

  2 v 4:
  Vzemimo $x \in H$.
  Vemo, da obstaja kvečjemu števno mnogo $e \in E$, da je $\sk{x,e} \ne 0$.
  Te vektorje oštevilčimo v $(e_n)_n$.
  Dokažimo, da je
  \[
	x = \sum_{n=1}^\infty \sk{x, e_n} e_n.
  \]
  Vrsta res konvergira, saj za delne vsote $s_n$ in $m > n$ velja
  \[
	\norm{s_m - s_n}^2 = \norm{\sum_{k=n+1}^m \sk{e, e_k} e_k}^2
	= \sum_{k=n+1}^m \abs{\sk{x, e_k}}^2
	\le \sum_{k=n+1}^\infty \abs{\sk{x, e_k}}^2.
  \]
  Ker ta vrsta konvergira po Besselovi neenakosti, je zaporedje $(s_n)_n$
  Cauchyjevo v $H$.
  Zato $s_n \to x_0 \in H$.
  Ker je
  \[
	\sk{x_0, e_j} = \sum_{k=1}^\infty \sk{x, e_k} \sk{e_k, e_j} = \sk{x, e_j},
  \]
  velja $x - x_0 \bot e_j$ za vsak $j$, torej $x = x_0$.

  4 v 5:
  Preprost račun.

  5 v 6:
  Preprost račun.

  6 v 2:
  Če je $E^\bot \ne \{0\}$, obstaja $x \in E^\bot$, različen od $0$.
  Po Parsevalu za $x$ in $E$ velja
  \[
	0 \ne \norm{x}^2 = \sum_{e \in E} \abs{\sk{x, e}}^2 = 0,
  \]
  kar je protislovno.
  \protislovje{}
\end{proof}

\begin{trditev}
  Poljubni ortonormirani bazi Hilbertovega prostora imata isto kardinalnost.
\end{trditev}

\begin{proof}
  Naj bosta $E, F \subseteq H$ bazi.
  Če je $\abs{E} < \infty$, rezultat vemo iz linearne algebre.
  Sicer za $e \in E$ tvorimo $F_e = \{f \in F \such \sk{e,f} \ne 0\}$.
  Ta množica je kvečjemu števno neskončna, po drugi strani pa velja
  \[
	F = \bigcup_{e \in E} F_e,
  \]
  saj je $E$ baza.
  Torej $\abs{F} \le \abs{E} \abs{\N} = \abs{E}$ in podobno v drugo smer.
\end{proof}

\begin{definicija}
  \pojem{Dimenzija} Hilbertovega prostora je enaka kardinalnosti katerekoli
  njene baze.
\end{definicija}

\begin{lema}
  V separabilnem metričnem prostoru je vsaka družina paroma disjunktnih odprtih
  krogel kvečjemu števno neskončna.
\end{lema}

\begin{proof}
  Naj bo $\{ \mathring{B}(x_i, \varepsilon_i) \such i \in I \}$ družina paroma
  disjunktnih odprtih krogel.
  Naj bo $S$ števna gosta množica v danem metričnem prostoru.
  Zaradi gostosti je $\mathring{B}(x_i, \varepsilon_i) \cap S \ne \varnothing$
  za vse $i$.
  Izberimo $y_i \in \mathring{B}(x_i, \varepsilon_i) \cap S$ in definiramo
  $\varphi: I \to S$ z $\varphi(i) = y_i$.
  Ker so krogle med seboj disjunktne, je $\varphi$ injekcija, torej $\abs{I} \le
  \abs{\N}$.
\end{proof}

\begin{trditev}
  Neskončnorazsežen Hilbertov prostor je separabilen natanko tedaj, ko je $\dim
  H = \aleph_0$.
\end{trditev}

\begin{proof}
  V desno:
  Naj bo $E$ KONS v $H$.
  Za $e, e' \in E$ velja
  \[
	\norm{e-e'}^2 = \sk{e-e', e-e'} = \norm{e}^2 - 2 \Real \sk{e, e'} +
	\norm{e'}^2 = 2.
  \]
  Tvorimo $S = \{ \mathring{B}(e, \frac{\sqrt{2}}{2}) \such e \in E\}$.
  Te krogle so paroma disjunktne, torej imamo injekcijo $\varphi: E \to S$.
  Po prejšnji lemi je $S$ največ števna, torej $\abs{E} \le \aleph_0$.

  V levo:
  Ker je $\dim H = \aleph_0$, obstaja števen KONS $(e_n)_n$.
  Vsak $x \in X$ lahko razvijemo v Fourierovo vrsto
  \[
	x = \sum_{n =1}^\infty \sk{x, e_n} e_n.
  \]
  Za poljuben $\varepsilon > 0$ potem obstaja $n_\varepsilon$, da za vse $n \ge
  n_\varepsilon$ velja
  \[
	\norm{x - \sum_{k=1}^n \sk{x, e_k} e_k} < \varepsilon.
  \]
  Skalarne produkte $\sk{x, e_k}$ lahko aproksimiramo z $\lambda_k \in \Q$, če
  je $\F = \R$, oziroma $\lambda_k \in \Q + i \Q$, če je $\F = \C$.
  V obeh primerih zahtevamo $\abs{\sk{x, e_k} - \lambda_k} <
  \nicefrac{\varepsilon}{n}$.
  Tako dobimo števno gosto množico
  \[
	\left\{ \sum_{k=1}^n \lambda_k e_k \such n \in \N, \lambda_k \in \Q + i \Q
	\right\}.
	\qedhere
  \]
\end{proof}

\begin{definicija}
  Linearna preslikava $U: H \to K$ je \pojem{izomorfizem} Hilbertovih
  prostorov (tudi \pojem{unitarni operator}), če je surjektivna in če velja
  \[
	\sk{Ux, Uy} = \sk{x,y}.
  \]
\end{definicija}

\begin{trditev}
  Naj bo $U: X \to Y$ linearna izometrija med prostoroma s skalarnim produktom.
  Tedaj $U$ ohranja skalarni produkt.
\end{trditev}

\begin{proof}
  Računamo
  \begin{gather*}
	\norm{U(x+y)}^2 = \norm{Ux}^2 + \norm{Uy}^2 + 2 \Real \sk{Ux, Uy} \\
	\norm{U(x+y)}^2 = \norm{x+y}^2 = \norm{x}^2 + \norm{y}^2 + 2 \Real \sk{Ux, Uy}
  \end{gather*}
  torej $\Real \sk{x,y} = \Real \sk{Ux, Uy}$.
  Če namesto $x$ pišemo $ix$, dobimo še $- \Imag \sk{x,y} = - \Imag \sk{Ux,
	Uy}$.
\end{proof}

\begin{lema}
  Naj bo $U: H \to K$ izomorfizem Hilbertovih prostorov.
  Potem $U$ slika KONS v KONS.
\end{lema}

\begin{proof}
  Naj bo $\{e_i\}_{i \in I}$ KONS za $H$.
  Potem je $\sk{Ue_i, U e_j} = \sk{e_i, e_j} = \delta_{ij}$.
  Če je $y \in K$ tak, da je $y \bot U e_i$ za vse $i$, potem je $0 = \sk{y, U
	e_i} = \sk{U^{-1} y, e_i}$, torej $y = 0$.
\end{proof}

\begin{izrek}
  Hilbertova prostora $H$ in $K$ sta izomorfna natanko tedaj, ko je $\dim H =
  \dim K$.
\end{izrek}

\begin{proof}
  V desno sledi iz leme.
  V levo:
  Dokazali bomo, da je $H \cong l^2(I)$, kjer je $(e_i)_{i \in I}$ KONS za $H$.
  Definiramo $U: H \to l^2(I)$ z $Ux = \hat{x}$, kjer je
  \[
	\hat{x}: i \mapsto \sk{x, e_i}.
  \]
  Potem je $U$ linearna izometrija, saj
  \[
	\norm{Ux}^2 = \sum_{i \in I} \abs{\hat{x}(i)}^2 = \sum_{i \in I} \abs{\sk{x,
	  e_i}}^2 = \norm{x}^2
  \]
  po Parsevalovi enakosti, hkrati pa je $U$ tudi surjektivna, saj za $f \in
  l^2(I)$ lahko definiramo
  \[
	x = \sum_{i \in I} f(i) e_i,
  \]
  kar konvergira, ker je $f \in l^2(I)$.
  Seveda $\hat{x} = f$.
  Bijekcijo med indeksnima množicama lahko razširimo do izomorfizma Hilbertovih
  prostorov.
\end{proof}

\begin{posledica}
  Neskončnorazsežen separabilen Hilbertov prostor je izomorfen $l^2$.
\end{posledica}

Naj bosta $H$ in $K$ Hilbertova prostora.
Produkt $H \times K$ opremimo s skalarnim produktom
\[
  \sk{(x_1,y_1), (x_2, y_2)} = \sk{x_1, x_2} + \sk{y_1, y_2},
\]
s čimer je prostor $H \times K$ poln, torej Hilbertov.
Označimo $H \times K = H \oplus K$, konstrukciji pravimo \pojem{ortogonalna
  direktna vsota}.

Za neskončne direktne vsote množico
\[
  \left\{ (x_n)_n \such x_n \in H_n, \sum_{n=1}^\infty \norm{x_n}^2 < \infty
  \right\},
\]
kjer so $H_n$ Hilbertovi prostori, opremimo s skalarnim produktom
\[
  \sk{(x_n)_n, (y_n)_n} = \sum_{n=1}^\infty \sk{x_n, y_n}.
\]
Dobljen prostor označimo z
\[
  \bigoplus_{n=1}^\infty H_n,
\]
in ga imenujemo \pojem{ortogonalna direktna vsota} prostorov $(H_n)_n$.
Je tudi Hilbertov prostor.

\podnaslov{Stone-Weierstrassov izrek}

\begin{lema}
  Naj bo $X$ neprazna množica in $V$ vektorski prostor funkcij na $X$, ki loči
  točke in vsebuje konstante.
  Tedaj za vsaka različna $x, y \in X$ in $\alpha, \beta \in \R$ obstaja $f \in
  V$, da je $f(x) = \alpha$ ter $f(y) = \beta$.
\end{lema}

\begin{lema}
  Naj bo $K$ kompakten Hausdorffov prostor in $V \subseteq \zvezne{K}$ realen
  vektorski prostor, ki je podmreža v $\zvezne{K}$.
  Naj $V$ vsebuje konstante in loči točke na $K$.
  Tedaj za vsak $g \in \zvezne{K}$, $a \in K$ ter $\varepsilon > 0$ obstaja $f
  \in V$, da je $f(a) = g(a)$ in $f(x) > g(x) - \varepsilon$ za vse $x \in K$.
\end{lema}

\begin{proof}
  Po prejšnji lemi za vsak $x \in K$ obstaja $f_x \in V$, da je $f_x(a) = g(a)$
  in $f_x(x) = g(x)$.
  Zaradi zveznosti zato obstaja odprta okolica $U_x \ni x$, da je $f_x(y) > g(y)
  - \varepsilon$ za vse $y \in U_x$.
  Dobimo odprto pokritje $K$, zaradi kompaktnosti obstaja končno podpokritje
  $U_{x_1} \cup \ldots \cup U_{x_n} = K$.
  Definiramo $f = \max\{f_{x_1}, \ldots, f_{x_n}\}$.
\end{proof}

\begin{izrek}[mrežni Stone-Weierstrass]
  Naj bo $K$ kompakten Hausdorffov prostor ter $V$ realen vektorski podprostor
  $\zvezne{K}$, ki je podmreža, loči točke $K$ in vsebuje konstante.
  Tedaj je $V$ gost v $\zvezne{K}$ glede na supremum normo.
\end{izrek}

\begin{proof}
  Naj bo $g \in \zvezne{K}$ in $\varepsilon > 0$.
  Iščemo $f \in V$, da je $\norm{f-g}_\infty < \varepsilon$, oziroma $f-g <
  \varepsilon$ in $g-f < \varepsilon$.
  Po prejšnji lemi za vsak $x \in K$ obstaja $f_x \in V$, da je $f_x > g -
  \varepsilon$ in $f_x(x) = g(x)$.
  Zaradi zveznosti obstaja odprta okolica $V_x$ za $x$, da je $\abs{f_x(y) -
	g(y)} < \varepsilon$ za vse $y \in V_x$.
  Množice $V_x$ tvorijo pokritje za $K$, torej obstaja končno podpokritje
  $V_{x_1} \cup \ldots \cup V_{x_n}$.
  Potem definiramo $f = \min \{ f_{x_1}, \ldots, f_{x_n} \}$.
  Za vsak $m = 1, \ldots, n$ velja $f_{x_m} > g - \varepsilon$, torej $f > g -
  \varepsilon$.
  Za poljuben $y \in K$ obstaja $m$, da je $y \in V_m$, torej $f(y) \le
  f_{x_m}(y) < g(y) + \varepsilon$.
\end{proof}

\begin{lema}
  Obstaja zaporedje polinomov, ki na $[0,1]$ konvergira enakomerno proti
  funkciji $\sqrt{x}$.
\end{lema}

\begin{proof}
  Funkcijo $x \mapsto 1 - \sqrt{1-x}$ razvijemo v Taylorjevo vrsto na $\zo{0,1}$.
  Velja
  \[
	1 - (1 - x)^{1/2} = 1 - \sum_{k=0}^\infty \binom{1/2}{k} (-1)^k x^k.
  \]
  S $S_n(x)$ označimo $n$-to delno vsoto zgornje vrste.
  Za $x \in (0,1)$ je zaporedje $(S_n(x))_n$ strogo naraščajoče in velja $\lim
  S_n(x) \le 1$, saj je $1 - (1 - x)^{1/2} \le 1$ za te $x$.
  Ker so $S_n$ zvezni, je
  \[
	S_n(x) = \lim_{x \to 1} S_n(x) \le 1,
  \]
  torej je zaporedje $(S_n(1))_n$ naraščajoče in navzgor omejeno.
  Definirajmo $f(x) = \lim S_n(x)$.

  Na $\zo{0,1}$ se $f$ in $1 - \sqrt{1-x}$ ujemata, hkrati pa je za $x \in
  [0,1]$ tudi
  \[
	0 \le f(x) - S_n(x) = \sum_{k=n+1}^\infty \binom{1/2}{k} (-1)^k x^k
	\le \sum_{k=n+1}^\infty \binom{1/2}{k} (-1)^k
	= f(1) - S_n(1) \xrightarrow[n \to \infty]{} 0.
  \]
  Torej $S_n$ konvergira k $f$ enakomerno na $[0,1]$, in je zato $f$ zvezna in
  se na tem intervalu ujema z $1 - \sqrt{1-x}$.
  Dokazali smo, da obstaja zaporedje polinomov, ki konvergirajo k $f$
  enakomerno.
  Enostavno ga lahko transformiramo v zaporedje, ki enakomerno konvergira k
  $\sqrt{x}$.
\end{proof}

\begin{izrek}[Stone-Weierstrass, realna verzija]
  Naj bo $K$ kompakten Hausdorffov prostor in $A \subseteq \zvezne{K}$
  podalgebra, ki loči točke in vsebuje konstante.
  Tedaj je $A$ gosta v $\zvezne{K}$.
\end{izrek}

\begin{proof}
  Pokazali bomo, da je $\cl{A}$ podmreža v $\zvezne{K}$.
  Potem bo po mrežni različici izreka gosta v $\zvezne{K}$.
  Ker bo zaprta, bo $\cl{A} = \zvezne{K}$.

  Očitno je $\cl{A}$ podalgebra.
  Ker velja $\max\{f,g\} = \pol (f + g + \abs{f-g})$ in $\min\{f,g\} = \pol(f+g
  - \abs{f-g})$, je dovolj pokazati, da je $\cl{A}$ zaprta za absolutne
  vrednosti.
  Naj bo $f \in \cl{A}$.
  Oglejmo si $g = f / \norm{f}_\infty$.
  Ker je $g^2 \le 1$, in ker po lemi obstaja zaporedje polinomov $(p_n)_n$, ki
  konvergirajo enakomerno na $[0,1]$ proti $\sqrt{x}$, velja $(p_n \circ g^2)(x)
  \to \abs{g(x)}$ enakomerno.
  Torej $\abs{g} \in \cl{A}$, ampak $\abs{g} = \abs{f} / \norm{f}_\infty$, torej
  tudi $\abs{f} \in \cl{A}$.
\end{proof}

\begin{izrek}[Weierstrass]
  Naj bo $K \subseteq \R$ kompakt in $f: K \to \R$ zvezna.
  Tedaj za vsak $\varepsilon > 0$ obstaja polinom $p$, da je $\abs{f(x) - p(x)}
  < \varepsilon$ za vse $x \in K$.
\end{izrek}

\begin{izrek}[Stone-Weierstrass, kompleksna verzija]
  Naj bo $K$ kompakten Hausdorffov prostor in $A \subseteq \zvezne{K}$
  podalgebra v algebri kompleksnih funkcij, ki
  \begin{itemize}
  \item loči točke $K$,
  \item vsebuje konstante,
  \item je sebi-adjungirana: $f \in A$ pomeni $\konj{f} \in A$.
  \end{itemize}
  Tedaj je $A$ gosta v $\zvezne{K}$.
\end{izrek}

\begin{proof}
  Naj bo $A_0$ algebra realnih funkcij, ki so vsebovane v $A$.
  Očitno $A_0$ vsebuje konstante.
  Za $x, y \in K$ obstaja $f \in A$, da je $f(x) \ne f(y)$, torej ena od realnih
  funkcij $\frac{f+\konj{f}}{2}, \frac{f-\konj{f}}{2i} \in A$ loči točki $x, y$.
  Torej $A$ loči točke.
  Po realni verziji izreka je $A_0$ gosta v $\zvezne{K, \R}$.

  Naj bo $f \in \zvezne{K, \C}$ poljubna in $\varepsilon >0$.
  Pišimo $f = g + ih$ za realni $g, h$.
  Po razno dokazanem obstajata $g_1, h_1 \in A_0$, da je $\norm{g - g_1} <
  \nicefrac{\varepsilon}{2}$ in $\norm{h-h_1} < \nicefrac{\varepsilon}{2}$.
  Potem je $\norm{f - (g_1 + i h_1)} < \varepsilon$.
\end{proof}

% LocalWords:  polskalarnim von paralelogramska paralelogramski paralelogramsko
% LocalWords:  Riesz Rieszovem Hahn-Banachovo Hahn-Banachova Rieszu Besselova
% LocalWords:  Besselovo KONS Besselovi Parsevalu separabilnem Parsevalovi
% LocalWords:  podmreža Weierstrass
