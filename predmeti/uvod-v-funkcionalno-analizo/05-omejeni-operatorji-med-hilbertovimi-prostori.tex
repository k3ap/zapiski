\naslov{Omejeni operatorji med Hilbertovimi prostori}

\begin{definicija}
  Naj bosta $H, K$ Hilbertova prostora.
  Preslikava $u: H \times K \to \F$ se imenuje \pojem{seskvilinearna forma}, če
  velja
  \begin{itemize}
  \item $u(\alpha x + \beta y, z) = \alpha u(x, z) + \beta u(y, z)$,
  \item $u(x, \alpha y + \beta z) = \konj{\alpha} u(x,y) + \konj{\beta} u(x,z)$.
  \end{itemize}
\end{definicija}

\begin{definicija}
  Seskvilinearna forma je \pojem{zvezna} (ali \pojem{omejena}), če obstaja $M
  \ge 0$, da $\abs{u(x,y)} \le M \norm{x} \norm{y}$.
\end{definicija}

\vprasanje{Kaj je seskvilinearna forma? Kdaj je omejena?}

\begin{primer}
  Za $A \in B(H, K)$ definiramo $u(x,y) = \sk{Ax,y}$.
  To je seskvilinearna forma, velja $\abs{u(x,y)} \le \norm{A} \norm{x}
  \norm{y}$.
\end{primer}

\begin{izrek}
  Naj bo $u: H \times K \to \F$ omejena seskvilinearna forma.
  Potem obstajata natanko določeni $A \in B(H,K)$ in $B \in B(K,H)$, da
  \[
	u(x,y) = \sk{Ax,y} = \sk{x, By}.
  \]
\end{izrek}

\begin{proof}
  Fiksiramo $x \in H$ in definiramo $f_x: K \to \F$ z $f_x(y) = \konj{u(x,y)}$.
  To je očitno omejen linearen funkcional na $K$.
  Po Rieszovem izreku obstaja natanko določen $z_x \in K$, da je
  \[
	\konj{u(x,y)} = f_x(y) = \sk{y, z_x}
  \]
  za vsak $y \in K$.
  Definiramo $z_x = Ax$.
  To nam da preslikavo $A: H \to K$, da je $\konj{u(x,y)} = \sk{y, Ax} =
  \konj{\sk{Ax, y}}$.
  Enostavno lahko preverimo, da je $A$ linearna, ker velja
  \[
	\norm{Ax} = \norm{z_x} = \norm{f_x} \le M \norm{x},
  \]
  pa je tudi omejena (tu $M$ pride iz definicije omejene seskvilinearne forme).
  Če je $\sk{Ax, y} = \sk{A' x, y}$, potem velja $\sk{Ax - A'x, y} = 0$ za vsak
  $y$, torej $Ax = A'x$ za vse $x$.
  Sledi, da je $A$ enolično določena.
  Podobno za $B$.
\end{proof}

\vprasanje{Pokaži, da lahko vsako omejeno seskvilinearno formo izrazimo s
  skalarnim produktom.}

\begin{definicija}
  Naj bo $A$ omejen operator $H \to K$.
  Tedaj operatorju $B: H \to K$, za katerega velja $\sk{Ax,y} = \sk{x,By}$,
  pravimo \pojem{adjungirani operator} operatorja $A$.
  Označimo ga z $A^*$.
\end{definicija}

\begin{trditev}
  Preslikava $U \in B(H,K)$ je izomorfizem Hilbertovih prostorov natanko tedaj,
  ko je $U$ obrnljiv in $U^* = U^{-1}$.
\end{trditev}

\begin{proof}
  V levo:
  Če je $U$ obrnljiv, je surjektiven.
  Potem je
  \[
	\sk{Ux,Uy} = \sk{x, U^*U y} = \sk{x, U^{-1} U y} = \sk{x,y},
  \]
  torej $U$ ohranja skalarni produkt in je zato izomorfizem.

  V desno:
  Če je $U$ izomorfizem, je obrnljiv.
  Velja
  \[
	\sk{x,y} = \sk{Ux, Uy} = \sk{x, U^* U y}
  \]
  za poljubna $x, y$, torej je $U^* = U^{-1}$.
\end{proof}

\vprasanje{Pokaži: $U$ je izomorfizem Hilbertovih prostorov natanko tedaj, ko je
  obrnljiv in $U^* = U^{-1}$.}

\begin{opomba}
  Če sta $A, B \in B(H,K)$, potem velja
  \begin{itemize}
  \item $(A + B)^* = A^* + B^*$,
  \item $(\alpha A)^* = \konj{\alpha} A^*$,
  \item $A^{**} = A$.
  \end{itemize}
  Zadnje je res, ker je
  \[
	\sk{Ax,y} = \sk{x, A^* y} = \konj{\sk{A^* y, x}} = \konj{\sk{y, A^{**} x}} =
	\sk{A^{**}x, y}.
  \]
\end{opomba}

\begin{opomba}
  Preslikava $i: A \mapsto A^*$ je involucija.
\end{opomba}

\begin{trditev}
  Naj bosta $A \in B(H,K)$ in $B \in B(K,L)$ omejena operatorja.
  Tedaj $(BA)^* = A^* B^*$.
\end{trditev}

\begin{proof}
  Preprost račun.
\end{proof}

\begin{posledica}
  Operator $A \in B(H,K)$ je obrnljiv natanko tedaj, ko je $A^* \in B(K,H)$
  obrnljiv.
  Če je $A$ obrnljiv, velja $(A^*)^{-1} = (A^{-1})^*$.
\end{posledica}

\begin{proof}
  Operator $A$ je obrnljiv natanko tedaj, ko obstaja $B \in B(K,H)$, da velja
  $AB = I_K$ in $BA = I_H$.
  Potem je
  \begin{gather*}
	(AB)^* = B^* A^* = I_K, \\
	(BA)^* = A^* B^* = I_H,
  \end{gather*}
  torej je $A^*$ obrnljiv in $(A^*)^{-1} = B^* = (A^{-1})^*$.
  Podobno v drugo smer, kjer upoštevamo še $A^{**} = A$.
\end{proof}

\begin{trditev}
  Če je $A \in B(H,K)$, je $\jedro A^* = (\im A)^\bot$.
\end{trditev}

\begin{proof}
  Velja
  \[
	x \in \jedro A^*
	\iff A^* x = 0
	\iff \forall y. \sk{A^* x, y} = 0
	\iff \forall y. \sk{x, Ay} = 0
	\iff x \in (\im A)^\bot.
	\qedhere
  \]
\end{proof}

\vprasanje{Kaj je $\jedro A^*$?}

\begin{posledica}
  Za $A \in B(H,K)$ velja
  \begin{itemize}
  \item $\jedro A = (\im A^*)^\bot$,
  \item $(\jedro A)^\bot = \cl{\im A^*}$,
  \item $(\jedro A^*)^\bot = \cl{\im A}$.
  \end{itemize}
\end{posledica}

\begin{proof}
  Prva točka je očitna.
  Če na njej uporabimo komplement in upoštevamo $X^{\bot \bot} =
  \cl{\operatorname{Lin} X}$, dobimo drugo točko.
  Podobno za tretje.
\end{proof}

\begin{definicija}
  Operator $A \in B(H)$ je \pojem{sebi adjungiran}, če je $A^* =A$.
  Je \pojem{normalen}, če je $A^* A = A A^*$.
  Je \pojem{unitaren}, če je $A^* A = A A^* = I$.
\end{definicija}

\begin{opomba}
  Operator $A$ je unitaren natanko tedaj, ko je $A$ izomorfizem prostora $H$.
\end{opomba}

\begin{opomba}
  Vsak unitaren operator je normalen.
\end{opomba}

\begin{opomba}
  Sebi adjungiran operator je vedno normalen.
\end{opomba}

Če je $A \in B(H)$, lahko definiramo
\begin{gather*}
  \Real A = \frac{A+A^*}{2}, \\
  \Imag A = \frac{A - A^*}{2i},
\end{gather*}
tako dobimo $A = \Real A + i \Imag A$.
Oba ta operatorja sta sebi adjungirana.

\begin{trditev}
  Naj bo $H$ kompleksen Hilbertov prostor in $A \in B(H)$.
  Tedaj je $A = A^*$ natanko tedaj, ko je $\sk{Ax,x} \in \R$ za vsak $x \in H$.
\end{trditev}

\begin{proof}
  V desno: $\sk{Ax,x} = \sk{x,Ax}= \konj{\sk{Ax, x}}$.

  V levo:
  Po predpostavki je $\sk{A(x+y), x+y} \in \R$.
  Velja
  \[
	\sk{A(x+y), x+y} = \sk{Ax, x} + \sk{Ay,y} + \sk{Ax,y} + \sk{Ay,x},
  \]
  torej $\sk{Ax,y} + \sk{Ay,x} \in \R$ za vsaka $x,y$.
  Pišimo $\sk{Ax,y} = \alpha + i \beta$ in $\sk{Ay,x} = \gamma - i \beta$.

  Če menjamo $y \to i y$, dobimo $i \sk{Ay, x} - i \sk{Ax,y} \in \R$.
  To je enako
  \[
	i \sk{Ay, x} - i \sk{Ax,y}
	= i(\gamma - i \beta - \alpha - i \beta)
	= i(\gamma - \alpha) + 2 \beta,
  \]
  torej $\gamma = \alpha$.
\end{proof}

\vprasanje{Pokaži: če je $\sk{Ax, x} \in \R$ za vsak $x$, je $A$ sebi
  adjungiran.}

\begin{izrek}
  Naj bo $H$ Hilbertov prostor in $A \in B(H)$ sebi adjungiran operator.
  Tedaj velja
  \[
	\norm{A} = w(A) := \sup_{\norm{x} = 1} \abs{\sk{Ax,x}}.
  \]
\end{izrek}

\begin{proof}
  Velja $\abs{\sk{Ax,x}} \le \norm{Ax} \norm{x} \le \norm{A} \norm{x}^2$, torej
  je $w(A) \le \norm{A}$.
  Računamo
  \begin{align*}
	\sk{A(x+y), x+y} - \sk{A(x-y), x-y}
	&= 2 \sk{Ax,y} + 2 \sk{Ay,x} \\
	&= 2 \sk{y,Ax} + 2 \sk{Ax,y} \\
	&= 4 \Real \sk{Ax,y},
  \end{align*}
  torej
  \begin{align*}
	4 \abs{\Real \sk{Ax,y}}
	&\le \abs{\sk{A(x+y), x+y}} + \abs{\sk{A(x-y), x-y}} \\
	&\le w(A) \left( \norm{x+y}^2 + \norm{x-y}^2 \right) \\
	&= 2 w(A) \left( \norm{x}^2 + \norm{y}^2 \right)
  \end{align*}
  oziroma
  \[
	\abs{\Real \sk{Ax,y}} \le \pol w(A) \left( \norm{x}^2 + \norm{y}^2 \right).
  \]
  Izberemo $x$ z $\norm{x} = 1$.
  Če je $A = 0$, izrek velja, sicer pa lahko vzamemo tak $x$, da $Ax \ne 0$.
  Potem nastavimo $y = \frac{Ax}{\norm{Ax}}$, in dobimo
  \[
	\abs{\Real \sk{Ax, \frac{Ax}{\norm{Ax}}}} \le w(A),
  \]
  zato $\norm{Ax} \le w(A)$ za vsak $x$ z $\norm{x} = 1$ in $Ax \ne 0$.
  Sedaj lahko naredimo supremum po enotski sferi.
\end{proof}

\vprasanje{Kako lahko izraziš normo sebi adjungiranega operatorja? Dokaži.}

\begin{trditev}
  Naj bo $H$ kompleksen Hilbertov prostor in $A \in B(H)$ tak, da je $\sk{Ax,x}
  = 0$ za vsak $x \in H$.
  Tedaj je $A = 0$.
\end{trditev}

\begin{proof}
  Ker je $H$ kompleksen, je $A = A^*$.
  Po izreku je
  \[
	\norm{A} = \sup_{\norm{x} = 1} \abs{\sk{Ax,x}} = 0.
	\qedhere
  \]
\end{proof}

\begin{izrek}
  Za $A \in B(H)$ velja $\norm{A^* A} = \norm{A}^2$.
\end{izrek}

\begin{proof}
  Računamo
  \[
	\norm{A^* A}
	= \sup_{\norm{x} = 1} \abs{\sk{A^* Ax, x}}
	= \sup_{\norm{x} = 1} \abs{\sk{Ax, Ax}}
	= \sup_{\norm{x} = 1} \norm{Ax}^2
	= \norm{A}^2.
	\qedhere
  \]
\end{proof}

\begin{opomba}
  Tej enakosti pravimo $C^*$-aksiom.
\end{opomba}

\vprasanje{Pokaži, da v $B(H)$ velja $C^*$-aksiom.}

\begin{trditev}
  Za $A \in B(H)$ so naslednje trditve ekvivalentne:
  \begin{itemize}
  \item $A$ je normalen,
  \item $\norm{A^* x} = \norm{Ax}$ za vsak $x \in H$,
  \item če je $\F = \C$, potem je $\Real A \Imag A = \Imag A \Real A$.
  \end{itemize}
\end{trditev}

\begin{proof}
  Ekvivalenca 1 in 2:
  Velja
  \[
	\norm{A^* x} = \norm{A x}
	\iff \sk{A^*x, A^* x} = \sk{Ax, Ax}
	\iff \sk{A A^* x, x} = \sk{A^* A x, x}
	\iff \sk{B x, x} = 0
  \]
  za $B = A A^* - A^* A$.
  Zgornje velja za vse $x \in H$ natanko tedaj, ko je $B = 0$.

  Ekvivalenca 1 in 3:
  Pišimo $A = H + iK$ za $H = \Real A$ in $K = \Imag A$.
  Potem je $A$ normalen natanko tedaj, ko je $(H - i K)(H + iK) = (H + iK)(H -
  iK)$, oziroma $HK = KH$.
\end{proof}

\vprasanje{Karakteriziraj normalnost operatorja in dokaži karakterizacijo.}

\begin{posledica}
  Naj bo $A \in B(H)$ normalen in $\lambda \in \F$.
  Tedaj je $\jedro (A^* - \lambda I) = \jedro (A - \konj{\lambda} I)$.
\end{posledica}

\begin{definicija}
  Operator $A \in B(H)$ je \pojem{pozitivno semidefiniten}, če je sebi
  adjungiran in $\sk{Ax,x} \ge 0$ za vsak $x \in H$.
\end{definicija}

\begin{opomba}
  Če je $A$ pozitivno definiten, je injektiven in zato
  \[
	\jedro A = \jedro A^* = (\im A)^\bot,
  \]
  torej $\cl{\im A} = (\im A)^{\bot \bot} = (\jedro A)^\bot = H$,
  torej je $\im A$ gost v $H$.
\end{opomba}

\begin{lema}
  Naj bosta $A, B \in B(H)$ pozitivno semidefinitna.
  Potem je $A+B \ge 0$.
  Če je $A > 0$, je $A+B > 0$.
\end{lema}

Definiramo množici
\begin{itemize}
\item $B(H)_{\text{sa}} = \{ A \in B(H) \such A = A^* \}$,
\item $B(H)_{\text{sa}}^+ = \{ A \in B(H)_{\text{sa}} \such A \ge 0 \}$.
\end{itemize}

\begin{trditev}
  Relacija $A \ge B \iff A-B \ge 0$ je delna urejenost na $B(H)_{\text{sa}}$.
\end{trditev}

\begin{definicija}
  Podmnožica $C \subseteq X$ realnega vektorskega prostora $X$ je
  \pojem{stožec}, če velja
  \begin{itemize}
  \item $C + C \subseteq C$,
  \item $\lambda C \subseteq C$ za vsak $\lambda \ge 0$,
  \item $C \cap (-C) = \{0\}$.
  \end{itemize}
\end{definicija}

\vprasanje{Kaj je stožec v realnem vektorskem prostoru?}

\begin{trditev}
  Množica $B(H)_{\text{sa}}^+$ je stožec v $B(H)_{\text{sa}}$.
\end{trditev}

\begin{trditev}
  Naj bo $A \in B(H)$ sebi adjungiran in $B \in B(K,H)$.
  \begin{itemize}
  \item če $A \ge 0$, potem $B^* A B \in B(K)_{\text{sa}}^+$,
  \item velja $-\norm{A} I \le A \le \norm{A} I$,
  \item $B^* B \in B(K)_{\text{sa}}^+$ in $BB^* \in B(H)_{\text{sa}}^+$.
  \end{itemize}
\end{trditev}

\begin{izrek}
  Naj bo $A \in B(H)$.
  Tedaj obstaja natanko en $B \in B(H)_{\text{sa}}^+$, da je $A = B^2$.
\end{izrek}

\podnaslov{Idempotenti in invariantni podprostori}

Če je $P$ idempotent, velja
\[
  x = \underbrace{x - Px}_{\in \jedro P} + \underbrace{Px}_{\in \im P}.
\]
Ker sta $\jedro P$ in $\im P$ zaprta v $H$, velja $H = \im P \oplus \jedro P$.

\begin{definicija}
  Idempotent $P \in B(H)$ je \pojem{ortogonalni projektor}, če je $\jedro P =
  (\im P)^\bot$.
\end{definicija}

\begin{izrek}
  Naj bo $P \in B(H)$ neničelni idempontent.
  Naslednje trditve so ekvivalentne:
  \begin{itemize}
  \item $P$ je ortogonalni projektor,
  \item $\norm{P} = 1$,
  \item $P^* = P$,
  \item $P$ je normalen,
  \item $\sk{Px, x} \ge 0$ za vsak $x \in H$.
  \end{itemize}
\end{izrek}

\begin{proof}
  1 v 2:
  Vemo že, da je $\norm{Px} \le \norm{x}$ za vsak $x \in H$.
  Če izberemo $x \in \im P$, je $Px = x$, torej $\norm{P} = 1$.

  2 v 1:
  Vzemimo $x \in (\jedro P)^\bot$, da je $x - Px \in \jedro P$.
  Sledi $\sk{x - Px, x} = 0$, oziroma $\sk{Px,x} = \norm{x}^2$.
  Torej
  \[
	\norm{x}^2 = \sk{Px, x} \le \norm{Px} \norm{x} \le \norm{P} \norm{x}^2 =
	\norm{x}^2,
  \]
  iz česar sledi $\norm{x} = \norm{Px} = \sqrt{\sk{Px,x}}$.
  Potem je
  \[
	\norm{x - Px}^2 = \norm{x}^2 + \norm{Px}^2 - 2 \Real \sk{x, Px}
	= \norm{x}^2 + \norm{Px}^2 - 2 \sk{Px,x} = 0.
  \]
  Torej $x = Px \in \im P$, oziroma $(\jedro P)^\bot \subseteq \im P$.

  Za drugo inkluzijo vzemimo $y \in \im P$ in zapišimo $y = y_1 + y_2$ za $y_1
  \in \jedro P$, $y_2 \in (\jedro P)^\bot$.
  Potem je
  \[
	y = Py = P(y_1 + y_2) = P y_2 = y_2 \in (\jedro P)^\bot,
  \]
  torej $\im P \subseteq (\jedro P)^\bot$.

  1 v 3:
  Za $x, y \in H$ zapišimo $x = x_1 + x_2$, $y = y_1 + y_2$, kjer sta $x_1, y_1
  \in \jedro P$ ter $x_2, y_2 \in (\jedro P)^\bot$.
  Računamo
  \begin{gather*}
	\sk{P^* x, y} = \sk{x, Py} = \sk{x_1 + x_2, P y_2} = \sk{x_1 + x_2, y_2} =
	\sk{x_2, y_2} \\
	\sk{Px, y} = \sk{Px_2, y_1 + y_2} = \sk{x_2, y_1 + y_2} = \sk{x_2, y_2}.
  \end{gather*}

  3 v 4 je očitno.
  4 v 1:
  Operator $P$ je normalen, torej je $(\im P)^\bot = (\jedro P)^* = \jedro P$.
  Torej $\im P = P^\bot$.

  3 v 5:
  Računamo
  \[
	\sk{Px,x} = \sk{P^2 x, x} = \sk{Px, P^* x} = \sk{Px, Px} \ge 0.
  \]
  5 v 1 izpustimo.
\end{proof}

\vprasanje{Kdaj je idempotent $P \in B(H)$ ortogonalni projektor? Karakteriziraj
  in dokaži.}

\begin{definicija}
  Zaprt podprostor $M$ normiranega prostora $X$ je \pojem{invarianten} za $A \in
  B(X)$, če je $AM \subseteq M$.
\end{definicija}

\begin{trditev}
  Zaprt podprostor $M$ je invarianten za $A \in B(H)$ natanko tedaj, ko je
  $M^\bot$ invarianten za $A^*$.
\end{trditev}

\begin{proof}
  Naj bo $M$ invarianten za $A$.
  Če je $y \in M^\bot$, potem je $\sk{A^* y, x} = \sk{y, Ax} = 0$, in zato $A^*
  y \in M^\bot$.
  Drugo implikacijo dobimo, ko zamenjamo vlogi $M$ in $M^\bot$ ter $A$ in $A^*$.
\end{proof}

\begin{posledica}
  Naj bo $A \in B(H)_{\text{sa}}$.
  Potem je zaprt podprostor $M$ invarianten za $A$ natanko tedaj, ko je $M^\bot$
  invarianten za $A$.
\end{posledica}

\podnaslov{Kompaktni operatorji}

\begin{definicija}
  Naj bo $T: X \to Y$ linearna preslikava med normiranima prostoroma.
  Operator $T$ je \pojem{kompakten}, če je $\cl{T(B_X)}$ kompakt v $Y$.
\end{definicija}

\begin{opomba}
  $B_X = \{ x \in X \such \norm{x} = 1 \}$
\end{opomba}

\begin{opomba}
  Vsak kompakten operator je omejen.
\end{opomba}

Označimo $K(X,Y) = \{ T: X \to Y \such \text{$T$ kompakten} \}$.

\begin{izrek}
  Za metrični prostor $M$ so naslednje trditve ekvivalentne:
  \begin{itemize}
  \item $M$ je kompakten,
  \item vsako neskončno zaporedje v $M$ ima stekališče,
  \item $M$ je poln in povsem omejen (za vsak $\varepsilon > 0$ lahko $M$
	pokrijemo s končno mnogo kroglami polmera $\varepsilon$).
  \end{itemize}
\end{izrek}

\begin{izrek}
  Naj bo $T: X \to Y$ linearen operator.
  Naslednje trditve so ekvivalentne:
  \begin{itemize}
  \item $T$ je kompakten,
  \item $T$ preslika omejene množice v relativno kompaktne množice,
  \item če je $(x_m)_m$ omejeno zaporedje v $X$, potem ima $(Tx_m)_m$ stekališče
	v $Y$.
  \end{itemize}
\end{izrek}

\vprasanje{Definiraj in karakteriziraj kompaktnost linearnega operatorja.}

\begin{trditev}
  Naj bosta $X, Y$ Banachova prostora.
  Potem velja:
  \begin{itemize}
  \item $K(X,Y)$ je zaprt podprostor v $B(X,Y)$,
  \item za vsake $K \in K(X,Y)$, $A \in B(X)$ ter $B \in B(Y)$ sta $KA, BK \in
	K(X,Y)$.
  \end{itemize}
\end{trditev}

\begin{proof}
  Samo prva točka.
  Naj bosta $K_1, K_2 \in K(X,Y)$ ter $\lambda, \mu \in \F$.
  Naj bo $(x_m)_m$ omejeno zaporedje v $X$.
  Ker je $K_1$ kompakten, obstaja podzaporedje $(x_{n_k})_k$, da $(K_1
  x_{n_k})_k$ konvergira.
  Podobno imamo podzaporedje $(x_{n_{k_j}})_j$, da je $(K_2 x_{n_{k_j}})_j$
  konvergentno.
  Potem $(\mu K_1 x_{n_{k_j}} + \lambda K_2 x_{n_{k_j}})_j$ konvergira v $Y$.

  Naj bo sedaj $(K_n)_n$ zaporedje kompaktnih operatorjev, ki konvergira k
  nekemu $K$.
  Pokažimo, da je $K$ kompakten.
  Ker je $Y$ Banachov, je $\cl{K(B_X)}$ kompakt natanko tedaj, ko je
  $\cl{K(B_X)}$ povsem omejena, oziroma ko je $K(B_X)$ povsem omejena.
  Naj bo $\varepsilon > 0$.
  Obstaja $n \in \N$, da je $\norm{K - K_n} < \nicefrac{\varepsilon}{3}$.
  Ker je $K_n$ kompakten, obstajajo $x_1, \ldots, x_m \in B_X$, da je
  \[
	K_n(B_X) \subseteq \bigcup_{i=1}^m \mathring{B}(Kx_i,
	\nicefrac{\varepsilon}{3}).
  \]
  Pokažimo še, da je
  \[
	K(B_X) \subseteq \bigcup_{i=1}^m \mathring{B}(x_i, \varepsilon).
  \]

  Če je $x \in B_X$, potem obstaja $j$, da je $\norm{K_n x_j - K_n x} <
  \nicefrac{\varepsilon}{3}$.
  Velja
  \begin{align*}
	\norm{K x_j - K x}
	&\le \norm{K x_j - K_n x_j} + \norm{K_n x_j - K_n x} + \norm{K_n x - Kx} \\
	&\le \norm{K - K_n} (\norm{x_j} + \norm{x}) + \norm{K_n(x_j - x)} \\
	&< \varepsilon.
	  \qedhere
  \end{align*}
\end{proof}

\vprasanje{Pokaži, da je $K(X,Y)$ zaprt podprostor $B(X,Y)$ za Banachova $X, Y$.}

\begin{trditev}
  Naj bo $A \in B(X)$ operator z $\dim \im A < \infty$.
  Potem je $A$ kompakten.
\end{trditev}

\begin{proof}
  Velja $A(B_X) \subseteq \im A$.
  Ker je množica omejena v končnorazsežnem prostoru, je relativno kompaktna.
\end{proof}

\begin{definicija}
  Operator $A \in B(X,Y)$ je \pojem{končnega ranga}, če je $\dim \im A <
  \infty$.
  V tem primeru je $\rang A = \dim \im A$.
\end{definicija}

\begin{definicija}
  Množico vseh operatorjev končnega ranga $X \to Y$ označimo z $F(X,Y)$.
\end{definicija}

\vprasanje{Pokaži, da je $F(X,Y) \subseteq K(X,Y)$.}

\begin{izrek}
  Naj bo $X$ normiran prostor.
  Tedaj je $B_X$ kompakt natanko tedaj, ko je $\dim X < \infty$.
\end{izrek}

\begin{posledica}
  Identiteta $\id: X \to X$ je kompakten operator natanko tedaj, ko je $\dim X <
  \infty$.
\end{posledica}

Naj bo $X$ neskončnorazsežen Banachov prostor.
Tedaj je $\id \in B(X) \setminus K(X)$.
Tvorimo prostor $B(X) / K(X)$.
Ker je $X$ Banachov, je $B(X)/K(X)$ Banachov, ker pa je $K(X)$ zaprti ideal v
Banachovi algebri $B(X)$, je tudi $B(X) / K(X)$ Banachova algebra.
Pravimo ji \pojem{Calkinova algebra}.

\begin{trditev}
  Naj bo $X$ normiran in $Y$ Banachov.
  Naj bo $A \in K(X,Y)$ ter $\cl{A}$ enolična razširitev $A$ na $\cl{X}$ po
  zveznosti.
  Tedaj je $\cl{A} \in K(\cl{X}, \cl{Y})$.
\end{trditev}

\podnaslov{Izrek Arzela-Ascoli}

\begin{definicija}
  Naj bo $K$ kompakten Hausdorffov prostor. Množica $H \subseteq \zvezne{K}$ je
  \pojem{enakozvezna}, če za vsaka $x \in K$ in $\varepsilon > 0$ obstaja odprta
  množica $U_x \ni x$, da je $\abs{f(y) - f(x)} < \varepsilon$ za vse $y \in
  U_x$ in $f \in H$.
\end{definicija}

\begin{izrek}[Arzela-Ascoli]
  Naj bo $K$ kompakten Hausdorffov prostor in $H \subseteq \zvezne{K}$ družina
  funkcij.
  Tedaj je $H$ relativno kompaktna natanko tedaj, ko je enakozvezna in po točkah
  omejena.
\end{izrek}

\vprasanje{Povej izrek Arzela-Ascoli.}

\begin{izrek}
  Naj bo $(f_n)_n \subseteq \zvezne{[a,b]}$ tako zaporedje, da za vse
  $\varepsilon > 0$ obstaja $\delta > 0$, da za vsak $n$ velja
  \[
	\abs{y-x} < \delta \implies \abs{f_n(x) - f_n(y)} < \varepsilon.
  \]
  Če je $(f_n)_n$ omejeno zaporedje, je relativno kompaktno.
\end{izrek}

\begin{proof}
  Naj bo $(f_n)_n$ kot v izreku.
  Naj bo $(x_n)_n$ množica racionalnih števil na $[a,b]$, postavljena v
  zaporedje.
  Zaporedje $(f_n(x_1))_n$ je omejeno v $\C$, torej ima konvergentno
  podzaporedje $(f_n^{(1)}(x_1))_n$.
  Zaporedje $(f_n^{(1)}(x_2))_n$ je omejeno, torej ima konvergentno podzaporedje
  $(f_n^{(2)}(x_2))_n$.
  Postopek ponavljamo, da dobimo neskončno zaporedij $(f_n^{(k)}(x_k))_n$.
  Tvorimo $\tilde{f}_n = f_n^{(n)}$.
  Po konstrukciji $(\tilde{f}_n(x_j))_n$ konvergira za poljuben $j \in \N$.

  Naj bo $\varepsilon > 0$.
  Obstaja $\delta > 0$, da velja
  \[
	\abs{y-x} < \delta \implies \abs{f_n(y) - f_n(x)} < \frac{\varepsilon}{3}.
  \]
  Ker je $[a,b]$ kompakt, obstaja $p \in \N$, da je
  \[
	[a,b] \subseteq \bigcup_{j=1}^p (x_j - \delta, x_j + \delta).
  \]
  Obstaja $n_\varepsilon \in \N$, da za vse $m,n \ge n_\varepsilon$ velja
  $\abs{\tilde{f}_n(x_j) - \tilde{f}_m(x_j)} < \nicefrac{\varepsilon}{3}$ za vse
  $j=1, \ldots, p$.
  Za $x \in [a,b]$ potem obstaja $j$, da $\abs{x_j -x} < \delta$, torej
  \[
	\abs{\tilde{f}_n(x) - \tilde{f}_m(x)}
	\le \abs{\tilde{f}_n(x) - \tilde{f}_n(x_j)} + \abs{\tilde{f}_m(x) -
	  \tilde{f}_m(x_j)} + \abs{\tilde{f}_n(x_j) - \tilde{f}_m(x_j)} <
	\varepsilon,
  \]
  torej je $\norm{\tilde{f}_n - \tilde{f}_m} \le \varepsilon$ za $m, n \ge
  n_\varepsilon$.
  Torej je zaporedje $(\tilde{f}_n)_n$ Cauchyjevo in zato konvergira.
  Limita je stekališče $(f_n)_n$.
\end{proof}

\vprasanje{Pokaži: če za omejeno zaporedje $(f_n)_n$ zveznih funkcij na
  kompaktnem intervalu za poljuben $\varepsilon > 0$ obstaja $\delta > 0$, da za
  vsak $n$ iz $\abs{x-y} < \delta$ sledi $\abs{f_n(x) - f_n(y)} < \varepsilon$,
  potem ima zaporedje stekališče.}

\begin{trditev}
  Naj bo $H \subseteq \zvezne{[a,b]}$ družina zvezno odvedljivih funkcij.
  Če je supremum množice $\{ \norm{f}_\infty \such f \in H \}$ končen, potem je
  $H$ enakozvezna.
\end{trditev}

\begin{proof}
  Za $f \in H$ je
  \[
	\abs{f(y) - f(x)} = \abs{f'(\xi) (y-x)} \le \sup_{g \in H} \norm{g}_\infty
	\abs{y-x}
  \]
  po Lagrangeovem izreku, za neki $\xi$.
\end{proof}

\begin{izrek}
  Naj bo $k \in \zvezne{[a,b] \times [a,b]}$ in $K$ integralski operator, dan s
  predpisom
  \[
	(Kf)(x) = \int_a^b k(x,y) f(y) dy.
  \]
  Tedaj sta $K: (\zvezne{[a,b]}, \norm{\cdot}_\infty) \to (\zvezne{[a,b]},
  \norm{\cdot}_\infty)$ in $K: (\zvezne{[a,b]}, \norm{\cdot}_2) \to
  (\zvezne{[a,b]}, \norm{\cdot}_2)$ kompaktna operatorja.
\end{izrek}

\begin{proof}
  Ker je $k$ enakomerno zvezna, za vsak $\varepsilon > 0$ obstaja $\delta > 0$,
  da iz $\abs{(x,y) - (z,w)} < \delta$ sledi $\abs{k(x,y) - k(z,w)} <
  \varepsilon$.
  Potem je
  \begin{align*}
	\abs{Kf(x) - Kf(y)}
	&\le \int_a^b \abs{k(x,z) - k(y,z)} \abs{f(z)} dz \\
	&< \varepsilon \int_a^b \abs{f(z)} dz \\
	&= \varepsilon \sk{1, f}_{L^2} \\
	&\le \varepsilon \sqrt{b-a} \norm{f}_2,
  \end{align*}
  torej je $K(B_{\zvezne{[a,b]}})$ je enakozvezna.
  Je tudi omejena:
  \[
	\abs{Kf(x)}
	\le \int_a^b \abs{k(x,y) f(y)} dy
	\le \norm{k}_\infty \norm{f}_\infty (b-a).
  \]
  Po izreku Arzela-Ascoli je slika krogle relativno kompaktna glede na
  $\norm{\cdot}_\infty$ normo.
  Za drugo normo vzemimo omejeno zaporedje $(f_n)_n \subseteq \zvezne{[a,b]}$.
  Od prej vemo, da je $(Kf_n)_n$ enakozvezna, podobno pokažemo tudi, da je
  omejena.
\end{proof}

\vprasanje{Pokaži, da je integralski operator z zveznih jedrom kompakten na
  $\zvezne{[a,b]}$ za drugo ali neskončno normo.}

\podnaslov{Kompaktnost adjungiranega operatorja}

\begin{izrek}
  Naj bo $T \in B(H,K)$.
  Naslednje trditve so ekvivalentne:
  \begin{itemize}
  \item $T$ je kompakten,
  \item $T^*$ je kompakten,
  \item obstaja zaporedje $(T_n)_n \subseteq F(H,K)$, da $T_n \to T$.
  \end{itemize}
\end{izrek}

\begin{proof}
  3 v 1 je jasno.
  1 v 3:
  Množica $\cl{T(B_H)}$ je kompaktna v $K$, torej je separabilna in velja
  \[
	\cl{\im T} \subseteq \bigcup_{n \in \N} n \cl{T(B_H)}.
  \]
  Torej je tudi $\cl{\im T}$ separabilen Hilbertov prostor, in ima števen KONS
  $(e_n)_n$.
  Naj bo $P_n$ ortogonalni projektor na linearno ogrinjačo prvih $n$ vektorjev
  $e_i$.
  Projektor je končnega ranga in zato kompakten.
  Definiramo $T_n = P_n T$.

  Če je $y \in \cl{\im T}$, potem je $y = \sum_m \sk{y,e_m} e_m$.
  Velja $P_n y \xrightarrow[n \to \infty]{} y$, torej za $x \in H$ velja $Tx \in
  \im T$, torej $T_n x \xrightarrow[n \to \infty]{} Tx$.
  Torej $T_n \to T$ po točkah.

  Pokazati moramo še, da $T_n \to T$ v $B(H,K)$.
  Ker je $T$ kompakten, za vsak $\varepsilon > 0$ obstajajo $x_1, \ldots, x_m
  \in B_H$. da je
  \[
	T(B_H) \subseteq \bigcup_{j=1}^m \mathring{B}(T x_j, \frac{\varepsilon}{3}).
  \]
  Za $x \in B_H$ obstaja $x_j$, da je $\norm{Tx - Tx_j} <
  \nicefrac{\varepsilon}{3}$.
  Potem je
  \[
	\norm{Tx - T_n x}
	\le \norm{Tx - T x_j} + \norm{T x_j - T_n x_j} + \norm{T_n x_j - T_n x}
	< \varepsilon,
  \]
  saj
  \[
	\norm{T_n x_j - T_n x} \le \norm{P_n} \norm{T x_j - Tx} <
	\frac{\varepsilon}{3}.
  \]

  1 v 2:
  Najprej izračunamo
  \[
	\norm{T^* x}^2
	= \sk{T^* x, T^* x}
	= \sk{T T^* x, x}
	\le \norm{T T^* x} \norm{x}.
  \]
  Ker je $T$ kompakten, je $T T^*$ kompakten.
  Naj bo $(x_n)_n$ omejeno v $K$, torej obstaja konvergentno podzaporedje
  $(x_{n_k})_k$.
  Velja
  \[
	\norm{T^* x_{n_k} - T^* x_{n_j}}^2
	\le \norm{T T^* (x_{n_k} - x_{n_j})} \norm{x_{n_k} - x_{n_j}}.
  \]
  Prvi člen konvergira k $0$ za $k, j \to \infty$, drugi člen pa je omejen.
  Torej je zaporedje $(T^* x_{n_k})_k$ Cauchyjevo v $H$, ki je poln, zato
  konvergira, in je $T^*$ kompakten.

  2 v 1:
  Velja $T^{**} = T$.
\end{proof}

\vprasanje{Pokaži: omejen operator $T$ je kompakten natanko tedaj, ko je $T^*$
  kompakten, kar je natanko tedaj, ko obstaja zaporedje $(T_n)_n$ operatorjev
  končnega ranga, ki konvergira k $T$.}

% LocalWords:  seskvilinearna seskvilinearne involucija semidefiniten Calkinova
% LocalWords:  Idempotenti Arzela-Ascoli enakozvezna Lagrangeovem KONS
