\naslov{Linearni funkcionali}

\begin{izrek}
  Naj bo $T: X \to Y$ linearna preslikava med normiranima prostoroma.
  Naslednje trditve so ekvivalentne:
  \begin{itemize}
  \item $T$ je zvezna na $X$
  \item $T$ je zvezna v $x_0 \in X$
  \item $T$ je zvezna v $0$
  \item obstaja $C > 0$, da za vse $x \in X$ velja $\norm{Tx} \le C \norm{x}$
  \item $T$ je Lipschitzova
  \item $T$ je enakomerno zvezna
  \end{itemize}
\end{izrek}

\vprasanje{Karakteriziraj zveznost linearnih preslikav med normiranima
  prostoroma.}

Za omejen operator $T: X \to Y$ definiramo
\[
  \norm{T} = \inf \{ C > 0 \such \forall x . \norm{Tx} \le C \norm{x} \}.
\]
Ta infimum obstaja in je dejansko minimum.
Potem velja
\[
  \norm{Tx} \le \norm{T} \norm{x}
\]
za vsak $x \in X$.
Velja
\[
  \norm{T} = \sup_{\norm{x} = 1} \norm{Tx}
  = \sup_{\norm{x} \le 1} \norm{Tx}
  = \sup_{\norm{x} < 1} \norm{Tx}.
\]
Množico vseh omejenih linearnih operatorjev $X \to Y$ označimo z $B(X,Y)$, in jo
opremimo z zgornjo normo.

\vprasanje{Kako definiraš normo na (omejenem) linearnem operatorju? Povej še
  vsaj eno izražavo.}

\begin{trditev}
  Naj bodo $X, Y, Z$ normirani prostori.
  \begin{itemize}
  \item $B(X,Y)$ je normiran prostor.
  \item Če $T \in B(X,Y)$ in $S \in B(Y,Z)$, potem je $ST \in B(X,Z)$ in
	$\norm{ST} \le \norm{S} \norm{T}$.
  \end{itemize}
\end{trditev}

\begin{proof}
  Samo druga točka.
  Ker je omejenost ekvivalentna zveznosti, je kompozitum v $B(X,Z)$.
  Za vsak $x \in X$ velja
  \[
	\norm{STx} \le \norm{S} \norm{Tx} \le \norm{S} \norm{T} \norm{x}.
	\qedhere
  \]
\end{proof}

\begin{definicija}
  \pojem{Dualni prostor} prostora $X$ je $X^* = B(X, \F)$.
\end{definicija}

\begin{izrek}
  Naj bo $X$ normiran in $Y$ Banachov prostor.
  Tedaj je $B(X,Y)$ Banachov prostor.
\end{izrek}

\begin{proof}
  Naj bo $(T_n)_n$ Cauchyjevo v $B(X,Y)$.
  Izberimo $\varepsilon > 0$.
  Obstaja $n_\varepsilon$, da za $m, n \ge n_\varepsilon$ velja
  \[
	\norm{(T_n - T_m) x} \le \norm{T_n - T_m} \norm{x} < \varepsilon \norm{x}
  \]
  za poljuben $x \in X$.
  Zaporedje $(T_n x)_n$ je Cauchyjevo, zato obstaja $Tx = \lim T_nx \in Y$.
  S tem dobimo po točkah definiran operator $T: X \to Y$.

  Enostavno se prepričamo, da je $T$ linearen.
  Ker je $\abs{\norm{T_n} - \norm{T_m}} \le \norm{T_n - T_m}$, je tudi zaporedje
  norm Cauchyjevo, in zato omejeno.
  Torej je $\norm{T_n x} \le \norm{T_n} \norm{x} \le M \norm{x}$, in je $T$
  omejen.

  Za konec za poljuben $x \in X$ in $n, m \ge n_\varepsilon$ dobimo $\norm{T_n x
	- T_m x} < \varepsilon \norm{x}$, in če vzamemo limito $n \to \infty$,
  \[
	\norm{Tx - T_m x} \le \varepsilon \norm{x}.
  \]
  Torej je $\norm{T - T_m} \le \varepsilon$ za $m \ge n_\varepsilon$ in zato
  $T_m \to T$.
\end{proof}

\vprasanje{Pokaži: če je $Y$ Banachov, je $B(X,Y)$ Banachov.}

\begin{posledica}
  Dualni prostor je vedno Banachov.
\end{posledica}

\begin{izrek}
  Naj bo $X$ normiran prostor in $Y \le X$.
  Naj bo $T: Y \to Z$ omejen operator in $Z$ poln.
  Tedaj obstaja natanko en omejen linearen operator $S: \cl{Y} \to Z$, da je
  $\left. S \right|_Y = T$.
  Velja še $\norm{S} = \norm{T}$.
\end{izrek}

\begin{proof}
  Naj bo $x \in \cl{Y}$.
  Radi bi definirali $Sx$.
  Obstaja zaporedje $(x_n)_n$ v $Y$, da bo $x_n \to x$.
  Definiramo $Sx = \lim T x_n$.

  Če je tudi $(x_n')_n$ zaporedje, ki konvergira k $x$, je
  \[
	\lim (T x_n - T x_n') = \lim T(x_n - x_n') = 0,
  \]
  ker je $T$ zvezen.
  Torej je $S$ dobro definiran.
  Očitno je tudi linearen, in velja $\left. S \right|_Y = T$.
  Za enoličnost predpostavimo, da je tudi $S'$ tak operator.
  Potem za $x \in \cl{Y}$ velja $S x_n = S' x_n$, in sta limiti $Sx$ in $S' x$
  posledično tudi enaki.

  Velja
  \[
	\norm{Sx} = \norm{\lim T x_n} = \lim \norm{T x_n}
	= \lim \norm{T} \norm{x_n} = \norm{T} \norm{x},
  \]
  torej je $S$ omejen in $\norm{S} \le \norm{T}$.
  Obrat je očiten.
\end{proof}

\vprasanje{Dokaži: linearen operator $T: Y \to Z$, ki slika v poln prostor,
  lahko enolično razširimo do operatorja $\cl{Y} \to Z$.}

\begin{posledica}
  Naj bosta $S, T: X \to Y$ omejena operatorja, ki se ujemata na gostem
  podprostoru.
  Potem je $S = T$.
\end{posledica}

\begin{posledica}
  Naj bo $X$ normiran prostor, ki je gost podprostor v Banachovem prostoru $Y$.
  Tedaj sta $\hat{X}$ in $Y$ izometrično izomorfna.
\end{posledica}

\begin{proof}
  Naj bosta $\iota_Y$ in $\iota_{\hat{X}}$ vložitvi.
  \[\begin{tikzcd}
	  X & {\hat{X}} \\
	  & {\iota_Y(X)}
	  \arrow["{\iota_{\hat{X}}}", hook, from=1-1, to=1-2]
	  \arrow["{\iota_Y^{-1}}", from=2-2, to=1-1]
	  \arrow[dashed, from=2-2, to=1-2]
	\end{tikzcd}\]
  Preslikavi $\iota_Y^{-1}$ in $\iota_{\hat{X}}$ sta izometriji, torej je tak
  tudi njun kompozitum.
  To preslikavo lahko enolično razširimo do izometrije prostorov $Y$ in
  $\hat{X}$.
\end{proof}

\begin{definicija}
  Naj bo $T: X \to X$ omejen operator med normiranima prostoroma.
  Tedaj je $T$ \pojem{obrnljiv}, če je bijektiven in $T^{-1}$ omejen.
\end{definicija}

\begin{definicija}
  Operator $T:X \to Y$ je \pojem{navzdol omejen}, če obstaja $c > 0$, da je
  $\norm{Tx} \ge c \norm{x}$ za vsak $x \in X$.
\end{definicija}

\begin{opomba}
  Vsak navzdol omejen operator je injektiven.
\end{opomba}

\begin{trditev}
  Naj bo $T \in B(X,Y)$.
  Naslednji trditvi sta ekvivalentni.
  \begin{itemize}
  \item Obstaja operator $T^{-1}: TX \to X$.
  \item $T$ je navzdol omejen.
  \end{itemize}
\end{trditev}

\podnaslov{Banachov izrek}

\begin{definicija}
  Naj bo $X$ vektorski prostor.
  Preslikava $p: X \to \R$ je \pojem{sublinearni funkcional}, če je $p(x+y) \le
  p(x) + p(y)$ in $p(\lambda x) = \lambda p(x)$ za poljubne $x, y \in X$ ter
  $\lambda \ge 0$.
\end{definicija}

\vprasanje{Kaj je sublinearni funkcional?}

\begin{primer}
  Vsaka polnorma je sublinearni funkcional.
\end{primer}

\begin{izrek}[realni Hahn-Banachov izrek]
  Naj bo $Y \le X$ vektorski prostor in $p: X \to \R$ sublinearni funkcional.
  Naj bo $f: Y \to \R$ tak linearni funkcional, da za vsak $y \in Y$ velja $f(y)
  \le p(y)$.
  Tedaj obstaja linearni funkcional $F: X \to \R$, da je $\left. F \right|_Y =
  f$ in $F(x) \le p(x)$ za vsak $x \in X$.
\end{izrek}

\begin{proof}
  Prvo obravnavajmo primer, kjer je $\dim X/Y = 1$.
  Tedaj je $X = Y \oplus \R x_0$ za neki $x_0 \in X \setminus Y$.
  Vse možne linearne razširitve $f$ do $F$ so oblike
  \[
	F(x) = F(y + \lambda x_0) = F(y) + \lambda F(x_0),
  \]
  torej je razširitev enolično določena z $F(x_0) =: \alpha$.
  Za $y_1, y_2 \in Y$ velja
  \[
	f(y_1 + y_2) \le p(y_1 + y_2) = p(y_1 + y_2 + x_0 - x_0)
	\le p(y_1 + x_0) + p(y_2 - x_0)
  \]
  oziroma
  \[
	f(y_2) - p(y_2 - x_0) \le p(y_1 + x_0) - f(y_1),
  \]
  torej je
  \[
	\sup_{y \in Y} \left( f(y) - p(y - x_0) \right)
	\le \inf_{y \in Y} \left( p(y + x_0) - f(y) \right).
  \]
  Za $\alpha$ lahko izberemo katerokoli vrednost med tema številoma.
  Potem definiramo $F(y + t x_0) = f(y) + t \alpha$ in ločimo primere.
  \begin{itemize}
  \item Če je $t = 0$, je $F(y) \le p(y)$ po predpostavki, saj je $F(y) = f(y)$.
  \item Če je $t > 0$, je $\alpha \le p(\nicefrac{y}{t} + x_0) -
	f(\nicefrac{y}{t})$, kar množimo s $t$, in dobimo $f(y) + t \alpha \le p(y +
	t x_0)$.
  \item Če je $t < 0$, je $\alpha \ge f(-\nicefrac{y}{t}) - p(-\nicefrac{y}{t} -
	x_0)$, kar množimo z $(-t)$ in zaključimo kot zgoraj.
  \end{itemize}

  Za splošen primer tvorimo množico
  \[
	\mathcal{A} = \{
	(Y_1, f_1) \such Y \le Y_1 \le X, \left. f_1 \right|_Y = f,
	\forall y_1 \in Y_1 . f_1(y_1) \le p(y_1)
	\}
  \]
  in definiramo relacijo
  \[
	(Y_1, f_1) \preccurlyeq (Y_2, f_2) \iff Y_1 \le Y_2 \land \left. f_2
	\right|_{Y_1} = f_1.
  \]
  To je očitno delna urejenost.
  Naj bo $\{(Y_i, f_i)\}_{i \in I}$ neka veriga v $\mathcal{A}$.
  Vzemimo $Z = \bigcup_{i \in I} Y_i$.
  To je podprostor, ker je veriga urejena.
  Definiramo še preslikavo $g: Z \to \R$ z $g(z) = f_i(z)$ za nek $i \in I$, za
  katerega je $z \in Y_i$.
  To je dobro definiran funkcional, za katerega velja $g(z) \le p(z)$ za vse $z
  \in Z$.

  Očitno je $(Z, g)$ zgornja meja za verigo.
  Po Zornovi lemi ima $\mathcal{A}$ maksimalen element $(\tilde{Y}, \tilde{f})$.
  Če je $\tilde{Y} \ne X$, potem obstaja $x \in X \setminus \tilde{Y}$.
  Po prvem koraku dokaza lahko $\tilde{f}$ razširimo na linearno ogrinjačo
  množice $\{\tilde{Y}, x\}$, kar je protislovje z maksimalnostjo.
  Torej $X = \tilde{Y}$.
\end{proof}

\vprasanje{Povej in dokaži realni Hahn-Banachov izrek.}

\begin{lema}
  Naj bo $X$ kompleksen vektorski prostor.
  Potem veljajo naslednje točke.
  \begin{itemize}
  \item Če je $f: X \to \R$ $\R$-linearen funkcional, je $\tilde{f}: X \to \C$,
	definiran z $\tilde{f}(x) = f(x) - i f(ix)$ $\C$-linearen funkcional, in
	velja $\Real \tilde{f} = f$.
  \item Če je $g: X \to \C$ $\C$-linearen funkcional in $\Real g = f$, potem je
	$g = \tilde{f}$.
  \item Če je $p$ polnorma, potem je $\abs{f(x)} \le p(x)$ za vse $x \in X$
	natanko tedaj, ko za vse $x \in X$ velja $\abs{\tilde{f}(x)} \le p(x)$.
  \item Če je $X$ normiran in $f: X \to \R$ omejen, je $\tilde{f}$ omejen in
	$\norm{\tilde{f}} = \norm{f}$.
  \end{itemize}
\end{lema}

\begin{proof}
  Prva točka je enostavna.
  Za drugo točko le pogledamo $g(x) = f(x) + i f_1(x)$ ter $i g(x) = g(ix)$, s
  čimer pokažemo $f_1(x) = -f(ix)$.

  Za tretjo točko opazimo
  \[
	\abs{f(x)} = \abs{\Real \tilde{f}(x)} \le \abs{\tilde{f}(x)} \le p(x),
  \]
  v drugo smer pa
  \[
	\abs{\tilde{f}(x)} = e^{i \varphi} \tilde{f}(x) = \tilde{f}(e^{i \varphi} x)
	= \Real \tilde{f}(e^{i \varphi} x)
  \]
  za nek kot $\varphi$.
  Potem
  \[
	\abs{\tilde{f}(x)} = f(e^{i \varphi} x) \le p(e^{i \varphi} x) = \abs{e^{i
		\varphi}} p(x) = p(x).
  \]

  Za zadnjo točko poglejmo $\abs{f(x)} \le \norm{f} \norm{x}$.
  Potem je $p(x) = \norm{f} \norm{x}$ polnorma na $X$, za katero velja
  $\abs{f(x)} \le p(x)$.
  Po prejšnji točki velja $\abs{\tilde{f}(x)} \le p(x)$, torej je $\tilde{f}$
  omejen z $\norm{f}$.
  Po drugi strani je $\norm{\tilde{f}(x)} \le q(x)$ za $q(x) = \norm{\tilde{f}}
  \norm{x}$, in zato po prejšnji točki $\abs{f(x)} \le q(x)$ in $\norm{f} \le
  \norm{\tilde{f}}$.
\end{proof}

\vprasanje{Pokaži, da lahko $\R$-linearni funkcional iz kompleksnega vektorskega
  prostora $X$ enolično razširimo do $\C$-linearnega funkcionala, in da
  razširitev ohranja normo.}

\begin{izrek}[kompleksni Hahn-Banach]
  Naj bo $X$ vektorski prostor nad $\F$, $Y \le X$ in $p$ polnorma na $X$.
  Če je $f: Y \to \F$ linearni funkcional, da za vse $y \in Y$ velja $\abs{f(y)}
  \le p(y)$, potem obstaja linearni funkcional $F: X \to \F$, za katerega je
  $\left. F \right|_Y = f$ in $\abs{F(X)} \le p(x)$ za vsak $x \in X$.
\end{izrek}

\begin{proof}
  Za $\F = \R$ po realni verziji izreka obstaja funkcional $F: X \to \R$, ki
  razširja $f$ in za katerega je $F(x) \le p(x)$ za $x \in X$.
  Velja $-F(x) = F(-x) \le p(-x) = p(x)$, torej $\abs{F(x)} \le p(x)$.

  Če pa je $\F = \C$, vzemimo $\R$-linearen funkcional $f_1 = \Real f$, za
  katerega po lemi velja $\abs{f_1(y)} \le p(y)$ za $y \in Y$.
  Tega lahko razširimo do $F_1: X \to \R$, nato pa vzamemo funkcional
  $\tilde{F}$, za katerega velja $\Real \tilde{F} = F$.
  S pomočjo leme hitro vidimo, da $\tilde{F}$ razširja $f$ in
  $\abs{\tilde{F}(x)} \le p(x)$.
\end{proof}

\vprasanje{Povej in dokaži kompleksni Hahn-Banachov izrek.}

\begin{izrek}[Hahn-Banachov izrek za normirane prostore]
  Naj bo $Y \le X$ podprostor normiranega prostora $X$ in $f: Y \to \F$ omejen.
  Tedaj obstaja $f: X \to \F$, da je $\left. F \right|_Y = f$ ter $\norm{F} =
  \norm{f}$.
\end{izrek}

\vprasanje{Povej Hahn-Banachov izrek za normirane prostore.}

\begin{posledica}
  Naj bo $X$ normiran in $x \in X$ neničeln vektor.
  Tedaj obstaja $F \in X^*$, da je $\norm{f} = 1$ in $f(x) = \norm{x}$.
\end{posledica}

\begin{proof}
  Naj bo $Y = \F \cdot x$.
  Funkcional $g(\lambda x) = \lambda \norm{x}$ je zvezen, linearen in definiran
  na $Y$ z $\norm{g} = 1$.
  Po Hahn-Banachu ga lahko razširimo na funkcional $X \to \F$.
\end{proof}

\begin{posledica}
  Naj bo $X$ normiran in $x \in X$.
  Tedaj je
  \[
	\norm{x} = \max \{\abs{f(x)} \such f \in X^*, \norm{f} = 1\}.
  \]
\end{posledica}

\vprasanje{Pokaži, da je $\norm{x} = \max \{ \abs{f(x)} \such f \in X^*,
  \norm{f} = 1 \}$.}

\begin{posledica}
  Naj bo $Y \le X$ zaprt podprostor normiranega prostora $X$.
  Naj bo $x_0 \in X \setminus Y$ ter $d = d(x,Y)$.
  Tedaj obstaja $f \in X^*$, da je $f(x_0) = 1$, $\left. f \right|_Y = 0$ in
  $\norm{f} = \nicefrac{1}{d}$.
\end{posledica}

\begin{proof}
  Naj bo $g \in (X/Y)^*$ tak, da je $\norm{g} = 1$ in $g(x_0 + Y) = \norm{x_0 +
	Y} = d$.
  Definirajmo $f = g \circ \pi$, kjer je $\pi$ kvocientna projekcija.
  Za $y \in Y$ velja $f(y) = 0$, ker je $y + Y = 0$ in $g$ linearna.
  Funkcional $f$ je omejen, ker je kompozitum omejenih funkcionalov, velja
  $f(x_0) = d$.
  Izračunamo
  \[
	\norm{f} = \sup_{\norm{x} < 1} \abs{f(x)}
	= \sup_{\norm{x} < 1} \abs{g(\pi(x))}
	= \sup_{\norm{x + Y} < 1} g(x + Y)
	= \norm{g} = 1,
  \]
  torej je iskani funkcional $\inv{d} f$.
\end{proof}

\vprasanje{Pokaži, da lahko z linearnim funkcionalom ločiš točko od zaprtega
  podprostora.}

\begin{izrek}
  Naj bo $Y \le X$ podprostor normiranega prostora $X$.
  Tedaj je
  \[
	\cl{Y} = \bigcap \{ \jedro f \such f \in X^*, Y \subseteq \jedro f \}.
  \]
\end{izrek}

\begin{proof}
  Naj bo $Z$ presek iz izreka.
  Če je $Y \subseteq \jedro f$, je $\cl{Y} \subseteq \jedro f$, saj je jedro
  zaprto.
  Torej je $\cl{Y} \subseteq Z$.
  Recimo, da $\cl{Y}$ ni enako $Z$.
  Tedaj obstaja $z \in Z \setminus \cl{Y}$, torej po prejšnji posledici obstaja
  $f \in X^*$, da je $f(z) = 1$ in $\left. f \right|_{\cl{Y}} = 0$.
  Velja $Y \subseteq \jedro f$, torej $z \in \jedro f$, kar je protislovje z
  $f(z) = 1$.
\end{proof}

\vprasanje{Kako lahko izraziš zaprtje podprostora z jedri linearnih
  funkcionalov? Dokaži enakost.}

\podnaslov{Adjungirani operator in drugi dual}

\begin{definicija}
  Za normiran prostor $X$ in $f \in X^*$ definiramo $\hat{x}(f) = f(x)$.
\end{definicija}

\begin{trditev}
  Velja $\hat{x} \in X^{**}$ in $\norm{\hat{x}} = \norm{x}$.
\end{trditev}

\begin{proof}
  Velja $\abs{\hat{x}(f)} = \abs{f(x)} \le \norm{f} \norm{x}$, zato
  $\norm{\hat{x}} \le \norm{x}$.
  Po Hahn-Banachu obstaja $f$ z $\norm{f} = 1$ in $f(x) = \norm{x}$.
  Tedaj $\hat{x}(f) = \norm{x}$, torej $\norm{\hat{x}} \ge \norm{x}$.
\end{proof}

Prostor $X$ vložimo v $X^{**}$ s preslikavo $i(x) = \hat{x}$.

\begin{trditev}
  Preslikava $i$ je linearna izometrična vložitev.
\end{trditev}

Naj bo $A: X \to Y$ omejen linearen operator.
Za $f \in Y^*$ definiramo adjungirani operator operatorja $A$ v smislu
Banachovih prostorov, $A^*$, z
\[
  A^* f = f \circ A.
\]

\begin{trditev}
  Naj bo $A: X \to Y$ omejen linearen operator med normiranima prostoroma.
  Tedaj je $A^* : Y^* \to X^*$ tudi omejen in $\norm{A^*} = \norm{A}$.
\end{trditev}

\begin{proof}
  Izračunamo
  \[
	\abs{A^* f(x)} = \abs{f(Ax)} \le \norm{f} \norm{A x} \le \norm{f} \norm{A}
	\norm{x},
  \]
  torej $\norm{A^* f} \le \norm{A} \norm{f}$.
  Iz tega sledi, da je tudi $A^{**}$ omejen z $\norm{A^{**}} \le \norm{A^*} \le
  \norm{A}$.
  Vemo pa
  \[
	\norm{Ax} = \norm{\widehat{A x}} = \norm{A^{**} \hat{x}} \le \norm{A^{**}}
	\norm{\hat{x}} = \norm{A^{**}} \norm{x}
  \]
  torej je $\norm{A} \le \norm{A^{**}}$ in so norme enake.
\end{proof}

\vprasanje{Kako je definiran adjungiran operator v smislu Banachovih prostorov?
  Pokaži, da adjungiranje ohranja normo operatorja.}

\begin{definicija}
  Normiran prostor je \pojem{refleksiven}, če je $i_X$ surjekcija.
\end{definicija}

\begin{opomba}
  Refleksivni prostori so Banachovi, saj je $X^{**}$ Banachov.
\end{opomba}

% LocalWords:  Hahn-Banachov maksimalnostjo Hahn-Banach Hahn-Banachu
% LocalWords:  adjungiranje
