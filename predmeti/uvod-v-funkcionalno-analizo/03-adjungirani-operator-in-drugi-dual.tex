\naslov{Adjungirani operator in drugi dual}

\begin{definicija}
  Za normiran prostor $X$ in $f \in X^*$ definiramo $\hat{x}(f) = f(x)$.
\end{definicija}

\begin{trditev}
  Velja $\hat{x} \in X^{**}$ in $\norm{\hat{x}} = \norm{x}$.
\end{trditev}

\begin{proof}
  Velja $\abs{\hat{x}(f)} = \abs{f(x)} \le \norm{f} \norm{x}$, zato
  $\norm{\hat{x}} \le \norm{x}$.
  Po Hahn-Banachu velja $f$ z $\norm{f} = 1$ in $f(x) = \norm{x}$.
  Tedaj $\hat{x}(f) = \norm{x}$, torej $\norm{\hat{x}} \ge \norm{x}$.
\end{proof}

Prostor $X$ vložimo v $X^{**}$ s preslikavo $i(x) = \hat{x}$.

\begin{trditev}
  Preslikava $i$ je linearna izometrična vložitev.
\end{trditev}

Naj bo $A: X \to Y$ omejen linearen operator.
Za $f \in Y^*$ definiramo adjungirani operator operatorja $A$ v smislu
Banachovih prostorov, $A^*$, z
\[
  A^* f = f \circ A.
\]


% LocalWords:  Hahn-Banachu Banachovih
