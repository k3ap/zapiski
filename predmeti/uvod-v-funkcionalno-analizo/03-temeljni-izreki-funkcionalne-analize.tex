\naslov{Temeljni izreki funkcionalne analize}

\begin{izrek}[Baire]
  Naj bo $(X, d)$ poln metrični prostor in $(U_n)_n$ števna družina odprtih
  gostih množic v $X$.
  Tedaj je presek $\bigcap_n U_n$ gost v $X$.
\end{izrek}

\begin{proof}
  Dokazujemo, da za vsak $x \in X$ in $r > 0$ velja
  \[
	\mathring{B}(x, r) \cap \bigcap_{n \in \N} U_n \ne \varnothing.
  \]
  Naj bosta $x \in X$ ter $r > 0$ poljubna.
  Induktivno bomo konstruirali zaporedji $(x_n)_n$ in $(r_n)_n$ z naslednjimi
  lastnostmi:
  \begin{itemize}
  \item $B(x_{n+1}, r_{n+1}) \subseteq U_n \cap \mathring{B}(x_n, r_n)$,
  \item $r_n \le \nicefrac{1}{n}$.
  \end{itemize}
  Postavimo $x_1 = x$ in $r_1 = \min\{1, r\}$.
  Recimo, da smo že konstruirali zaporedji do $n$-tega člena.
  Ker je $U_n$ gosta odprta množica, je $U_n \cap \mathring{B}(x_n, r_n)$
  neprazna odprta množica, zato obstajata $r_{n+1} \le \nicefrac{1}{n+1}$ in
  $x_{n+1}$, da je $\mathring{B}(x_{n+1}, 2r_{n+1}) \subseteq U_n \cap
  \mathring{B}(x_n, r_n)$.
  Prva lastnost sedaj velja, ker je $B(x_{n+1}, r_{n+1}) \subseteq
  \mathring{B}(x_{n+1}, 2 r_{n+1})$.

  S tem smo konstruirali želeni zaporedji.
  Ker je
  \[
	x_n \in B(x_n, r_n) \subseteq U_{n-1} \cap \mathring{B}(x_{n-1}, r_{n-1})
	\subseteq \mathring{B}(x_{n-1}, r_{n-1}) \subseteq \mathring{B}(x_m, r_m)
  \]
  za $m < n$, je $d(x_m, x_n) \le r_m \le \nicefrac{1}{m}$.
  Torej je zaporedje Cauchyjevo in obstaja limita $x_0$ v $X$.
  Velja
  \[
	x_0 \in \bigcap_{n \in \N} B(x_{n+1}, r_{n+1})
	\subseteq \bigcap_{n \in \N} U_n \cap \mathring{B}(x_n, r_n)
	\subseteq \mathring{B}(x_1, r_1) \cap \bigcap_{n \in \N} U_n
	= \mathring{B}(x, r) \cap \bigcap_{n \in \N} U_n,
  \]
  s čimer je dokaz zaključen.
\end{proof}

\vprasanje{Povej in dokaži Bairov izrek.}

\begin{posledica}
  Naj bo $X$ poln metrični prostor in $(A_n)_n$ zaporedje zaprtih množic, da je
  $X = \bigcup_n A_n$.
  Tedaj obstaja $m \in \N$, da je $\mathring{A}_m \ne \varnothing$.
\end{posledica}

\begin{proof}
  Če je $\mathring{A}_j = \varnothing$ za vse $j$, potem je $A_j\complement$
  odprta in gosta.
  Potem je $\bigcap_j A_j\complement$ gost, torej $\bigcup_j A_j = \left(
	\bigcap_j A_j\complement \right)\complement \ne X$.
\end{proof}

\vprasanje{Povej in dokaži posledico Bairovega izreka z zaprtimi množicami.}

Naj bosta $X$ in $Y$ normirana prostora ter $\mathcal{F} \subseteq B(X,Y)$.
Recimo, da obstaja $M \ge 0$, za katerega je $\norm{T} \le M$ za vse $T \in
\mathcal{F}$.
Tedaj za vsak $x \in X$ velja
\[
  \norm{Tx} \le \norm{T} \norm{x} \le M \norm{x},
\]
oziroma $\mathcal{F}x \subseteq B(0, M \norm{x})$.
Pravimo, da je $\mathcal{F}$ \pojem{omejena po točkah}.

\begin{izrek}[princip enakomerne omejenosti]
  Naj bo $X$ Banachov in $Y$ normiran prostor.
  Naj bo $\mathcal{F} \subseteq B(X,Y)$ po točkah omejena družina.
  Potem je kot množica operatorjev enakomerno omejena v $B(X,Y)$.
\end{izrek}

\begin{proof}
  Definiramo
  \[
	A_n = \{ x \in X \such \forall T \in \mathcal{F} . \norm{Tx} \le n \}
	= \bigcap_{T \in \mathcal{F}} f_T^{-1}([0,n])
  \]
  za $f_T(x) = \norm{Tx}$.
  Ker je $A_n$ presek zaprtih množic, je zaprt.
  Ker je $\mathcal{F}$ omejena po točkah, za poljuben $x \in X$ obstaja $m \in
  \N$, da je $\norm{Tx} \le m$ za poljuben $T \in \mathcal{F}$, oziroma $x \in
  A_m$.
  Torej je $X$ enak uniji množic $A_n$, in po posledici Bairovega izreka obstaja
  tak $n_0 \in \N$, da ima $A_{n_0}$ notranjo točko, in posledično vsebuje
  odprto kroglo $\mathring{B}(x_0, r)$.

  Izberimo poljuben $x \in B(0,1)$ ter definirajmo $y = x_0 + \frac{r}{2} x$.
  Velja $y \in \mathring{B}(x_0, r)$, torej je $\norm{Ty} \le n_0$ za poljuben
  $T \in \mathcal{F}$.
  Sledi
  \[
	\norm{Tx} = \norm{T \frac{2y - x_0}{r}}
	\le \frac{2}{r} \norm{Ty} + \inv{r} \norm{T x_0}
	\le \frac{2}{r} n_0 + \inv{r} \norm{T x_0}
	=: M.
  \]
  Torej je $\norm{T} \le M$ za vsak $T \in \mathcal{F}$.
\end{proof}

\vprasanje{Povej in dokaži princip enakomerne omejenosti.}

\begin{izrek}[o šibki omejenosti]
  Naj bo $A \subseteq X$ podmnožica normiranega prostora $X$.
  Potem je $A$ omejena v $X$ natanko tedaj, ko je za vsak $f \in X^*$ množica
  $\{f(x) \such x \in A\}$ omejena v $\F$.
\end{izrek}

\begin{proof}
  V desno:
  Ker je $A$ omejena, obstaja $M$, da je $\norm{x} \le M$ za vsak $x \in A$.
  Potem je $\abs{f(x)} \le \norm{f} \norm{x} \le M \norm{f}$ za $f \in X^*$.

  V levo:
  Oglejmo si vložitev v drugi dual, $i(x) = \hat{x}$.
  Množica $\{f(x) \such x \in A\}$ je omejena natanko tedaj, ko je omejena
  množica $\{\hat{x}(f) \such \hat{x} \in i(A)\}$, ker pa je $i(A)$ omejena po
  točkah, je po principu enakomerne omejenosti $i(A)$ tudi enakomerno omejena,
  torej obstaja $M \ge 0$, da je $\norm{x} = \norm{\hat{x}} \le M$ za vse $x \in
  A$.
\end{proof}

\vprasanje{Povej in dokaži izrek o šibki omejenosti.}

\begin{posledica}
  Naj bo $X$ Banachov prostor in $Y$ normiran prostor.
  Naj bo $A \subseteq B(X,Y)$ taka, da za vsak $f \in Y^*$ in vsak $x \in X$
  obstaja $M_{f,x} \ge 0$, da je $\abs{f(Tx)} \le M_{f,x}$ za vsak $T \in A$.
  Tedaj je $A$ omejena.
\end{posledica}

\begin{proof}
  Po izreku o šibki omejenosti je množica $\{Ty \such T \in A\}$ omejena v $Y$.
  Torej je $A$ omejena po točkah, ker pa je $X$ Banachov, je $A$ omejena po
  principu enakomerne omejenosti.
\end{proof}

\begin{lema}
  Naj bo $X$ Banachov prostor in $(x_n)_n$ zaporedje vektorjev, za katere velja
  $\sum \norm{x_n} < \infty$.
  Tedaj vrsta $\sum x_n$ konvergira v $X$.
\end{lema}

\begin{proof}
  Označimo $s_n = \sum_{i=1}^n x_i$.
  Za $n > m$ velja
  \[
	\norm{s_n - s_m} = \norm{x_n + x_{n-1} + \cdots + x_{m+1}}
	\le \norm{x_{m+1}} + \cdots + \norm{x_n}
	\le \sum_{k \ge m} \norm{x_k}
	\xrightarrow[m \to \infty]{} 0.
  \]
  Torej je zaporedje $(s_n)_n$ Cauchyjevo in zato konvergentno.
\end{proof}

\begin{izrek}[o odprti preslikavi]
  Naj bo $T$ omejen surjektiven linearen operator med Banachovima prostoroma $X$
  in $Y$.
  Tedaj je $T$ odprta preslikava.
\end{izrek}

\begin{proof}
  Dokaz poteka v štirih korakih.
  V prvem koraku dokažimo, da če odprta podmnožica $U \subseteq X$ vsebuje $0$,
  ima $\cl{T(U)}$ notranjo točko.
  Ker je $U$ odprta, obstaja $\delta > 0$, da je
  \[
	\delta \mathring{B}(0, 1) = \mathring{B}(0,\delta) \subseteq U.
  \]
  Poljuben $x \in X$ je v
  \[
	x \in 2 \norm{x} \mathring{B}(0,1)
	= \frac{2 \norm{x}}{\delta} \delta \mathring{B}(0,1) \subseteq m U
  \]
  za $m \ge \frac{2 \norm{x}}{\delta}$, torej je
  \[
	X \subseteq \bigcup_{n \in \N} n U.
  \]
  Velja
  \[
	Y = TX = \bigcup_{n \in \N} T(n U)
	= \bigcup_{n \in \N} \cl{T(n U)}
  \]
  in po Bairovem izreku obstaja $n_0$, da ima $\cl{T(n_0 U)}$ notranjo točko.
  Množenje s skalarjem je homeomorfizem, zato velja $\cl{T(n_0 U)} = n_0
  \cl{TU}$ in ima tudi $\cl{TU}$ notranjo točko.

  V drugem koraku pokažimo, da je ob isti predpostavki točka $0$ notranja za
  $\cl{TU}$.
  Ker je $U$ odprta v $X$, obstaja $\mathring{B}(0,\varepsilon) \subseteq U$.
  Vzemimo $V = \mathring{B}(0, \nicefrac{\varepsilon}{2})$.
  Potem je množica $V - V \subseteq \mathring{B}(0, \varepsilon)$, kjer smo
  vzeli definicijo
  \[
	A - B := \{ a - b \such a \in A, b \in B \}.
  \]
  Po dokazanem v prvem koraku ima $\cl{TV}$ notranjo točko, torej obstaja odprta
  množica $W \subseteq \cl{TV}$.
  Množica
  \[
	\bigcup_{w \in W} (W - \{w\}) = W - W
	\subseteq \cl{TV} - \cl{TV}
	\subseteq \cl{TV - TV}
	= \cl{T(V - V)}
	\subseteq \cl{TU}
  \]
  je unija odprtih množic in vsebuje $0$.
  Druga vključitev zgoraj velja, ker je funkcija $m: Y \times Y \to Y$, $m(y_1,
  y_2) = y_1 - y_2$, zvezna.

  V tretjem koraku spet ob isti predpostavki pokažimo, da je $TU$ odprta okolica
  za $0$.
  Najprej pokažimo poseben primer, če je $U = \mathring{B}(0, \varepsilon)$.
  Definiramo $\varepsilon_0 = \nicefrac{\varepsilon}{2}$ in zapišemo
  \[
	\varepsilon_0 = \sum_{i=1}^\infty \varepsilon_i
  \]
  za neko zaporedje $(\varepsilon_i)_i$ pozitivnih števil.
  Po drugem koraku za vsak $i$ obstaja $\eta_i > 0$, da je $\mathring{B}(0,
  \eta_i) \subseteq \cl{T(\mathring{B}(0, \varepsilon_i))}$.
  Sedaj izberimo $y \in \mathring{B}(0, \eta_0)$ in pokažimo $y = Tx$ za neki $x
  \in \mathring{B}(0, \varepsilon)$.

  Če je $\norm{x} < \varepsilon_i$, velja $\norm{Tx} \le \norm{T}
  \varepsilon_i$, torej je
  \[
	\mathring{B}(0, \eta_i)
	\subseteq T(\mathring{B}(0, \varepsilon_i))
	\subseteq \mathring{B}(0, \varepsilon_i \norm{T}).
  \]
  Sledi $\eta_i \le \varepsilon_i \norm{T}$ in zaporedje $\eta_i$ konvergira k
  $0$.
  Po predpostavki je $y \in \mathring{B}(0,\eta_i) \subseteq
  \cl{T(\mathring{B}(0, \varepsilon_0))}$.
  Po definiciji zaprtja $\eta_1$-okolica za $y$ seka $T(\mathring{B}(0,
  \varepsilon_0))$, torej obstaja $x_0 \in X$, da je $\norm{x_0} <
  \varepsilon_0$ in $\norm{y - Tx_0} < \eta_1$.
  Sedaj velja $y - Tx_0 \in \mathring{B}(0, \eta_1) \subseteq
  \cl{T(\mathring{B}(0, \varepsilon_1))}$, in spet po definiciji zaprtja
  $\eta_2$-okolica za $y - Tx_0$ seka $\cl{T(\mathring{B}(0, \varepsilon_1))}$,
  torej obstaja $x_1 \in X$ z $\norm{x_1} < \varepsilon_1$ ter $\norm{y - Tx_0 -
	Tx_1} < \eta_2$.

  Postopek ponavljamo, s čimer dobimo zaporedje $(x_n)_n$, za katerega velja
  $\norm{x_n} < \varepsilon_n$ ter $\norm{y - T(x_1 + \cdots + x_n)} < \eta_{n+1}$.
  Ker je $\sum \norm{x_n} < \sum \varepsilon_n < \infty$, vrsta $\sum x_n$
  konvergira absolutno in zato v Banachovem prostoru $X$ konvergira proti nekemu
  vektorju $x$.
  Ker je $T$ omejen operator, konvergira tudi vrsta $\sum T x_n$ in velja $Tx =
  \sum T x_n$.
  To mora biti enako $y$, saj $\eta_i \to 0$.
  Dodatno je $\norm{x} \le \sum \norm{x_n} < 2 \varepsilon_0 = \varepsilon$,
  torej je $T(\mathring{B}(0, \varepsilon))$ okolica za $0$.

  Če pa $U$ ni take oblike, obstaja $\varepsilon > 0$, da je $\mathring{B}(0,
  \varepsilon) \subseteq U$, in po ravno dokazanem $T$-slika te množice vsebuje
  odprto kroglo okoli $0$.
  Zato jo vsebuje tudi $TU$.

  V zadnjem koraku dokažimo, da je slika odprte $U \subseteq X$ odprta v $Y$.
  Naj bo $y \in TU$.
  Obstaja $x \in U$, da je $y = Tx$, torej je $V = U - x$ odprta okolica za $0$
  in je po prejšnjem koraku $TV$ okolica za $0$ v $Y$.
  Potem je $TU = TV + Tx$ okolica za $y$.
\end{proof}

\vprasanje{Povej in dokaži izrek o odprti preslikavi.}

\begin{posledica}
  Naj bo $T$ omejen linearen bijektiven operator med Banachovima prostoroma.
  Potem je njegov inverz tudi omejen.
\end{posledica}

\vprasanje{Povej posledico izreka o odprti preslikavi.}

\begin{posledica}
  Naj bo $X$ vektorski prostor in $\norm{\cdot}_1, \norm{\cdot}_2$ dve normi na
  $X$, za kateri je $X$ Banachov prostor.
  Potem sta normi bodisi ekvivalentni bodisi neprimerljivi.
\end{posledica}

\begin{proof}
  Če je $\norm{x}_1 \le c_1 \norm{x_2}$, je identiteta $I: (X, \norm{\cdot}_2)
  \to (X, \norm{\cdot}_1)$ omejena, in je njen inverz tudi omejen.
\end{proof}

\vprasanje{Dokaži: dve normi, za kateri je $X$ Banachov prostor, sta bodisi
  ekvivalentni bodisi neprimerljivi.}

\begin{lema}
  Naj bosta $X, Y$ normirana prostora in $f: X \to Y$ zvezna preslikava.
  Tedaj je $\Gamma_f \zapp X \times Y$, če ta produkt opremimo z normo
  $\norm{(x,y)}_\infty = \max \{\norm{x}, \norm{y}\}$.
\end{lema}

\begin{proof}
  Naj gre $(x_n, f(x_n)) \to (x,y)$.
  Velja $x_n \to x$ in $f(x_n) \to y$, ker pa je $f$ zvezna, tudi $f(x_n) \to
  f(x)$.
\end{proof}

\begin{izrek}[o zaprtem grafu]
  Naj bo $T: X \to Y$ linearna preslikava, $X$ in $Y$ Banachova prostora ter
  $\Gamma_T$ zaprt v $X \times Y$.
  Potem je $T$ omejena.
\end{izrek}

\begin{proof}
  Graf je zaprt, torej je Banachov.
  Oglejmo si spodnji diagram.
  \[\begin{tikzcd}
	  X & {\Gamma_T} \\
	  & Y
	  \arrow["{(x, Tx)}", from=1-1, to=1-2]
	  \arrow["T"', from=1-1, to=2-2]
	  \arrow["{\operatorname{pr}_2}", from=1-2, to=2-2]
	\end{tikzcd}\]
  Projekcija $\operatorname{pr}_1: \Gamma_T \to X$ je omejena in bijektivna,
  torej je tudi njen inverz $x \mapsto (x, Tx)$ omejen po posledici izreka o
  odprti preslikavi.
  Velja $T = \operatorname{pr}_2 \circ \operatorname{pr}_1^{-1}$.
\end{proof}

\vprasanje{Povej in dokaži izrek o zaprtem grafu.}

% LocalWords:  Bairovega Bairovem Banachovima Bairov
