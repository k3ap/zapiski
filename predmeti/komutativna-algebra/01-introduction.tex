\naslov{Introduction}

\podnaslov{Rings}

\begin{definition}
  A \pojem{ring} is a set $A$, combined with operations $+$ and $\cdot$, such
  that $(A, +)$ is an Abelian group, $(A, \cdot)$ is a monoid, and
  distributivity holds: $a(b+c) = ab+ac$, $(b+c)a = ba+ca$ for any $a, b, c \in
  A$.
\end{definition}

\begin{remark}
  We will assume that all rings are commutative.
\end{remark}

\begin{example}
  We know of a few simple rings:
  \begin{itemize}
  \item $\Z$,
  \item $\Z[i] = \{a + bi \such a, b \in \Z\}$,
  \item the \pojem{zero ring} $\underline{0} = \{0\}$,
  \item if $A$ is a ring, then $A[X], A[X_1, \ldots, X_n]$ are also rings,
  \item if $M$ is an Abelian group, $\End(M)$ is (usually) a non-commutative
	ring,
  \item if $K$ is a field and $V$ is a $K$-vector space of finite dimension
	$d>1$, then $\End(V) \cong M_d(K)$ is a non-commutative ring,
  \item if $X$ is a topological space, $\zvezne{X, \R}$ is a ring with pointwise
	operations,
  \item if $\{A_i\}_{i \in I}$ is a family of rings, then
	\[
	  \prod_{i \in I} A_i
	\]
	is a ring,
  \item if $A$ is a ring, the set of formal power series' $A\llbracket X
	\rrbracket$ consists of sequences $f: \N_0 \to A$ with operations $(f+g)(n)
	= f(n) + g(n)$ and
	\[
	  (fg)(n) = \sum_{k=0}^n f(k) g(n-k).
	\]
  \item We can also introduce other products for sequences:
	\begin{gather*}
	  (fg)(n) = f(n) g(n) \qquad \text{(pointwise product)} \\
	  (f * g)(n) = \sum_{d | n} f(d) g(n / d) \qquad \text{(Dirichlet's
		convolution)}
	\end{gather*}
  \item if $\{A_i\}_{i \in I}$ is a family of subrings of $A$, then their
	intersection is a subring of $A$.
  \end{itemize}
\end{example}

\begin{definition}
  Let $A$ be a ring and $a \in A$.
  Then
  \begin{itemize}
  \item $a$ is \pojem{invertible} (or a \pojem{unit}) if there exists
	$b \in A$ such that $ab = 1$,
  \item $a$ is a \pojem{zero divisor} if there exists an element $b \in A \setminus
	\{0\}$ such that $ab = 0$,
  \item $a$ is \pojem{nilpotent} if $a^n = 0$ for some $n$.
  \end{itemize}
\end{definition}

\begin{definition}
  We label the group of units of $A$ by $A^*$ or $A^x$, and we label the set of
  non-zero-divisors by $A^\cdot$.
\end{definition}

\begin{definition}
  A ring $A$ is an \pojem{(integral) domain} if $A \ne \underline{0}$ and $ab =
  0$ implies $a = 0$ or $b = 0$ for any $a, b \in A$.
\end{definition}

\begin{remark}
  This is equivalent to the statement that $0$ is the only zero-divisor of $A$.
\end{remark}

\begin{example}
  Let $f = \sum f(n) x^n \in A\llbracket x \rrbracket$.
  Then $f$ is invertible if and only if $f(0) \in A^x$.
\end{example}

\begin{definition}
  If $A \le B$ are rings and $S \subseteq B$ is a subset, then
  \[
	A[S] = \bigcap_{A \le A' \le B, S \subseteq A'} A'
  \]
  is the subring of $B$ obtained by adjoining $S$ to $A$.
\end{definition}

\begin{remark}
  Note that
  \[
	A[S] = \left\{ \sum_{i=1}^m a_i s_i \such a_i \in A, s_i \in S, m \in \N
	\right\}.
  \]
\end{remark}

\begin{definition}
  A subset $I \subseteq A$ is an \pojem{ideal} if $I \ne \varnothing$ and $I$
  is closed under addition and (left) multiplication with elements of $A$.
  We denote $I \trianglelefteq A$.
\end{definition}

\begin{remark}
  Given an ideal $I$ of $A$, the quotient $A/I = \{ a + I \such a \in A \}$ is a
  ring.
  We sometimes also label $a + I = \bar{a} = [a] = [a]_I$.
\end{remark}

\begin{definition}
  If $X \subseteq A$, then $(X)$ (or $\sk{X}$) is the ideal generated by $X$, so
  the smallest ideal of $A$ which includes $X$.
\end{definition}

\begin{definition}
  The \pojem{principal ideals} are those generated by a single element, $(a) =
  Aa$.
\end{definition}

\begin{example}
  The ideals of $\Z$ are $\{n \Z \such n \in \N_0 \}$.
\end{example}

\begin{example}
  The ideal $(x,y) \trianglelefteq K[x,y]$ is not principal.
\end{example}

\begin{remark}
  If $I \trianglelefteq A$, then $A = I$ if and only if $1 \in I$.
  This is also equivalent to $I \cap A^x \ne \varnothing$.
\end{remark}

\begin{definition}
  An ideal $I \trianglelefteq A$ is
  \begin{itemize}
  \item a \pojem{prime ideal} if $I$ is proper and for any $a, b \in A$, if $ab
	\in I$, then $a \in I$ or $b \in I$,
  \item a \pojem{maximal ideal} if $I$ is proper and if for all $J
	\trianglelefteq A$, $I \subseteq J$ implies $J = A$.
  \end{itemize}
  The spectrum $\Spec(A)$ is the set of all prime ideals.
  The set of all maximal ideals is denoted by $\Max(A)$.
\end{definition}

\begin{theorem}
  An ideal $I$ of $A$ is prime if and only if $A/I$ is a domain.
  It is maximal if and only if $A/I$ is a field.
\end{theorem}

\begin{corollary}
  All maximal ideals are prime.
\end{corollary}

\begin{remark}
  Every ideal is contained in a maximal ideal (if Zorn's lemma is true).
\end{remark}

\begin{example}
  The prime ideals of $\Z$ are $\{ p \Z \such \text{$p$ prime} \} \cup \{ \{0\}
  \}$.
  Each of $p\Z$ is also maximal, as $\Z / p \Z = \Z_p$ is a field.
\end{example}

\begin{example}
  In $K[x,y]$, we have $\underline{0} \subsetneq (x) \subsetneq (x,y)$.
  Each of these are prime ideals, and $(x,y)$ is maximal.
\end{example}

\begin{remark}
  The trivial ideal $\underline{0}$ is prime if ad only if $A$ is a domain.
\end{remark}

\begin{definition}
  If $I, J \trianglelefteq A$, then the following are also ideals:
  \begin{itemize}
  \item $I + J = \{ a + b \such a \in I, b \in J \} = (I,J)$,
  \item $I \cap J$,
  \item $I \cdot J = (\{ ab \such a \in I, b \in J \})$ (ideal generated by the
	products),
  \item radial ideal: $\sqrt{I} = \{ a \in A \such \exists n \ge 1 . a^n \in I
	\}$.
  \end{itemize}
\end{definition}

\begin{remark}
  The non-trivial step in proving that the radical ideal really is an ideal is
  that it is closed under addition.
  Let $a, b \in \sqrt{I}$.
  Then let $n$ be large enough that $a^n, b^n \in I$.
  Compute
  \[
	(a+b)^{2n} = \sum_{i=0}^{2n} a^i b^{2n-i} \binom{2n}{i}.
  \]
  If $i \ge n$, then $a^i \in I$, but if $i \le n$, then $b^{2n-i} \in I$.
  So in every case, the terms of the sum are in $I$, which means the entire sum
  is in $I$.
\end{remark}

\begin{remark}
  Note that $\sqrt{I}$ is not necessarily an ideal in a non-commutative ring.
\end{remark}

\begin{definition}
  The \pojem{nilradical} of $A$ is $N(A) = \sqrt{\underline{0}}$ (the set of all
  nilpotents).
  The \pojem{Jacobson radical} $J(A)$ is the intersection of all maximal ideals.
\end{definition}

\begin{remark}
  If $A \ne \underline{0}$, then $N(A), J(A)$ are proper ideals.
\end{remark}

\begin{lemma}
  For any ring $A$, $N(A) \subseteq J(A)$.
  Also, $J(A) = \{ a \in A \such \forall b \in A . 1 - ba \in A^x \}$.
\end{lemma}

\begin{proof}
  For the first point, take $a \in N(A)$, so $a^n = 0$ for some $n \ge 1$.
  If $M \in \Max(A)$, then $a^n \in M$, as $0$ is in every ideal.
  Since $M$ is prime, this implies $a \in M$.

  For the second point, let $a \in J(A)$ and $b \in A$.
  For any $M \in \Max(A)$, $1 - ab \notin M$, as since $a \in M$, also $ab \in
  M$, but then if $1 - ab \in M$, this implies $1 \in M$.
  So $1 - ab$ cannot be in any maximal ideal, which means $(1-ab) = A$.
  But then $1 - ab \in A^x$, as also $1 \in (1-ab) A = A$.

  For the other inclusion, let $a \in A$ be such that for every $b \in A$, we
  have $1 - ab \in A^x$.
  Let $M \in \Max(A)$ and suppose $a \notin M$.
  Then $(M, a) = A$, so $1 = m + xa$ for some $m \in M, x \in A$.
  This means $m = 1 - xa \in M$, but this element is invertible by assumption,
  so $M = A$.
  \protislovje{}
\end{proof}

\begin{lemma}[prime avoidance]
  Let $I$ be an ideal of $A$ and let $P_1, \ldots, P_n \in \Spec(A)$.
  If $I \subseteq P_1 \cup \ldots \cup P_n$, then there exists some $k$ such
  that $I \subseteq P_k$.
\end{lemma}

\begin{proof}
  Induction on $n$.
  If $n = 1$, there is nothing to do.
  Let $n > 1$.
  Suppose that for any $i$, there is an element $a_i \in I \setminus \bigcup_{j
	\ne i} P_j$, so $a_i \in P_i$.
  Consider
  \[
	a = \sum_{j=1}^n a_1 \ldots \hat{a}_j \ldots a_n
	= \underbrace{a_1 \ldots \hat{a}_i \ldots a_n}_{\notin P_i} +
	\underbrace{\sum_{j \ne i} a_1 \ldots \hat{a}_j \ldots a_n}_{\in P_i}
  \]
  for some arbitrary $i$, because otherwise there would exist some $j \ne i$
  such that $a_j \in P_i$ by primality of $P_i$, contradicting the choice of
  $a_j$.

  So $a \notin P_i$ for all $i$.
  This is a contradiction, as $a \in I \subseteq P_1 \cup \ldots \cup P_n$.
\end{proof}

\begin{lemma}
  Let $I_1, \ldots, I_n$ be ideals of $A$ and let $P \in \Spec(A)$.
  If $I_1 \cap \ldots \cap I_n \subseteq P$, then there exists some $k$ such
  that $I_k \subseteq P$.
\end{lemma}

\begin{proof}
  Suppose that for every $j$, $I_j \nsubseteq P$, so there exist $a_j \in I_j
  \setminus P$.
  Then $a_1 \ldots a_n \in I_1 \ldots I_n \subseteq I_1 \cap \ldots \cap I_n
  \subseteq P$.
  Since $P$ is prime, this implies there exists a $j$ such that $a_j \in P$,
  which is a contradiction.
\end{proof}

\begin{remark}
  If $f: A \to B$ is a ring homomorphism, then $\jedro f \trianglelefteq A$.
  More generally, if $I \trianglelefteq B$, then $f^{-1}(I) \trianglelefteq A$.
  Also, if $P \in \Spec(B)$, then $f^{-1}(P) \in \Spec(A)$.
\end{remark}

\begin{proposition}[universal property for quotients]
  Let $I \trianglelefteq A$ and $\pi : A \to A/I$ be the canonical epimorphism.
  For every ring homomorphism $f: A \to B$ with $I \subseteq \jedro f$, there
  exists a unique ring homomorphism $\hat{f}: A/I \to B$ such that $f = \hat{f}
  \circ \pi$.
\end{proposition}

\begin{corollary}[first isomorphism theorem]
  If $f: A \to B$ is a ring homomorphism, then $A / \jedro f \cong f(A)$.
\end{corollary}

\begin{theorem}[isomorphism theorems]
  Let $I \trianglelefteq A$.
  Then
  \begin{itemize}
  \item $\{ J \trianglelefteq A \such I \subseteq J \subseteq A \}$ is in a
	bijective correspondence with the ideals of $A/I$, with the map
	\[
	  J \mapsto J/I = \{ a + I \such a \in J \},
	\]
  \item if $J \trianglelefteq A$ is such that $I \subseteq J \subseteq A$,
	\[
	  A/J \cong \frac{A/I}{J/I},
	\]
  \item if $B \subseteq A$ is a subring, then $B+I$ is a subring of $A$, $B \cap
	I \trianglelefteq B$, and
	\[
	  \frac{B+I}{I} \cong \frac{B}{B \cap I}.
	\]
  \end{itemize}
\end{theorem}

\begin{theorem}[Chinese remainder theorem]
  If $I_1, \ldots, I_n \trianglelefteq A$ are pairwise comaximal (so $I_i + I_j
  = A$ for any $i, j$), then
  \[
	A / I_1 \cap \ldots \cap I_n \cong A/I_1 \times \cdots \times A / I_n
  \]
  and $I_1 \cap \ldots \cap I_n = I_1 \cdot \cdots \cdot I_n$.
\end{theorem}

\begin{example}
  $\Z / p^n q^m \Z \cong \Z/ p^n \Z \times \Z/q^m \Z$ for primes $p, q$.
\end{example}

\podnaslov{Modules}

\begin{definition}
  Let $A$ be a ring. A \pojem{module} $M$ is an additive Abelian group with a
  scalar multiplication operation $\cdot: A \times M \to M$, such that
  \begin{itemize}
  \item $1 m = m$,
  \item $(ab)m = a(bm)$,
  \item $(a+b)m = am + bm$,
  \item $a(m+n) = am + an$.
  \end{itemize}
\end{definition}

If $M$ is an Abelian group, an $A$-module structure can equivalently be
described by a ring homomorphism $\varepsilon: A \to \End(M)$, called the
\pojem{structure homomorphism}.
Given a module, we can define
\[
  \varepsilon(a)(m) = am,
\]
and in the other direction, the scalar product is given by $am :=
\varepsilon(a)(m)$.

\begin{example}
  The $\Z$-modules are precisely the Abelian groups.
\end{example}

\begin{example}
  If $K$ is a field, then $K$-modules are precisely the $K$-vector spaces.
\end{example}

\begin{remark}
  The submodules of $A$ are precisely the ideals of $A$.
\end{remark}

\begin{example}
  If $V$ is a $K$-vector space, and $\varphi \in \End_K(V)$, then we can
  construct a ring homomorphism $\phi: K[x] \to \End_K(V)$ by
  \[
	\sum_{i=0}^n a_i x^i \mapsto \sum_{i=0}^n a_i \varphi^i.
  \]
  This induces a $K[x]$-module structure on $V$, with $x$ acting as $\varphi$,
  $x \cdot v = \varphi(v)$.
  Since $K[x]$ is a principal ideal domain, $\jedro \phi = (m_\varphi)$ for some
  $m_\varphi$, which turns out to be the minimal polynomial of $\varphi$.
\end{example}

Let $M$ be an $A$-module and $E \subseteq M$.
Then we define
\[
  \sk{E}_A = \left\{
	\sum_{i=1}^n a_i m_i \such a_i \in A, m_i \in E
  \right\}
\]
as the $A$-module generated by $E$ (it is of course the smallest submodule that
includes $E$).

If $\{N_i\}_{i\in I}$ is a family of submodules, then the intersection of all
$N_i$ is a submodule, as is
\[
  \sum_{i \in I} M_i = \sk{\bigcup_{i \in I} M_i}.
\]

If $I \trianglelefteq A$, then
\[
  IM = \left\{
	\sum_{i=1}^n a_i m_i \such a_i \in I, m_i \in M
  \right\}
\]
is also a submodule of $M$.

\begin{remark}
  Let $\varphi: A \to B$ be a ring homomorphism.
  If $N$ is a $B$-module, then it is also an $A$-module via $an := \varphi(a)
  n$.
  Similarly, if $\varphi$ is an epimorphism and $M$ is an $A$-module such that
  $IM = 0$ for $I = \jedro \varphi$, then $M$ is also a $B$-module via $bm :=
  am$ for any $a \in \varphi^{-1}(b)$.
\end{remark}

\begin{remark}
  In particular, if $M$ is an $A$-module and $I \trianglelefteq A$, then $M /
  IM$ is an $A/I$-module via $(a+I)(m + IM) = am + IM$.
\end{remark}

\begin{remark}
  This construction gives a category equivalence between the category of
  $A$-modules $M$ with $IM = 0$ and the category of $A/I$-modules.
\end{remark}

\begin{example}
  If $M$ is an Abelian group (so $\Z$-module) and $p$ is prime, then $M/pM$ is a
  $\Z/p\Z$-vector space.
\end{example}

\begin{theorem}[Isomorphism theorems]
  Let $M$ be a module and $N, N'$ submodules.
  \begin{itemize}
  \item $\im f \cong M / \jedro f$ for any module homomorphism $f$,
  \item there is a bijection between $\{ X \such N \le X \le M \}$ and the
	submodules of $M/N$,
  \item if $N \le X \le M$, then
	\[
	  M/N \cong \frac{M/N}{X/N},
	\]
  \item $(N + N')/N \cong N' / (N \cap N')$.
  \end{itemize}
\end{theorem}

Let $\{M_i\}_{i \in I}$ be a family of $A$-modules.
Then the product is defined as
\[
  \prod_{i \in I} M_i = \{ \{m_i\}_{i \in I} \such m_i \in M_i \}
\]
and the direct sum (coproduct) as
\[
  \bigoplus_{i \in I} M_i = \{ \{m_i\}_{i \in I} \such m_i \in M_i, \text{$m_i
	\ne 0$ for finitely many $i$} \}.
\]
These have the usual (universal) properties.

% LocalWords:  nilradical nilpotents primality
