\naslov{Localisations}

Given a ring $A$, we wish to adjoin some inverses (fractions), but not all of
them.

\begin{definition}
  A subset $S \subseteq A$ is a \pojem{multiplicatively closed set} (or
  \pojem{mc set}) if $1 \in S$ and for any $s, s' \in S$, their product $s s'
  \in S$.
\end{definition}

\begin{example}
  For $a \in A$, $S = \{1, a, a^2, \ldots\}$ is an mc set.
\end{example}

\begin{example}
  The set $S = A \setminus P$, where $P \in \Spec(A)$ is multiplicatively
  closed.
\end{example}

\begin{example}
  The set of all non-zero divisors of $A$ is multiplicatively closed.
\end{example}

\begin{example}
  If $I \ideal A$ is an ideal, then $S = 1 + I$ is multiplicatively
  closed.
\end{example}

\begin{definition}
  Given an mc set $S \subseteq A$, the \pojem{localisation} $S^{-1} A$ of $S$ is
  defined as follows:
  \begin{itemize}
  \item As a set, $S^{-1} A = A \times S / \sim$, where
	\[
	  (a, s) \sim (b, t) :\iff \exists u \in S . atu = bsu.
	\]
  \item We introduce the notation
	\[
	  \frac{a}{s} = [(a,s)]_\sim.
	\]
  \item We introduce the operations
	\[
	  \frac{a}{s} \cdot \frac{a'}{s'} = \frac{aa'}{ss'},
	  \qquad
	  \frac{a}{s} + \frac{a'}{s'} = \frac{as' + sa'}{ss'},
	\]
	which makes $S^{-1} A$ a ring with zero $\frac{0}{1}$ and one $\frac{1}{1}$.
  \end{itemize}
\end{definition}

We need to check that $\sim$ defined above is an equivalence relation.
It is clearly reflexive and symmetric.
If $(a,s) \sim (a', s')$ and $(a', s') \sim (a'', s'')$.
Then there exist $u, v \in S$ such that $as'u = a'su$ and $a's''v = a''s'v$, but
then
\[
  (a s'') s'uv = as'u s''v = a'su s''v = a'' s s' v u = (a'' s) s'uv,
\]
so $(a, s) \sim (a'', s'')$ and $\sim$ is an equivalence relation.

We introduce the ring homomorphism $j = j_S: A \to S^{-1} A$, defined with $a
\mapsto \frac{a}{1}$.
Note that $j(S) \subseteq (S^{-1} A)^x$.
Then we have the following universal property:
For any $\varphi: A \to B$ with $\varphi(S) \subseteq B^x$, there exists a
unique ring homomorphism $\bar{\varphi}: S^{-1} A \to B$ such that $\varphi =
\bar{\varphi} \circ j$.

\begin{proof}[Sketch of the proof]
  For uniqueness, note that
  \[
	\bar{\varphi}\left( \frac{a}{s} \right)
	= \bar{\varphi}\left( \frac{a}{1} \right) \varphi\left( \inv{s} \right).
  \]
  We can multiply both sides with $\bar{\varphi}(s)$ to find $\varphi(a) =
  \bar{\varphi}\left( \frac{a}{s} \right) \varphi(s)$.
  This also leads us to define $\bar{\varphi}(\nicefrac{a}{s}) = \varphi(a)
  \varphi(s)^{-1}$.
\end{proof}

Note that $\varphi(a) \varphi(s)^{-1} = 0$ if and only if $\varphi(a) = 0$, so
\[
  \jedro \bar{\varphi} = \{ \frac{a}{s} \such a \in \jedro \varphi, s \in S \}.
\]
Similarly, $j(a) = 0$ if and only if $\frac{a}{1} = 0$, or if there exists an $s
\in S$ for which $as = 0$.
Then
\[
  \jedro j = \{ a \in A \such \exists s \in S \velja as = 0 \}
\]
can be nontrivial.

\begin{definition}
  If $S$ is the set of non-zero-divisors of $A$, then $S^{-1} A =
  \mathcal{F}(A)$ is the \pojem{total ring of fractions} of $A$.
  If $A$ is a domain, we also call it the \pojem{field of fractions} or
  \pojem{quotient field}.
\end{definition}

If $S$ does not contain zero-divisors, then $S^{-1} A \to \mathcal{F}(A)$ embeds
canonically by the universal property.

\begin{definition}
  If $S = A \setminus P$ for a prime ideal $P$, write $A_P = S^{-1} A$.
  If $S = \{1, a, a^2, \ldots\}$ for some $a \in A$, we write $A_a = S^{-1} A$.
\end{definition}

\begin{example}
  If $p$ is a prime number, then
  \[
	\Z_{(p)} = \left\{ \frac{a}{b} \such a \in \Z, p \nshortmid b \right\}
  \]
  but
  \[
	\Z_p = \left\{
	  \frac{a}{p^k} \in \Q \such a \in \Z, k \ge 0
	\right\}
  \]
\end{example}

Recall:
An $A$-algebra is a ring $B$ that is simultaneously an $A$-module (with the same
addition), for which there is a ring homomorphism $A \to B$.

\begin{example}
  The polynomial ring $A[X]$ is an $A$-algebra.
\end{example}

\begin{example}
  Rings are precisely $\Z$-algebras.
\end{example}

\begin{example}
  If $I \ideal A$, then $A \to A/I$ gives an $A$-algebra.
\end{example}

\begin{example}
  If $S \subseteq A$ is an mc set, then $j: A \to S^{-1} A$ gives an
  $A$-algebra.
\end{example}

\begin{remark}
  We say that the homomorphism \emph{is} an algebra, even though is only gives
  us an algebra.
\end{remark}

If $f: A \to B$ is an $A$-algebra, we can
\begin{itemize}
\item extend ideals: $I \ideal A \mapsto f(I) B = \sk{f(I)}_B$,
\item contract ideals: $J \ideal B \mapsto f(J)^{-1} \ideal A$.
\end{itemize}

For the localisation $j: A \to S^{-1} A$ and an ideal $I \ideal A$,
we write
\[
  I \cdot S^{-1} A = j(I) S^{-1} A = \left\{ \frac{a}{s} \such s \in S, a \in I
  \right\}.
\]

\begin{definition}
  Let $I \ideal A$ and $X \subseteq A$, define the \pojem{colon ideal}
  \[
	(I:X) = \{ a \in A \such a X \subseteq I \}.
  \]
\end{definition}

\begin{remark}
  It is easy to check that this is an ideal.
\end{remark}

\begin{proposition}
  Let $S \subseteq A$ be an mc set and $j: A \to S^{-1} A$ the localisation.
  Then

  \begin{itemize}
  \item For all $J \ideal S^{-1} A$, we have $J = j(j^{-1}(J)) S^{-1}A$.
	In particular, every ideal of $S^{-1} A$ is an extension of an ideal of $A$.
  \item For any $I \ideal A$,
	\[
	  j^{-1}(j(I) S^{-1} A) = j^{-1}(I S^{-1} A) = \bigcup_{s \in S} (I:s)
	  \supseteq I.
	\]
  \item $I \cdot S^{-1} A = S^{-1} A$ if and only if $I \cap S^{-1} A \ne
	\varnothing$.
  \end{itemize}
\end{proposition}

\begin{proof}
  For the first point, the inclusion $\supseteq$ is clear.
  The second inclusion:
  If $\nicefrac{a}{s} \in J$, then $j(a) = \frac{a}{1} \in J$, meaning $a \in
  j^{-1}(J)$, so $\frac{a}{1} \in j(j^{-1}(J))$ and $\frac{a}{s} \in
  j(j^{-1}(J)) S^{-1}A$.

  For the second point, consider
  \begin{align*}
	a \in j^{-1}(I \cdot S^{-1} A)
	&\iff \exists x \in I, s \in S \velja \frac{a}{1} = \frac{x}{s} \\
	&\iff \exists x \in I, s, t \in S \velja ast = xt \\
	&\iff \exists s' \in S \velja as' \in I \\
	&\iff \exists s' \in S \velja a \in (I : s')
  \end{align*}
  and the inclusion holds since $1 \in S$ and $(I:1) = I$.

  For the third point, the right-to-left implication holds because $j(S)
  \subseteq (S^{-1} A)^x$, and for the left-to-right implication, $\frac{1}{1} =
  \frac{x}{s}$ with $x \in I$, $s \in S$, so there exists $t \in S$ such that
  $st = xt \in S \cap I$.
\end{proof}

\begin{corollary}
  If $P \in \Spec(A)$, then $A_P$ is a local ring with unique maximal ideal $P
  A_P$.
\end{corollary}

\begin{proof}
  By the first and third point of the proposition, with $S = A \setminus P$.
\end{proof}

\begin{example}
  Let $p \in \Prime$.
  Then $\Z_{(p)}$ has a maximal ideal
  \[
	\left\{ \frac{a}{b} \in \Q \such p \nshortmid b, p \shortmid a \right\}
  \]
\end{example}

\begin{example}
  If $K$ is a field and $P =(x,y) \ideal K[x,y]$, then
  \[
	K[x,y]_{(x,y)} = \left\{ \frac{f}{g} \such f, g \in K[x,y], g(0,0) \ne 0
	\right\}
  \]
  is local, and has maximal ideal
  \[
	(x,y) K[x,y]_{(x,y)} = \left\{ \frac{f}{g} \such g(0,0) \ne 0, f(0,0) = 0
	\right\}.
  \]
\end{example}

\begin{example}
  Different ideals can localise to the same ideal.
  For example, $2 \Z_{(2)} = 6 \Z_{(2)}$, because $3 \in \Z_{(2)}^x$.
\end{example}

\begin{corollary}
  There is a bijection
  \[
	\{ P \in \Spec(A) \such P \cap S = \varnothing \}
	\to \Spec(S^{-1}A),
  \]
  given with $P \mapsto P \cdot S^{-1} A$.
\end{corollary}

\begin{proof}
  The inverse of this map is $Q \mapsto j^{-1}(Q)$.
  We need to check that this inverse actually maps into $\Spec A$, which is true
  since the preimages of prime ideals under ring homomorphisms are prime,
  and by the previous proposition, $j^{-1}(Q) S^{-1} A = Q$ and $j^{-1}(Q) \cap
  S = \varnothing$.
  For the other direction, let $P \in \Spec A$ be such that $P \cap S =
  \varnothing$.
  Then $P S^{-1} A$ is prime:
  It is equal to
  \[
	P S^{-1} A = \left\{ \frac{x}{s} \such x \in P, s \in S \right\}.
  \]
  Suppose that $\frac{a}{s} \cdot \frac{b}{t} \in P S^{-1} A$, so $\frac{ab}{st}
  = \frac{x}{s'}$ with $x \in P$, $s' \in S$.
  Then there exists $u \in S$ such that $abs'u = xstu \in P$, but since $s'u
  \notin P$, we have $ab \in P$, so $a \in P$ or $b \in P$.
  Therefore $\frac{a}{s} \in P S^{-1} A$ or $\frac{b}{t} \in PS^{-1} A$.

  Now,
  \[
	j^{-1} (P S^{-1} A) = \bigcup_{s \in S} (P:s) = P
  \]
  since $(P:s) = P$ for any $s$, as $s \notin P$.
\end{proof}

\begin{corollary}
  The following statements hold:
  \begin{itemize}
  \item the nilradical is equal to
	\[
	  N(A) = \bigcap_{P \in \Spec(A)} P,
	\]
  \item if $I \ideal A$, then
	\[
	  \sqrt{I} = \bigcap_{P \in \Spec(A), I \subseteq P} P.
	\]
  \end{itemize}
\end{corollary}

\begin{proof}
  For the first point, the $\subseteq$ inclusion holds since for $P \in
  \Spec(A)$ and $a \in N(A)$, so there exists $n \in \N$ such that $a^n = 0 \in
  P$ and $a \in P$.

  The other inclusion:
  Let $a$ be in the intersection.
  Then $A_a$ is a ring with $\Spec(A) = \varnothing$, meaning $A_a =
  \underline{0}$ (since otherwise, we have maximal ideals), so $1 \in \jedro (j:
  A \to A_a)$, meaning there exists $n \in \N_0$ for which $a^n 1 = 0$, so $a
  \in N(A)$.

  For the second point, apply the first to $A/I$, noting that $N(A/I) =
  \sqrt{I}/I$.
\end{proof}

\podnaslov{Localisation of modules}

Let $M$ be an $A$-module and $S \subseteq A$ an mc set.
Define
\[
  (m, s) \sim (m', s') \iff \exists t \in S \velja ms't = m'st.
\]
This is an equivalence relation on $M \times S$, which induces the set
\[
  S^{-1} M = \left\{ \frac{m}{s} := [(m,s)]_\sim \such m \in M, s \in S \right\}.
\]
This is an $S^{-1}A$-module via
\[
  \frac{a}{s} \frac{m}{s'} = \frac{am}{ss'}.
\]

As before, we have an $A$-homomorphism $j: M \to S^{-1} M$, defined with $m
\mapsto \frac{m}{1}$.
It has kernel $\jedro j = \{ m \in M \such \exists s \in S \velja sm = 0 \}$.

If $f: M \to N$ is an $A$-homomorphism, then $S^{-1} f: S^{-1}M \to S^{-1}N$ is
an $S^{-1}A$-homomorphism for $\frac{m}{s} \mapsto \frac{f(m)}{s}$.
This means that localisation is a functor from the category of $A$-modules to
the category of $S^{-1}A$-modules.

\begin{proposition}
  Localisation is an exact functor.
  If $M \xrightarrow{f} N \xrightarrow{g} P$ is exact, then so is $S^{-1} M
  \xrightarrow{S^{-1}f} S^{-1}N \xrightarrow{S^{-1}g} S^{-1} P$.
\end{proposition}

\begin{proof}
  Since $g \circ f = 0$, we have $0 = S^{-1}(g \circ f) = S^{-1} g \circ S^{-1}
  f$, so $\im S^{-1} f \subseteq \jedro S^{-1} g$.
  Let $\frac{n}{s} \in \jedro S^{-1} g$.
  Then $\frac{g(n)}{s} = 0$ in $S^{-1} P$, so there exists $t \in S$ for which
  $0 = g(n) t = g(nt)$, meaning $nt \in \jedro g = \im f$.
  So we have $nt = f(m)$ for some $m \in M$.
  Then
  \[
	\frac{n}{s} = \frac{nt}{st} = \frac{f(m)}{st} = S^{-1} f (\frac{m}{st}).
  \]
  This shows the exactness of the sequence.
\end{proof}

In particular, if $M \le N$, then without loss of generality, $S^{-1} M \le
S^{-1} N$ and we can think of it as
\[
  S^{-1} M = \left\{ \frac{m}{s} \in S^{-1} N \such m \in M \right\}.
\]
Also, if $I \ideal A$, then $S^{-1} I = I S^{-1} A \ideal S^{-1} A$.

\begin{corollary}
  If $N$ is an $A$-module and $M, M' \le N$, then
  \begin{itemize}
  \item $S^{-1} (M + M') = S^{-1} M + S^{-1} M'$,
  \item $S^{-1}(M \cap M') = S^{-1} M \cap S^{-1} M'$,
  \item $S^{-1}(M/N) \cong S^{-1} N / S^{-1} M$ as $S^{-1}A$-modules.
  \end{itemize}
\end{corollary}

\begin{proof}
  The first point is trivial.
  For the second point, $\subseteq$ is easy, and for the other inclusion:
  Let $\frac{m}{s} = \frac{m'}{t}$ with $s, t \in S$, $m \in M$ and $m' \in M'$.
  There exists $u \in S$ for which $mtu = m'su \in M \cap M'$, so $\frac{m}{s} =
  \frac{mtu}{stu} qin S^{-1}(M \cap M')$.

  For the third point, the exactness of $0 \to M \to N \to N/M \to 0$ induces
  the short exact sequence $0 \to S^{-1} M \to S^{-1} N \to S^{-1}(N/M) \to 0$,
  meaning $S^{-1} (N/M) \cong S^{-1}N / S^{-1}M$.
\end{proof}

\begin{remark}
  If $I \ideal A$, then $S^{-1} \sqrt{I} = \sqrt{S^{-1} I} \ideal S^{-1} A$ and
  in particular $S^{-1} N(A) = N(S^{-1} A)$.
\end{remark}

\begin{lemma}
  Let $I \ideal A$ and $\pi: A \to A/I$ the canonical epimorphism.
  Let $S \subseteq A$ be an mc set.
  Then $T = \pi(S)$ is an mc set and $S^{-1} M \cong T^{-1} M$ canonically.
\end{lemma}

\begin{proof}
  Define $\varphi: S^{-1} M \to T^{-1} M$ mapping $\frac{m}{s} \mapsto
  \frac{m}{\pi(s)}$.
  We can easily show that this is an $A$-module epimorphism.
  It is also injective:
  Suppose $\frac{m}{\pi(s)} = \frac{0}{1} \in T^{-1} M$, so there exists $t \in
  T$ such that $mt = 0$, and $s' \in S$ for which $t = \pi(s')$.
  This means $0 = m \pi(s') = m(s' + I) = ms' + MI$, so $ms' \in MI = 0$.
  So $\frac{m}{s} = 0$.
\end{proof}

% LocalWords:  multiplicatively mc
