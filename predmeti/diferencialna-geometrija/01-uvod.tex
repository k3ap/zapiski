\naslov{Osnovni pojmi}

Naj bo $X$ topološka mnogoterost ($2$-števen lokalno evklidski Hausdorffov
prostor).

\begin{definicija}
  \pojem{Gladek atlas} na $X$ je družina $U = \{ (U_\alpha, \varphi_\alpha)
  \such \alpha \in A \}$ kjer je $A$ neka indeksna množica,
  $\varphi_\alpha : U_\alpha \to V_\alpha \subseteq \R^n$ homomorfizmi,
  $\bigcup_\alpha U_\alpha = X$ in kjer je za poljubna $\alpha, \beta$ z
  $U_\alpha \cap U_\beta \ne \varnothing$ preslikava $\varphi_\beta \circ
  \varphi_\alpha^{-1}$ difeomorfizem $\varphi_\alpha(U_\alpha \cap U_\beta) \to
  \varphi_\beta(U_\alpha \cap U_\beta)$.
\end{definicija}

\begin{definicija}
  Naj bosta $U = \{(U_\alpha, \varphi_\alpha)\}_{\alpha \in A}$ ter $V =
  \{(V_\beta, \psi_\beta)\}_{\beta \in B}$ dva atlasa na $X$. Atlasa sta
  \pojem{ekvivalentna}, če je $\psi_\beta \circ \varphi_\alpha^{-1}$
  difeomorfizem $\varphi_\alpha(U_\alpha \cap V_\beta) \to \psi_\beta(U_\alpha
  \cap V_\beta)$ za vsak par $\alpha, \beta$, za katera je $U_\alpha \cap
  V_\beta$.
\end{definicija}

\begin{definicija}
  Naj bo $U$ atlas na $X$.
  Ekvivalenčni razred $[U]$ glede na relacijo ekvivalentnosti atlasov se imenuje
  \pojem{gladka struktura} na $X$.
  Mnogoterost, opremljena z gladko strukturo, je \pojem{gladka mnogoterost}.
\end{definicija}

\begin{opomba}
  Obstajajo topološke mnogoterosti, na katerih je več neekvivalentnih gladkih
  struktur.
\end{opomba}

Naj bo sedaj $X \subseteq \R^N$ (gladka) mnogoterost, vložena v $\R^N$, za
katero velja $\dim X = n < N$.
Dodatno naj bo $m \in X$ ter $\gamma: (-\varepsilon, \varepsilon) \to X$ gladka
krivulja z $\gamma(0) = m$.
Potem je
\[
  V = \left. \frac{d}{dt} \right|_{t=0} \gamma(t)
\]
eden od tangentnih vektorjev na $X$ v točki $m$.

\begin{definicija}
  \pojem{Tangentni prostor} $T_m X$ na $X$ v točki $m$ je množica
  \[
	T_m X = \left\{ \left. \frac{d}{dt} \right|_{t=0} \gamma(t) \such \gamma:
	  (-\varepsilon, \varepsilon) \to X, \gamma(0) = m \right\}.
  \]
  To je vektorski podprostor dimenzije $n$ v $\R^N$.
\end{definicija}

Naj bo sedaj $X$ ponovno abstraktna gladka mnogoterost.

\begin{definicija}
  Krivulja $\gamma: (a,b) \to X$ je \pojem{gladka}, če je vsak kompozitum
  $\varphi_\alpha \circ \gamma$ gladka preslikava za vsak $\alpha$ (na primernem
  definicijskem območju).
\end{definicija}

\begin{definicija}
  Krivulji $\gamma_1, \gamma_2: (-\varepsilon, \varepsilon) \to X$, za kateri
  velja $\gamma_1(0) = \gamma_2(0) = m$ sta \pojem{ekvivalentni}, če velja
  \[
	\left. \frac{d}{dt} \varphi_\alpha \right|_{t=0} \varphi_\alpha(\gamma_1(t))
	= \left. \frac{d}{dt} \varphi_\alpha \right|_{t=0} \varphi_\alpha(\gamma_2(t))
  \]
  za poljuben $\alpha$, za katerega je $m \in U_\alpha$.
  Označimo $\gamma_1 \sim \gamma_2$.
\end{definicija}

\begin{definicija}
  \pojem{Tangentni vektor} na $X$ v točki $m$ je eden od ekvivalenčnih razredov
  za zgornjo relacijo.
  \pojem{Tangentni prostor} $T_m X$ je prostor vseh tangentnih vektorjev na $X$
  v točki $m$.
\end{definicija}

\begin{opomba}
  Izkaže se, da je ekvivalenčna relacija $\sim$ neodvisna od karte.
\end{opomba}

Predstavnik tangentnega vektorja $[\gamma]$ označimo z
\[
  \tangentni{\varphi_\alpha(\gamma)} = \left. \frac{d}{dt} \right|_{t=0}
  \varphi_\alpha(\gamma(t)).
\]
To je vektor v $\R^n$.
Poglejmo si, kaj se zgodi z $\tangentni{\varphi_\alpha(\gamma)}$, če zamenjamo karto.
Naj bo $(U_\beta, \varphi_\beta)$ še ena karta, ki vsebuje $m$.
Potem je
\[
  \tangentni{\varphi_\beta(\gamma)}
  = \left. \frac{d}{dt} \right|_{t=0} \varphi_\beta(\gamma(t))
  = \left. \frac{d}{dt} \right|_{t=0} (\varphi_\beta \varphi_\alpha^{-1}
  \varphi_\alpha)(\gamma(t))
  = D_{\varphi_\alpha(m)} (\varphi_\beta \varphi_\alpha^{-1}) \left.
	\frac{d}{dt} \right|_{t=0} \varphi_\alpha(\gamma(t))
\]
po verižnem pravilu za odvajanje, kar pa je enako
\[
  \tangentni{\varphi_\beta(\gamma)}
  = D_{\varphi_\alpha(m)} (\varphi_\beta \varphi_\alpha^{-1})
  \tangentni{\varphi_\alpha(\gamma)}
\]
Preslikava $D_{\varphi_\alpha(m)}$ je linearni izomorfizem $\R^n \to \R^n$.

V tangentnem prostoru so nam na voljo običajni operaciji vektorskih prostorov:
\begin{itemize}
\item Seštevanje: $[\gamma_1] + [\gamma_2] = [ t \mapsto
  \varphi_\alpha^{-1}(\varphi_\alpha(m) + t
  (\tangentni{\varphi_\alpha(\gamma_1)} + \tangentni{\varphi_\alpha(\gamma_2)}
  )) ]$.
\item Množenje s skalarjem: $a [\gamma] = [\gamma^a]$ za $\gamma^a(t) =
  \gamma(at)$.
\end{itemize}

Vsakemu tangentnemu vektorju lahko priredimo diferencialni operator prvega reda.
Naj bo $f: X \to \R$ gladka funkcija.
V smeri $[\gamma]$ ga odvajamo s smernim odvodom
\[
  (Vf)_{(m)} [\gamma] = \left. \frac{d}{dt} \right|_{t=0} (f \circ \gamma)(t).
\]
To je neodvisno od izbire predstavnika $\gamma$:
če sta $\gamma_1, \gamma_2 \in [\gamma]$, velja
\begin{align*}
  \left. \frac{d}{dt} \right|_{t=0} (f \circ \gamma_i)(t)
  &= \left. \frac{d}{dt} \right|_{t=0} (f \varphi_\alpha^{-1} \varphi_\alpha
  \gamma_i)(t) \\
  &= \grad(f \varphi_\alpha^{-1}) \left. \frac{d}{dt} \right|_{t=0}
  (\varphi_\alpha \gamma_i)(t) \\
  &= \grad(f \varphi_\alpha^{-1}) \tangentni{\varphi_\alpha(\gamma_i)}.
\end{align*}
Ker sta $\gamma_1$ in $\gamma_2$ ekvivalentna, je dobljeni izraz neodvisen od
izbire $i$.

\setmainlanguage{english}

Tangent vectors can be considered to be first order partial differential
operators.
Let $[\gamma] = V_m \in T_m X$ be a tangent vector.
The partial differential operator associated with $V$ is the directional
derivative; let $f: X \to \R$ be a smooth function.
We define
\[
  V_m(f) = \left. \frac{d}{dt} \right|_{t=0} f(\gamma(t)).
\]
In local coordinates given by a chart $(U_\alpha, \varphi_\alpha)$, we can
write
\[
  V_m(f) = \left. \frac{d}{dt} \right|_{t=0} f(\gamma(t))
  = \left. \frac{d}{dt} \right|_{t=0} f \circ \varphi_\alpha^{-1} \circ
  \varphi_\alpha \circ \gamma(t)
  = \sum_{i=1}^n \frac{\partial (f \varphi_\alpha^{-1})}{\partial x_i}
  v_i
\]
where $\frac{d}{dt}(\varphi_\alpha \gamma)(0) = (v_1, \ldots, v_n)$ and $x_1,
\ldots, x_n$ are the chosen coordinates in $\R^n$.
This is obviously a linear operator, and $V(fg) = V(f) g + f V(g)$ holds.

\podnaslov{Vector bundles}

Consider an ODE $\dot{\vec{x}} = V(\vec{x})$, where $V$ is a vector field on
some subset of $\R^n$.
We know that for a given initial value problem, we can find an integral curve
$\gamma$ which satisfies $\dot{\gamma} = V(\gamma)$.

Now let $X$ be a manifold.
A vector field on $X$ is defined as a map
\[
  V: X \to \bigsqcup_{m \in X} T_m X
\]
to the disjoint union of the tangent spaces on $X$.

\begin{definition}
  Let $X$ and $E$ be smooth manifolds such that there exists a projection map
  $\pi: E \to X$ with the following properties:
  \begin{itemize}
  \item For every $b \in X$, $\pi^{-1}(b) \subseteq E$ is linearly isomorphic to
	$\F^k$.
  \item For every $b \in X$, there exists an open set $b \in U \subseteq X$ and
	diffeomorphism $T_U: \pi^{-1}(U) \to U \times \F^k$ such that every
	restriction of $T_U$ to $\pi^{-1}(b)$ is a linear isomorphism $\pi^{-1}(b)
	\to b \times \F^k$.
	We call $T_U$ a \pojem{local trivialization}.
  \end{itemize}
\end{definition}

\begin{definition}
  A bundle $\pi: E \to X$ is \pojem{trivial} if it is diffeomorphic to $E = X
  \times V$.
\end{definition}

We introduce the following terminology:
\begin{itemize}
\item $\pi$ is the \pojem{bundle projection},
\item $V(\F^k)$ is the \pojem{fiber},
\item $X$ is the \pojem{base space},
\item $E$ is the \pojem{total space},
\item $T_U$ is the \pojem{trivialization}.
\end{itemize}

We can construct a smooth atlas on $E$.
Let $U = \{ (U_\alpha, \varphi_\alpha) \}_{\alpha \in A}$ be a smooth atlas on
$X$ such that the restricted bundle $E/U_\alpha$ is trivial.

% LocalWords:  neekvivalentnih
