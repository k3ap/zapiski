\naslov{Osnovni pojmi}

Naj bo $X$ topološka mnogoterost ($2$-števen lokalno evklidski Hausdorffov
prostor).

\begin{definicija}
  \pojem{Gladek atlas} na $X$ je družina $U = \{ (U_\alpha, \varphi_\alpha)
  \such \alpha \in A \}$ kjer je $A$ neka indeksna množica,
  $\varphi_\alpha : U_\alpha \to V_\alpha \subseteq \R^n$ homomorfizmi,
  $\bigcup_\alpha U_\alpha = X$ in kjer je za poljubna $\alpha, \beta$ z
  $U_\alpha \cap U_\beta \ne \varnothing$ preslikava $\varphi_\beta \circ
  \varphi_\alpha^{-1}$ difeomorfizem $\varphi_\alpha(U_\alpha \cap U_\beta) \to
  \varphi_\beta(U_\alpha \cap U_\beta)$.
\end{definicija}

\begin{definicija}
  Naj bosta $U = \{(U_\alpha, \varphi_\alpha)\}_{\alpha \in A}$ ter $V =
  \{(V_\beta, \psi_\beta)\}_{\beta \in B}$ dva atlasa na $X$. Atlasa sta
  \pojem{ekvivalentna}, če je $\psi_\beta \circ \varphi_\alpha^{-1}$
  difeomorfizem $\varphi_\alpha(U_\alpha \cap V_\beta) \to \psi_\beta(U_\alpha
  \cap V_\beta)$ za vsak par $\alpha, \beta$, za katera je $U_\alpha \cap
  V_\beta$.
\end{definicija}

\begin{definicija}
  Naj bo $U$ atlas na $X$.
  Ekvivalenčni razred $[U]$ glede na relacijo ekvivalentnosti atlasov se imenuje
  \pojem{gladka struktura} na $X$.
  Mnogoterost, opremljena z gladko strukturo, je \pojem{gladka mnogoterost}.
\end{definicija}

\begin{opomba}
  Obstajajo topološke mnogoterosti, na katerih je več neekvivalentnih gladkih
  struktur.
\end{opomba}

Naj bo sedaj $X \subseteq \R^N$ (gladka) mnogoterost, vložena v $\R^N$, za
katero velja $\dim X = n < N$.
Dodatno naj bo $m \in X$ ter $\gamma: (-\varepsilon, \varepsilon) \to X$ gladka
krivulja z $\gamma(0) = m$.
Potem je
\[
  V = \left. \frac{d}{dt} \right|_{t=0} \gamma(t)
\]
eden od tangentnih vektorjev na $X$ v točki $m$.

\begin{definicija}
  \pojem{Tangentni prostor} $T_m X$ na $X$ v točki $m$ je množica
  \[
	T_m X = \left\{ \left. \frac{d}{dt} \right|_{t=0} \gamma(t) \such \gamma:
	  (-\varepsilon, \varepsilon) \to X, \gamma(0) = m \right\}.
  \]
  To je vektorski podprostor dimenzije $n$ v $\R^N$.
\end{definicija}

Naj bo sedaj $X$ ponovno abstraktna gladka mnogoterost.

\begin{definicija}
  Krivulja $\gamma: (a,b) \to X$ je \pojem{gladka}, če je vsak kompozitum
  $\varphi_\alpha \circ \gamma$ gladka preslikava za vsak $\alpha$ (na primernem
  definicijskem območju).
\end{definicija}

\begin{definicija}
  Krivulji $\gamma_1, \gamma_2: (-\varepsilon, \varepsilon) \to X$, za kateri
  velja $\gamma_1(0) = \gamma_2(0) = m$ sta \pojem{ekvivalentni}, če velja
  \[
	\left. \frac{d}{dt} \varphi_\alpha \right|_{t=0} \varphi_\alpha(\gamma_1(t))
	= \left. \frac{d}{dt} \varphi_\alpha \right|_{t=0} \varphi_\alpha(\gamma_2(t))
  \]
  za poljuben $\alpha$, za katerega je $m \in U_\alpha$.
  Označimo $\gamma_1 \sim \gamma_2$.
\end{definicija}

\begin{definicija}
  \pojem{Tangentni vektor} na $X$ v točki $m$ je eden od ekvivalenčnih razredov
  za zgornjo relacijo.
  \pojem{Tangentni prostor} $T_m X$ je prostor vseh tangentnih vektorjev na $X$
  v točki $m$.
\end{definicija}

\begin{opomba}
  Izkaže se, da je ekvivalenčna relacija $\sim$ neodvisna od karte.
\end{opomba}

Predstavnik tangentnega vektorja $[\gamma]$ označimo z
\[
  \tangentni{\varphi_\alpha(\gamma)} = \left. \frac{d}{dt} \right|_{t=0}
  \varphi_\alpha(\gamma(t)).
\]
To je vektor v $\R^n$.
Poglejmo si, kaj se zgodi z $\tangentni{\varphi_\alpha(\gamma)}$, če zamenjamo karto.
Naj bo $(U_\beta, \varphi_\beta)$ še ena karta, ki vsebuje $m$.
Potem je
\[
  \tangentni{\varphi_\beta(\gamma)}
  = \left. \frac{d}{dt} \right|_{t=0} \varphi_\beta(\gamma(t))
  = \left. \frac{d}{dt} \right|_{t=0} (\varphi_\beta \varphi_\alpha^{-1}
  \varphi_\alpha)(\gamma(t))
  = D_{\varphi_\alpha(m)} (\varphi_\beta \varphi_\alpha^{-1}) \left.
	\frac{d}{dt} \right|_{t=0} \varphi_\alpha(\gamma(t))
\]
po verižnem pravilu za odvajanje, kar pa je enako
\[
  \tangentni{\varphi_\beta(\gamma)}
  = D_{\varphi_\alpha(m)} (\varphi_\beta \varphi_\alpha^{-1})
  \tangentni{\varphi_\alpha(\gamma)}
\]
Preslikava $D_{\varphi_\alpha(m)}$ je linearni izomorfizem $\R^n \to \R^n$.

V tangentnem prostoru so nam na voljo običajni operaciji vektorskih prostorov:
\begin{itemize}
\item Seštevanje: $[\gamma_1] + [\gamma_2] = [ t \mapsto
  \varphi_\alpha^{-1}(\varphi_\alpha(m) + t
  (\tangentni{\varphi_\alpha(\gamma_1)} + \tangentni{\varphi_\alpha(\gamma_2)}
  )) ]$.
\item Množenje s skalarjem: $a [\gamma] = [\gamma^a]$ za $\gamma^a(t) =
  \gamma(at)$.
\end{itemize}

Vsakemu tangentnemu vektorju lahko priredimo diferencialni operator prvega reda.
Naj bo $f: X \to \R$ gladka funkcija.
V smeri $[\gamma]$ ga odvajamo s smernim odvodom
\[
  (Vf)_{(m)} [\gamma] = \left. \frac{d}{dt} \right|_{t=0} (f \circ \gamma)(t).
\]
To je neodvisno od izbire predstavnika $\gamma$:
če sta $\gamma_1, \gamma_2 \in [\gamma]$, velja
\begin{align*}
  \left. \frac{d}{dt} \right|_{t=0} (f \circ \gamma_i)(t)
  &= \left. \frac{d}{dt} \right|_{t=0} (f \varphi_\alpha^{-1} \varphi_\alpha
  \gamma_i)(t) \\
  &= \grad(f \varphi_\alpha^{-1}) \left. \frac{d}{dt} \right|_{t=0}
  (\varphi_\alpha \gamma_i)(t) \\
  &= \grad(f \varphi_\alpha^{-1}) \tangentni{\varphi_\alpha(\gamma_i)}.
\end{align*}
Ker sta $\gamma_1$ in $\gamma_2$ ekvivalentna, je dobljeni izraz neodvisen od
izbire $i$.

Tangentne vektorje lahko obravnavamo kot parcialne diferencialne operatorje
prvega reda.
Naj bo $[\gamma] = V_m \in T_m X$ tangentni vektor.
PDO, ki ga pripišemo temu vektorju, je smerni odvod; če je $f: X \to \R$ gladka
funkcija, definiramo
\[
  V_m(f) = \left. \frac{d}{dt} \right|_{t=0} f(\gamma(t)).
\]
V lokalnih koordinatah, danih z atlasom $(U_\alpha, \varphi_\alpha)$, lahko zapišemo
\[
  V_m(f) = \left. \frac{d}{dt} \right|_{t=0} f(\gamma(t))
  = \left. \frac{d}{dt} \right|_{t=0} f \circ \varphi_\alpha^{-1} \circ
  \varphi_\alpha \circ \gamma(t)
  = \sum_{i=1}^n \frac{\partial (f \varphi_\alpha^{-1})}{\partial x_i}
  v_i
\]
kjer je $\frac{d}{dt}(\varphi_\alpha \gamma)(0) = (v_1, \ldots, v_n)$ in $x_1,
\ldots, x_n$ izbrane koordinate v $\R^n$.
To je očitno linearen operator, velja $V(fg) = V(f) g + f V(g)$.

\podnaslov{Vektorski svežnji}

Oglejmo si navadno diferencialno enačbo $\dot{\vec{x}} = V(\vec{x})$, kjer je
$V$ vektorsko polje na neki podmnožici $\R^n$.
Vemo:
Za dan Cauchyjev problem lahko najdemo krivuljo $\gamma$, za katero velja
$\dot{\gamma} = V(\gamma)$.
Naj bo sedaj $X$ gladka mnogoterost.
Vektorsko polje na $X$ je definirano kot preslikava
\[
  V: X \to \bigsqcup_{m \in X} T_m X,
\]
ki slika v disjunktno unijo tangentnih prostorov na $X$.

\begin{definicija}
  Naj bosta $X$ in $E$ taki gladki mnogoterosti, da obstaja projekcija
  $\pi: E \to X$ z naslednjimi lastnostmi:
  \begin{itemize}
  \item Za vsak $b \in X$ je $\pi^{-1}(b) \subseteq E$ linearno izomorfna
	$\F^k$.
  \item Za vsak $b \in X$ obstaja odprta okolica $b \in U \subseteq X$ in
	difeomorfizem $T_U: \pi^{-1}(U) \to U \times \F^k$, za katerega je vsaka
	skrčitev $T_U$ na $\pi^{-1}(b)$ linearen izomorfizem $\pi^{-1}(b)
	\to b \times \F^k$.
	Potem je $T_U$ \pojem{lokalna trivializacija}.
  \end{itemize}
\end{definicija}

\begin{definition}
  Sveženj $\pi: E \to X$ je \pojem{trivialen}, če je difeomorfen $E = X
  \times V$.
\end{definition}

Vpeljemo še naslednjo terminologijo:
\begin{itemize}
\item $\pi$ je \pojem{projekcija na sveženj},
\item $V(\F^k)$ je \pojem{vlakno},
\item $X$ je \pojem{osnovni prostor},
\item $E$ je \pojem{celoten prostor},
\item $T_U$ je \pojem{trivializacija}.
\end{itemize}

Na $E$ lahko konstruiramo gladek atlas.
Naj bo $U = \{ (U_\alpha, \varphi_\alpha) \}_{\alpha \in A}$ tak gladek atlas na
$X$, da je restringiran sveženj $E/U_\alpha = \pi^{-1}(U_\alpha)$ trivialen.
Za odprte množice v novem atlasu $\tilde{U}$ vzamemo množice
$\pi^{-1}(U_\alpha)$ za $\alpha \in A$, karte pa so dane s preslikavami
\begin{gather*}
  \tilde{\tau}_\alpha: \pi^{-1}(U_\alpha) \to \F^{n+k} \\
  \tilde{\tau}_\alpha(v)
  = ( \varphi_\alpha(\pi(v)), v_1, \ldots, v_k )
  = ( \varphi_\alpha(\pi(v)), \operatorname{pr}_2(\tau_\alpha(v)) ),
\end{gather*}
kjer je $\tau_\alpha$ difeomorfizem $\pi^{-1}(U_\alpha) \to U_\alpha \times
\F^k$, ki slika $v \mapsto (\pi(v), (v_1, \ldots, v_k))$.
Včasih pišemo tudi
\[
  \tau_\alpha(m) = (\pi(m), v_1(\pi(m)), \ldots, v_k(\pi(m)))
  = (b, v_1(b), \ldots, v_k(b)).
\]
Oglejmo si prehodno preslikavo
\[
  \tau_\beta \circ \tau_\alpha^{-1}
  : (b, v_1(b), \ldots, v_k(b)) \mapsto (b, w_1(b), \ldots, w_k(b)).
\]
Zapišemo jo lahko v obliki
\[
  \tau_\beta \circ \tau_\alpha^{-1}
  \left(
	b,
	\begin{bmatrix}
	  v_1(b) \\ \vdots \\ v_k(b)
	\end{bmatrix}
  \right)
  = \left(
	b,
	g_{\beta \alpha}(b)
	\begin{bmatrix}
	  v_1(b) \\ \vdots \\ v_k(b)
	\end{bmatrix}
  \right)
\]
oziroma $\tau_\beta \circ \tau_\alpha^{-1} = (\id, g_{\beta \alpha}(b))$.
Za vsak par $\alpha, \beta$, za katera je $U_\alpha \cap U_\beta \ne
\varnothing$, torej dobimo gladko preslikavo $g_{\beta \alpha} \in \GL(k, \F)$.
Dodano velja $g_{\gamma \beta} g_{\beta \alpha} = g_{\gamma \alpha}$.

\begin{definicija}
  \pojem{Kocikel}, prirejen atlasu $U = \{(U_\alpha, \varphi_\alpha)\}_\alpha$,
  z vrednostmi v $\GL(k, \F)$, je družina preslikav $g_{\beta \alpha}: U_\alpha
  \cap U_\beta \to \GL(k, \F)$, za katere za vsak primeren $b$ velja
  \begin{itemize}
  \item $g_{\alpha \gamma}(b) g_{\gamma \beta}(b) g_{\beta \alpha}(b) = I$,
  \item $g_{\alpha \beta}(b) = g_{\beta \alpha}(b)^{-1}$.
  \end{itemize}
\end{definicija}

\begin{opomba}
  Če imamo dan $X$ in kocikel, lahko do izomorfizma natančno konstruiramo
  pripadajoči sveženj.
\end{opomba}

\begin{definicija}
  Naj bo $\pi: E \to X$ sveženj.
  \pojem{Gladek lokalni prerez} svežnja $E$ nad $U_\alpha$ je gladka preslikava
  $s: U_\alpha \to E/U_\alpha$.
\end{definicija}

\begin{opomba}
  Ker je $E/U_\alpha = \pi^{-1}(b)$, je $\pi \circ s(b) = b$.
\end{opomba}

\begin{definicija}
  \pojem{Lokalno ogrodje} svežnja $\pi: E \to M$ nad $U_\alpha$ je $k$-terica
  prerezov $(s_1, \ldots, s_k): U_\alpha \to E/U_\alpha$, ki so med seboj
  linearno neodvisni, torej da so za vsak $b \in U_\alpha$ vektorji $s_1(b),
  \ldots, s_k(b)$ linearno neodvisni v $\pi^{-1}(b) \cong \F^k$.
\end{definicija}

\begin{definicija}
  \pojem{Tangentni sveženj} $TX$ gladke mnogoterosti $X$ je disjunktna unija
  \[
	TX = \bigsqcup_{m \in X} T_m X.
  \]
\end{definicija}

Opremimo $TX$ z gladko strukturo.
Naj bo $U = \{ (U_\alpha, \varphi_\alpha) \}_\alpha$ atlas za $X$.
Potem definiramo
\[
  T U_\alpha = \bigsqcup_{b \in U_\alpha} T_b X := TX / U_\alpha.
\]
Naj bo $U_\alpha$ kontraktibilna.
Vpeljemo lokalno trivializacijo $\tau_\alpha: TU_\alpha \to V_\alpha \times
\R^n$,
\[
  \tau_\alpha(v(b)) = (\varphi_\alpha(b), D_b \varphi_\alpha (v(b)))
  = (\varphi_\alpha(b), v_1(b), \ldots, v_n(b)).
\]
Torej imamo atlas $TU = \{ (TU_\alpha, \tau_\alpha) \}_\alpha$.
Za bazo lokalnih prerezov v točki $b_0 = \varphi_\alpha^{-1}(x_0)$ vzamemo
vektorje $[\gamma_i]_{b_0}$, pri katerih so reprezentativne krivulje koordinatne
premice $t \mapsto \varphi_\alpha^{-1}(x_1^0, \ldots, x_i^0  + t, \ldots,
x_n^0)$ za $x_0 = (x_1^0, \ldots, x_n^0)$.
Tak tangentni vektor označimo z $\frac{\partial}{\partial x_i^\alpha}(b)$.

\podnaslov{Frobeniusov izrek}

Naj bo $X$ gladka mnogoterost in $TX$ njen tangentni sveženj.
Naj bodo $V_1, \ldots, V_r$ lokalna gladka vektorska polja v $TX$ in $r < n =
\dim TX$.
Označimo z $D_V X$ lokalni vektorski podsveženj v $TX$.
Polja $V_1, \ldots, V_r$ naj bodo v vsaki točki, kjer so definirana, linearno
neodvisna.
Zanimalo nas bo, kdaj obstajajo integralske podmnogoterosti v $X$, ki
\enquote{pointegrirajo} $D_V X$.

\begin{definicija}
  Podmnogoterost $N \subseteq X$ dimenzije $r$ je \pojem{integralska
	podmnogoterost} za distribucijo $D_V X$, če za vsako točko $m = (b, \ldots)
  \in D_V(X)$ velja $T_b N = \operatorname{Lin} \{ V_1(b), \ldots, V_r(b) \}$.
\end{definicija}

% LocalWords:  neekvivalentnih trivializacija restringiran Kocikel kocikel
% LocalWords:  kontraktibilna trivializacijo Frobeniusov podsveženj
% LocalWords:  pointegrirajo
