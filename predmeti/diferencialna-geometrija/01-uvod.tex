\naslov{Osnovni pojmi}

Naj bo $X$ topološka mnogoterost ($2$-števen lokalno evklidski Hausdorffov
prostor).

\begin{definicija}
  \pojem{Gladek atlas} na $X$ je družina $U = \{ (U_\alpha, \varphi_\alpha)
  \such \alpha \in A \}$ kjer je $A$ neka indeksna množica,
  $\varphi_\alpha : U_\alpha \to V_\alpha \subseteq \R^n$ homomorfizmi,
  $\bigcup_\alpha U_\alpha = X$ in kjer je za poljubna $\alpha, \beta$ z
  $U_\alpha \cap U_\beta \ne \varnothing$ preslikava $\varphi_\beta \circ
  \varphi_\alpha^{-1}$ difeomorfizem $\varphi_\alpha(U_\alpha \cap U_\beta) \to
  \varphi_\beta(U_\alpha \cap U_\beta)$.
\end{definicija}

\begin{definicija}
  Naj bosta $U = \{(U_\alpha, \varphi_\alpha)\}_{\alpha \in A}$ ter $V =
  \{(V_\beta, \psi_\beta)\}_{\beta \in B}$ dva atlasa na $X$. Atlasa sta
  \pojem{ekvivalentna}, če je $\psi_\beta \circ \varphi_\alpha^{-1}$
  difeomorfizem $\varphi_\alpha(U_\alpha \cap V_\beta) \to \psi_\beta(U_\alpha
  \cap V_\beta)$ za vsak par $\alpha, \beta$, za katera je $U_\alpha \cap
  V_\beta$.
\end{definicija}

\begin{definicija}
  Naj bo $U$ atlas na $X$.
  Ekvivalenčni razred $[U]$ glede na relacijo ekvivalentnosti atlasov se imenuje
  \pojem{gladka struktura} na $X$.
  Mnogoterost, opremljena z gladko strukturo, je \pojem{gladka mnogoterost}.
\end{definicija}

\begin{opomba}
  Obstajajo topološke mnogoterosti, na katerih je več neekvivalentnih gladkih
  struktur.
\end{opomba}

Naj bo sedaj $X \subseteq \R^N$ (gladka) mnogoterost, vložena v $\R^N$, za
katero velja $\dim X = n < N$.
Dodatno naj bo $m \in X$ ter $\gamma: (-\varepsilon, \varepsilon) \to X$ gladka
krivulja z $\gamma(0) = m$.
Potem je
\[
  V = \left. \frac{d}{dt} \right|_{t=0} \gamma(t)
\]
eden od tangentnih vektorjev na $X$ v točki $m$.

\begin{definicija}
  \pojem{Tangentni prostor} $T_m X$ na $X$ v točki $m$ je množica
  \[
	T_m X = \left\{ \left. \frac{d}{dt} \right|_{t=0} \gamma(t) \such \gamma:
	  (-\varepsilon, \varepsilon) \to X, \gamma(0) = m \right\}.
  \]
  To je vektorski podprostor dimenzije $n$ v $\R^N$.
\end{definicija}

Naj bo sedaj $X$ ponovno abstraktna gladka mnogoterost.

\begin{definicija}
  Krivulja $\gamma: (a,b) \to X$ je \pojem{gladka}, če je vsak kompozitum
  $\varphi_\alpha \circ \gamma$ gladka preslikava za vsak $\alpha$ (na primernem
  definicijskem območju).
\end{definicija}

\begin{definicija}
  Krivulji $\gamma_1, \gamma_2: (-\varepsilon, \varepsilon) \to X$, za kateri
  velja $\gamma_1(0) = \gamma_2(0) = m$ sta \pojem{ekvivalentni}, če velja
  \[
	\left. \frac{d}{dt} \right|_{t=0} \varphi_\alpha(\gamma_1(t))
	= \left. \frac{d}{dt} \right|_{t=0} \varphi_\alpha(\gamma_2(t))
  \]
  za poljuben $\alpha$, za katerega je $m \in U_\alpha$.
  Označimo $\gamma_1 \sim \gamma_2$.
\end{definicija}

\begin{definicija}
  \pojem{Tangentni vektor} na $X$ v točki $m$ je eden od ekvivalenčnih razredov
  za zgornjo relacijo.
  \pojem{Tangentni prostor} $T_m X$ je prostor vseh tangentnih vektorjev na $X$
  v točki $m$.
\end{definicija}

\begin{opomba}
  Izkaže se, da je ekvivalenčna relacija $\sim$ neodvisna od karte.
\end{opomba}

Predstavnik tangentnega vektorja $[\gamma]$ označimo z
\[
  \tangentni{\varphi_\alpha(\gamma)} = \left. \frac{d}{dt} \right|_{t=0}
  \varphi_\alpha(\gamma(t)).
\]
To je vektor v $\R^n$.
Poglejmo si, kaj se zgodi z $\tangentni{\varphi_\alpha(\gamma)}$, če zamenjamo karto.
Naj bo $(U_\beta, \varphi_\beta)$ še ena karta, ki vsebuje $m$.
Potem je
\[
  \tangentni{\varphi_\beta(\gamma)}
  = \left. \frac{d}{dt} \right|_{t=0} \varphi_\beta(\gamma(t))
  = \left. \frac{d}{dt} \right|_{t=0} (\varphi_\beta \varphi_\alpha^{-1}
  \varphi_\alpha)(\gamma(t))
  = D_{\varphi_\alpha(m)} (\varphi_\beta \varphi_\alpha^{-1}) \left.
	\frac{d}{dt} \right|_{t=0} \varphi_\alpha(\gamma(t))
\]
po verižnem pravilu za odvajanje, kar pa je enako
\[
  \tangentni{\varphi_\beta(\gamma)}
  = D_{\varphi_\alpha(m)} (\varphi_\beta \varphi_\alpha^{-1})
  \tangentni{\varphi_\alpha(\gamma)}
\]
Preslikava $D_{\varphi_\alpha(m)}$ je linearni izomorfizem $\R^n \to \R^n$.

V tangentnem prostoru so nam na voljo običajni operaciji vektorskih prostorov:
\begin{itemize}
\item Seštevanje: $[\gamma_1] + [\gamma_2] = [ t \mapsto
  \varphi_\alpha^{-1}(\varphi_\alpha(m) + t
  (\tangentni{\varphi_\alpha(\gamma_1)} + \tangentni{\varphi_\alpha(\gamma_2)}
  )) ]$.
\item Množenje s skalarjem: $a [\gamma] = [\gamma^a]$ za $\gamma^a(t) =
  \gamma(at)$.
\end{itemize}

Vsakemu tangentnemu vektorju lahko priredimo diferencialni operator prvega reda.
Naj bo $f: X \to \R$ gladka funkcija.
V smeri $[\gamma]$ ga odvajamo s smernim odvodom
\[
  (Vf)_{(m)} [\gamma] = \left. \frac{d}{dt} \right|_{t=0} (f \circ \gamma)(t).
\]
To je neodvisno od izbire predstavnika $\gamma$:
če sta $\gamma_1, \gamma_2 \in [\gamma]$, velja
\begin{align*}
  \left. \frac{d}{dt} \right|_{t=0} (f \circ \gamma_i)(t)
  &= \left. \frac{d}{dt} \right|_{t=0} (f \varphi_\alpha^{-1} \varphi_\alpha
  \gamma_i)(t) \\
  &= \grad(f \varphi_\alpha^{-1}) \left. \frac{d}{dt} \right|_{t=0}
  (\varphi_\alpha \gamma_i)(t) \\
  &= \grad(f \varphi_\alpha^{-1}) \tangentni{\varphi_\alpha(\gamma_i)}.
\end{align*}
Ker sta $\gamma_1$ in $\gamma_2$ ekvivalentna, je dobljeni izraz neodvisen od
izbire $i$.

Tangentne vektorje lahko obravnavamo kot parcialne diferencialne operatorje
prvega reda.
Naj bo $[\gamma] = V_m \in T_m X$ tangentni vektor.
PDO, ki ga pripišemo temu vektorju, je smerni odvod; če je $f: X \to \R$ gladka
funkcija, definiramo
\[
  V_m(f) = \left. \frac{d}{dt} \right|_{t=0} f(\gamma(t)).
\]
V lokalnih koordinatah, danih z atlasom $(U_\alpha, \varphi_\alpha)$, lahko zapišemo
\[
  V_m(f) = \left. \frac{d}{dt} \right|_{t=0} f(\gamma(t))
  = \left. \frac{d}{dt} \right|_{t=0} f \circ \varphi_\alpha^{-1} \circ
  \varphi_\alpha \circ \gamma(t)
  = \sum_{i=1}^n \frac{\partial (f \varphi_\alpha^{-1})}{\partial x_i}
  v_i
\]
kjer je $\frac{d}{dt}(\varphi_\alpha \gamma)(0) = (v_1, \ldots, v_n)$ in $x_1,
\ldots, x_n$ izbrane koordinate v $\R^n$.
To je očitno linearen operator, velja $V(fg) = V(f) g + f V(g)$.

\podnaslov{Vektorski svežnji}

Oglejmo si navadno diferencialno enačbo $\dot{\vec{x}} = V(\vec{x})$, kjer je
$V$ vektorsko polje na neki podmnožici $\R^n$.
Vemo:
Za dan Cauchyjev problem lahko najdemo krivuljo $\gamma$, za katero velja
$\dot{\gamma} = V(\gamma)$.
Naj bo sedaj $X$ gladka mnogoterost.
Vektorsko polje na $X$ je definirano kot preslikava
\[
  V: X \to \bigsqcup_{m \in X} T_m X,
\]
ki slika v disjunktno unijo tangentnih prostorov na $X$.

\begin{definicija}
  Naj bosta $X$ in $E$ taki gladki mnogoterosti, da obstaja projekcija
  $\pi: E \to X$ z naslednjimi lastnostmi:
  \begin{itemize}
  \item Za vsak $b \in X$ je $\pi^{-1}(b) \subseteq E$ vektorski prostor,
	izomorfen $\F^k$.
  \item Za vsak $b \in X$ obstaja odprta okolica $b \in U \subseteq X$ in
	difeomorfizem $T_U: \pi^{-1}(U) \to U \times \F^k$, za katerega je vsaka
	skrčitev $T_U$ na $\pi^{-1}(a)$ (lokalen) linearen izomorfizem $\pi^{-1}(a)
	\to \{a\} \times \F^k$ za poljuben $a \in U$.
	Potem je $T_U$ \pojem{lokalna trivializacija}.
  \end{itemize}
  Potem je $(E, X, \pi)$ \pojem{vektorski sveženj} ranga $k$.
\end{definicija}

\begin{definicija}
  Sveženj $\pi: E \to X$ je \pojem{trivialen}, če je difeomorfen $E = X
  \times V$.
\end{definicija}

Vpeljemo še naslednjo terminologijo:
\begin{itemize}
\item $\pi$ je \pojem{projekcija na sveženj},
\item $V(\F^k)$ je \pojem{vlakno},
\item $X$ je \pojem{osnovni prostor},
\item $E$ je \pojem{celoten prostor},
\item $T_U$ je \pojem{trivializacija}.
\end{itemize}

Na $E$ lahko konstruiramo gladek atlas.
Naj bo $U = \{ (U_\alpha, \varphi_\alpha) \}_{\alpha \in A}$ tak gladek atlas na
$X$, da je restringiran sveženj $E/U_\alpha = \pi^{-1}(U_\alpha)$ trivialen.
Za odprte množice v novem atlasu $\tilde{U}$ vzamemo množice
$\pi^{-1}(U_\alpha)$ za $\alpha \in A$, karte pa so dane s preslikavami
\begin{gather*}
  \tilde{\tau}_\alpha: \pi^{-1}(U_\alpha) \to \F^{n+k} \\
  \tilde{\tau}_\alpha(v)
  = ( \varphi_\alpha(\pi(v)), v_1, \ldots, v_k )
  = ( \varphi_\alpha(\pi(v)), \operatorname{pr}_2(\tau_\alpha(v)) ),
\end{gather*}
kjer je $\tau_\alpha$ difeomorfizem $\pi^{-1}(U_\alpha) \to U_\alpha \times
\F^k$, ki slika $v \mapsto (\pi(v), (v_1, \ldots, v_k))$.
Včasih pišemo tudi
\[
  \tau_\alpha(m) = (\pi(m), v_1(\pi(m)), \ldots, v_k(\pi(m)))
  = (b, v_1(b), \ldots, v_k(b)).
\]
Oglejmo si prehodno preslikavo
\[
  \tau_\beta \circ \tau_\alpha^{-1}
  : (b, v_1(b), \ldots, v_k(b)) \mapsto (b, w_1(b), \ldots, w_k(b)).
\]
Zapišemo jo lahko v obliki
\[
  \tau_\beta \circ \tau_\alpha^{-1}
  \left(
	b,
	\begin{bmatrix}
	  v_1(b) \\ \vdots \\ v_k(b)
	\end{bmatrix}
  \right)
  = \left(
	b,
	g_{\beta \alpha}(b)
	\begin{bmatrix}
	  v_1(b) \\ \vdots \\ v_k(b)
	\end{bmatrix}
  \right)
\]
oziroma $\tau_\beta \circ \tau_\alpha^{-1} = (\id, g_{\beta \alpha}(b))$.
Za vsak par $\alpha, \beta$, za katera je $U_\alpha \cap U_\beta \ne
\varnothing$, torej dobimo gladko preslikavo $g_{\beta \alpha} \in \GL(k, \F)$.
Dodano velja $g_{\gamma \beta} g_{\beta \alpha} = g_{\gamma \alpha}$.

\begin{definicija}
  \pojem{Kocikel}, prirejen atlasu $U = \{(U_\alpha, \varphi_\alpha)\}_\alpha$,
  z vrednostmi v $\GL(k, \F)$, je družina preslikav $g_{\beta \alpha}: U_\alpha
  \cap U_\beta \to \GL(k, \F)$, za katere za vsak primeren $b$ velja
  \begin{itemize}
  \item $g_{\alpha \gamma}(b) g_{\gamma \beta}(b) g_{\beta \alpha}(b) = I$,
  \item $g_{\alpha \beta}(b) = g_{\beta \alpha}(b)^{-1}$.
  \end{itemize}
\end{definicija}

\begin{opomba}
  Če imamo dan $X$ in kocikel, lahko do izomorfizma natančno konstruiramo
  pripadajoči sveženj.
\end{opomba}

\begin{definicija}
  Naj bo $\pi: E \to X$ sveženj.
  \pojem{Gladek lokalni prerez} svežnja $E$ nad $U_\alpha$ je gladka preslikava
  $s: U_\alpha \to E/U_\alpha$.
\end{definicija}

\begin{opomba}
  Ker je $E/U_\alpha = \pi^{-1}(U_\alpha)$, za primeren $b$ velja $\pi \circ s(b) = b$.
\end{opomba}

\begin{definicija}
  \pojem{Lokalno ogrodje} svežnja $\pi: E \to X$ nad $U_\alpha$ je $k$-terica
  prerezov $(s_1, \ldots, s_k): U_\alpha \to E/U_\alpha$, ki so med seboj
  linearno neodvisni, torej da so za vsak $b \in U_\alpha$ vektorji $s_1(b),
  \ldots, s_k(b)$ linearno neodvisni v $\pi^{-1}(b) \cong \F^k$.
\end{definicija}

\begin{definicija}
  \pojem{Tangentni sveženj} $TX$ gladke mnogoterosti $X$ je disjunktna unija
  \[
	TX = \bigsqcup_{m \in X} T_m X.
  \]
\end{definicija}

Opremimo $TX$ z gladko strukturo.
Naj bo $U = \{ (U_\alpha, \varphi_\alpha) \}_\alpha$ atlas za $X$.
Potem definiramo
\[
  T U_\alpha = \bigsqcup_{b \in U_\alpha} T_b X := TX / U_\alpha.
\]
Naj bo $U_\alpha$ kontraktibilna.
Vpeljemo lokalno trivializacijo $\tau_\alpha: TU_\alpha \to V_\alpha \times
\R^n$,
\[
  \tau_\alpha(v(b)) = (\varphi_\alpha(b), D_b \varphi_\alpha (v(b)))
  = (\varphi_\alpha(b), v_1(b), \ldots, v_n(b)).
\]
Torej imamo atlas $TU = \{ (TU_\alpha, \tau_\alpha) \}_\alpha$.
Za bazo lokalnih prerezov v točki $b_0 = \varphi_\alpha^{-1}(x_0)$ vzamemo
vektorje $[\gamma_i]_{b_0}$, pri katerih so reprezentativne krivulje koordinatne
premice $t \mapsto \varphi_\alpha^{-1}(x_1^0, \ldots, x_i^0  + t, \ldots,
x_n^0)$ za $x_0 = (x_1^0, \ldots, x_n^0)$.
Tak tangentni vektor označimo z $\frac{\partial}{\partial x_i^\alpha}(b)$.

\podnaslov{Liejev odvod}

Spomnimo se:
Nad lokalno karto $(U_\alpha, \varphi_\alpha)$, nad katero je $T U_\alpha
\subseteq TX$ trivialen, imamo odlikovana bazo prerezov $T U_\alpha$, ki jo
označimo z
\[
  \frac{\partial}{\partial x_1^\alpha}, \frac{\partial}{\partial x_2^\alpha},
  \ldots, \frac{\partial}{\partial x_n^\alpha}.
\]
Velja $\frac{\partial}{\partial x_i^\alpha} = [c_i^\alpha]_m$, kjer je
$c_i^\alpha$ $i$-ta koordinatna krivulja na $U_\alpha$.
Vsako vektorsko polje $V$ lahko lokalno zapišemo v obliki
\[
  V(m) = \sum_{i=1}^n a_i^\alpha(m) \frac{\partial}{\partial x_i^\alpha}(m)
\]
za neke $a_i^\alpha$.

\begin{definicija}
  \pojem{Smerni odvod} funkcije $f: X \to \R$ v smeri vektorskega polja $V$ je
  funkcija $V(f): X \to \R$, podana s predpisom
  \[
	V(f)(m) = \left. \frac{d}{dt} \right|_{t=0} f(\gamma(t)),
  \]
  kjer je $\gamma$ integralska krivulja $V$ skozi točko $\gamma(0) = m$.
\end{definicija}

Velja potem
\[
  V(f)(a) = \sum_{i=1}^n \frac{\partial f}{\partial x_i^\alpha}(x)
  a_i^\alpha(x).
\]

Naj bosta $V$ in $W$ gladki vektorski polji na $U \subseteq \R^n$.
Opazujemo vrednosti $W$ vzdolž integralske krivulje $\gamma^V(t)$ polja $V$.
Naj bo $\phi^V_t: U_\alpha \to \phi^V_t(U_\alpha)$ tok vektorskega polja $V$.
Definiramo
\[
  \mathcal{L}_V W(m)
  = \left. \frac{d}{dt} \right|_{t=0} \left( D_{\phi^V_t(m)} \phi^V_t
  \right)^{-1} W(\gamma^V(t))
\]
kjer je $\gamma^V(0) = m$.

Izračunajmo $\mathcal{L}_V W$ v koordinatah.
Če označimo $\phi^V_t(x) = ((\phi^V_t)_1, \ldots, (\phi^V_t)_n)(x_1, \ldots,
x_n)$, potem je
\[
  D \phi^V_t =
  \begin{bmatrix}
	\frac{\partial (\phi^V_t)_1}{\partial x_1} & \cdots & \frac{\partial
														  (\phi^V_t)_1}{\partial
														  x_n} \\
	\vdots & \ddots & \vdots \\
	\frac{\partial (\phi^V_t)_n}{\partial x_1} & \cdots & \frac{\partial
														  (\phi^V_t)_n}{\partial
														  x_n} \\
  \end{bmatrix}.
\]
Če na definiciji $\mathcal{L}_V W$ uporabimo produktno pravilo za odvajanje,
dobimo
\[
  \mathcal{L}_V W(m)
  = \left( \left. \frac{d}{dt} \right|_{t=0} \left( D \phi_t^V \right)^{-1}
  \right) W(\gamma^V(0))
  + \left. \left( D \phi_t^V \right)^{-1} \right|_{t=0}
  \left. \frac{d}{dt} \right|_{t=0} W(\gamma^V(t))
\]
V našem primeru je $\phi^V_t = \id$ pri $t = 0$.
Potem je
\[
  \left.\frac{d}{dt} \right|_{t=0} D \phi^V_t =
  \begin{bmatrix}
	\frac{\partial V_1}{\partial x_1} & \cdots & \frac{\partial V_1}{\partial
												 x_n} \\
	\vdots & \ddots & \vdots \\
	\frac{\partial V_n}{\partial x_1} & \cdots & \frac{\partial V_n}{\partial
												 x_n} \\
  \end{bmatrix}(m),
\]
saj odvod po času in po $x_i$ komutirata.
Potem imamo, upoštevaje $\partial_t (A^{-1}) = -A^{-1} \dot{A} A^{-1}$,
\[
  \mathcal{L}_V W =
  -
  \begin{bmatrix}
	\frac{\partial V_1}{\partial x_1} & \cdots & \frac{\partial V_1}{\partial
												 x_n} \\
	\vdots & \ddots & \vdots \\
	\frac{\partial V_n}{\partial x_1} & \cdots & \frac{\partial V_n}{\partial
												 x_n} \\
  \end{bmatrix}
  \cdot
  \begin{bmatrix}
	W_1 \\ \vdots \\ W_n
  \end{bmatrix}
  +
  \begin{bmatrix}
	\sum_{j=1}^n \frac{\partial W_1}{\partial x_j} V_j \\
	\vdots \\
	\sum_{j=1}^n \frac{\partial W_n}{\partial x_j} V_j \\
  \end{bmatrix},
\]
kjer so vsa polja evaluirana v točki $m$.
Potem je $i$-ta komponenta tega vektorja enaka
\[
  - \sum_{j=1}^n \frac{\partial V_i}{\partial x_j} W_j + \sum_{j=1}^n
  \frac{\partial W_i}{\partial x_j} V_j,
\]
torej je zapis $\mathcal{L}_V W$ antisimetričen v $V$ in $W$.

Rezultat nas navede na drugo definicijo Liejevega odvoda.
Naj bo $f: X \to \R$ poljubna funkcija.
Potem
\[
  (\mathcal{L}_V W) (f) = \left[ V, W \right] (f)
  = V(W(f)) - W(V(f))
  = \sum_{i,j} \left( V_j \frac{\partial W_i}{\partial x_j} - W_j \frac{\partial
	V_i}{x_j} \right) \left( \frac{\partial f}{\partial x_i} \right).
\]

Naj bosta $\phi_t^V$ in $\phi_t^W$ tokova $V$ in $W$.
Pričakovali bi, da mora veljati $\phi_s^W \circ \phi_t^V = \phi_t^V \circ
\phi_s^W$, vendar to ni nujno res.
Izračunamo lahko
\[
  \left. \frac{d^2}{dt ds} \right|_{s=t=0} f(\phi_t^V(\phi_s^W(m))) -
  f(\phi_s^W(\phi_t^V(m)))
  = \left[ W, V \right](f) (m).
\]
Če tokova komutirata, potem torej velja $\mathcal{L}_V W = 0$.

\podnaslov{Frobeniusov izrek}

Naj bo $X$ gladka mnogoterost in $TX$ njen tangentni sveženj.
Naj bodo $V_1, \ldots, V_r$ lokalna gladka vektorska polja v $TX$ in $r < n =
\dim TX$.
Označimo z $D_V X$ lokalni vektorski podsveženj v $TX$.
Polja $V_1, \ldots, V_r$ naj bodo v vsaki točki, kjer so definirana, linearno
neodvisna.
Zanimalo nas bo, kdaj obstajajo integralske podmnogoterosti v $X$, ki
\enquote{pointegrirajo} $D_V X$.

\begin{definicija}
  Podmnogoterost $N \subseteq X$ dimenzije $r$ je \pojem{integralska
	podmnogoterost} za distribucijo $D_V X$, če za vsako točko $m = (b, \ldots)
  \in D_V(X)$ velja $T_b N = \operatorname{Lin} \{ V_1(b), \ldots, V_r(b) \}$.
\end{definicija}

\begin{izrek}[Frobenius]
  Naj bo $X$ mnogoterost dimenzije $n$ in $V_1, \ldots, V_r$ vektorska polja, ki
  so v vsaki točki linearno neodvisna.
  Naj za ta polja velja
  \[
	\left[ V_i, V_j \right](m)
	= \sum_{k=1}^n c_{i,j}^k(m) V_k,
  \]
  kjer so $C_{i,j}^k: X \to \R$ gladke funkcije.
  Potem za vsak $p \in X$ obstaja podmnogoterost $N \subseteq X$, da
  je $p \in N$ in da za vsak $q \in N$ velja $T_q N = \operatorname{Lin}
  \{V_1(q), \ldots, V_r(q)\}$.
\end{izrek}

\begin{proof}
  Ideja.
  Z Gaussovim postopkom nadomestimo polja $V_1, \ldots, V_r$ s polji
  $\tilde{V}_1, \ldots, \tilde{V}_r$, za katera velja $\left[ \tilde{V}_i,
	\tilde{V}_j \right] = 0$.
  Definiramo preslikavo
  \[
	\varphi(t_1, \ldots, t_r)
	= \phi_{t_r}^{\tilde{V}_r} \circ \phi_{t_{r-1}}^{\tilde{V}_{r-1}} \circ \cdots \circ
	\phi_{t_2}^{\tilde{V}_2} \circ \phi_{t_1}^{\tilde{V}_1}(p).
  \]
  Potem moramo pokazati, da za vsako točko $q \in N$ velja $T_q N =
  \operatorname{Lin}\{ \tilde{V}_1(q), \ldots, \tilde{V}_r(q) \}$.
  Izberimo $i = \{1, \ldots, r\}$ in si oglejmo
  \[
	\left. \frac{d}{ds} \right|_{s=0} \varphi(t_1^0, \ldots, t_{i-1}^0, t_i^0 +
	s, t_{i+1}^0, \ldots, t_r^0)
	= \left. \frac{d}{ds} \right|_{s=0} \phi_s^{\tilde{V}_i} \circ
	\phi_{t_r^0}^{\tilde{V}_r} \circ \cdots \circ \phi_{t_1^0}^{\tilde{V}_1}(p)
	= \tilde{V}_1(q).
  \]
\end{proof}

\podnaslov{Liejeve grupe}

Če je $G$ hkrati mnogoterost in grupa, ter sta množenje in invertiranje gladki
preslikavi, je $G$ Liejeva grupa.
Grupna struktura nam poda veliko poceni difeomorfizmov, na primer leve in desne
translacije ter adjungiranje
\begin{gather*}
  L_g: h \mapsto gh \\
  R_g: h \mapsto hg \\
  \tilde{A}_g: h \mapsto ghg^{-1}
\end{gather*}

\begin{definicija}
  Vektorsko polje $X_\xi$ je \pojem{levo invariantno vektorsko polje}, če je
  podano s predpisi
  \[
	X_\xi(g) = D_e L_g (\xi),
  \]
  kjer je $\xi \in T_e G$ in $e$ enota grupe.
\end{definicija}

Naj bo $\{\xi_1, \ldots, \xi_n\}$ neka baza $T_e G$.
Ker je za vsak $g \in G$ preslikava $D_e L_g: T_e G \to T_g G$ linearni
izomorfizem, je $\{ D_e L_g(\xi_1), \ldots, D_e L_g(\xi_n) \} = \{ g \xi_1,
\ldots, g \xi_n \}$ baza $T_g G$.
Iz tega sledi, da je tangentni sveženj $T G$ trivialen.

Za $x \in T_h G$ pišemo $g \cdot x := D_h L_g(x)$ in podobno $x \cdot g$.

Oglejmo si preslikavo $\tilde{A}: (g, h) \mapsto ghg^{-1}$.
Če nadomestimo $h$ s krivuljo $h(t)$, za katero je $h(0) = e$ in posledično
\[
  \left. \frac{d}{dt} \right|_{t=0} h(t) = \xi \in T_e G,
\]
lahko definiramo preslikavo $\operatorname{Ad}_g: T_e G \to T_e G$ s predpisom
\[
  \xi \mapsto \left. \frac{d}{dt} \right|_{t=0} \tilde{A}_g(h(t)).
\]
Opazimo
\[
  \operatorname{Ad}_g(\xi) = (D_e R_g)^{-1} (D_e L_g) (\xi)
  = g \cdot \xi \cdot g^{-1}.
\]

Integralske krivulje levo invariantnih vektorskih polj so rešitve začetnih
problemov $\dot{\gamma}(t) = X_\xi (\gamma(t)) = \gamma(t) \cdot \xi$ z
$\gamma(0) = e$.
Podobno za desno invariantna polja rešujemo $\dot{\gamma}(t) = \xi \cdot
\gamma(t)$.
Če je $G$ matrična Liejeva grupa (torej $\xi$ matrika), so rešitve teh enačb
oblike
\[
  \gamma(t) = \sum_{n=0}^\infty \frac{t^n}{n!} \xi^n = \exp(t \xi).
\]

Običajno pišemo $\varphi_\xi(t) := \gamma(t)$.
Potem so preslikave $\varphi_\xi : (-\varepsilon, \varepsilon) \mapsto G$
homomorfizmi, kar sledi iz izrekov o tokovih za sisteme NDE\@.
Slike $\im \varphi_\xi$ so torej enodimenzionalne podgrupe v $G$.

\begin{definicija}
  Splošna eksponentna preslikava $\exp: T_e G \to G$ je podana s predpisom
  $\exp(\xi) = \varphi_\xi(1)$.
\end{definicija}

\begin{opomba}
  Velja $\exp(t \xi) = \varphi_{t \xi}(1) = \varphi_\xi(t)$.
\end{opomba}

V prostor $\mathfrak{g} = T_e G$ bomo vpeljali operacijo
\[
  \llbracket \xi, \eta \rrbracket
  = \mathcal{L}_{X_\xi} (X_\eta)(e).
\]

\begin{trditev}
  Naj bosta $\xi, \eta \in \mathfrak{g}$ poljubna.
  Potem velja
  \[
	\llbracket \xi, \eta \rrbracket
	= \left. \frac{d}{dt} \right|_{t=0}
	\operatorname{Ad}_{\varphi_\eta(t)}(\xi).
  \]
\end{trditev}

\begin{proof}
  Integralska krivulja polja $X_\xi$ skozi $g$ je podana s predpisom
  \[
	\gamma_g^{X_\xi}(t) = g \cdot \varphi_\xi(t), \quad \gamma^{X_\xi}(0) = g.
  \]
  Definirajmo $Y_\eta = X_\eta$.
  Potem je $Y_\eta(g \cdot \varphi_\xi(t)) = g \cdot \varphi_\xi(t) \cdot \eta$
  in
  \[
	\phi_t^{X_\xi}(h) = \gamma_h^{X_\xi}(t) = h \cdot \varphi_\xi(t),
  \]
  torej
  \[
	(\phi_t^{X_\xi})^{-1} (gh) = gh \cdot \varphi_\xi(-t)
	= gh \varphi_\xi(t)^{-1}.
  \]
  Sedaj lahko izračunamo
  \begin{align*}
	\mathcal{L}_{X_\xi} (Y_\eta)(g)
	&= \left. \frac{d}{dt} \right|_{t=0} \left( g \cdot \varphi_\xi(t) \eta \varphi_\xi(t)^{-1} \right) \\
	&= \left. \frac{d}{dt} \right|_{t=0} g \operatorname{Ad}_{\varphi_\xi(t)} (\eta).
  \end{align*}
  Vstavimo $g = e$ in dobimo
  \[
	\llbracket \xi, \eta \rrbracket
	= \left. \frac{d}{dt} \right|_{t=0} \operatorname{Ad}_{\varphi_\xi(t)}
	(\eta).
	\qedhere
  \]
\end{proof}

Zgornji dokaz nam tudi pove
\[
  [X_\xi, Y_\eta](g)
  = g \left. \frac{d}{dt} \right|_{t=0} \operatorname{Ad}_{\varphi_\xi(t)}(\eta)
  = g \llbracket \xi, \eta \rrbracket.
\]
Spotoma smo torej dokazali, da je Liejev produkt levo invariantnih vektorskih
polj levo invariantno vektorsko polje.

Kako pa ta formula izgleda v primeru matričnih Liejevih grup?
Za matrične grupe velja
\[
  \varphi_\xi(t) = \exp(t\xi) = I + t \xi + \cdots + \frac{t^n}{n!} \xi^n + \cdots,
\]
iz česar sledi
\[
  \llbracket \xi, \eta \rrbracket
  = [X_\xi, X_\eta](e)
  = \left. \frac{d}{dt} \right|_{t=0} \exp(t\xi) \eta \exp(-t \xi)
  = \left. \frac{d}{dt} \right|_{t=0} (I + t \xi) \eta (I - t \xi),
\]
to pa je enako
\[
  \llbracket \xi, \eta \rrbracket
  = \xi \eta - \eta \xi
  = [\xi, \eta].
\]

\begin{primer}
  Če je $G = SU(n)$, je tangentni prostor enak
  \[
	\mathfrak{su}(n) = \left\{ \xi \in \mathfrak{gl}(n, \C) \such \xi^* = -\xi,
	  \sled(\xi) = 0 \right\}.
  \]
  Res; naj bo $g(t)$ krivulja z $g(0) = I$.
  Potem za vsak $t$ velja $g^*(t)g(t) = I$, kar odvajamo v
  \[
	\frac{d}{dt}(g^*(t)) g(t) + (g(t))^* \frac{d}{dt} (g(t)) = 0,
  \]
  kar pa še evaluiramo v $0$, in dobimo.
\end{primer}

% LocalWords:  neekvivalentnih trivializacija restringiran Kocikel kocikel
% LocalWords:  kontraktibilna trivializacijo Frobeniusov podsveženj Liejev
% LocalWords:  pointegrirajo evaluirana Liejevega Frobenius Liejeve Liejeva
% LocalWords:  invertiranje difeomorfizmov invariantnih evaluiramo
