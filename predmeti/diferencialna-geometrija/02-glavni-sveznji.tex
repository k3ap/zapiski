\naslov{Glavni svežnji}

Glavni svežnji so objekti, sorodne vektorskim svežnjem.
Vlakna v glavnih svežnjih so Liejeve grupe in ne vektorski prostori.

\begin{definicija}
  Gladka mnogoterost $P$, skupaj s preslikavo $\pi: P \to M$, je \pojem{glavni
	$G$-sveženj}, če velja:
  \begin{itemize}
  \item grupa $G$ gladko deluje na $P$ z desne z $\rho: P \times G \to P$,
	$\rho(p, g) = p \cdot g = \rho_g(p)$,
  \item prostor orbit $M = P/G$ je gladka mnogoterost in naravna projekcija
	\[
	  \pi: P \to P/G =M
	\]
	je gladka,
  \item sveženj $P$ je lokalno trivialen.
	To pomeni, za vsako točko $m \in M$ obstaja odprta okolica $m \in U
	\subseteq M$, da obstaja difeomorfizem $\varphi: \pi^{-1}(U) \subseteq P \to
	U \times G$ oblike $\varphi(p) = (\pi(p), s(p))$, kjer je $s$ ekvivariantna
	preslikava; $s(\rho_g(p)) = s(p) \cdot g$.
  \end{itemize}
\end{definicija}

\begin{primer}
  Naj bo $M$ mnogoterost in $G$ Liejeva grupa.
  Potem je $P = M \times G$ glavni sveženj.
\end{primer}

\begin{primer}
  Naj bo $P = S^3 = \{ (z_1, z_2) \in \C^2 \such \abs{z_1}^2 + \abs{z_2}^2 = 1
  \}$.
  Liejeva grupa $U(1) = \{ e^{i \varphi} \such \varphi \in [0, 2 \pi] \}$ deluje
  na $S^3$ z
  \[
	\rho_{e^{i \varphi}} (z_1, z_2) = (z_1 e^{i \varphi}, z_2 e^{i \varphi}).
  \]
  Prostor orbit $S^3 / U(1)$ je tedaj
  \[
	M = \{ [z_1, z_2] \such (z_1, z_2) \in S^3 \}
	= \CP{1} \approx S^2.
  \]
\end{primer}

\begin{opomba}
  Velja $S^3 \approx SU(2)$.
\end{opomba}

Naj bo $\{U_\alpha\}_\alpha$ atlas na $M$ in naj bo $\pi^{-1}(U_\alpha)$ za vsak
$\alpha$ trivialen.
Potem je $\varphi_\alpha: \pi^{-1}(U_\alpha) \to U_\alpha \times G$
\[
  p \mapsto (\pi(p), s_\alpha(p))
\]
trivializacija.
Recimo, da je $\pi(p) \in U_\alpha \cap U_\beta$.
Definiramo lahko preslikavo
\[
  g_{\beta \alpha}: U_\alpha \cap U_\beta \to G,
\]
podano s predpisom $g_{\beta \alpha}(m) = s_\beta(p) s_\alpha^{-1}(p)$ za
poljuben $p \in \pi^{-1}(m)$.
Ta je res neodvisna od izbire $p$:
Če je tudi $q \in \pi^{-1}(m)$, obstaja $g \in G$, da je $q = p \cdot g$, zaradi
ekvivariantnosti $s$ pa je potem $s_\beta(p) s_{\alpha}^{-1}(p) = s_\beta(q)
s_\alpha^{-1}(q)$.
Očitno za $g_{\beta \alpha}$ veljata lastnosti kocikla.

Kakor pri vektorskih svežnjih tudi tu velja:
Če imamo gladko mnogoterost $M$ z atlasom $\{(U_\alpha,
\varphi_\alpha)\}_\alpha$, Liejevo grupo $G$ kocikel $\{g_{\beta
  \alpha}\}_{\beta, \alpha}$, lahko konstruiramo glavni sveženj.
Definiramo lahko namreč
\[
  \tilde{P} = \prod_{\alpha \in A} (U_\alpha \times G)
\]
in $P = \tilde{P} / \sim$ za
\[
  (\alpha, m_1, h) \sim (\beta, m_2, g)
  \iff m_1 = m_2 \land g_{\beta \alpha}(m_1) h = g.
\]

% LocalWords:  Liejeve ekvivariantna Liejeva trivializacija ekvivariantnosti
% LocalWords:  kocikla Liejevo kocikel
