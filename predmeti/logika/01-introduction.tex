\naslov{Introduction}

First-order logic is a formal language for expressing properties of mathematical
statements.
The language is determined by a \enquote{signature.}

\begin{example}
  Consider the following arithmetic signature, consisting of:
  \begin{itemize}
  \item two constants $0$, $1$,
  \item two binary function symbols $+$, $\cdot$.
  \end{itemize}
  Given the signature, we have a derived notions of \enquote{terms} and
  \enquote{formulas.}
  In this case, terms represent numbers, for example $0, 1, x, x \cdot (y+z),
  x \cdot x \cdot x + 1 \cdot 0, \ldots$ are all terms of this language.
  On the other hand, formulas represent properties of their free variables.
  As an example, the formula $\exists z \velja y = x + z$ expresses the property
  $x \le y$.
  Here, $z$ is a bound variable, and $x, y$ are free variables.
  \boxdot{}
\end{example}

A formula with free variables is called a \enquote{sentence}.
When a sentence is interpreted in a structure, it is either true or false.

\begin{example}
  Fermat's last theorem can be expressed as
  \[
	\forall n \velja n > 2 \implies \lnot \left(
	  \exists x, y, z \velja x > 0 \land y > 0 \land z > 0
	  \land x^n + y^n = z^n
	\right)
  \]
  This uses exponentiation, which we can expresses in the arithmetic language,
  as was shown by Gödel.
  The sentence above is true, and the proof is left as an exercise for the
  reader.
  \boxdot{}
\end{example}

\begin{example}
  The sentence
  \[
	x > 1 \land \forall y, z \velja x = y \cdot z \implies x = y \lor x = z
  \]
  encodes \enquote{$x$ is prime.}
  We can use this for example to express some open problems, such as the twin
  prime conjecture,
  \[
	\forall x \exists y \velja y > x \land \operatorname{Prime}(y) \land
	\operatorname{Prime}(y+2).
  \]
  I need some text to put a boxdot here.
  \boxdot{}
\end{example}

The richness of the examples suggests that the following is a difficult
question:
\begin{quote}
  Given an input sentence $A$, decide whether $A$ is true or false in
  $\N$, i.e.~if $\N \models A$.
\end{quote}

\begin{theorem}[Church, Turing]
  The above question is undecidable.
\end{theorem}

We can interpret our arithmetic signature in any ring with a unit, such as in
the real numbers.
Then, a formula with $n$ free variables describes an $n$-dimensional geometric
shape (actually, a subset of $\R^n$).
This signature, interpreted in $\R$, gives a language for Euclidean geometry.
But more important is the following.

\begin{theorem}[Tarski]
  The problem, \enquote{given a sentence $A$, decide if $\R \models A$,}
  is decidable.
\end{theorem}

\begin{definition}
  A \pojem{signature} is a combination of a set of predicate (or relation)
  symbols $\mathcal{P}$ and function symbols $\mathcal{F}$, together with an
  arity function $\Ar: \mathcal{P} \dcup \mathcal{F} \to \N$, which returns the
  number of arguments that a symbol takes.
\end{definition}

\begin{example}
  The signature discussed above has $\mathcal{P} = \varnothing$, $\mathcal{F} =
  \{ 0, 1, \cdot, + \}$, and $\Ar$ defined with
  \begin{equation*}
	0 \mapsto 0 \qquad
	1 \mapsto 0 \qquad
	+ \mapsto 2 \qquad
	\cdot \mapsto 2
  \end{equation*}
\end{example}

\begin{definition}
  A function symbol of arity $0$ is called a \pojem{constant}.
\end{definition}

\begin{example}
  Another important signature is $(\{E\}, \varnothing)$ with $\Ar(E) = 2$.
  This is the signature for interpreting in graphs.
\end{example}

\begin{example}
  Yet another important signature is $(\{\in\}, \varnothing)$ with
  $\Ar(\in) = 2$.
  This is the signature of set theory.
  Note that this is just the same signature as above.
\end{example}

\begin{definition}
  \pojem{Terms} are defined by the following two rules:
  \begin{itemize}
  \item a variable is a term,
  \item if $t_1, \ldots, t_n$ are terms and $f \in \mathcal{F}$ is a function
	symbol of arity $n$, then $f(t_1, \ldots, t_n)$ is a term.
  \end{itemize}
\end{definition}

\begin{definition}
  \pojem{Formulas} are defined by the following rules:
  \begin{itemize}
  \item if $t_1, t_2$ are terms, then $t_1 = t_2$ is a formula,
  \item if $t_1, \ldots, t_n$ are terms and $p \in \mathcal{P}$ is a predicate
	symbol of arity $n$, then $p(t_1, \ldots, t_n)$ is a formula,
  \item if $A, B$ are formulas, then so are
	$\lnot A$, $A \land B$, $A \lor B$ and $A \implies B$,
  \item if $A$ is a formula, then so are $\exists x \velja A$ and $\forall x
	\velja A$.
  \end{itemize}
  Formulas of the first two types are called \pojem{atomic formulas}.
\end{definition}

\begin{definition}
  A \pojem{structure} for a signature $(\mathcal{P}, \mathcal{F})$ is a triple
  \[
	\mathcal{M} = \left( M, (p^{\mathcal{M}})_{p \in \mathcal{P}},
	  (f^{\mathcal{M}})_{f \in \mathcal{F}} \right),
  \]
  where $M$ is the underlying set, $p^{\mathcal{M}}: M^{\Ar p} \to \{\top,
  \bot\}$ is the interpretation of predicate $p$, and $f^{\mathcal{M}}: M^{\Ar
	f} \to M$ is the interpretation of function symbol $f$.
\end{definition}

A term $t$ is interpreted in a structure $\mathcal{M}$ as follows.
We assume an \pojem{environment} $\rho: \Vars \to M$, then define
$t_{\mathcal{M}}^\rho$ by
\begin{itemize}
\item a variable is interpreted as $x^\rho = \rho(x)$,
\item a function symbol is interpreted as $\left( f(t_1, \ldots, t_n)
  \right)^\rho = f^{\mathcal{M}}(t_1^\rho, \ldots, t_n^\rho)$.
\end{itemize}

The \pojem{satisfaction relation} $\mathcal{M} \models_\rho A$ then states
that formula $A$ is true in $\mathcal{M}$ under environment $\rho$.
We define it with
\begin{itemize}
\item $\mathcal{M} \models_\rho t_1 = t_2$ if and only if $t_1^\rho = t_2^\rho$,
\item for a predicate $p$, $\mathcal{M} \models_\rho p(t_1, \ldots, t_n)$ if and
  only if $p^{\mathcal{M}}(t_1^\rho, \ldots, t_n^\rho) = \top$,
\item the logical operations are interpreted as usual, so for example
  \[
	(\mathcal{M} \models_\rho A \implies B)
	:\iff \left(
	  \mathcal{M} \nmodels_\rho A \lor \mathcal{M} \models_\rho B
	\right)
  \]
\item The universal quantifier is interpreted as
  \[
	(\mathcal{M} \models_\rho \forall x \velja A)
	:\iff \forall d \in M \pvelja \mathcal{M} \models_{\rho[x \mapsto d]} A
  \]
  where $\rho[x \mapsto d]$ means that we use the environment $\rho$, except we
  replace the value of $x$ with $d$, so
  \[
	\rho[x \mapsto d](y) =
	\begin{cases}
	  \rho(y) & \text{if $y$ is not $x$}, \\
	  d & \text{if $y$ is $x$}
	\end{cases}
  \]
  We define the interpretation of the existential quantifier similarly.
\end{itemize}

\begin{example}
  The structure discussed above is $\mathcal{N} = (\N, 0, +, 1, \cdot)$.
  We also have a related structure, $\mathcal{R} = (\R, 0, +, 1, \cdot)$.
\end{example}

\begin{lemma}
  Let $A$ be formula in structure $\mathcal{M}$ and $\FV(A)$ the set of free
  variables in $A$.
  If $\left. \rho \right|_{\FV(A)} = \left. \rho' \right|_{\FV(A)}$, then
  $\mathcal{M} \models_\rho A$ if and only if $\mathcal{M} \models_{\rho'} A$.
\end{lemma}

\begin{corollary}
  If $A$ is a sentence, then $\mathcal{M} \models_\rho A$ is independent of
  $\rho$.
  In this case, we write $\mathcal{M} \models A$.
\end{corollary}

\begin{definition}
  The \pojem{theory} $\Th(\mathcal{M})$ of a structure $\mathcal{M}$ is the set
  of all sentences that $\mathcal{M}$ satisfies.
\end{definition}

\begin{remark}
  We can restate the Church-Turing theorem to \enquote{$\Th(\mathcal{N})$ is
	undecidable,} and Tarski's theorem to \enquote{$\Th(\mathcal{R})$ is
	decidable.}
\end{remark}

\begin{proposition}[law of the excluded middle]
  For any sentence $A$, either $M \models A$ or $M \models \lnot A$.
\end{proposition}

\begin{remark}
  This says that the theory of an individual structure is \pojem{complete}.
\end{remark}

\begin{example}
  Consider the signature $(e, m)$, where $e$ is a constant and $m$ is a function
  symbol of arity $2$.
  We have several interesting classes of structures on this signature:
  $\cls{Group}$ (the class of groups), $\cls{AbGroup}$ (the class of Abelian
  groups), \ldots
\end{example}

\begin{example}
  For the signature $(\le)$, where $\le$ is a binary relation, we also have
  several classes of structures: $\cls{PartOrd}$ (the partial orderings),
  $\cls{LinOrd}$ (the linear orderings), \ldots
\end{example}

\begin{definition}
  Let $\cls{M}$ be a class of structures.
  Then $\cls{M} \models A$ if and only if for every $\mathcal{M} \in \cls{M}$,
  we have $\mathcal{M} \models A$.
  We say that $A$ is \pojem{valid} in $\cls{M}$.
\end{definition}

\begin{definition}
  The \pojem{theory} $\Th(\cls{M})$ of a collection of structures $\cls{M}$ is
  the set of all valid sentences in $\cls{M}$.
\end{definition}

\begin{example}
  For $\cls{Group}$, the sentence $\forall x \exists y \velja m(x,y) = e \land
  m(y,x) = e$ is valid, but the sentence $\exists x \velja x \ne e \land m(x,x)
  = e$ is not valid, and neither is its negation.
\end{example}

So for the theory of a collection, it is in general not (necessarily) true that
either $A$ is valid or $\lnot A$ is valid.
The theory of a collection is not complete.

\begin{theorem}[Szmielew]
  The theory $\Th(\cls{AbGroup})$ is decidable.
\end{theorem}

\begin{theorem}[Novikov]
  The theory $\Th(\cls{Group})$ is undecidable.
\end{theorem}

\begin{remark}
  Novikov proved that even a simple case, the \enquote{word problem for groups}
  is undecidable.
\end{remark}

A set of \pojem{axioms} is a set $S$ of sentences which determine a class of
structures
\[
  \cls{Mod}(S) = \{ \mathcal{M} \such \forall A \in S \pvelja \mathcal{M}
  \models A \}
\]
called the \pojem{collection of models} of $S$.

\begin{example}
  If
  \[
	S_G =
	\begin{Bmatrix}
	  \forall x \velja  m(x,e) = x \land m(e,x) = x, \\
	  \forall x, y, z \velja m(x, m(y,z)) = m(m(x,y), z), \\
	  \forall x \exists y \velja  m(x,y) = e \land m(y,x) = e
	\end{Bmatrix},
  \]
  then $\cls{Mod}(S) = \cls{Group}$.
\end{example}

\begin{definition}
  A sentence $A$ is a \pojem{logical consequence} if
  \[
	S \models A \iff \forall \mathcal{M} \in \cls{Mod}(S) \pvelja \mathcal{M}
	\models A
  \]
  holds.
  We say that $S$ \pojem{entails} $A$.
\end{definition}

\begin{remark}
  A sentence $A$ is a logical consequence of $S$ if and only if $A \in
  \Th(\cls{Mod}(S))$.
\end{remark}

\begin{remark}
  For a set of axioms $S$, we define $\Th(S)$ to be $\Th(\cls{Mod}(S))$.
\end{remark}

\begin{definition}
  A sentence $A$ is \pojem{logically valid} if it satisfies the empty set of
  axioms, $\varnothing \models A$, so if $\mathcal{M} \models A$ for every
  $\mathcal{M}$.
  We label this with $\models A$.
\end{definition}

\begin{definition}
  A set of sentences $T$ is a \pojem{theory} if it is closed under logical
  consequence, i.e.~if for all sentences $A$, if $T \models A$, then $A \in T$.
\end{definition}

\begin{definition}
  A theory $T$ is \pojem{complete} if for every sentence $A$ either $A \in T$ or
  $\lnot A \in T$.
\end{definition}

\begin{definition}
  A theory $T$ is \pojem{consistent} if for every sentence $A$ at most one of
  $A, \lnot A$ belong to $T$.
\end{definition}

\begin{remark}
  A theory $T$ is inconsistent if and only if $T$ is the set of all sentences.
\end{remark}

A great contribution of logic is the notion of formal proof.
Using this we write precise finite proofs showing that a sentence $A$ follows
from an agreed set of axioms.
We define the provability relation $S \vdash A$ which states that \enquote{$A$
  is derivable from the axioms $S$.}

\begin{theorem}[soundness theorem]
  For all axioms $S$ and sentences $A$, if $S \vdash A$, then $S \models A$.
\end{theorem}

\begin{theorem}[completeness theorem]
  For all axioms $S$ and sentences $A$, if $S \models A$, then $S \vdash A$.
\end{theorem}

\begin{remark}
  This is the completeness of the proof system; it means that we haven't
  forgotten any means of proof.
  It is different than the incompleteness theorem (both are Gödel's).
\end{remark}

\begin{remark}
  As an application, to verify a sentence $A$ for $\Th(\cls{Group})$, we just
  need to construct a formal proof showing that $S_G \vdash A$.
  We know that such a proof exists.
\end{remark}

We've seen that the theory of natural numbers is undecidable.
Can we at least find a complete set of axioms like we did for groups?
There is a natural set of axioms $\mathit{PA}$ (Peano's) for $\Th(\mathcal{N})$,
which axiomatises Peano arithmetic.
In practice, many known theorems of number theory are provable in $\mathit{PA}$,
for example that there are infinitely many primes, or the prime number theorem.

\begin{theorem}[Gödel's first incompleteness theorem]
  One can find a sentence $G$ such that
  \begin{itemize}
  \item $\mathcal{N} \models G$,
  \item $PA \nvdash G$,
  \item $PA \nvdash \lnot G$.
  \end{itemize}
\end{theorem}

\begin{remark}
  The theory $\Th(\mathit{PA})$ is incomplete.
\end{remark}

One possible interpretation of this is that perhaps we've just missed some
necessary axioms.
But as it turns out, whatever axioms $S$ we end up with, if $S$ is decidable,
then we can repeat the theorem.

% LocalWords:  Gödel Tarski Tarski's Szmielew Novikov Gödel's Peano's Peano
% LocalWords:  axiomatises axiomatise
