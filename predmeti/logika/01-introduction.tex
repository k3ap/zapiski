\naslov{Introduction}

First-order logic is a formal language for expressing properties of mathematical
statements.
The language is determined by a \enquote{signature.}

\begin{example}
  Consider the following arithmetic signature, consisting of:
  \begin{itemize}
  \item two constants $0$, $1$,
  \item two binary function symbols $+$, $\cdot$.
  \end{itemize}
  Given the signature, we have a derived notions of \enquote{terms} and
  \enquote{formulas.}
  In this case, terms represent numbers, for example $0, 1, x, x \cdot (y+z),
  x \cdot x \cdot x + 1 \cdot 0, \ldots$ are all terms of this language.
  On the other hand, formulas represent properties of their free variables.
  As an example, the formula $\exists z. y = x + z$ expresses the property $x
  \le y$.
  Here, $z$ is a bound variable, and $x, y$ are free variables.
  \boxdot{}
\end{example}

A formula with free variables is called a \enquote{sentence}.
When a sentence is interpreted in a structure, it is either true or false.

\begin{example}
  Fermat's last theorem can be expressed as
  \[
	\forall n. n > 2 \implies \lnot \left(
	  \exists x, y, z. x > 0 \land y > 0 \land z > 0
	  \land x^n + y^n = z^n
	\right)
  \]
  This uses exponentiation, which we can expresses in the arithmetic language,
  as was shown by Gödel.
  The sentence above is true, and the proof is left as an exercise for the
  reader.
  \boxdot{}
\end{example}

\begin{example}
  The sentence
  \[
	x > 1 \land \forall y, z . x = y \cdot z \implies x = y \lor x = z
  \]
  encodes \enquote{$x$ is prime.}
  We can use this for example to express some open problems, such as the twin
  prime conjecture,
  \[
	\forall x . \exists y. y > x \land \operatorname{Prime}(y) \land
	\operatorname{Prime}(y+2).
  \]
  I need some text to put a boxdot here.
  \boxdot{}
\end{example}

The richness of the examples suggests that the following is a difficult
question:
\begin{quote}
  Given an input sentence $\phi$, decide whether $\phi$ is true or false in
  $\N$, i.e.~if $\N \models \phi$.
\end{quote}

\begin{theorem}[Church, Turing]
  The above question is undecidable.
\end{theorem}

We can interpret our arithmetic signature in any ring with a unit, such as in
the real numbers.
Then, a formula with $n$ free variables describes an $n$-dimensional geometric
shape (actually, a subset of $\R^n$).
This signature, interpreted in $\R$, gives a language for Euclidean geometry.
But more important is the following.

\begin{theorem}[Tarski]
  The problem, \enquote{given a sentence $\phi$, decide if $\R \models \phi$,}
  is decidable.
\end{theorem}

\begin{definition}
  A \pojem{signature} is a combination of a set of predicate (or relation)
  symbols $\mathcal{P}$ and function symbols $\mathcal{F}$, together with an
  arity function $\Ar: \mathcal{P} \dcup \mathcal{F} \to \N$, which returns the
  number of arguments that a symbol takes.
\end{definition}

\begin{example}
  The signature discussed above has $\mathcal{P} = \varnothing$, $\mathcal{F} =
  \{ 0, 1, \cdot, + \}$, and $\Ar$ defined with
  \begin{equation*}
	0 \mapsto 0 \qquad
	1 \mapsto 0 \qquad
	+ \mapsto 2 \qquad
	\cdot \mapsto 2
  \end{equation*}
\end{example}

\begin{definition}
  A function symbol of arity $0$ is called a \pojem{constant}.
\end{definition}

\begin{example}
  Another important signature is $(\{E\}, \varnothing)$ with $\Ar(E) = 2$.
  This is the signature for interpreting in graphs.
\end{example}

\begin{example}
  Yet another important signature is $(\{\in\}, \varnothing)$ with
  $\Ar(\in) = 2$.
  This is the signature of set theory.
  Note that this is just the same signature as above.
\end{example}

\begin{definition}
  \pojem{Terms} are defined by the following two rules:
  \begin{itemize}
  \item a variable is a term,
  \item if $t_1, \ldots, t_n$ are terms and $f \in \mathcal{F}$ is a function
	symbol of arity $n$, then $f(t_1, \ldots, t_n)$ is a term.
  \end{itemize}
\end{definition}

\begin{definition}
  \pojem{Formulas} are defined by the following rules:
  \begin{itemize}
  \item if $t_1, t_2$ are terms, then $t_1 = t_2$ is a formula,
  \item if $t_1, \ldots, t_n$ are terms and $p \in \mathcal{P}$ is a predicate
	symbol of arity $n$, then $p(t_1, \ldots, t_n)$ is a formula,
  \item if $\phi, \psi$ are formulas, then so are
	$\lnot \phi$, $\phi \land \psi$, $\phi \lor \psi$ and $\phi \implies \psi$,
  \item if $\phi$ is a formula, then so are $\exists x . \phi$ and $\forall x . \phi$.
  \end{itemize}
  Formulas of the first two types are called \pojem{atomic formulas}.
\end{definition}

\begin{definition}
  A \pojem{structure} for a signature $(\mathcal{P}, \mathcal{F})$ is a triple
  \[
	\mathcal{M} = \left( M, (p^{\mathcal{M}})_{p \in \mathcal{P}},
	  (f^{\mathcal{M}})_{f \in \mathcal{F}} \right),
  \]
  where $M$ is the underlying set, $p^{\mathcal{M}}: M^{\Ar p} \to \{\top,
  \bot\}$ is the interpretation of predicate $p$, and $f^{\mathcal{M}}: M^{\Ar
	f} \to M$ is the interpretation of function symbol $f$.
\end{definition}

A term $t$ is interpreted in a structure $\mathcal{M}$ as follows.
We assume an \pojem{environment} $\rho: \Vars \to M$, then define
$t_{\mathcal{M}}^\rho$ by
\begin{itemize}
\item a variable is interpreted as $x^\rho = \rho(x)$,
\item a function symbol is interpreted as $\left( f(t_1, \ldots, t_n)
  \right)^\rho = f^{\mathcal{M}}(t_1^\rho, \ldots, t_n^\rho)$.
\end{itemize}

The \pojem{satisfaction relation} $\mathcal{M} \models_\rho \phi$ then states
that formula $\phi$ is true in $\mathcal{M}$ under environment $\rho$.
We define it with
\begin{itemize}
\item $\mathcal{M} \models_\rho t_1 = t_2$ if and only if $t_1^\rho = t_2^\rho$,
\item for a predicate $p$, $\mathcal{M} \models_\rho p(t_1, \ldots, t_n)$ if and
  only if $p^{\mathcal{M}}(t_1^\rho, \ldots, t_n^\rho) = \top$,
\item the logical operations are interpreted as usual, so for example
  \[
	(\mathcal{M} \models_\rho \phi \implies \psi)
	:\iff \left(
	  \mathcal{M} \nvDash_\rho \phi \lor \mathcal{M} \models_\rho \psi
	\right)
  \]
\item The universal quantifier is interpreted as
  \[
	(\mathcal{M} \models_\rho \forall x. \phi)
	:\iff \forall d \in M . \mathcal{M} \models_{\rho[x \mapsto d]} \phi
  \]
  where $\rho[x \mapsto d]$ means that we use the environment $\rho$, except we
  replace the value of $x$ with $d$, so
  \[
	\rho[x \mapsto d](y) =
	\begin{cases}
	  \rho(y) & \text{if $y$ is not $x$}, \\
	  d & \text{if $y$ is $x$}
	\end{cases}
  \]
  We define the interpretation of the existential quantifier similarly.
\end{itemize}

% LocalWords:  Gödel Tarski
