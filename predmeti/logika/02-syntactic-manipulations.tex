\naslov{Syntactic Manipulations}

Consider the language of natural numbers.
We use the following abbreviations:
\begin{itemize}
\item $\exists x > t \velja A$ abbreviates $\exists x \velja x > t \land A$,
\item $\forall x > t \velja A$ abbreviates $\forall x \velja x > t \implies A$
\end{itemize}
Here, $t$ is not a logical variable, but rather a meta-variable; that is, we can
substitute it for any term.
You can of course use meta-variables in different situations, for example, to be
substituted by formulas or just regular variables.
We will use $s, t, u, \ldots$ as meta-variables for terms, $A, B, C, \ldots$ for
formulas, $x, y, z, \ldots$ for variables, $P, Q, R, \ldots$ for relation
symbols, $f, g, \ldots$ for function symbols etc.
Note that $y > x$ is itself an abbreviation of $\exists z \velja y = x+z+1$.

Let $A$ be a formula, $t$ a term and $x$ a free variable.
By $A[t/x]$, we denote the formula we get from substituting $t$ for $x$
in $A$.
Similarly, if $s$ is a term, then $s[t/x]$ denotes the term we get by
substituting $t$ for the variable $x$ in $s$.

\begin{example}
  The abbreviation $\operatorname{Prime}(y)$ means $\lnot(y=1) \land \forall x
  \forall z \velja y = x \cdot z \implies (x = 1 \lor z = 1)$.
  We can substitute $\operatorname{Prime}(y)[y+2/y]$, but we must be careful
  to rename the bound variable $x$ when substituting
  $\operatorname{Prime}(y)[x/y]$.
\end{example}

More formally, we define $s[t/x]$ as the term we get when we literally replace
all occurrences of the variable $x$ with the term $t$.
For $A[t/x]$, we rename all bound variables in $A$ so that no bound
variable occurs in the set $\Vars(t) \cup \{x\}$.
Then we replace all remaining occurrences of $x$ by $t$.

Even more formally, for a variable $y$, we define
\[
  y[t/x] :=
  \begin{cases}
	t & \text{if $y$ is syntactically equal to $x$} \\
	y & \text{otherwise}
  \end{cases}
\]
and for a function symbol,
\[
  (f(s_1, \ldots, s_k))[t/x] := f(s_1[t/x], \ldots, s_k[t/x]),
\]
which covers $s[t/x]$ for any term $s$.
For a formula $A$, we can write a similar recursive definition of $A[t/x]$.
The only interesting part of the definition are the quantifiers,
\[
  (\forall y \velja A)[t/x] := \forall y \velja A[t/x]
\]
assuming we've already renamed the bound variables.
If we haven't, we can use the following definition:
\[
  (\forall y \velja A)[t/x]
  :=
  \begin{cases}
	\forall y \velja A & \text{if $y$ is literally $x$} \\
	\forall y \velja A[t/x] & \text{if $y \notin \Vars(t)$ and $y$ is not $x$} \\
	\forall w \velja A[w/y][t/x] & \text{otherwise}
  \end{cases}
\]
where $w$ is some chosen fresh $w$ (that isn't $x$ or used in $A$ or $t$).

Substitution has a semantic interpretation,
\[
  \left( S[t/x] \right)^\rho_{\mathcal{M}}
  = S^{\rho[x \mapsto t^\rho]}_{\mathcal{M}},
\]
and
\[
  \mathcal{M} \models_\rho A[t/x]
  \iff \mathcal{M} \models_{\rho[x \mapsto t^\rho]} A.
\]

\begin{definition}
  Two formulas $A, B$ are \pojem{logically equivalent} ($A \equiv B$) if the
  formula $A \iff B$ is valid.
  Equivalently, if for all $\mathcal{M}$ and $\rho$, $\mathcal{M} \models_\rho
  A$ if and only if $\mathcal{M} \models_\rho B$.
\end{definition}

\begin{example}
  For any formula $A$, $\top \equiv A \lor \lnot A$ and $\bot \equiv A \land
  \lnot A$.
\end{example}

\begin{example}
  For any formulas $A, B$, we have $A \iff B \equiv (A \implies B) \land (B
  \implies A)$.
\end{example}

\begin{example}
  If we have negation, then we need only one of $\land, \lor, \implies$, we can
  express one with another.
  We also only need one quantifier.
\end{example}

As discussed above, combining $\lnot$ with one of $\land, \lor, \implies$ and
one of $\forall, \exists$, will give us a \pojem{sufficient set of connectives
  and quantifiers}.

\podnaslov{Normal forms}

A formula is in \pojem{prenex normal form} (PNF) if it has the form $Q_1 x_1 Q_2
x_2 \ldots Q_k x_k B$, where $Q_i$ are quantifiers and $B$ is quantifier free.
We include the possibility of $k = 0$.

\begin{proposition}
  Every formula $A$ is equivalent to a formula $A^{\text{PNF}}$ in prenex normal
  form.
\end{proposition}

\begin{proof}
  We restrict to formulas over the sufficient set $\lnot, \lor, \exists$ (for
  the input only).
  We will describe a recursive algorithm that computes an equivalent formula in
  prenex formal form.

  For $(A \lor A')^{\text{PNF}}$, first recursively compute $A^{\text{PDF}} = Q_1 x_1
  \ldots Q_k x_K B$ and $A'^{\text{PNF}} = Q_1' x_1' \ldots Q_{k'}' x_{k'}' B'$,
  then return $Q_1 x_1 \ldots Q_k x_k Q_1' x_1' \ldots Q_{k'}' x_{k'}' (B \lor
  B')$.
  We of course assume there is no overlap in bound variables.

  The PNF of $\exists x \velja A$ is $\exists x \velja A^{\text{PNF}}$.
  Finally, for negation $\lnot A$, recursively compute $A^{\text{PNF}} = Q_1 x_1
  \ldots Q_k x_k B$, then switch each quantifier to $\bar{Q}_1 x_1 \ldots
  \bar{Q}_k x_k (\lnot B)$.
\end{proof}

For quantifier free formulas, we consider normal forms with connectives $\land,
\top, \lor, \bot, \lnot$.
Recall that an atomic formula is one of $t_1 = t_2$ or $P(t_1, \ldots, t_n)$.
We say that a \pojem{literal} is either an atomic formula or its negation.

A formula is in \pojem{negation normal form} if it is built from literals
using only conjunctions and disjunctions, i.e.~if $\lnot$ occurs only before an
atomic formula.
We consider $\top$ as a conjunction and $\bot$ as a disjunction.

\begin{proposition}
  Every quantifier free formula $A$ is equivalent to a formula $A^{\text{NNF}}$
  in negation normal form.
\end{proposition}

\begin{proof}
  Use De Morgan's laws.
\end{proof}

A formula is in \pojem{conjunctive normal form} if it is a conjunction of
disjunctions of literals, i.e.~of the form $D_1 \land \ldots \land D_k$, where
$k \ge 0$ and each $D_i$ is of the form $L_{i1} \lor \ldots \lor L_{i l_i}$,
with $L_{ij}$ being literals.
Similarly, a formula is in \pojem{disjunctive normal form} if it is a
disjunction of conjunctions of literals.

\begin{proposition}
  Every quantifier free formula $A$ is equivalent to formulas $A^{\text{DNF}}$
  and $A^{\text{CNF}}$ in disjunctive and conjunctive normal forms,
  respectively.
\end{proposition}

\begin{remark}
  While the negation normal form can be found efficiently, finding conjunctive
  and disjunctive normal forms is an NP-complete problem.
\end{remark}

% LocalWords:  prenex PNF disjunctions disjunction
