\naslov{Connectivity}

\begin{definition}
  The \pojem{connectivity number} $\kappa(G)$ is the minimum number of vertices
  in $S \subseteq V(G)$ such that $G - S$ is either disconnected or contains
  only one vertex.
\end{definition}

\begin{definition}
  A graph $G$ is \pojem{$k$-connected} if $\kappa(G) \ge k$ or if the removal of
  $k-1$ vertices always results in a connected graph with at least two vertices.
\end{definition}

\begin{remark}
  In any graph, $\kappa(G) \le \delta(G)$.
  If $G$ is complete, then $\delta(G) = n-1 = \kappa(G)$.
\end{remark}

\begin{remark}
  If $A$ is an independent set, removing every other vertex gives a disconnected
  graph, so $\kappa(G) \le n-\alpha(G) = \beta(G)$.
\end{remark}

\begin{theorem}
  The minimum number of edges in a $k$-connected graph of order $n$ is
  $\ceil{\frac{nk}{2}}$, if $n > k \ge 2$.
\end{theorem}

\begin{proof}
  If the graph is $k$-connected, then $k \le \kappa(G) \le \delta(G)$, so
  \[
	m(G) = \pol \sum_{v \in V} \deg_G(v) \ge \frac{nk}{2}.
  \]
  We will show that there exists a $k$-connected graph of order $n$ with the
  specified number of edges.
  For this, we define Harary graphs $H_{n,k}$ as follows.
  \begin{itemize}
  \item If $k$ is even, $H_{n,k} = C_n^{k/2}$ (i.e.~the graph we get by
	connecting all vertices from $C_n$ which are at a distance of at most $k/2$).
  \item If $k$ is odd and $n$ is even, $H_{n,k}$ is $C_n^{(k-1)/2}$, along with
	all edges between two opposing vertices of the cycle.
  \item If both $n$ and $k$ are odd, then we again take $C_n^{(k-1)/2}$ and add
	the edges between $i$ and $i + \frac{n-1}{2}$ (if we label the vertices $0,
	1, \ldots, n-1$).
  \end{itemize}
  All these graphs have $m(H_{n,k}) = \ceil{\frac{nk}{2}}$.

  We will show that all these graphs are $k$-connected.
  For the first case, if $k$ is even, let $S$ be a vertex set with $\abs{S} =
  k-1$.
  We define a big gap as $\frac{k}{2}$ consecutive vertices in $S$, and claim
  the following:
  \begin{itemize}
  \item If there is no big gap between $u$ and $v$ in a certain direction, then
	we may find a $uv$-path in that direction.
	This is clear from the construction.
  \item For any $u,v \in V(G) \setminus S$, there is a $uv$-path in $G - S$.
	Since there can be only one big gap between them, we can just avoid it by
	going in the other direction, so this is also clear.
  \end{itemize}

  If $k$ is odd and $n$ even, we define a big gap as $\frac{k-1}{2}$ consecutive
  missing vertices.
  Similarly as before, if there is no big gap on a path from $u$ to $v$, we can
  find a path after removing $S$.
  But in this case, both paths may contain a big gap.
  Let $P$ and $Q$ be the two paths along the cycle.
  If there are big gaps along both, we know the length of both is at least
  $\frac{k-1}{2} + 1$.
  Let $u'$ and $v'$ be the opposite vertices of $u$ and $v$.
  Suppose that $Q$ is longer than $P$, and split it into paths $Q_1$, $Q_2$ and
  $Q_3$ by $v'$ and $u'$.

  Note that by symmetry, the length of $Q_2$ (the center region) is also at
  least $\frac{k+1}{2}$, so the big gap in $Q$ cannot cover both $u'$ and $v'$.
  We can then find a $u,v$-path in $G - S$ using one of these vertices.

  We can consider the case of odd $k$ and odd $n$ similarly, it's just more
  annoying to write down.
  In all cases, all graphs have precisely $\ceil{\frac{kn}{2}}$ edges.
\end{proof}

\begin{definition}
  A set $F \subseteq E(G)$ is a \pojem{disconnecting set} if $G - F$ is
  disconnected.
\end{definition}

\begin{definition}
  An \pojem{edge cut} of $A$ is the set $E(A, \cl{A})$ of edges between $A$ and
  $\cl{A}$.
\end{definition}

\begin{remark}
  An edge cut is a disconnecting set.
  A minimal disconnected set is an edge cut.
\end{remark}

\begin{definition}
  The \pojem{edge-connectivity} number of $G$ is the minimum number of edges in
  a disconnecting set.
  We denote it by $\kappa'(G)$.
  A graph is \pojem{$k$-edge-connected} if the removal of less than $k$ edges
  always leaves a connected graph.
\end{definition}

\begin{theorem}
  Suppose $G$ is a simple graph with $n(G) \ge 2$.
  Then $\kappa(G) \le \kappa'(G) \le \delta(G)$.
\end{theorem}

\begin{proof}
  The second inequality is clear.
  Consider a minimum edge cut $E(A, \cl{A})$ in $G$.
  By definition $\abs{E(A, \cl{A})} = \kappa'(G)$.
  We will show that there is a vertex cut with at most $\kappa'(G)$ vertices.
  For that, consider two cases.
  If $E(A, \cl{A})$ forms a complete bipartite graph, then
  \[
	\abs{E(A, \cl{A})} =
	\abs{A} \abs{\cl{A}} = \abs{A} (n - \abs{A}).
  \]
  Since $0 < \abs{A} < n$, we have $\abs{E(A, \cl{A})} \ge n-1$, so $\kappa'(G)
  \ge n-1$, but $\kappa(G) \le n-1$ and $\kappa(G) \le \kappa'(G)$.

  In the second case, if there are vertices $x \in A$ and $y \in \cl{A}$ which
  are nonadjacent, then we may choose an endpoint different from $x$ and $y$ in
  each edge in the edge cut.
  This gives us a vertex cut with at most $\abs{E(A, \cl{A})}$ vertices, in
  which $x$ and $y$ are not cut, and are disconnected.
\end{proof}

\begin{corollary}
  If $G$ is $k$-connected, then $G$ is $k$-edge-connected.
\end{corollary}

\begin{corollary}
  The minimum number of edges in a $k$-edge-connected graph on $n$ vertices is
  $\ceil{\frac{kn}{2}}$.
\end{corollary}

\begin{proof}
  We know that $k \le \kappa'(G) \le \delta(G)$, so
  \[
	m(G) = \pol \sum_{v \in V} \deg_G(v) \ge \pol n \delta(G)
  \]
  which means $m(G) \ge \ceil{\frac{nk}{2}}$.
  For the other direction, note that the Harary graphs are $k$-edge-connected.
\end{proof}

\begin{theorem}[Whitney]
  If $G$ is a $2$-connected graph, then for every $u, v \in V(G)$, there are two
  internally disjoint $u,v$-paths.
  The converse also holds.
\end{theorem}

\begin{proof}
  For the left implication, suppose $u, v \in V(G - x)$ for some vertex $x$.
  There are two disjoint paths from $u$ to $v$ in $G$, and at most one of them
  includes $x$, so there is a path from $u$ to $v$ in $G - x$.
  This means our graph has no cut vertex, so it is $2$-connected.

  Now consider the right implication.
  Let $u, v$ be vertices in $G$.
  Induction on $d = d(u,v)$.
  For $d = 1$, we know that $G$ is $2$-edge-connected (since it is
  $2$-connected), so there is no bridge in $G$.
  If we remove $uv$, then the graph must still be connected.
  Therefore, there is another $u,v$-path in $G$.

  For a general $d$, let $w$ be the neighbour of $v$ on the shortest $u,v$-path.
  Then $d(u,w) = d-1$.
  By the induction hypothesis, there are two internally disjoint paths $P, Q$
  from $u$ to $w$.
  Consider two cases.
  If $v \in V(P)$ (or $Q$, symmetrically), then we have two internally disjoint
  $u,v$-paths in
  \begin{alignat*}{2}
	u \xrightarrow{P} v,
	& \qquad &
			   u \xrightarrow{Q} w \to v.
  \end{alignat*}

  Otherwise, if $v$ is on neither path, then consider the graph $G - w$.
  It is still connected, so there is at least one $u,v$-path $R$ in this graph.
  If $R$ shares no internal vertex with $P$ or $Q$, then we may choose $u
  \xrightarrow{P} w \to v$ and $R$ as the two paths.
  Finally, if $R$ intersects $P$ and/or $Q$, identify the last intersection with
  either, and label it $z$.
  Without loss of generality, $z \in Q \cap R$.
  Then we have two paths
  \begin{alignat*}{2}
	u \xrightarrow{P} w \to v,
	& \qquad &
			   u \xrightarrow{Q} z \xrightarrow{R} v.
  \end{alignat*}
\end{proof}

\begin{theorem}[Expansion lemma]
  If $G$ is $k$-connected and we add a new vertex $v$ and $k$ incident edges to
  the graph, then we obtain a $k$-connected graph.
\end{theorem}

\begin{proof}
  Call the new vertex $y$, and let $G'$ be the new graph.
  We will prove that every vertex cut in $G'$ contains at least $k$ vertices.
  For that, let $S$ be a vertex cut in $G'$.
  Consider three cases.
  \begin{itemize}
  \item If $y \in S$, then let $V_1$ and $V_2$ be components of $G \setminus S$.
	Every path from $V_1$ to $V_2$ passes $S$, which is also true in $G$, since
	$y \in S$.
	In $G$, every vertex cut contains at least $k$ vertices, so $S \setminus
	\{y\}$ contains at least $k$ vertices, and $S$ contains at least $k+1$
	vertices.

  \item If $y \notin S$ and $N(y) \subseteq S$, then $y$ is its own component in
	$G' \setminus S$.
	This means $\abs{S} \ge k$.

  \item Otherwise, there is a vertex $y'$ which is a neighbour of $y$ and
	belongs to the same component in $G' \setminus S$.
	If we remove $y$ from $G'$, $S$ remains a vertex cut in $G$, as no path
	exists between the components which avoids $S$.
	Note that we don't remove the component of $y$ if we delete the vertex, as
	the component has at least one other vertex ($y'$).
	Since $G$ is $k$-connected, every vertex cut contains at least $k$ vertices,
	so $\abs{S} \ge k$.
  \end{itemize}
\end{proof}

\begin{theorem}
  If $G$ is a graph with $n(G) \ge 3$, then the following statements are
  equivalent:
  \begin{itemize}
  \item $G$ is $2$-connected,
  \item $G$ is connected and there is no cut vertex,
  \item for every $u,v \in V(G)$, there are at least two internally disjoint
	$u,v$-paths,
  \item for every $u,v \in V(G)$, there is a cycle through $u$ and $v$,
  \item $\delta(G) \ge 1$ and, for every two edges $e_1, e_2$ there is a cycle
	containing $e_1$ and $e_2$.
  \end{itemize}
\end{theorem}

\begin{proof}
  We already know the first four statements are equivalent.
  Suppose $G$ is $2$-connected.
  Take edges $e, f \in E(G)$ and label $e = uv, f = u'v'$.
  Now add two vertices $w, w'$ to $G$, $w$ connected to $u$ and $v$, and $w'$
  connected to $u'$ and $v'$.
  Let $G'$ be the resulting graph.
  By the expansion lemma, $G'$ is $2$-connected, and satisfies the fourth
  condition, so there is a cycle though $w$ and $w'$.
  This cycle must contain all the new edges and the vertices $u,v,u'v'$.
  We can now replace the added edges with $e$ and $f$.

  Conversely, take vertices $u,v \in V(G)$.
  Now let $e = uu'$ and $f = vv'$ be two different edges.
  There is a cycle though $e$ and $f$ by assumption, so there is a cycle though
  $u$ and $v$.
\end{proof}

\begin{proposition}[subdivision lemma]
  Suppose $G'$ is obtained from $G$ by subdividing an edge $uv \in E(G)$ with a
  vertex $w$.
  Then $G$ is $2$-connected if and only if $G'$ is $2$-connected.
\end{proposition}

\begin{proof}
  Left to right:
  Consider two vertices $x, y \in V(G')$.
  If neither is equal to $w$, then we can take the cycle though $xy$
  in $G$.
  If it contains $uv$, we can replace it with the path $uwv$.
  If one of the vertices $x,y$ is $w$, then we find a cycle through $u$
  and $v$, and subdivide the edge.

  Right to left:
  Any cycle in $G'$ that contains $w$ must also contain $u$ and $v$.
\end{proof}

\podnaslov{Ear decomposition of a graph}

\begin{definition}
  In a graph $G$, a path $P$ is an \pojem{(open) ear} if all internal vertices
  of $P$ are of degree $2$ in $G$, and for the end vertices of the path, the
  degree is at least $3$.
\end{definition}

\begin{definition}
  An \pojem{(open) ear decomposition} of $G$ is a sequence $P_0, P_1, \ldots,
  P_k$, where $P_0$ is a cycle in $G$, and every other $P_i$ is an ear in the
  graph $G_i := P_0 \cup P_1 \cup \ldots \cup P_i$.
  We also require $G_k = G$.
\end{definition}

\begin{theorem}
  A graph $G$ is $2$-connected if and only if it has an ear decomposition.
\end{theorem}

\begin{proof}
  Right to left:
  We will prove that $G_i$ is $2$-connected for every $i$ by induction.
  For $i=0$, we know that a cycle is $2$-connected.
  For the induction step, let $u,v$ be the endpoints of $P_{i+1}$.
  Add an edge $uv$ to $G_i$.
  It is still $2$-connected, as adding an edge cannot decrease connectivity.
  Now we can repeatedly subdivide this edge, and the resulting graph is still
  $2$-connected by the subdivision lemma.

  Left to right:
  Since $G$ is $2$-connected, there exists a cycle $C$, which we can take as
  $P_0$.
  We will construct the decomposition inductively.
  If $G_i$ is obtained from $P_0$ by adding the ears $P_1, \ldots, P_i$, then:
  \begin{itemize}
  \item If $G = G_i$, we're done.
  \item If $G_i$ is not an induced subgraph of $G$, we may add an edge from
	$E(G) \setminus E(G_i)$ with both endpoints in $V(G_i)$.
	This is a valid ear (as since $G_i$ is $2$-connected, $\delta(G_i) \ge 2$),
	so we can continue the decomposition.
  \item If $G_i$ is an induced subgraph of $G$, then there exists a vertex $v
	\in V(G) \setminus V(G_i)$ which is connected to a vertex $u \in V(G_i)$.
	Since $G$ is $2$-connected, there is a cycle $C$ though the edge $uv$ and
	some other edge $f \in E(G_i)$.
	Let $u'$ be the first vertex on this cycle which is in $G_i$ on a $u \to v
	\to \cdots$ path.
	We can add this path between $u$ and $u'$ as $P_{i+1}$.
	\qedhere
  \end{itemize}
\end{proof}

\begin{proposition}
  A graph $G$ is $2$-edge-connected if and only if $G$ is connected and there is
  no cut edge in $G$.
\end{proposition}

\begin{definition}
  A \pojem{closed ear} in a graph $G$ is a cycle for which all but one vertex
  has degree $2$, and the exception has degree at least $4$.
\end{definition}

\begin{definition}
  A \pojem{closed ear decomposition} of $G$ is a sequence $P_0, P_1, \ldots,
  P_k$ such that $P_0$ is a cycle and $P_i$ is either an open or closed ear in
  $G_i = P_0 \cup P_1 \cup \ldots \cup P_i$ for all $i \ge 1$.
\end{definition}

\begin{theorem}
  A graph $G$ is $2$-edge-connected if any only if it has a closed ear decomposition.
\end{theorem}

\begin{proof}
  Right to left:
  Since $P_0$ is a cycle, it is $2$-edge-connected.
  If $G_i$ is $2$-edge-connected, so is $G_{i+1}$:
  Take any edge $uv \in E(G_{i+1})$.
  If $uv \in E(G_i)$, then it is part of a cycle in $G_i$.
  If $uv \in E(P_{i+1})$, then either $P_{i+1}$ is a closed ear, so a cycle
  containing $uv$, or it is an open ear between some vertices $x,y$, in which
  case there is an $x,y$-path in $G_i$, which forms a cycle with $P_{i+1}$.

  Left to right:
  Start the closed ear decomposition with a cycle $P_0$, and build inductively.
  If $G_i$ is not an induced subgraph of $G$, then we can add edges with open
  ears.
  If it is, then let $v$ be a vertex in $V(G) \setminus V(G_i)$ connected to a
  vertex $u \in V(G_i)$.
  We know $uv$ is in a cycle in $G$.
  If $u$ is the only vertex of the cycle in $G_i$, then we can add it as a
  closed ear.
  Otherwise, we can add it as an open ear as in the previous theorem.
\end{proof}

% LocalWords:  Harary
