\naslov{Connectivity}

\begin{definition}
  The \pojem{connectivity number} $\kappa(G)$ is the minimum number of vertices
  in $S \subseteq V(G)$ such that $G - S$ is either disconnected or contains
  only one vertex.
\end{definition}

\begin{definition}
  A graph $G$ is \pojem{$k$-connected} if $\kappa(G) \ge k$ or if the removal of
  $k-1$ vertices always results in a connected graph with at least two vertices.
\end{definition}

\begin{remark}
  In any graph, $\kappa(G) \le \delta(G)$.
  If $G$ is complete, then $\delta(G) = n-1 = \kappa(G)$.
\end{remark}

\begin{remark}
  If $A$ is an independent set, removing every other vertex gives a disconnected
  graph, so $\kappa(G) \le n-\alpha(G) = \beta(G)$.
\end{remark}

\begin{theorem}
  The minimum number of edges in a $k$-connected graph of order $n$ is
  $\ceil{\frac{nk}{2}}$, if $n > k \ge 2$.
\end{theorem}

\begin{proof}
  If the graph is $k$-connected, then $k \le \kappa(G) \le \delta(G)$, so
  \[
	m(G) = \pol \sum_{v \in V} \deg_G(v) \ge \frac{nk}{2}.
  \]
  We will show that there exists a $k$-connected graph of order $n$ with the
  specified number of edges.
  For this, we define Harary graphs $H_{n,k}$ as follows.
  \begin{itemize}
  \item If $k$ is even, $H_{n,k} = C_n^{k/2}$ (i.e.~the graph we get by
	connecting all vertices from $C_n$ which are at a distance of at most $k/2$).
  \item If $k$ is odd and $n$ is even, $H_{n,k}$ is $C_n^{(k-1)/2}$, along with
	all edges between two opposing vertices of the cycle.
  \item If both $n$ and $k$ are odd, then we again take $C_n^{(k-1)/2}$ and add
	the edges between $i$ and $i + \frac{n-1}{2}$ (if we label the vertices $0,
	1, \ldots, n-1$).
  \end{itemize}
  All these graphs have $m(H_{n,k}) = \ceil{\frac{nk}{2}}$.

  We will show that all these graphs are $k$-connected.
  For the first case, if $k$ is even, let $S$ be a vertex set with $\abs{S} =
  k-1$.
  We define a big gap as $\frac{k}{2}$ consecutive vertices in $S$, and claim
  the following:
  \begin{itemize}
  \item If there is no big gap between $u$ and $v$ in a certain direction, then
	we may find a $uv$-path in that direction.
	This is clear from the construction.
  \item For any $u,v \in V(G) \setminus S$, there is a $uv$-path in $G - S$.
	Since there can be only one big gap between them, we can just avoid it by
	going in the other direction, so this is also clear.
  \end{itemize}
\end{proof}

% LocalWords:  Harary
