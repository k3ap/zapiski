\naslov{Planar graphs}

\begin{definition}
  If $G$ is a graph, then the \pojem{drawing} of $G$ into the plane is a
  function $h(x)$ defined on $V(G) \cup E(G)$ such that for every vertex $v$,
  $h(v)$ is a point and for every edge $uv$, $h(uv)$ is a $h(u),h(v)$-curve.
\end{definition}

\begin{remark}
  We can assume without loss of generality that the image of an edge is a
  polygonal curve, so composed of finitely many line segments.
\end{remark}

\begin{definition}
  The \pojem{planar embedding} of a graph $G$ is a drawing where the curves
  corresponding to the edges of $G$ do not intersect, except in common end
  vertices.
\end{definition}

\begin{definition}
  A \pojem{plane graph} is a particular embedding of a planar graph.
\end{definition}

\begin{theorem}[Jordan]
  Every closed simple curve in the plane divides it into exactly two regions, a
  bounded inside region and an unbounded outside region.
\end{theorem}

\begin{definition}
  If $G$ is a plane graph, then a \pojem{face} of $G$ is a maximal region that
  does not contain any points from the image of the embedding function $f$.
\end{definition}

\begin{definition}
  The \pojem{dual graph} $G^*$ of a plane graph $G$ is obtained by switching the
  role of the vertices and faces of $G$, with two faces being connected if and
  only if they share a common border.
  If there are multiple common edges on the boundary between two faces, make
  that many edges.
  If there is an edge which borders twice on the same face, add a loop.
\end{definition}

\begin{remark}
  Two different embeddings of a planar graph may have non-isomorphic dual
  graphs.
\end{remark}

\begin{proposition}
  Dual graphs are always connected.
\end{proposition}

\begin{proposition}
  If $G$ is a connected plane graph, then $(G^*)^* \cong G$.
\end{proposition}

\begin{remark}
  Note that dual graphs are always planar.
\end{remark}

\begin{definition}
  The \pojem{length} $l(F)$ of a face $F$ in a plane graph $G$ is the total
  length of the walk(s) along the boundary of $F$, in units of number of edges.
\end{definition}

\begin{remark}
  With this definition, cut edges are counted twice.
\end{remark}

\begin{remark}
  For any plane graph,
  \[
	\sum_{\text{$F$ face}} l(F) = 2 m(G).
  \]
\end{remark}

\begin{theorem}
  Suppose $G$ is a plane graph.
  The following statements are equivalent:
  \begin{itemize}
  \item $G$ is bipartite,
  \item every face of $G$ has an even length,
  \item $G^*$ is Eulerian (connected and all vertices are of even degree).
  \end{itemize}
\end{theorem}

\begin{proof}
  1 to 2:
  The length of a face is the length of a cycle in $G$, and possibly some cut
  edges, which are counted twice.

  2 to 1:
  Suppose $G$ is not bipartite, so there is an odd cycle $C$.
  Consider the sum of the lengths of faces inside $C$.
  Every edge inside $C$ (but not part of $C$) is counted twice, but every edge
  of $C$ is counted only once.
  So the sum is odd.
  Then there must be a face $F$ in the cycle with an odd length, which is a
  contradiction.

  Equivalence between 2 and 3:
  Note that $G^*$ is connected and $\deg_{G^*}(x) = l(x)$.
\end{proof}

\begin{theorem}
  If $G$ is a plane graph and $D \subseteq E(G)$, then $D$ is a set of edges of
  a cycle if and only if the corresponding dual edge set $D^*$ is a minimal edge
  cut in $G^*$.
\end{theorem}

\begin{proof}
  Consider three cases.
  \begin{itemize}
  \item If $D$ is exactly the edge set of a cycle $C$, then $D^*$ contains all
	edges between the inside and outside faces of $C$, so it is an edge cut in
	$G^*$ and it is minimal, as the faces inside $C$ are all connected in $G^*$.
  \item If $D$ is a proper subset of the set of edges of a cycle, we can prove
	that $D^*$ is an edge cut, but not minimal.
  \item If $D$ does not contain the edge set of a cycle, we can show that $D^*$
	is not an edge cut. \qedhere
  \end{itemize}
\end{proof}

\begin{definition}
  A planar graph $G$ is \pojem{outerplanar} if there exists an embedding such
  that the boundary of the outer face contains all vertices.
\end{definition}

\begin{theorem}
  If $G$ is a simple outerplanar graph, then $\delta(G) \le 2$.
\end{theorem}

\begin{proof}
  If $n \le 4$, then this is simple casework.
  For $n \ge 4$, we will prove that additionally, there are at least two
  nonadjacent such vertices.
  Induction on $n = n(G)$.
  We covered the base case $n=4$ before, just do the induction step, in which we
  consider two cases.

  In the first case, if there is a cut vertex $v$, then $G-v$ splits into two
  graphs $G_1, G_2$.
  Both $G_1 + v$ and $G_2 + v$, with the original edges from $v$, are
  outerplanar graphs.
  By the induction hypothesis, there are two nonadjacent vertices in $G_1 + v$
  of degree at most $2$, so there is a vertex $z_1 \ne v$ in $G_1 + v$ with
  degree at most $2$ in $G_1 + v$.
  Similarly, there is a vertex $z_2$ in $G_2 + v$ for which the same holds.
  These vertices are of the same degree in $G$.

  In the second case, if there is no cut vertex, then there is no cut edge, and
  the boundary of the outer face is a Hamiltonian cycle.
  If there is no chord in this cycle, the statement is proven.
  Otherwise, let $xy$ be a chord.
  Consider the two cycles $H_1, H_2$ that the chord divides our Hamiltonian
  cycle into.
  Both are outerplanar, so using the induction hypothesis on each, we can find
  two nonadjacent vertices of degree at most $2$.
  They can't be adjacent to a vertex in the other part of the Hamiltonian cycle,
  as that edge would cross $xy$.
\end{proof}

\begin{theorem}[Euler]
  If $G$ is a connected plane graph, then
  \[
	n(G) + f(G) - m(G) = 2.
  \]
\end{theorem}

% LocalWords:  Eulerian outerplanar
