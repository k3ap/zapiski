\naslov{Matchings}

\begin{definition}
  A vertex set $S \subseteq V$ is an \pojem{independent set} of the graph $G =
  (V, E)$ if the induced subgraph $G[S]$ is empty.
  The maximum cardinality of an independent set is the \pojem{independence
	number} $\alpha(G)$.
\end{definition}

\begin{definition}
  A vertex set $T \subseteq V$ is a \pojem{vertex cover} if every edge has at least one
  of its endings in $T$.
  The maximum cardinality of a vertex cover is the \pojem{vertex cover number}
  $\beta(G)$.
\end{definition}

\begin{definition}
  An edge set $M \subseteq E$ is a \pojem{matching} if for every distinct $e_1,
  e_2 \in M$, edges $e_1$ and $e_2$ have no common ending.
  The maximum cardinality of a matching is the \pojem{matching number}
  $\alpha'(G)$.
\end{definition}

\begin{definition}
  An edge set $C \subseteq E$ is an \pojem{edge cover} if every vertex of $G$ is
  covered by at least one edge from $C$.
  If the minimum degree $\delta(G)$ is at least $1$, we can define the
  \pojem{edge cover number} $\beta'(G)$ as the minimum cardinality of an edge
  cover.
\end{definition}

\vprasanje{Define the independence number, the vertex and edge cover number, and
  the matching number.}

\begin{remark}
  The complement of an independent set is a vertex cover, so $\alpha(G) +
  \beta(G) = n(G)$ in every graph $G$.
  In a maximum matching, every edge must be covered by different vertices, so
  $\alpha'(G) \le \beta(G)$.
  We can similarly argue that $\alpha'(G) \le \nicefrac{n(G)}{2} \le \beta'(G)$
  and $\alpha(G) \le \beta'(G)$.
\end{remark}

\begin{theorem}[Gallai]
  If $\delta(G) \ge 1$, then $\alpha'(G) + \beta'(G) = n(G)$.
\end{theorem}

\begin{proof}
  Take a maximum matching $M$ in $G$ and let $V(M)$ be the vertices covered by
  $M$.
  For every vertex not covered by $M$, we can take an incident edge and add it
  to $M$.
  This gives an edge cover with
  \[
	\abs{M} + \abs{\comp{V(M)}} = \abs{M} + (n - 2 \abs{M}) = n - \abs{M}
  \]
  edges.
  Since $\abs{M} = \alpha'(G)$, this implies $\beta'(G) + \alpha'(G) \le n(G)$.

  Now take a minimum edge cover $C$.
  We claim that for every edge in $C$, at least one end is covered only once by
  $C$.
  For suppose that $uv \in C$ is an edge and both $u$ and $v$ are covered by
  other edges in $C$.
  If we remove $uv$, then $C \setminus \{uv\}$ is a smaller cover, which is a
  contradiction.

  The induced subgraph $G[C]$ is then a forest of stars.
  Suppose it consists of $k$ components.
  We get $\abs{C} = n-k$, since in a tree, the number of vertices is $1$ more
  than the number of edges.
  A matching is obtained by choosing one edge from every star, which gives
  $\alpha'(G) + \beta'(G) \ge n(g)$, thus completing the proof.
\end{proof}

\vprasanje{State and prove Gallai's theorem.}

\begin{definition}
  Let $M$ be a matching.
  A path $v_1 v_2 \ldots v_k$ is an \pojem{$M$-alternating path} if
  the edges along the path alternate between $M$ and $\comp{M}$.
  An $M$-alternating path is \pojem{$M$-augmenting} if neither end of the path
  is covered by $M$.
\end{definition}

\begin{remark}
  Such a path cannot start or end with an edge from $M$, and the endpoints
  cannot be part of an edge in $M$.
\end{remark}

\begin{proposition}
  If $G$ is a graph, $M$ is a matching and there exists an $M$-augmenting path
  $P$, then $M$ is not a maximum matching.
\end{proposition}

\begin{proof}
  Suppose $P = v_1 \ldots v_k$.
  We know the first and last edge are not in $M$, so $\abs{E(P) \cap \comp{M}} =
  \abs{E(P) \cap M} + 1$.
  Now let $M' = M \oplus E(P)$ be the symmetric difference of $M$ and $E(P)$.
  This is clearly a matching.
  We know that $\abs{M'} = \abs{M} + 1$, so $M$ cannot be maximum.
\end{proof}

\vprasanje{How can you construct a larger matching from an augmenting path?}

\begin{definition}
  A \pojem{König-Egerváry} graph is a graph $G$ with $\alpha'(G) = \beta(G)$.
\end{definition}

\begin{theorem}[König]
  Let $G$ be a bipartite graph.
  Then $\alpha'(G) = \beta(G)$.
  Additionally, if $M$ is a matching in $G$ and there is no $M$-augmenting
  path, $M$ is a maximum matching.
\end{theorem}

\begin{proof}
  Let the partite classes of $G$ be $A$ and $B$.
  Suppose that $M$ is a matching for which there is no $M$-augmenting path in
  $G$.
  Define $X$ as the set of all vertices in $A$ that are not covered by the
  matching, and $Y$ the vertices in $B$ not covered by $M$.
  Additionally, let $B_1$ be the set of vertices in $B$ that can be reached by
  an $M$-alternating path from $X$, and similarly, let $A_1$ be the set of
  vertices of $A$ which can be reached from $X$ by an $M$-alternating path.
  Finally, define $B_2 = B \setminus (B_1 \cup Y)$ and $A_2 = A \setminus (A_1
  \cup X)$.

  Observe that on an $M$-alternating path from $X$, any edge from $A$ to $B$ is
  in $\comp{M}$ and any edge from $B$ to $A$ is in $M$.
  Out matching provides a one-to-one mapping between $A_1$ and $B_1$ and between
  $A_2$ and $B_2$, so $\abs{A_1} = \abs{B_1}$ and $\abs{A_2} = \abs{B_2}$.
  We also know that $\abs{A_1} + \abs{A_2} = \abs{M}$.
  Now consider the possible edges between the defined vertex sets.

  There is no edge between $X$ and $Y$, since that would be a (trivial)
  $M$-augmenting path.
  There are also no edges between $X$ and $B_2$, since an edge $xb$ is an
  $M$-alternating path, implying $b \in B_1$.
  So the only edges from $X$ lead to $B_1$.

  There are also no edges between $A_1$ and $Y$, because we can construct an
  $M$-augmenting path with such an edge.
  If $a \in A_1$, then there is an alternating path from $X$ to $a$, which we
  could extend with a $a$-to-$Y$ edge to get an augmenting path.
  Finally, there are no edges between $A_1$ and $B_2$, since that would give an
  alternating path from $X$ to the $B_2$-vertex as before.

  Then $T = B_1 \cup A_2$ is a vertex cover with
  $\abs{T} = \abs{B_1} + \abs{A_2} = \abs{A_1} + \abs{A_2} = \abs{M}$.
  We have thus constructed a vertex cover with $\abs{M}$ vertices, giving
  \[
	\beta(G) \le \abs{T} = \abs{M} \le \alpha'(G).
  \]
  The other inequality holds in the general case, so this completes the proof.
\end{proof}

\vprasanje{State and prove König's theorem.}

\begin{corollary}
  If $G$ is a bipartite graph, then $\alpha(G) = \beta'(G)$.
\end{corollary}

\begin{definition}
  Let $G$ be a bipartite graph with partite classes $A$ and $B$.
  \pojem{Hall's condition} holds for the set $A$ if for every $S \subseteq A$,
  \[
	\abs{S} \le \abs{N(S)} = \abs{ \bigcup_{u \in S} N(u) }.
  \]
\end{definition}

\begin{theorem}[Hall]
  If $G$ is a bipartite graph with partite classes $A$ and $B$, then there
  exists a matching that covers $A$ if and only if Hall's condition holds for
  $A$.
\end{theorem}

\begin{proof}
  Let $M$ be a matching covering $A$ and $S \subseteq A$.
  We can take the pairs matched by $M$
  \[
	B_S = \{ v \in B \such \text{$v$ is covered by an edge in $M$} \}.
  \]
  Clearly, $\abs{S} = \abs{B_S}$ and $B_S \subseteq N(S)$, so $\abs{S} \le
  \abs{N(S)}$.

  For the other implication, suppose there is no matching covering $A$.
  Divide the sets $A$ and $B$ as in the proof of König's theorem, using some
  matching $M$.
  Since the matching doesn't cover $A$, $X$ is not empty.
  Now consider $S = A_1 \cup X$.
  All edges from $S$ lead into $B_1$, so $N(S) = B_1$, but $\abs{S} = \abs{A_1}
  + \abs{X} > \abs{B_1} = \abs{N(S)}$.
\end{proof}

\vprasanje{State and prove Hall's theorem.}

\begin{definition}
  A matching $M$ is a \pojem{perfect matching} if it covers all vertices.
\end{definition}

\begin{corollary}
  In a bipartite graph $G$, there is a perfect matching if and only if $\abs{A}
  = \abs{B}$ and $A$ satisfies Hall's condition.
\end{corollary}

\begin{definition}
  Let $G$ be a bipartite graph with partite classes $A,B$, and $S \subseteq A$.
  The \pojem{deficiency} of $S$ is $\operatorname{def}(S) = \abs{S} -
  \abs{N(S)}$.
\end{definition}

\begin{theorem}
  Let $G$ be a bipartite graph with partite classes $A$ and $B$.
  If $M$ is a maximum matching in $G$, it covers
  \[
	\alpha'(G) = \abs{A} - \max_{S \subseteq A} \operatorname{def}(S)
  \]
  vertices of $A$.
\end{theorem}

\begin{theorem}
  If $G$ is a regular bipartite graph, then $G$ has a perfect matching.
\end{theorem}

\begin{proof}
  The number of edges in the graph is $k \cdot \abs{A} = k \cdot \abs{B}$, so
  $\abs{A} = \abs{B}$.
  Let $S \subseteq A$.
  The number of edges between $S$ and $N(S)$ is exactly $k \cdot \abs{S}$.
  Every neighbour $u \in N(S)$ has exactly $k$ neighbours, at most $k$ are in
  $S$.
  So at most $k \cdot \abs{N(S)}$ edges are between $S$ and $N(S)$, implying
  $\abs{S} \le \abs{N(S)}$.
\end{proof}

\vprasanje{Show that a regular bipartite graph has a perfect matching.}

\begin{theorem}
  Let $M$ be a matching in $G$.
  Then there is an $M$-augmenting path in $G$ if and only if $M$ is not a
  maximum matching in $G$.
\end{theorem}

\begin{proof}
  We've already proved the right implication.
  Suppose there is a matching $M'$ with $\abs{M'} > \abs{M}$.
  Consider the symmetric difference $M \triangle M'$ and denote $G' = G[M
  \triangle M']$.
  Clearly the maximum degree $\Delta(G') \le 2$, from which we know that the
  components of $G'$ are all paths or cycles.

  Any cycle must alternate between edges from $M$ and $M'$, so it is an even
  cycle, and contains the same number of edges from the two matchings.
  In any path of even length, there must be the same number of edges in $M$ and
  in $M'$.
  And finally, in a path of odd length, one of the two sets has an extra edge
  compared to the other.
  Since $\abs{M'} > \abs{M}$, there must be a path with more edges in $M'$ than
  in $M$.
  Label it $G_1'$.

  This component is an $M$-augmenting path in $G$, since if either of its
  endpoints are covered by $M$, they must also be covered by the same edge in
  $M'$, but then $M'$ wouldn't be a matching.
\end{proof}

We can find the maximum matching in polynomial time, with so-called
\enquote{Blossom algorithms}, which find an augmenting path in $O(m \sqrt{n})$.
This also means we can determine the edge-cover number $\beta'(G)$ in polynomial
time.

\podnaslov{Tutte's theorem}

\begin{definition}
  A component of a graph $G$ is \pojem{odd} if it has an odd number of vertices.
  We denote the number of odd components in $G$ with $o(G)$.
\end{definition}

\begin{theorem}[Tutte]
  A graph $G$ has a perfect matching if and only if for any $S \subseteq V(G)$,
  $\abs{S} \ge o(G - S)$ holds.
  This is called Tutte's condition.
\end{theorem}

\begin{proof}
  Left to right: Let $S \subseteq V(G)$ and $M$ be a perfect matching in $G$.
  Let $H_1, \ldots, H_k$ be the components of $G - S$.
  If $H_i$ is an odd component, then there exists at least one $M$-edge between
  $V(H_i)$ and $S$.
  Therefore $\abs{S} \ge o(G-S)$.

  Right to left:
  Suppose that Tutte's condition holds for a graph $F$ but there is no perfect
  matching in $F$.
  Now add edges to $F$ so that this still holds after the addition, and let $G$
  be a maximal such graph on $n(F)$ vertices.
  If we consider Tutte's condition on $S = \varnothing$, we see that there are
  no odd components in $G$, which means $n(G)$ is even.

  Now take any edge in the complement of $G$.
  Since $G$ is maximal, $G+e$ is not a counterexample, so it either has a
  perfect matching, or Tutte's condition does not hold for it.
  We know that for any $S \subseteq V(G)$, $\abs{S} \ge o(G-S)$.
  After having added an edge, at most one pair of components in $G-S$ is joined.
  If this was a pair of odd components, $o(G-S)$ has reduced, and in all other
  cases, it has remained the same.
  This means that $G+e$ must satisfy Tutte's condition, so it must have a
  perfect matching as it is not a counterexample.

  We will consider two cases.
  Let $U$ be the set of universal vertices in $G$, that is, the vertices
  adjacent to every other vertex, and let $H_1, \ldots, H_k$ be the components
  of $G - U$.
  In the first case, suppose every $H_i$ induces a complete graph.
  We will construct a perfect matching.
  For the even components, we may create a matching within the component, and
  for the odd components, we may connect the one remaining vertex with $U$.
  We are left over with an even number of vertices in $U$ (since $n$ is even),
  which we may form a matching with, as $G[U]$ is a complete graph.
  So in this case, $G$ is not a counterexample, which is a contradiction.

  Now suppose there is a non-complete component $H_i$.
  Then there exist vertices $x, y, z$ for which $xy \notin E(H_1)$ but both $xz$
  and $yz$ are in $E(H_1)$.
  There is also another vertex $w \in V(G)$ for which $zw \notin E(G)$, since
  $z$ is not a universal vertex.
  Let $G_1 = G + xy$ and $G_2 = G + zw$.
  We know both these graphs have perfect matchings, which must include $xy$ and
  $zw$ respectively.
  Denote the perfect matchings with $M_1$ and $M_2$ and consider $G[M_1
  \triangle M_2]$.
  Note that every non-isolated vertex in this graph is of degree $2$, since it
  is covered by both perfect matchings.
  These vertices must of course appear in even cycles.

  If $xy$ and $zw$ belong to different components, then we may choose edges in
  each component, and the edges deleted by the symmetric difference, to form a
  perfect matching, and we can avoid taking $xy$ or $zw$.
  This is a contradiction.
  Alternatively, if $xy$ and $zw$ belong to the same component, they appear in
  the same cycle.
  Without loss of generality they appear in the order $zwxy$ (but not
  necessarily adjacent).
  To form a new perfect matching, we will take the edge $xz$, and one edge set
  from each side of the cycle.
  This avoids both new edges $xy$ and $zw$, so we have a contradiction.
\end{proof}

We also have the Berge-Tutte formula, which states that a maximum matching in
$G$ leaves uncovered exactly
\[
  \max_{S \subseteq V(G)} \{o(G - S) - \abs{S}\}
\]
vertices.

% LocalWords:  Egerváry König König's Tutte's Tutte
