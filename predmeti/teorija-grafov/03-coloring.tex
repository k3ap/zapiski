\naslov{Coloring}

\begin{remark}
  In any graph,
  $\omega(G) \le \chi(G) \le \Delta(G)+1$.
\end{remark}

\begin{remark}
  As the color classes are independent sets, the number of vertices in a color
  class is at most $\alpha(G)$, so
  \[
	\chi(G) \ge \frac{n(G)}{\alpha(G)}.
  \]
\end{remark}

\begin{theorem}[Welsh-Powel]
  If $d_1 \ge d_2 \ge \cdots \ge d_n$ is the degree sequence of the vertices of
  $G$, then
  \[
	\chi(G) \le 1 + \max_{i=1, \ldots, n} \{ \min \{d_i, i-1\} \}.
  \]
\end{theorem}

\begin{theorem}[Brooks]
  If $G$ is connected and not a complete graph or odd cycle, then $\chi(G) \le
  \Delta(G)$.
\end{theorem}

\begin{proof}
  Let $k = \Delta(G)$.
  Consider two cases.
  First, if $G$ is not $k$-regular, then there is a vertex $v_n$ with degree at
  most $k-1$.
  Define the following vertex order for a greedy coloring.
  Consider a breadth-first search tree, rooted in $v_n$.
  Order the vertices such that every child of the tree precedes its parent.
  This way, each vertex is preceded by at most $k-1$ of its neighbours, so we
  can color the entire graph with at most $k$ colors.

  In the second case, if $G$ is $k$-regular, we can quickly write off $k=1$ as
  then $G = P_2$, which is a complete graph, and the case $k=2$, where $G$ must
  be a cycle.
  We have excluded odd cycles, and for even cycles, $\chi(G) = \Delta(G) = 2$.
  So let $k \ge 3$.
  We will consider three subcases.

  If $\kappa(G) = 1$, then there exists a cut vertex $x \in V(G)$.
  Let $V_1, V_2$ be disconnected vertex sets in $G - x$ such that $V_1 \cup V_2
  = V(G-x)$, and let $G_i = G[V_i \cup \{x\}]$.
  Since $x$ has neighbours in both $V_1$ and $V_2$, the degree of $x$ in either
  $G_i$ is at most $k-1$ and $G_1, G_2$ are not regular.
  So we can color them with at most $k$ colors.
  We can permute the colors in one of the colorings so that $x$ has the same
  color in both.

  In the second subcase, suppose $\kappa(G) = 2$.
  Then we have $x,y \in V(G)$ such that $G - \{x,y\}$ is disconnected.
  Define $V_1$ and $V_2$ as before and $G_i = G[V_i \cup \{x,y\}]$.
  We have $\deg_{G_i}(x) < \deg_G(x) = k$, so neither $G_i$ is regular.
  Therefore, they are $k$-colorable, so we have colorings $\varphi_1,
  \varphi_2$.
  If $\varphi_1(x) = \varphi_1(y)$ and $\varphi_2(x) = \varphi_2(y)$, or if both
  of these pairs are nonequal, then we can define a combined $k$-coloring.

  If we cannot find such $\varphi_1, \varphi_2$ however, then without loss of
  generality every $k$-coloring of $G_1$ assigns the same color to $x$ and $y$,
  and every $k$-coloring of $G_2$ assigns different colors to those vertices.
  Equivalently, $G_1 + xy$ is not $k$-colorable, and $G_2 + xy$ is.
  So $G_1 + xy$ must be $k$-regular, in which case, we will prove that $G_2$ can
  be $k$-colored with $\varphi_2(x) = \varphi_2(y)$.
  Let $G_2' = G_2 - \{x,y\}$.
  It is $k$-colorable.
  Since $G_1 + xy$ is $k$-regular, $x$ and $y$ have precisely one neighbour each
  in $G_2$, meaning we can choose a color different from the colors of those
  neighbours (as $k \ge 3$) and find a $k$-coloring for $G_2$ with $\varphi_2(x)
  = \varphi_2(y)$.

  Finally, consider the case $\kappa(G) \ge 3$.
  We have vertices $u,v \in V(G)$ for which $d(u,v) = 2$, since $G$ is not
  complete.
  Let $z$ be a common neighbour for them.
  We want a vertex order such that in a greedy coloring, $\varphi(u) =
  \varphi(v)$ and the last vertex in the ordering is $z$.
  Let $G' = G - \{u,v\}$.
  Consider a breadth-first search in $G'$, rooted in $z$.
  Since $\kappa(G) \ge 3$, it will not stop immediately.
  Again, take the order in which every vertex is preceded by at most $k-1$
  children, and $z$ is at the end.
  Now a greedy coloring $u, v, v_1, v_2, \ldots, v_{n-2}=z$ of $G$ will assign
  $\varphi(u) = \varphi(v) = 1$, and for $v_1, \ldots, v_{n-3}$, at most $k-1$
  neighbours are colored before $v_i$, so it is assigned a color $\le k$.
  We know that $z$ has two neighbours with the same color, so we can find a
  $k$-coloring for $G$.
\end{proof}

\vprasanje{State and prove Brooks' theorem.}

\podnaslov{Mycielski's construction}

The Mycielskian $M(G)$ of a graph $G$ is a graph defined as follows:
\begin{itemize}
\item label the vertices of $G$ as $v_1, \ldots, v_n$,
\item create $n+1$ new vertices $u_1, \ldots, u_n, z$,
\item add connections $u_i v_j$ for all pairs $v_i v_j \in E(G)$,
\item add connections $u_i z$ for all $i$.
\end{itemize}
Label $V = \{v_1, \ldots, v_n\}$ and $U = \{u_1, \ldots, u_n\}$.

\begin{theorem}
  If $G$ is a graph with at least one edge, then $\chi(M(G)) = \chi(G) + 1$ and
  $\omega(M(G)) = \omega(G)$.
\end{theorem}

\begin{proof}
  Since $G$ is a subgraph of $M(G)$, we have $\omega(G) \le \omega(M(G))$.
  If $z$ is in a clique of $M(G)$, then this is a clique of order at most $2$,
  which appears in $G$ as well.
  If $u_i$ is in a clique of $M(G)$, then $v_i$ can't be in the same clique, so
  we can replace $u_i$ with $v_i$, and find a clique in $G$ of the same size.
  Therefore $\omega(M(G)) \le \omega(G)$.

  If we have a coloring of $G$, then we can paint $u_i$ with the same color as
  $v_i$, and use a new color for $z$.
  So $\chi(M(G)) \le \chi(G)+1$.
  Now suppose that there exists a $\chi(G)$-coloring of $M(G)$ and label it
  $\varphi$.
  Without loss of generality, $\varphi(z) = k := \chi(G)$.
  Then this color does not appear in $U$, so $U$ is colored with $k-1$ colors.
  But then since $\chi(G) = k$, the color $k$ must appear in $V$.
  We can replace the colors for those $v_i$ which have $\varphi(v_i) = k$ with
  the color of $u_i$, and get a proper $(k-1)$-coloring of $G$, which is
  impossible as $\chi(G) = k$.
  \protislovje{}
\end{proof}

\vprasanje{Show that the difference between $\chi(G)$ and $\omega(G)$ can be
  arbitrarily large.}

\begin{theorem}
  If $G$ is a graph on $n$ vertices and $\chi(G) = k$, then $m(G) \ge
  \binom{k}{2}$.
  This is sharp for any $n \ge k$.
\end{theorem}

\begin{proof}
  There is a partition of $G$ into $k$ color classes.
  Since $\chi(G) = k$, there must be at least one edge between any pair of color
  classes.
  This bound is sharp, as we can take $K_k$ and add isolated vertices as
  required.
\end{proof}

\vprasanje{State and prove a lower bound for $m(G)$ in a graph with $\chi(G) =
  k$.}

\podnaslov{Turàn's theorem}

\begin{definition}
  A graph $G$ is \pojem{$k$-partite} if the vertex set can be partitioned into
  $k$ classes $V_1, V_2, \ldots, V_k$ such that every edge is between different
  classes.
\end{definition}

\begin{remark}
  A graph $G$ is $k$-partite if and only if $G$ is $k$-colorable.
\end{remark}

\begin{definition}
  $G$ is a \pojem{complete $k$-partite graph} if every edge between the partite
  classes is present in $G$.
\end{definition}

\begin{definition}
  \pojem{The Turàn graph} $T_{n,k}$ is the complete $k$-partite graph on $n$
  vertices such that the partite classes are of nearly equal size, i.e.~if
  $\abs{\abs{V_i} - \abs{V_j}} \le 1$ for all $i, j$.
\end{definition}

\begin{theorem}[Turàn]
  If $G$ is a graph of order $n$ and $\omega(G) \le r$, then the number of edges
  in $G$ is at most the number of edges in $T_{n,r}$.
\end{theorem}

\begin{proof}
  Induction on $r$.
  If $r = 1$, there are no edges, so $G = T_{n,1}$.
  For $r > 1$, let $k = \Delta(G)$ and let $v$ be a vertex with degree $k$.
  Let $G' = G[N(v)]$.
  We know $\omega(G') \le r-1$, so by the induction hypothesis, $m(G') \le
  m(T_{k,r-1})$.

  Consider the following construction.
  Let $H$ be the graph obtained by a complete join of a graph with $n-k$
  independent vertices and the graph $T_{k,r-1}$.
  Clearly, $n(H) = n$ and $\omega(H) = r$, since $\omega(T_{k,r-1}) = r-1$.
  Compute
  \[
	m(H) = m(T_{k,r-1}) + (n-k)k \ge m(G') + (n-k)k
  \]
  and $m(G) \le m(G') + k(n-k)$, noting that in the remainder of the graph,
  every vertex has degree at most $k$, and there are $n-k$ vertices there.

  We see that $H$ is an $r$-partite graph.
  We claim that among the $r$-partite graphs on $n$ vertices, $T_{n,r}$ has the
  most edges.
  Suppose that $F$ is an $r$-partite graph with $n(F) = n$.
  If it is not a complete $r$-partite graph, then we may add edges until it is.
  If the sizes of the partite classes in $F$ differ by at least $2$, then take a
  vertex $u$ from the larger class $V_i$ and put it into the smaller class
  $V_j$.
  For this, we break $\abs{V_j}$ edges and add $\abs{V_i}-1$ edges, so we have
  increased their total number.
\end{proof}

\vprasanje{What is a Turàn graph? State and prove Turàn's theorem.}

\begin{remark}
  As $T_{n,r}$ satisfies the theorem's conditions, the bound is sharp.
  It is actually the unique graph with the maximum number of edges.
\end{remark}

\begin{remark}
  A similar result holds for graphs with $\chi(G) = r$.
\end{remark}

\podnaslov{Chordal graphs}

\begin{definition}
  A graph $G$ is \pojem{chordal} if there is no induced subgraph that is
  isomorphic to a cycle $C_n$ for $n \ge 4$.
\end{definition}

\begin{definition}
  A vertex $v$ is a \pojem{simplicial} vertex in $G$ if $N_G[v]$ induces a
  clique.
\end{definition}

\begin{lemma}[Voloshin]
  If $G$ is a chordal graph, then for every $x \in V(G)$ there exists a
  simplicial vertex among the vertices farthest from $x$.
\end{lemma}

\begin{proof}
  Induction on $n(G)$.
  If $n(G) = 1$, the statement is obvious.
  Let $n(G) > 1$.
  If $x$ is a universal vertex in $G$, then remove $x$ and apply the induction
  hypothesis to $G-x$.
  Otherwise, let $T$ be the set of vertices farthest from $x$, and let $H$ be a
  component in $T$.
  Take $S$ to be the neighbours of vertices in $H$ which are not themselves in
  $H$, so $S = N(H) \setminus H$.
  Finally, let $Q$ be the vertices in the component of $G-S$ which contains $x$.

  We will prove that $S$ induces a clique.
  Take $u, v \in S$.
  Then both vertices have neighbours in $H$ and in $Q$, so there exist two
  $u,v$-paths through $H$ and $Q$, respectively.
  Consider the shortest such paths $P_H, P_Q$.
  Combined, they form a cycle of order at least $4$.
  There cannot be a chord in $\{u/v\} \cup H$ or $\{u/v\} \cup Q$, since that
  would give a shorter path, and there is also no edge between $Q$ and $H$, so
  there must be a cord $uv \in E(G)$.
  So any two vertices in $S$ are connected, meaning $S$ is a clique.
  By the induction hypothesis on $G[S \cup H]$ with a vertex from $S$, there is
  a simplicial vertex in $H$.
\end{proof}

\vprasanje{State and prove Voloshin's lemma.}

\begin{theorem}
  A graph $G$ is chordal if and only if there is a simplicial elimination
  ordering of the vertices of $G$.
\end{theorem}

\begin{proof}
  Right-to-left:
  If $G$ is not chordal, then we will show there is no simplicial elimination
  ordering.
  Suppose there is one, $v_1, v_2, \ldots, v_n$.
  Since $G$ is not chordal, it has a cycle $C_k$ with $k \ge 4$.
  Let $v_i$ be the first vertex of this cycle in the ordering.
  Then $v_i$ has two neighbours along the cycle with no edge between them, which
  contradicts the definition of the simplicial elimination ordering.

  Left-to-right:
  By Voloshin's lemma, there is a simplicial vertex $v_1$ in $G$.
  Then $G - v_1$ is also a chordal graph, so it has a simplicial vertex $v_2$.
  Then $G - v_1 - v_2$ is a chordal graph, and we keep going.
  We get a simplicial elimination ordering $v_1, \ldots, v_n$.
\end{proof}

\vprasanje{Characterize chordal graphs and prove the characterization.}

\begin{theorem}
  If $G$ is chordal, then $\chi(G) = \omega(G)$.
\end{theorem}

\begin{proof}
  In general $\omega(G) \le \chi(G)$.
  Let $v_1, \ldots, v_n$ be a simplicial elimination ordering in $G$.
  Consider the greedy coloring in the reverse of that order.
  When we color a vertex $v_i$, then the neighbours that are already covered are
  $R_i = N(v_i) \cap \{ v_{i+1}, \ldots, v_n \}$.
  As it is a simplicial elimination ordering, $R_i$ is a clique, and $R_i \cup
  \{v_i\}$ is also a clique.
  Since $\abs{R_i \cup \{v_i\}} \le \omega(G)$, at most $\omega(G) - 1$
  neighbours of $v_i$ are colored before $v_i$, so we can color $v_i$ with one
  of $\{1, 2, \ldots, \omega(G)\}$.
  Since this is true for any $v_i$, we have $\chi(G) \le \omega(G)$.
\end{proof}

\podnaslov{Perfect graphs}

\begin{definition}
  A graph $G$ is a \pojem{perfect graph} if $\chi(H) = \omega(H)$ holds for
  every induced subgraph $H$ of $G$.
\end{definition}

\begin{theorem}
  Chordal graphs are perfect.
\end{theorem}

\begin{proof}
  An induced subgraph of a chordal graph is chordal.
\end{proof}

\vprasanje{Show that chordal graphs are perfect.}

\begin{theorem}
  Bipartite graphs are perfect.
\end{theorem}

\begin{proof}
  If $G$ is bipartite, then either $E(G) = \varnothing$, in which case
  $\omega(G) = \chi(G) = 1$, or $E(G) \ne \varnothing$, in which case $\chi(G) =
  \omega(G) = 2$.
  Every subgraph of a bipartite graph is bipartite.
\end{proof}

\begin{theorem}[Vizing]
  For every graph $G$, we have $\Delta(G) \le \chi'(G) \le \Delta(G)+1$.
\end{theorem}

\begin{theorem}
  If $G$ is bipartite, then $\chi'(G) = \Delta(G)$.
\end{theorem}

\begin{proof}
  Let $k = \Delta(G)$.
  Consider the following two cases.
  First, if $G$ is $k$-regular, there exists a perfect matching $M_1$ in $G$.
  The graph $G - M_1$ is $(k-1)$-regular and bipartite, so there exists a perfect
  matching $M_2$ in $G - M_1$.
  We can continue this procedure to find matchings $M_1, \ldots, M_k$.
  Now assign color $i$ to $M_i$.

  In the second case, if $G$ is not $k$-regular, we will construct a $k$-regular
  bipartite graph which has $G$ as a subgraph.
  We add edges to $G$ such that the maximum degree remains $k$ and the graph
  remains bipartite.
  If this is not possible but there are still vertices of degree $< k$, then we
  can follow the following construction.
  \begin{itemize}
  \item Create a copy $G'$ of $G$, except it has partite classes switched.
  \item Create a new graph by combining $G$ and $G'$, and adding connections $u
	\sim u'$ for $u \in G$ with $\deg(u) < k$.
  \end{itemize}
  This increases the minimum degree of $G$ by at least one, and we still have a
  bipartite graph.
  If we continue this process, we will get a $k$-regular bipartite graph.
\end{proof}

\vprasanje{Show that bipartite graphs are Vizing class one.}

\begin{theorem}
  If $G$ is bipartite, then its line graph $L(G)$ is perfect.
\end{theorem}

\begin{proof}
  For every graph, $\chi'(G) = \chi(L(G))$, as every edge coloring of $G$
  corresponds to a vertex coloring of $L(G)$.
  If $G$ is bipartite, $\chi'(G) = \Delta(G)$.
  The maximal cliques in $L(G)$ corresponds to the vertices of $G$ (since there
  are no triangles).
  If a vertex is of degree $k$, then the corresponding clique in $L(G)$ is also
  of size $k$, so $\Delta(G) = \omega(L(G))$.
  This means $\chi(L(G)) = \omega(L(G))$.

  The induced subgraphs of $L(G)$ are still line graphs of bipartite graphs, as
  if $H = L(G)[S]$, then $H$ is the line graph of the subgraph of $G$ obtained
  by only keeping the edges corresponding to the vertices in $S$.
\end{proof}

\vprasanje{Show that the line graph of a bipartite graph is perfect.}

\begin{theorem}[Perfect graph theorem]
  A graph $G$ is perfect if and only if its complement $\cl{G}$ is perfect.
\end{theorem}

\begin{theorem}[Strong perfect graph theorem]
  A graph $G$ is perfect if and only if neither $G$ nor $\cl{G}$ have an induced
  odd cycle of size at least $5$.
\end{theorem}

\vprasanje{State the weak and strong perfect graph theorems.}

\begin{definition}
  A \pojem{$(\beta, \alpha')$-perfect graph} is a graph for which $\beta(H) =
  \alpha'(H)$ for every induced subgraph $H$.
\end{definition}

\begin{theorem}
  A graph is $(\beta, \alpha')$-perfect if and only if it is bipartite.
\end{theorem}

\podnaslov{Gallai-Roy-Vitaver theorem}

\begin{theorem}
  If $G$ is a simple graph and $\vec{D}$ is an orientation of $G$, and
  $l(\vec{D})$ is the longest path in $\vec{D}$, then $\chi(G) \le l(\vec{D}) +
  1$.
  Further, there is an orientation for which the equality holds.
\end{theorem}

\begin{proof}
  Consider an arbitrary orientation $\vec{D}$ of $G$.
  Let $\pvec{D}'$ be a maximal subdigraph of $\vec{D}$ that does not contain any
  directed cycles.
  Define a vertex coloring $c$ for $G$ with $c(v)$ being one plus the length of
  the longest directed path in $\pvec{D}'$ that ends in $v$.
  This is a proper vertex coloring; consider an edge $\overrightarrow{uv} \in
  \vec{D}$.
  Consider two cases.
  \begin{itemize}
  \item If $\overrightarrow{uv} \in \pvec{D}'$, then $l'(u) < l'(v)$, since $v$
	cannot be on the longest path ending at $u$ (as there are no directed cycles).
  \item If $\overrightarrow{uv} \notin \pvec{D'}$, then there is a directed path
	in $\pvec{D}'$ between $v$ and $u$ (which forms a cycle with
	$\overrightarrow{uv}$).
	Then the coloring $c$ gives increasing colors to the vertices along this
	path, and $c(u) > c(v)$.
  \end{itemize}
  We skip the proof of the second statement.
\end{proof}

\vprasanje{State the Gallai-Roy-Vitaver theorem. Prove the $\chi(G) \le$
  inequality.}

% LocalWords:  neighbours colorings subcase colorable nonequal Mycielski's
% LocalWords:  Mycielskian Turàn's Turàn simplicial Voloshin Voloshin's Vizing
% LocalWords:  Gallai Vitaver subdigraph
