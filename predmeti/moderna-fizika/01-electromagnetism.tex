\naslov{Electromagnetism}

\podnaslov{A note on fields}

A \pojem{field} is defined as a quantity that takes a value at every point in
space and time, the value being some element of a particular vector space.
Fields are usually \pojem{local}, which means they only depend on one point in
space-time.
They influence particles, and particles influence fields.

There are four known interactions:
\begin{itemize}
\item \pojem{Electromagnetism} is the force between charged particles.
  Quantum mechanically, it is carried by photons, and it is responsible for most
  of our experience of the world.
\item The \pojem{weak} and \pojem{strong nuclear forces} govern the behaviour
  inside atomic cores.
\item \pojem{Gravity} is a geometric force, which we model slightly differently.
  Compared to the other forces, it is extremely weak.
\end{itemize}

Electromagnetism is a unified theory connecting the electric and magnetic
fields.
There is also a successful unification of EM and the weak force, called
\pojem{electroweak theory}.
We may further unify this with the strong force, giving the \pojem{grand
  unification theory} (GUT).
It is mathematically consistent, but none of its predictions have ever been
measured.
This is a real shame, since among other things, it predicts magnetic monopoles
and proton decay.

All these unifications are performed with the same techniques, but these fail
when trying to unify GUT with gravity.

\podnaslov{A Note on Notation}

We use Einstein's summation convention:
Whenever there is a repeated index in the expression, assume that we are summing
over it.
For example, the dot product of two vectors $\vec{w}$ and $\vec{v}$ can be
written as
\[
  \sk{\vec{v}, \vec{w}} = v^i w^i = \sum_{i=1}^3 v^i w^i,
\]
and similarly, the divergence of a vector field $\vec{E}$ is
\[
  \div \vec{E} = \partial_i E^i = \sum_{i=1}^3 \partial_i E^i.
\]
Note that here, $v^i$ denotes the $i$-th component of vector $\vec{v}$.

We also define two symbols, the \pojem{Kronecker delta}
\[
  \delta_{ij} =
  \begin{cases}
	1 & i = j, \\
	0 & \text{otherwise}
  \end{cases}
\]
and the \pojem{Levi-Chivita symbol}
\[
  \varepsilon_{ijk} =
  \begin{cases}
	0 & \text{one of the indices is repeated}, \\
	\sgn (i\,j\,k) & \text{all indices are distinct}.
  \end{cases}
\]
Here, $\sgn \pi$ denotes the sign of permutation $\pi$ as an element of $\{\pm
1\}$.
With these definitions, the following useful identities hold:
\begin{itemize}
\item for two vectors $\vec{v}, \vec{w}$, the dot product is
  \[
	\sk{\vec{v}, \vec{w}} = \delta_{ij} v^i w^j
  \]
  and the cross product is, component-wise
  \[
	(\vec{v} \times \vec{w})^i = \varepsilon_{ijk} v^j w^k,
  \]
\item $\delta_{ii} = 3$,
\item $\varepsilon_{ijk} \varepsilon_{imn} = \delta_{jm} \delta{kn} -
  \delta_{jn} \delta_{km}$,
\item $\varepsilon_{ijk} \varepsilon_{ijn} = 2 \delta_{kn}$,
\item $\varepsilon_{ijk} \varepsilon_{ijk} = 6$.
\end{itemize}

\podnaslov{Maxwell's Equations}

We define two fields, the electric field $\vec{E}(\vec{x}, t)$ and the magnetic
field $\vec{B}(\vec{x}, t)$.
For these fields, we assume the following equations to be true:
\begin{alignat*}{2}
  \div \vec{E} = \frac{\rho}{\varepsilon_0} & \qquad & \rot \vec{E} = -
													   \partial_t \vec{B} \\
  \div \vec{B} = 0 & & \rot \vec{B} = \varepsilon_0 \mu_0 \partial_t \vec{E} +
					   \mu_0 \vec{J}
\end{alignat*}
These are called \pojem{Maxwell's equations}.
Along with the aforementioned $\vec{E}$ and $\vec{B}$, they also contain the
following fields;
\begin{itemize}
\item the electric charge density $\rho$ is a scalar field,
\item the electric current density $\vec{J}$ is a vector field,
\end{itemize}
and two constants, the \pojem{permittivity of free space} $\varepsilon_0$ and
the \pojem{permeability of free space} $\mu_0$.

% LocalWords:  monopoles unifications Chivita th
