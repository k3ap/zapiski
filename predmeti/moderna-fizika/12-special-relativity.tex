\naslov{Special relativity}

\podnaslov{The Michelson-Morley Experiment}

Suppose there is some fluid medium at rest in a frame $S$.
Now, we create a plane wave in this medium, traveling in the $\vec{e}_x$
direction, with a velocity of $c_1$ as measured in $S$.
If we switch to a different frame $S'$ with a Galilean transformation $\vec{x}'
= \vec{x} + v t \vec{e}_x$, then we expect the plane wave to have a different
velocity $c_1 - v$ in $S'$.
This is true for waves in water, or sound waves in air, but, as famously
measured by Michelson and Morley, it is not true for light.
Their experiment, described below, showed that the speed of light $c$ is the
same in any inertial reference frame --- a result that contradicts entirely with
classical mechanics.

The Michelson-Morley experiment splits a beam of light into two beams traveling
orthogonally to each other, which then bounce off a pair of mirrors, before
being recombined and the interference pattern of this recombination being
measured.
Assume that $S$ is the rest frame of the ether, in which light waves are
traveling at $c$.
Suppose that we now switch to a frame $S'$, with a relative velocity of $v$ as
measured in $S$.
If one of the arms is parallel to the relative velocity vector, then the time of
flight of a wave of light should be
\[
  t_1 = \frac{L}{c + v} + \frac{L}{c-v} = \frac{2 c L}{c^2 - v^2},
\]
where $L$ is the distance between the beam splitter and the mirror.
In $S$, the other beam of light travels a path that is not parallel to
$\vec{e}_x$, as shown in figure~\ref{fig:michelson-morley-beam-path}.

\begin{figure}[h!]
  \centering
  \begin{tikzpicture}[>=Stealth]
	
	% Horizontal base line
	\draw[dotted] (-3,0) -- (3,0);
	\draw[thick] (-3.5,0) -- (-2.5,0);
	\draw[thick] (2.5,0) -- (3.5,0);

	% Wavy arrow for the beam of light
	\draw[thick,decorate,decoration={snake,amplitude=1mm,segment length=3mm}] (-3,0) -- (-3,1);

	% Vertical line in the center
	\draw[thin,<->] (0,0.1) -- (0,2.4);
	\node at (0.2,1.25) {$L$};

	% Arrow for light reflection
	\draw[dotted,-{Stealth[length=3mm]}] (0,2.5) -- (3,0);
	\draw[dotted,{Stealth[length=3mm]}-] (0,2.5) -- (-3,0);

	% Mirror at the top
	\draw[thick] (-1.5,2.5) -- (1.5,2.5);
	\node[above] at (0,2.5) {mirror};

	% Formulas at the bottom
	\draw[thin,<->] (-2.9,-0.2) -- (-0.1,-0.2);
	\node at (-1.5,-0.6) {$\frac{vt_2}{2}$};

	\draw[thin,<->] (2.9,-0.2) -- (0.1,-0.2);
	\node at (1.5,-0.6) {$\frac{vt_2}{2}$};
  \end{tikzpicture}
  \caption{Beam path in $S$}%
  \label{fig:michelson-morley-beam-path}
\end{figure}

From Pythagoras' theorem, the length of the beam path is
\[
  c t_2 = 2 \sqrt{L^2 + v^2 t_2^2 / 4}
\]
so
\[
  t_2 = \frac{2 L}{\sqrt{c^2 - v^2}}.
\]

As mentioned, Michelson and Morley conducted this experiment, while using the
Earth's rotation to perform many measurements in many directions.
They found that their experiment always gave the same result, no matter the
relative velocity of the second frame and no matter the orientation of the
testbed.
This demonstrated beyond any doubt that Max Planck's advisor Philipp von Jolly
was very wrong when he stated to Planck in 1874 that there was essentially
nothing left to discover in theoretical physics,%
\footnote{Source: \url{https://hsm.stackexchange.com/questions/2129/}}
since, as Feynman stated,
\begin{quote}
  \enquote{It doesn't matter how beautiful your theory is, it doesn't matter how smart
  you are. If it doesn't agree with experiment, it's wrong.}
\end{quote}
Classical mechanics and electromagnetism, as complete as might seem, is wrong.

\podnaslov{Lorentz, the new Galileo}

Enter Einstein, who formulated special relativity with the following postulates:
\begin{enumerate}
\item The principle of relativity: The laws of physics are identical in all
  inertial frames.
  This is already a postulate of classical mechanics, but we now have an
  additional one.
\item The speed of light in vacuum is a law of physics, so it is also the same
  in all inertial reference frames.
\end{enumerate}
%
Using these notions, we can find the expressions for the Lorentz
transformations, the replacement of Galilean transformations in classical
mechanics.
Suppose we have two reference frames, one stationary and one moving with
velocity $v \vec{e}_x$ relative to the first frame.
Let $x' = f(x, t)$ be the position of an event in frame $S'$ and $t' = g(x,t)$
be the time of that event, as measured by a clock in $S'$.
Without loss of generality, we can limit our consideration to the case where $x
= x' = 0$ at times $t = t' = 0$.

With some inspiration from dimensional analysis, we set $x' = \gamma(v) (x -
vt)$ for some function $\gamma$.
By the principle of relativity, rotating the system should have no effect on the
value of $\gamma$, but if we rotate frame $S'$ around, the same expression gives
$x' = \gamma(-v) (x + vt)$.
So $\gamma$ must be an even function of $v$.

Now consider a light wave in $S$, moving with speed $c$ towards the right.
Then $x = ct$, and by Einstein's second principle, $x' = ct'$.
The above transformation gives us
\[
  ct' = \gamma(v) (ct - vt).
\]
The same transformation formula should also work in reverse, $ct = \gamma(v)
(ct' + vt')$.
Solving these two equations gives us
\[
  \gamma(v) = \inv{\sqrt{1 - \frac{v^2}{c^2}}}.
\]

We still need to find $t'(x, t)$.
For this, again solve the equations $x' = \gamma (x - vt)$ and $x = \gamma (x' +
v t')$, but not for light.
We get
\[
  t' = \gamma (t - \frac{v}{c^2} x),
\]
which is the second part of a Lorentz transformation.

\begin{remark}
  Note that if $v$ is small compared to $c$, then $\gamma \approx 1$, so we
  recover Galilean transformations.
\end{remark}

\begin{remark}
  The transformation of time can be tricky to understand.
  Think about it this way:
  Time is what is measured by a clock.
  While in classical mechanics, every pair of clocks which ever agreed will
  always agree, in special relativity, this only holds for pairs of clocks which
  are stationary relative to each other (i.e.~there exists an inertial reference
  frame $S$ in which both clocks are stationary).
  Since the position of a clock doesn't matter, we can think of every reference
  frame as having a clock at its center, which measures the frame's time.
\end{remark}

\begin{definition}
  An \pojem{event} is a point $(t, \vec{x}) \in \R^{1+3}$.
\end{definition}

\begin{definition}
  A \pojem{worldline} is a function $(t, \vec{x}): \R \to \R^{1+3}$, for which
  $(t(\lambda), \vec{x}(\lambda))$ is an event for any $\lambda \in \R$.
  We call $\lambda$ the \pojem{affine parameter}.
\end{definition}

\begin{example}[time dilation]
  Let $S$ be a rest frame and $S'$ another frame moving with velocity $v$
  relative to $S$.
  Consider the time measured by a clock at $x' = 0$.
  Since $x = vt$, the Lorentz transformation of time gives us
  \[
	t' = \gamma(v) \left(1 - \frac{v^2}{c^2} \right) t
	= \inv{\gamma} t.
  \]
  Note that $\gamma > 1$, so in the moving system, the clock shows a lower time
  than in the non-moving system.
\end{example}

\begin{example}[length contraction]
  Consider a stick of length $l'$ in $S'$.
  We consider the ends of the stick to be two worldlines with $x_1' = 0$ and
  $x_2' = l'$.
  We wish to measure the length of the stick in the non-moving frame $S$.
  For this, we need to find two simultaneous events in $S$, one belonging to the
  first worldline and one belonging to the second.

  We'll take the measurement at $t=0$.
  From the Lorentz transformation of space, we then get
  \[
	l' = \gamma l,
  \]
  so the stick is shorter in $S$ than in $S'$.
\end{example}

Note that the expression $s^2 = c^2 t^2 - \vec{x}^2$ is invariant under Lorentz
transformations, as is the difference $\Delta s^2 = c^2 \Delta t^2 - \Delta
\vec{x}^2$ between two events.
This allows us to define $\Delta s^2$ as a sensible distance-like measure on
$\R^{1+3}$, called the \pojem{Minkowski metric}, though it is not a metric.
Depending on the value of $\Delta s^2$, we call a pair of events
\begin{itemize}
\item \pojem{timelike} if $\Delta s^2 > 0$,
\item \pojem{null} or \pojem{lightlike} if $\Delta s^2 = 0$, and
\item \pojem{spacelike} if $\Delta s^2 < 0$.
\end{itemize}
Lightlike events are those which can be connected with a light-speed signal.
Two timelike events can then be causally connected, as there was enough time
between them to transfer and process information.
Spacelike events, on the other hand, cannot be causally connected, since no
signal can travel faster than light, so there is no way for the information of
the first event to reach the second.
We can draw these events on a spacetime diagram, as shown in
figure~\ref{fig:spacetime-diagram-events}.

\begin{figure}[h!]
  \centering
  \begin{tikzpicture}[scale=0.7]
	\draw[dashed] (-5,-5) -- (5,5);
	\draw[dashed] (-5,5) -- (5,-5);

	\draw (-5,0) -- (5,0);
	\draw (0,-5) -- (0,5);

	\node[circle,fill,inner sep=1pt] (A) at (2, 3) {};
	\node[circle,fill,inner sep=1pt] (B) at (4, 1) {};
	\node[circle,fill,inner sep=1pt] (C) at (-3,3) {};

	\node[above] at (A) {timelike};
	\node[above] at (B) {spacelike};
	\node[below left] at (C) {lightlike};
  \end{tikzpicture}
  \caption{Spacetime diagram}%
  \label{fig:spacetime-diagram-events}
\end{figure}

% LocalWords:  von Philipp worldline worldlines Minkowski timelike spacelike
% LocalWords:  lightlike
