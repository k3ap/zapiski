\naslov{Special relativity}

\podnaslov{The Michelson-Morley Experiment}

Suppose there is some fluid medium at rest in a frame $S$.
Now, we create a plane wave in this medium, traveling in the $\vec{e}_x$
direction, with a velocity of $c_1$ as measured in $S$.
If we switch to a different frame $S'$ with a Galilean transformation $\vec{x}'
= \vec{x} + v t \vec{e}_x$, then we expect the plane wave to have a different
velocity $c_1 - v$ in $S'$.
This is true for waves in water, or sound waves in air, but, as famously
measured by Michelson and Morley, it is not true for light.
Their experiment, described below, showed that the speed of light $c$ is the
same in any inertial reference frame --- a result that contradicts entirely with
classical mechanics.

The Michelson-Morley experiment splits a beam of light into two beams traveling
orthogonally to each other, which then bounce off a pair of mirrors, before
being recombined and the interference pattern of this recombination being
measured.
Assume that $S$ is the rest frame of the ether, in which light waves are
traveling at $c$.
Suppose that we now switch to a frame $S'$, with a relative velocity of $v$ as
measured in $S$.
If one of the arms is parallel to the relative velocity vector, then the time of
flight of a wave of light should be
\[
  t_1 = \frac{L}{c + v} + \frac{L}{c-v} = \frac{2 c L}{c^2 - v^2},
\]
where $L$ is the distance between the beam splitter and the mirror.
In $S$, the other beam of light travels a path that is not parallel to
$\vec{e}_x$, as shown in figure~\ref{fig:michelson-morley-beam-path}.

\begin{figure}[h!]
  \centering
  \begin{tikzpicture}[>=Stealth]
	
	% Horizontal base line
	\draw[dotted] (-3,0) -- (3,0);
	\draw[thick] (-3.5,0) -- (-2.5,0);
	\draw[thick] (2.5,0) -- (3.5,0);

	% Wavy arrow for the beam of light
	\draw[thick,decorate,decoration={snake,amplitude=1mm,segment length=3mm}] (-3,0) -- (-3,1);

	% Vertical line in the center
	\draw[thin,<->] (0,0.1) -- (0,2.4);
	\node at (0.2,1.25) {$L$};

	% Arrow for light reflection
	\draw[dotted,-{Stealth[length=3mm]}] (0,2.5) -- (3,0);
	\draw[dotted,{Stealth[length=3mm]}-] (0,2.5) -- (-3,0);

	% Mirror at the top
	\draw[thick] (-1.5,2.5) -- (1.5,2.5);
	\node[above] at (0,2.5) {mirror};

	% Formulas at the bottom
	\draw[thin,<->] (-2.9,-0.2) -- (-0.1,-0.2);
	\node at (-1.5,-0.6) {$\frac{vt_2}{2}$};
	
	\draw[thin,<->] (2.9,-0.2) -- (0.1,-0.2);
	\node at (1.5,-0.6) {$\frac{vt_2}{2}$};
  \end{tikzpicture}
  \caption{Beam path in $S$}%
  \label{fig:michelson-morley-beam-path}
\end{figure}

From Pythagoras' theorem, the length of the beam path is
\[
  c t_2 = 2 \sqrt{L^2 + v^2 t_2^2 / 4}
\]
so
\[
  t_2 = \frac{2 L}{\sqrt{c^2 - v^2}}.
\]

As mentioned, Michelson and Morley conducted this experiment, while using the
Earth's rotation to perform many measurements in many directions.
They found that their experiment always gave the same result, no matter the
relative velocity of the second frame and no matter the orientation of the
testbed.
This demonstrated beyond any doubt that Max Planck's advisor Philipp von Jolly
was very wrong when he stated to Planck in 1874 that there was essentially
nothing left to discover in theoretical physics,%
\footnote{Source: \url{https://hsm.stackexchange.com/questions/2129/}}
since, as Feynman stated,
\begin{quote}
  \enquote{It doesn't matter how beautiful your theory is, it doesn't matter how smart
  you are. If it doesn't agree with experiment, it's wrong.}
\end{quote}
Classical mechanics and electromagnetism, as complete as might seem, is wrong.

\podnaslov{Lorentz, the new Galileo}

Enter Einsten, who formulated special relativity with the following postulates:
\begin{enumerate}
\item The principle of relativity: The laws of physics are identical in all
  inertial frames.
  This is already a postulate of classical mechanics, but we now have an
  additional one.
\item The speed of light in vacuum is a law of physics, so it is also the same
  in all inertial reference frames.
\end{enumerate}

% LocalWords:  von Philipp
