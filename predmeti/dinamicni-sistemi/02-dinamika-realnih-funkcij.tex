\naslov{Dinamika realnih funkcij}

\podnaslov{Osnovni pojmi}

Za začetek se bomo omejili na funkcije $f: I \to I$, kjer je $I$ interval v
$\R$.
Obravnavali bomo zaporedja $x_{n+1} = f(x_n)$ pri različnih začetnih pogojih
$x_0 \in I$.
Uvedimo oznako
\[
  f^n = \underbrace{f \circ f \circ \cdots \circ f}_{\text{$n$ pojavitev $f$}}.
\]

\begin{definicija}
  \pojem{Orbita} točke $x_0$ pri funkciji $f$ je množica
  $O_f(x_0) = \{ f^n(x_0) \such n \in \N_0 \}$.
  Množici vseh orbit za $x_0 \in I$ pravimo \pojem{dinamika funkcije $f$}.
\end{definicija}

\begin{primer}
  Za $x_{n+1} = x_n^2 + 1$ je orbita točke $1$ enaka $\{ 1, 2, 5, 26, \ldots \}$.
\end{primer}

Orbite lahko (za funkcije ene spremenljivke) vizualiziramo s
\pojem{pajčevinastim diagramom} (angl.~\textit{cobweb diagram}), prikazan na
sliki~\ref{fig:ds-02-pajcevinasti-diagram}.

\begin{figure}[h!]
  \centering
  {
  \newcommand{\startv}{.8}
  \begin{tikzpicture}[scale=5,declare function={f(\x)=\x*\x*\x;}]
	\draw[<->](0,1.1)|-(1.1,0);
	\draw[green!70!black] (0,0)--(1,1);
	\draw[blue, domain=0:1, smooth] plot (\x,{f(\x)});
	\foreach \t[evaluate=\t as \newv using f(\startv)] in {1,...,4}{
	  \draw[red] (\startv,\startv)|-(\newv,\newv);
	  \xdef\startv{\newv}
	}
  \end{tikzpicture}
  }
  \caption[Pajčevinasti diagram]{Pajčevinasti diagram za $f(x) = x^3$ z začetno
	točko $x_0 = 0.8$.\footnotemark{}}%
  \label{fig:ds-02-pajcevinasti-diagram}
\end{figure}

\footnotetext{Prirejeno po:
  \href{https://tex.stackexchange.com/questions/640022/how-to-draw-a-cobweb-graph-like-this-one}{stackexchange}}

\begin{definicija}
  Točka $x_0$ je \pojem{periodična} s periodo $n \in \N$ za $f$, če zanjo velja
  $f^n(x_0) = x_0$ in je $n$ najmanjše število s to lastnostjo.
  Če je $n = 1$, taki točki pravimo \pojem{fiksna točka}.
\end{definicija}

\begin{primer}
  Funkcija $f(x) = -x^3$ ima le eno fiksno točko, $0$.
  Izračunamo lahko, da imamo tudi dve točki periode $2$, to sta $1$ in $-1$.
\end{primer}

\begin{definicija}
  Orbiti $n$-periodične točke pravimo \pojem{$n$-cikel}.
\end{definicija}

\begin{definicija}
  Naj bo $x_0 \in I$ fiksna točka za $f: I \to I$.
  \begin{itemize}
  \item Točka $x_0$ je \pojem{šibko privlačna}, če obstaja okolica $U \ni x$, da
	za vsak $y_0 \in U$ velja $y_n = f^n(x_0) \to x_0$.
	Okolici $U$ pravimo \pojem{območje privlaka} za $x_0$, največjemu intervalu
	znotraj $U$ pa \pojem{neposredno območje privlaka} za $x_0$.
  \item Točka $x_0$ je \pojem{šibko odbojna}, če obstaja okolica $U \ni x$, da
	za vsak $y_0 \in U \setminus \{x_0\}$ obstaja $m \in \N$, da $f^m(y_0)
	\notin U$.
  \end{itemize}
\end{definicija}

\begin{definicija}
  Če je $f$ zvezno odvedljiva, uvedemo dodatne pojme. Točka $x_0$ je
  \pojem{privlačna}, če je $\abs{f'(x_0)} < 1$, \pojem{odbojna}, če je
  $\abs{f'(x_0)} > 1$, oziroma \pojem{nevtralna}, če je $\abs{f'(x_0)} = 1$.
\end{definicija}

\begin{primer}
  Funkcije $f_1(x) = x+x^3$, $f_2(x) = x - x^3$ in $f(x) = x + x^2$ imajo vse
  fiksno točko v $0$, a se obnaša drugače.
  V prvem primeru je $0$ šibko odbojna točka, v drugem šibko privlačna, v zadnjem pa
  niti ena niti druga.
  Preverimo lahko, da je v vseh primerih $\abs{f_i'(0)} = 1$, torej so vse točke
  odbojne.
\end{primer}

\begin{figure}[h!]
  \centering
  \begin{subfigure}{0.3\textwidth}
	\centering
	\begin{tikzpicture}[scale=1.9,declare function={f(\x)=\x*\x*\x+\x;}]
	  \draw[<->](0,1.1)|-(1.1,0);
	  \draw[<->](0,-1.1)|-(-1.1,0);
	  \draw[green!70!black] (-1,-1)--(1,1);
	  \draw[blue, domain=-0.7:0.7, smooth] plot (\x,{f(\x)});
	\end{tikzpicture}
	\caption{$f_1(x) = x + x^3$}
  \end{subfigure}
  \begin{subfigure}{0.3\textwidth}
	\centering
	\begin{tikzpicture}[scale=1.9,declare function={f(\x)=-\x*\x*\x+\x;}]
	  \draw[<->](0,1.1)|-(1.1,0);
	  \draw[<->](0,-1.1)|-(-1.1,0);
	  \draw[green!70!black] (-1,-1)--(1,1);
	  \draw[blue, domain=-0.9:0.9, smooth] plot (\x,{f(\x)});
	\end{tikzpicture}
	\caption{$f_2(x) = x - x^3$}
  \end{subfigure}
  \begin{subfigure}{0.3\textwidth}
	\centering
	\begin{tikzpicture}[scale=1.9,declare function={f(\x)=\x*\x+\x;}]
	  \draw[<->](0,1.1)|-(1.1,0);
	  \draw[<->](0,-1.1)|-(-1.1,0);
	  \draw[green!70!black] (-1,-1)--(1,1);
	  \draw[blue, domain=-0.9:0.7, smooth] plot (\x,{f(\x)});
	\end{tikzpicture}
	\caption{$f_3(x) = x + x^2$}
  \end{subfigure}
  \caption{Funkcije iz primera}%
  \label{fig:ds-02-primer}
\end{figure}

\begin{izrek}
  Naj bo $f \in \zvodv{1}{I}$ in $x_0 \in I$ fiksna točka.
  Potem velja:
  \begin{enumerate}
  \item Če je $\abs{f'(x_0)} < 1$, je $x_0$ šibko privlačna.
  \item Če je $\abs{f'(x_0)} > 1$, je $x_0$ šibko odbojna.
  \end{enumerate}
\end{izrek}

\begin{opomba}
  Če je točka privlačna, je šibko privlačna.
  Če je odbojna, je šibko odbojna.
\end{opomba}

\begin{proof}
  Naj bo $\lambda$ tak, da je $\abs{f'(x_0)} < \lambda < 1$.
  Potem obstaja $\delta > 0$, da je $\abs{f'(x)} < \lambda$ za vse $x \in (x_0 -
  \delta, x_0 + \delta)$.
  Po Lagrangeovem izreku obstaja $c \in (x, x_0)$, da velja
  \[
	\frac{f(x) - f(x_0)}{x - x_0} = f'(c).
  \]
  Velja $f(x_0) = x_0$, torej je $\abs{f(x) - x_0} = \abs{f'(c)} \abs{x - x_0} <
  \lambda \abs{x - x_0}$ in posledično $f(x) \in (x_0 - \delta, x_0 + \delta)$.
  Induktivno potem $\abs{f^n(x) - f(x_0)} < \lambda^n \abs{x - x_0}$, kar
  konvergira k $0$ za $n \to \infty$.

  Podobno za drugo točko.
\end{proof}

\begin{definition}
  Fiksna točka $x_0 \in I$ je \pojem{stabilna} za $f: I \to I$, če za vsako
  njeno okolico $U \subseteq I$ obstaja manjša okolica $U' \subseteq U$, da za
  vsak $x \in U'$ velja $O_f(x) \subseteq U$.
\end{definition}

\begin{opomba}
  Vse privlačne točke so stabilne (glej zgornji dokaz).
\end{opomba}

\begin{opomba}
  Stabilne šibke privlačne točke imenujemo \pojem{asimptotstko stabilne}.
\end{opomba}

\begin{primer}
  Poglejmo si $f(x) = 1 - x$.
  Ker je $f^2(x) = x$, ima $f$ fiksno točko v $x_0 = \pol$, vse ostale točke pa
  so periodične s periodo $2$.
  Točka $\pol$ je nevtralna (ni privlačna/odbojna), je pa stabilna.
  \begin{center}
	\renewcommand{\startv}{.8}
	\begin{tikzpicture}[scale=3,declare function={f(\x)=1-\x;}]
	  \draw[<->](0,1.1)|-(1.1,0);
	  \draw[green!70!black] (0,0)--(1,1);
	  \draw[blue, domain=0:1, smooth] plot (\x,{f(\x)});

	  \foreach \t[evaluate=\t as \newv using f(\startv)] in {1,...,3}{
		\draw[red] (\startv,\startv)|-(\newv,\newv);
		\xdef\startv{\newv}
	  }
	  \xdef\startv{0.6}
	  \foreach \t[evaluate=\t as \newv using f(\startv)] in {1,...,3}{
		\draw[red] (\startv,\startv)|-(\newv,\newv);
		\xdef\startv{\newv}
	  }
	\end{tikzpicture}
  \end{center}
\end{primer}

\begin{opomba}
  Če je $f$ zvezna in gre $f^n(x) \to x_0$, je $x_0$ nujno fiksna točka.
\end{opomba}

\begin{definicija}
  Naj bo $x_0$ $n$-periodična točka zvezne funkcije $f$.
  Pripadajoči $n$-cikel je \pojem{šibko privlačen}, če je vsak njegov element
  šibko privlačna točka za $f^n$ in \pojem{šibko obojen}, če je vsak njegov
  element šibko odbojna točka za $f^n$.
  Podobno definiramo privlačnost, odbojnost in nevtralnost cikla za zvezno
  odvedljivo $f$.
\end{definicija}

\begin{izrek}
  Če je $f \in \zvodv{1}{I}$, so vse periodične točke $n$-cikla istega tipa za
  $f^n$.
\end{izrek}

\begin{proof}
  Naj bo $\{x_1, \ldots, x_n\}$ cikel funkcije $f$.
  Recimo, da je $x_n$ šibko privlačna za $f^n$, torej obstaja okolica $U_n \ni
  x_n$, da za $x \in U$ velja $f^{nk}(x) \xrightarrow[k \to \infty]{} x_n$.
  Definiramo $U_{n-j}$ kot tisto povezano komponento praslike $f^{-j}(U_n)$, ki
  vsebuje $x_j$.

  Vemo, da za vsak $x \in U_{n-j}$ po konstrukciji velja $f^j(x) \in U_n$, torej
  $f^{nk+j}(x) \xrightarrow[k \to \infty]{} x_n$, iz česar sledi $f^{nk}(x)
  \xrightarrow[k \to \infty]{} x_{n-j}$ (uporabimo $f^{n-j}$ na prejšnji
  limiti).

  Dokaz za šibko odbojnost izpustimo.
  Za preostale tri lastnosti računamo
  \begin{align*}
	\abs{(f^n)'(x_n)}
	&= \abs{(f \circ f \circ \cdots \circ f)'(x_n)} \\
	&= \abs{f'(f^{n-1}(x_n))} \abs{f'(f^{n-2}(x_n))} \cdots \abs{f'(x_n)} \\
	&= \abs{f'(x_1)} \abs{f'(x_2)} \cdots \abs{f'(x_n)}.
  \end{align*}
  Podoben razvoj lahko naredimo v poljubni točki cikla, in dobimo enak rezultat.
\end{proof}

\begin{primer}
  Poglejmo si $f(x) = x^2 - 1$.
  Opazimo, da je $f(0) = -1$ in $f(-1) = 0$, torej imamo $2$-cikel.
  Izračunamo lahko $(f^2)'(0) = 0$, torej je cikel privlačen.
  Iz slike pa lahko vidimo, da se bližnje točke ne obnašajo kot cikel;
  privlačnost cikla ne pomeni, da privlači orbite.

  \begin{center}
	\renewcommand{\startv}{-.9}
	\begin{tikzpicture}[scale=2,declare function={f(\x)=\x*\x-1;}]
	  \draw[<->](0,1.5)|-(1.5,0);
	  \draw[<->](0,-1.5)|-(-1.5,0);
	  \draw[green!70!black] (-1.5,-1.5)--(1.5,1.5);
	  \draw[blue, domain=-1.3:1.3, smooth] plot (\x,{f(\x)});

	  \foreach \t[evaluate=\t as \newv using f(\startv)] in {1,...,20}{
		\draw[red] (\startv,\startv)|-(\newv,\newv);
		\xdef\startv{\newv}
	  }
	\end{tikzpicture}
  \end{center}

  Zanimiva je tudi točka $x = 1$, za katero je $f(1) = 0$, torej po prvi
  iteraciji postane ciklična.
  Takim točkam rečemo \pojem{predperiodične}.
  \boxdot{}
\end{primer}

\begin{opomba}
  Ta primer ravno nasprotno pokaže?
  Morda je treba vzeti polinome višje stopnje?
\end{opomba}

\podnaslov{Definicija kaosa}

\begin{definicija}[Devaney]
  Dinamični sistem, podan z $f: I \to I$ je \pojem{kaotičen}, če zanj veljajo
  naslednje lastnosti:
  \begin{enumerate}
  \item[(C1)] Množica periodičnih točk je gosta v $I$,
  \item[(C2)] Tranzitivnost: za poljubna odprta intervala $U_1, U_2 \subseteq I$
	obstajata $x_0 \in U_1$ ter $n \in \N$, da je $f^n(x_0) \in U_2$.
  \item[(C3)] Občutljivostna konstanta: Obstaja $\beta > 0$, da v poljubni
	okolici $U$ poljubne točke $x_0$ najdemo tudi točko $y_0 \in U$, za katero
	je $\abs{f^n(x_0) - f^n(y_0)} > \beta$ za nek $n \in \N$.
  \end{enumerate}
\end{definicija}

\begin{opomba}
  Tretja točka definicije je zanimiva pri izbiri majhnih okolic $U$, saj pomeni,
  da lahko vsaki točki poljubno blizu najdemo točko s popolnoma drugačno
  dinamiko.
  Temu pravimo \enquote{metuljev pojav.}
  Pomeni, da je orbita občutljiva na začetne pogoje.
\end{opomba}

\begin{opomba}
  Izkaže se, da je druga točka ekvivalentna obstoju goste orbite.
\end{opomba}

\begin{opomba}
  Če je $I$ kompakt, se izkaže, da prvi dve točki implicirata tretjo.
\end{opomba}

\begin{primer}[Podvojitvena preslikava]
  Definiramo $D: \zo{0,1} \to \zo{0,1}$ z
  \[
	D(x) = 2x - \lfloor 2x \rfloor
	=
	\begin{cases}
	  2x & x < \pol \\
	  2x-1 & x \ge \pol
	\end{cases}
  \]
  Za dokaz kaotičnosti uporabimo dvojiški zapis, kjer dodatno prepovemo
  neskončen niz enic.
  Naša preslikava potem število $0.x_1 x_2 x_3 \ldots$ slika v $0.x_2 x_3 x_4
  \ldots$, zato tej preslikavi pravimo tudi \enquote{operator zamika.}

  Naj bo $x \in \zo{0,1}$ poljubna.
  Potem je za dovolj velik $N \in \N$ točka
  \[
	\tilde{x} = 0. x_1 x_2 \ldots x_N x_1 x_2 \ldots x_N \ldots
  \]
  periodična in se nahaja v majhni okolici točke $x$.
  \boxdot{}
\end{primer}

% LocalWords:  predperiodične Devaney Podvojitvena
