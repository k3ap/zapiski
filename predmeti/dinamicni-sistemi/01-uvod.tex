\naslov{Uvod}

Dinamični sistem je kombinacija množice možnih stanj in evolucijskega pravila.
Obravnavamo delec v množici možnih stanj, pravilo pa nam pove, kaj se z njim
dogaja \enquote{v naslednjem trenutku.}
Pravilo je \pojem{deterministično}, kar pomeni, da je naslednji \enquote{korak}
odvisen le od trenutnega stanja delca.

Dinamične sisteme delimo na \pojem{diskretne} in \pojem{zvezne}.
Prvi so rekurzivne zveze $x_{n+1} = F(x_n)$ za neko funkcijo $F: S \to S$ in
evklidski prostor $S$, drugi pa so (nelinearni) sistemi navadnih diferencialnih
enačb $\dot{x} = F(x)$.
V obeh primerih gre za t.i.~\pojem{avtonomne sisteme}, v katerih pravilo ni
odvisno od časa.

\begin{opomba}
  Vsak sistem lahko prevedemo na avtonomen sistem, če zapakiramo čas kot dodatno
  spremenljivko.
\end{opomba}

\begin{primer}
  Sherlock Holmes najde truplo v jezeru, in izmeri njegovo temperaturo.
  Ob prihodu izmeri $\qty{9}{\celsius}$, eno uro kasneje $\qty{7}{\celsius}$,
  temperatura jezera pa je $\qty{5}{\celsius}$.
  Vprašanje je, kdaj je prišlo do umora.

  Fizikalni zakon pravi, da telesna temperatura pada sorazmerno z razliko do
  temperature jezera.
  Uporabimo lahko diskretni model,
  \[
	T_{n+1} = T_n - k (T_n - T_J),
  \]
  kjer je $T_J$ temperatura jezera, $k$ konstanta, $T_n$ pa temperatura ob
  $n$-tem trenutku.
  Če vstavimo meritve v to enačbo, dobimo $k = \pol$.
  Sistem lahko potem rešimo eksplicitno, in sicer dobimo
  \[
	T_n = \frac{C}{2^n} + \qty{5}{\celsius}
  \]
  za neko konstanto $C$.
  Če je $T_0 = \qty{37}{\celsius}$, izračunamo $C = \qty{32}{\celsius}$, iz
  česar dobimo, da je $T_3 = \qty{9}{\celsius}$, torej se je umor zgodil pred
  tremi urami.
  \boxdot{}
\end{primer}

\begin{primer}
  Zgornji problem lahko rešujemo tudi z zveznim modelom,
  \[
	\dot{T} = -k (T - T_J)
  \]
  za nek drug $k$ kot prej.
  Rešitev te diferencialne enačbe je $T = T_J + (T_0 - T_J) \exp(-kt)$, kjer smo
  že upoštevali začetni pogoj.
  Uporabiti moramo še meritve, za katere dobimo
  \begin{gather*}
	T(n) = (37 - 5) e^{-kn} + 5 = 9, \\
	T(n+1) = (37 - 5) e^{-k(n+1)} + 5 = 7,
  \end{gather*}
  oziroma $k = \log 2$ in $n = 3$.
  Enako kot prej torej sklepamo, da se je umor zgodil $3$ ure pred začetkom
  opazovanja.
  \boxdot{}
\end{primer}