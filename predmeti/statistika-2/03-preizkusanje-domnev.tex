\naslov{Preizkušanje domnev}

Obravnavamo model $\mathcal{P}$ za slučajni vektor $X$.
Privzemimo, da obstaja dekompozicija $\mathcal{P} = \mathcal{H} \cup
\mathcal{A}$ na neprazni disjunktni množici.
\pojem{Nerandomiziran poizkus domneve} $\mathcal{H}$ proti $\mathcal{A}$ je
odločitveno pravilo $\phi: \R^n \to \{0,1\}$, ki ga izvedemo tako:
\begin{itemize}
\item če je $\phi(x) = 1$, domnevo $\mathcal{H}$ zavrnemo,
\item če je $\phi(x) = 0$, domneve $\mathcal{H}$ ne zavrnemo.
\end{itemize}
Če je $\mathcal{P}$ parametriziran, torej v bijektivni korespondenci z množico
$\Theta$, potem označimo sliki $\mathcal{H}$ in $\mathcal{A}$ pod to bijekcijo s
$H$ in $A$.
Pravimo, da preizkušamo $H$ proti $A$.

Poznamo dve vrsti napake.
Če je v resnici $P_X \in H$, mi pa domnevo vseeno zavržemo, temu pravimo
\pojem{napaka prve vrste}, če pa $P_X \notin H$, a mi domneve ne zavržemo, je to
\pojem{napaka druge vrste}.
Pri primernih zveznostnih predpostavkah tipično velja
\[
  \sup \{ P_\theta(\text{napaka prve vrste}) \such \theta \in H \}
  + \sup \{ P_\theta(\text{napaka druge vrste}) \such \theta \in A \}
  = 1,
\]
torej popolnega preizkusa ni.
V praksi odločitev o zavrnitvi napravimo na podlagi neke testne statistike $T$ v
smislu
\[
  \phi(x) =
  \begin{cases}
	1, & T(x) \in B \\
	0, & T(X) \notin B
  \end{cases}
\]
za neko zavrnitveno območje $B$.

Zaradi komplementarnosti maksimalnih vrednosti napak obeh vrst izberemo
pomembnejšo in jo skušamo omejiti.
Po potrebi zamenjamo $H$ in $A$, da je to napaka prve vrste.
\pojem{Velikost preizkusa} je potem
\[
  \sup_{\theta \in H} P_\theta(\text{napaka prve vrste}).
\]
Pravimo, da ima preizkus \pojem{stopnjo značilnosti} ali \pojem{stopnjo
  tveganja} $\alpha$, če je njegova velikost največ $\alpha$.
Standardne izbire so $\alpha \in \{ 0.05, 0.1, 0.01 \}$.

\podnaslov{Preizkušanje na podlagi razmerja verjetij}

Privzamemo parametrični model s prostorom parametrov $\Theta \subseteq \R^d$ in
gladka verjetja $L(x,\theta)$.
Naj bo $H \subseteq \Theta$ preizkušana domneva.
\pojem{Razmerje verjetij} je $H$ je funkcija $\lambda: \R^n \to [0,1]$,
definirana z
\[
  \lambda(x) = \frac{
	\sup \{ L(x,\theta) \such \theta \in H \}
  }{
	\sup \{ L(x,\theta) \such \theta \in \Theta \}
  }.
\]
Vedno lahko potem konstruiramo preizkus oblike
\[
  \phi(x) =
  \begin{cases}
	1, & \lambda(x) < D \\
	0, & \lambda(x) \ge D
  \end{cases}
\]
za neko primerno konstanto $D$.

% LocalWords:  Nerandomiziran zveznostnih preizkušana
