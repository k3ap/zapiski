\naslov{Ocenjevanje za velike vzorce}

Naj bodo $X, X_1, X_2, \ldots$ slučajni vektorji, definirani na nekem skupnem
verjetnostnem prostoru.
\begin{itemize}
\item $(X_n)_n$ konvergira k $X$ \pojem{skoraj gotovo}, če je $P(\lim X_n = X) =
  1$.
\item $(X_n)_n$ konvergira k $X$ \pojem{v verjetnosti}, če za vsak $\varepsilon
  > 0$ velja $\lim P(\norm{X_n - X} > \varepsilon) = 0$.
\item $(X_n)_n$ konvergira k $X$ \pojem{v $L^2$}, če je $\lim E(\norm{X_n -
	X}_2^2) = 0$ ($\norm{\cdot}_2$ je funkcijska norma).
\item $(X_n)_n$ konvergira k $X$ \pojem{v porazdelitvi}, če velja $\lim
  F_{X_n}(x) = F_X(x)$ za vsako točko $x$, v kateri je $F_X$ zvezna.
\end{itemize}

\begin{lema}
  Naj bodo $X, X_1, X_2, \ldots$ slučajni $r$-vektorji.
  Potem velja naslednje.
  \begin{itemize}
  \item $X_n$ konvergira k $X$ v verjetnosti natanko tedaj, ko ima vsako
	podzaporedje zaporedja $(X_n)_n$ nadaljnje podzaporedje, ki konvergira
	skoraj gotovo.
  \item $X_n$ konvergira k $X$ v porazdelitvi natanko tedaj, ko za vsak $\xi \in
	\R^r$ velja $\sk{X_n, \xi} \xrightarrow{d} \sk{X, \xi}$.
  \end{itemize}
\end{lema}

\begin{trditev}
  Z enakimi oznakami, če $X_n$ konvergira k $X$ skoraj gotovo ali v $L^2$
  smislu, potem konvergira tudi v verjetnosti.
  Če konvergira v verjetnosti, potem konvergira tudi v porazdelitvi.
  Če je $X$ konstanten vektor, potem sta konvergenca v porazdelitvi in v
  verjetnosti ekvivalentni.
\end{trditev}

\begin{trditev}
  Naj bodo $X, X_1, X_2, \ldots$ slučajni $r$-vektorji in naj bo $g: \R^r \to
  \R^s$ Borelova funkcija, ki je zvezna skoraj povsod glede na $P_X$.
  Potem, če $X_n$ konvergira k $X$ skoraj gotovo, ali v verjetnosti, ali v
  porazdelitvi, potem tudi $g(X_n)$ konvergira k $g(X)$ v enakem smislu.
\end{trditev}

\begin{trditev}[Slucki]
  Naj bodo $X, X_1, X_2, \ldots$ in $Y_1, Y_2, \ldots$ slučajne spremenljivke.
  Dalje naj bo $c \in \R$.
  Privzemimo $X_n \xrightarrow{d} c$ in $Y_n \xrightarrow{d} c$.
  Tedaj $X_n + Y_n \xrightarrow{d} X+c$, $X_n Y_n \xrightarrow{d} cX$ in, če $c
  \ne 0$, $X_n / Y_n \xrightarrow{d} X/c$.
\end{trditev}

\begin{remark}
  Trditev lahko posplošimo na slučajne vektorje.
\end{remark}

\begin{primer}
  Naj bodo $X_1, X_2, \ldots$ neodvisne enako porazdeljene slučajne
  spremenljivke z disperzijo $\sigma^2$ in pričakovano vrednostjo $\mu$.
  Po centralnem limitnem izreku velja
  \[
	\frac{\cl{X} - \mu}{\sigma / \sqrt{n}} \xrightarrow[n \to \infty]{d} N(0,1).
  \]
  Potem po krepkem zakonu velikih števil
  \[
	\left(
	  \inv{n} \sum_i X_i,
	  \inv{n} \sum_i X_i^2
	\right)
	\xrightarrow[n \to \infty]{s.g.}
	(\mu, \sigma^2 + \mu^2).
  \]
  Upoštevaje zveznost potem velja
  \[
	S^2 = \frac{n}{n-1} \left( \inv{n} \sum_i X_i^2 - \cl{X}^2 \right)
	\xrightarrow[n \to \infty]{s.g.}
	\sigma^2
  \]
  za
  \[
	S := \sqrt{\inv{n-1} \sum_i (X_i - \cl{X})^2}.
  \]
  Po izreku Sluckega je potem
  \[
	\frac{\cl{X} - \mu}{S / \sqrt{n}} \xrightarrow[n \to \infty]{d} N(0,1).
  \]
\end{primer}

\begin{trditev}
  Naj bodo $Y, X_1, X_2, \ldots$ slučajni $r$-vektorji, $c \in \R^r$ ter
  $(a_n)_n$ zaporedje pozitivnih realnih števil z $a_n \to \infty$.
  Naj bo $g: \R^r \to \R$ Borelova funkcija, diferenciabilna pri $c$.
  Če velja
  \[
	a_n (X_n - c) \xrightarrow[n \to \infty]{d} Y
  \]
  in $dg(c) \ne 0$, potem velja tudi
  \[
	a_n (g(X_n) - g(c)) \xrightarrow[n \to \infty]{d} \sk{\grad g(c), Y} Y.
  \]
\end{trditev}

\begin{proof}
  Spomnimo se:
  Za vsak $\varepsilon > 0$ obstaja $\delta > 0$, da iz $\norm{x -c} \le \delta$
  sledi
  \[
	\abs{g(x) - g(c) - dg(c)(x-c)} \le \varepsilon \norm{x-c}.
  \]
  Izberimo neki $\varepsilon > 0$ in mu pridružimo $\delta$.
  Pišimo $Z_n = a_n (g(X_n) - g(X)) - a_n dg(c) (X_n - c)$.
  Pokazali bomo $Z_n \xrightarrow{p} 0$.
  Ko to vemo, lahko $Z_n$ prištejemo $a_n dg(c) (X_n - c)$ in bo rezultat
  sledil.

  Pokažimo, da za poljuben $\eta > 0$ velja $\lim P(\abs{Z_n} > \eta) = 0$.
  Velja
  \begin{align*}
	P(\abs{Z_n} > \eta)
	&= P(\abs{Z_n} > \eta \land \norm{X_n - c} > \delta)
	+ P(\abs{Z_n} > \eta \land \norm{X_n - c} \le \delta) \\
	&\le P(\norm{X_n - c} > \delta) + P(\abs{Z_n} > \eta
	  \land \abs{Z_n} \le a_n \varepsilon \norm{X_n - c}) \\
	&\le P(\norm{X_n - c} > \delta) + P(a_n \varepsilon \norm{X_n - c} > \eta).
  \end{align*}
  Ker $a_n^{-1} \to 0$ in $a_n (X_n - c) \xrightarrow{d} Y$, po Sluckem velja
  $X_n - c \xrightarrow{d} 0$.
  Sledi $P(\norm{X_n - c} > \delta) \xrightarrow{p} 0$.
  Ker je norma zvezna, velja tudi $\norm{a_n (X_n - c)} \xrightarrow{d}
  \norm{Y}$, zato
  \[
	\limsup P(\abs{Z_n} > \eta)
	\le 0 + \limsup P(\norm{a_n (X_n -c)} > \eta / \varepsilon).
  \]
  To je enako $1 - F_{\norm{Y}}(\eta / \varepsilon)$, če je $F_{\norm{Y}}$
  zvezna v točki $\eta / \varepsilon$.

  Točk nezveznosti je lahko največ števno mnogo.
  Izberemo lahko torej tako zaporedje $(\varepsilon_i)_i$, ki pada proti $0$, za
  katero je $\eta / \varepsilon_i$ točka zveznosti $F_{\norm{Y}}$.
  Za vsak $\varepsilon_i$ potem velja
  \[
	\limsup P(\abs{Z_n} > \eta) \le P(\norm{Y} > \eta / \varepsilon_i)
	\xrightarrow[i \to \infty]{} 0.
  \]
  Torej limita $\lim P(\abs{Z_n} > \eta)$ obstaja in je enaka $0$.
\end{proof}

\begin{posledica}[metoda $\delta$]
  Z enakimi oznakami, če $a_n (X_n - c) \xrightarrow[n \to \infty]{d} N(0,
  \Sigma)$,
  potem $a_n(g(X_n) - g(c)) \xrightarrow[n \to \infty]{d} N(0, dg(c) \Sigma
  dg(c)^T)$.
\end{posledica}

\begin{izrek}[krepki zakon velikih števil]
  Naj bodo $X_1, X_2, \ldots$ neodvisni enako porazdeljeni slučajni
  vektorji s pričakovano vrednostjo $\mu \in \R^r$.
  Potem
  \[
	\inv{n} \sum_i X_i \xrightarrow[n \to \infty]{s.g.} \mu.
  \]
\end{izrek}

\begin{izrek}[centralni limitni izrek]
  Naj bodo $X_1, X_2, \ldots$ neodvisni enako porazdeljeni slučajni vektorji z
  variančno matriko $\Sigma > 0$ ter pričakovano vrednostjo $\mu \in \R^r$.
  Tedaj
  \[
	\sqrt{n} \left( \inv{n} \sum_i X_i - \mu \right)
	\xrightarrow[n \to \infty]{d}
	N(0, \Sigma).
  \]
\end{izrek}

% LocalWords:  Borelova Sluckem
